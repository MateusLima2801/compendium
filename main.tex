\documentclass[12pt,a4paper]{article}
\usepackage[utf8]{inputenc}
\usepackage{amsmath}
\usepackage{chemfig}%coloca ligações e faz estruturas
\usepackage{amsfonts}
\usepackage{amssymb}
\usepackage{siunitx}
%http://www.bakoma-tex.com/doc/latex/siunitx/siunitx.pdf
\usepackage[portuguese]{babel}
\usepackage{setspace}%espaçamento
\usepackage{indentfirst}%indentação
\usepackage{enumitem}%enumerar personalizado
\usepackage{xcolor}%para por cor na pag
\usepackage{afterpage}%para colocar a pag amarela só em uma
\usepackage{graphicx}%colocar imagens
\usepackage{mathtools}%for curvy l \ell
\usepackage{textcomp}%degree symbol
\usepackage{tasks}%enumerate na horizontal
%\usepackage{tikz} %checkmark symbol \checkmark
\usepackage{multicol}%oara colocar multicolunas
\usepackage{color}  
\usepackage{hyperref}
\hypersetup{
    colorlinks=true, 
    linktoc=all,     
    linkcolor=black,
    urlcolor=blue,
}
\usepackage[a4paper, total={6in, 8in}]{geometry}
\usepackage{chemformula}


%comandos/macros
\newcommand{\R}{\mathbb{R}}
\newcommand{\C}{\mathbb{C}}
\newcommand{\modu}[1]{\vert #1 \vert}
\newcommand{\norm}[1]{\Vert #1 \Vert}
\newcommand{\parallelsum}{\mathbin{\!/\mkern-5mu/\!}}
\newcommand{\sen}{\mathrm{sen}}
\newcommand{\dd}{\mathrm{d}}
\newcommand{\Ln}{\ell\mathrm{n}}
\newcommand{\intin}{\displaystyle\int}
\newcommand{\intdef}[2]{\displaystyle\int\limits_{#1}^{#2}}
\newcommand{\original}[1]{\tiny \href{#1}{Original} \normalsize}

%Comandos para imagem
\newcommand{\imgh}[2]{\begin{figure}[h]
\centering
\includegraphics[width=#2cm]{#1.png}
\end{figure}}

\newcommand{\img}[3]{\begin{figure}[#3]
\centering
\includegraphics[width=#2cm]{#1.png}
\end{figure}}

%Comando pra por multiline comment
\newcommand{\hide}[1]{\iffalse #1 \fi}

%fontes estilizadas
\font\myfont=cmr12 at 32pt
\font\caiba=cmr12 at 10pt

%environments
\newenvironment{caixa}[1]
    {\begin{center}
    #1\\[1ex]
    \begin{tabular}{|p{0.9\textwidth}|}
    \hline\\
    }
    { 
    \\\\\hline
    \end{tabular} 
    \end{center}
    }
\newenvironment{caixa2}
    {\begin{center}
    \begin{tabular}{|p{0.5\textwidth}|p{0.36\textwidth}|}
    \hline\\
    }
    { 
    \\\\\hline
    \end{tabular} 
    \end{center}
    }
%---========-------=========--------========----
%Título -----------------------------

\title{{\myfont IME Compendium\\\Large Primeiro Período}} 
\author{IME XXI \\ IME XXII \\ IME XXIII}
\vspace{1cm}

\date{}

%==----=====----=====------=====----=======
%===============BEGIN======================
%========----=========-----================

\begin{document}

\onehalfspacing
\begin{titlepage}

\clearpage\maketitle
\thispagestyle{empty}

\pagecolor{purple}\afterpage{\nopagecolor}

\iffalse
\begin{figure}[h]
\centering
\includegraphics[width=8cm]{logo.png}
\end{figure}
\fi

{
\img{logo}{8}{h}
}

\begin{tasks}[label=\large{@}](3)
\task {\large{Little Frozen}}
\task {\large{Mocador do Bigode}}
\task {\large{Desmocador Camarada}}
\task {\large{Geniozinho do CSI}}
\task {\large{Hobbit da ZO}}
\task {\large{Terrorista do Sanhaço}}
\task {\large{Recruta Arregaçado}}
\task {\large{CJolo}}
\end{tasks}

\iffalse
\centering{\large{Little Frozen}}\\
\centering{\large{Henriques}}\\
\centering{\large{Hobbit da ZO}}\\
\centering{\large{Geniozinho do CSI}}\\
\centering{\large{Mocador do bigode}}\\
\centering{\large{Desmocador camarada}}\\
\centering{\large{Terrorista do Sanhaço}}\\
\centering{\large{CJolo}}
\fi
\end{titlepage}

\newpage

\tableofcontents

\newpage

\section{Prefácio}

A ideia do \textit{IME Compendium} é criar a cultura de guardar em \LaTeX \ as provas realizadas, de modo que turmas seguintes tenham acesso a essas provas de forma mais organizada.
O nome foi inspirado no \textit{IMO Compendium}, um compilado de questões de olimpíadas de matemática. Logo, apesar de a primeira versão ter apenas questões de provas antigas, o \textit{IME Compendium} poderá, posteriormente, ter também sugestões de questões que ajudaram alguns a estudar.

Importante ressaltar que o \textit{IME Compendium} não foi feito para fins lucrativos, tampouco busca utilizar o nome do IME para benefício pessoal de ninguém. Tem apenas por objetivo facilitar o estudo dos futuros alunos do IME: é um presente da IME XXI para as turmas posteriores.

Queremos agradecer a todos os alunos que contribuíram direta ou indiretamente para a confecção desse compilado (especialmente aos que atualizam o \textit{drive} com as provas).

\newpage

\section{Notas da 1ª Edição}

O \textit{IME Compendium} nos surpreendeu com mais de 300 páginas de provas antigas, superando as expectativas iniciais de "apenas" \ 200! Entretanto, ainda há pendências, que ficarão a cargo das próximas turmas: \textbf{adicionar as soluções das provas} e \textbf{adicionar as provas que virão a ser feitas}. É muito importante manter o Compendium vivo!

\subparagraph{Importante não comentar a existência desse documento com nenhum professor! Muitos não sabem (e não aprovam!) que os alunos possuem um banco de questões antigas.}

\newpage

\section{Cálculo I}

\subsection{VE 1}

\subsubsection{2019 \original{https://drive.google.com/open?id=1GLY88aSS4XACsZc02IgCHsqeOmPx6P1V}}

\textcolor{red}{Lembre-se : 
\begin{itemize}
    \item O uso da Regra de L'Hôpital, números complexos, aproximações e/ou integração não é permitido em qualquer questão da prova.
    \item Teoremas que não foram ministrados no curso devem ser provados.
    \item Todas as questões devem ser justificadas matematicamente.
\end{itemize}
}

\paragraph{Questão 1.} (3,0 pontos - 1,5 cada) Calcule ou mostre que não existe (Não use qualquer artifício que envolva derivada).
\begin{enumerate}[label = (\alph*)]
    \item $\lim \limits_{x \to 0} \left(\dfrac{\sen{(x)}}{x} \right)^{\frac{\sen{(x)}}{x - \sen{(x)}}} $
    \item $\lim \limits_{x \to 0} \dfrac{\sen{(4 \lfloor{x}\rfloor + x)}}{x}$ 
\end{enumerate}

\paragraph{Questão 2.} (2,0 pontos - 1,0 cada) \\
Dê o que se pede:
\begin{enumerate} [label = (\alph*)]
    \item Determine as constantes $a, b \in \mathbb{R}$ tais que: 
        \begin{equation*}
            \lim \limits_{x \to +\infty} \left[ax + b - \frac{x^{1000} + 1}{x^{999} + 1}\right] = 0 
        \end{equation*}
    \item Prove ou dê um contraexemplo:
    
    \centering{``Seja $f$ uma função definida e derivável em todo $\mathbb{R}$. Se a derivada de $f$ for contínua, então $f$ é derivável até a segunda ordem em todo $\mathbb{R}$''.}
\end{enumerate}

\paragraph{Questão 3.} (2,0 pontos) \\
Considere um cubo inscrito numa esfera. Conforme o lado do cubo aumenta, a esfera também aumenta, de forma que o primeiro esteja sempre inscrito na segunda. Sabendo que o lado do cubo varia a uma taxa de $3 \mathrm{cm/min}$, qual a taxa de variação do volume da esfera quando o cubo alcançar $10 \mathrm{cm}$?

\paragraph{Questão 4.} (3,0 pontos)
Seja $P(x)$ uma polinomial de grau ímpar. Seja $a$ um número real fixo tal que $P''(a) \neq 0$. Prove que para qualquer $y \in \left( 0, \frac{1}{2}\right)$ existe um número real $b \left( b \neq a\right)$ tal que:
    \begin{equation*}
        \frac{P(b) - P(a)}{b - a} = P'[yb + (1 -y)a]
    \end{equation*}
Dica: Analise a equação $P(x + a) - P(a) - xP'(yx + a)$.

\newpage

\subsubsection{2018 \original{}}

\textcolor{red}{Nessa prova não foi autorizado o uso de L'Hopital, muito menos integração, aproximação de Taylor e/ou complexos}

\paragraph{Questão 1: }(3,0 pontos - 1,5 cada) \\
Calcule ou mostre que não existe. (Não use qualquer artifício que envolva derivação).

\begin{enumerate}[label = (\alph*)]
\item $\lim \limits_{x \to 0^+} (1+x)^{\frac{1}{\arctan{x}}}$
\item $\lim \limits_{x \to 2} \dfrac{\mathrm{\tan {x - 2}}}{x^{\frac{1}{4}} - 2^{\frac{1}{4}}}$
\end{enumerate}

\paragraph{Questão 2:} (2,0 pontos - 1,0 ponto cada)

\begin{enumerate}[label = (\alph*)]
    \item Seja $f(x) = \mid x \mid $ se $x$ for racional, e $f(x) = 0 $ se $x$ for irracional. Verifique se $f$ é contínua em $x = 0$
    \item Considere a função $f(x) = a_{\textnormal{1}} + a_{\textnormal{2}} \cos{2x} + a_{\textnormal{3}} \cos{4x}$, onde $a_{\textnormal{1}}, a_{\textnormal{2}}, a_{\textnormal{3}}$ são números reais. Sabendo que $f(0) = f'(0) = f''(0) = f'''(0)$ e $f(\pi/2) = 1$, determine $a_{\textnormal{1}}, a_{\textnormal{2}}, a_{\textnormal{3}}$.
\end{enumerate}

\paragraph{Questão 3:}(2,0 pontos) Um cilindro circular reto é obtido girando-se um retângulo de diagonal constante e igual a \SI{5}{ \centi \meter}, em torno de um dos seus lados. Determine a taxa de variação do volume do cilindro no instante que a altura do cilindro é 
$\sqrt{5}cm$ 
e está aumentando à razão de $\SI{2}{\centi \meter \per \second}$.

\newpage

\paragraph{Questão 4:} (3,0 pontos) \\
Dado que:
\begin{align*}
u(x) = 1 + \frac{x^3}{3!} + \frac{x^6}{6!} + \frac{x^9}{9!} + \dots  \\
v(x) = x + \frac{x^4}{4!} + \frac{x^7}{7!} + \frac{x^{10}}{10!} + \dots  \\
w(x) = \frac{x^2}{2!} + \frac{x^5}{5!} + \frac{x^8}{8!} + \dots  \\
\end{align*}
Prove que $ u^3 + v^3 + w^3 - 3uvw = 1$. 

\newpage


%%%%%%%%%%%%%%
\subsubsection{2017 \original{https://drive.google.com/file/d/1IsTA8j9YEY4dMirQQ2oATMbmVwdLysBI/view?usp=sharing}}

\textcolor{red}{Nessa prova, não foi permitido utilizar derivadas, ou seja, nada de L'Hôpital! Também não valeu usar integral, complexos, nem aproximação de Taylor.}

\paragraph{Questão 1: }(3,0 pontos - 1,5 cada) \\
Calcule ou mostre que não existe. (Não use qualquer artifício que envolva derivação).

\begin{enumerate}[label=(\alph*)]

\item $\lim\limits_{x\to 0} \dfrac{2^{\frac{1}{x}}-1}
{2^{\frac{1}{x}}+1}$

\item $\lim\limits_{x \to 0} \dfrac{1-\cosh^2 x}{x^2\mathrm{arccos(senh}x)}$

\end{enumerate}

\paragraph{Questão 2: }(2,0 pontos - 1,0 cada)
\begin{enumerate}[label=(\alph*)]
\item Determine, sem usar qualquer artifício que envolva derivação, o valor da constante real $p$ para que o limite abaixo exista e seja finito. Calcule o valor do limite também.
$$\lim\limits_{x\to 0} \dfrac{\sqrt{1+x} - (1+px)}{x^2}$$

\item Prove ou dê um contraexemplo:
$$\text{"Se }\lim\limits_{x\to 0} f(x^2) \text{ existe e é finito, então }\lim\limits_{x\to 0}f(x) \text{ também existe e é finito".}$$

\end{enumerate}

\paragraph{Questão 3: }(2,0 pontos) \\
Considere um círculo inscrito em triângulo equilátero. Conforme o raio do círculo aumenta, o triângulo também aumenta, de forma que o primeiro esteja sempre inscrito no segundo. Sabendo que o raio do círculo varia a uma taxa de $\mathit{ cm/min}$, qual a taxa de variação da área do triângulo quando o raio do círculo alcançar $35 \mathit{  cm}$?

\paragraph{Questão 4: }(3,0 pontos - a) 0,5; b) 2,5) \hfill \scriptsize \hypertarget{calculove12017q4volta}{* }\hyperlink{calculove12017q4ida}{Solução}\normalsize\\
Seja $f: \R^+ \rightarrow \R^+$ uma função diferenciável em seu domínio. Considere que a igualdade abaixo é válida para quaisquer dois números $x$ e $y$ do domínio de $f$:

$$f(x)f\left(yf(x)\right) = f(x+y)$$

Considere também que:

$$\lim\limits_{x\to 0^+} \dfrac{f(x)-1}{x} = k$$

Onde $k\in \R^-$.

\begin{tasks}(2)
\task Determine, se possível, $\lim\limits_{x\to 0^+} f(x)$
\task Determine, se possível, $\dfrac{f^2(x)}{f^{\prime}(x)}$.
\end{tasks}


\newpage

\subsubsection{2016 \original{https://drive.google.com/file/d/11dNK93dA3wdr6dby8Vvfe-wC1w1neVfH/view?usp=sharing}}


\paragraph{$1^a$ Questão)} (3,0 pontos - 1,5 cada)

Calcule ou mostre que não existe. (Não use qualquer artifício que envolva derivação).

\begin{tasks}(2)
\task $\lim\limits_{x \rightarrow +\infty} \dfrac{\textrm{arctg}x}{\sqrt{x+\sqrt{x+\sqrt{x}}}-\sqrt{x}}$
\task $\lim\limits_{x\rightarrow 0}\dfrac{2x\textrm{cossec}^2x}{\textrm{tg}{\vert x \vert}}  $
\end{tasks}

\paragraph{$2^a$ Questão:}  (2,0 pontos - 1,0 cada)

Dê o que se pede, ou mostre não ser possível:

\begin{enumerate}[label=(\alph*)]

\item  Seja $f(x)=
\begin{cases}
 \dfrac{\textrm{sen}(11x-22)}{3x-6}, \ x>2 \\
 \dfrac{x^3+5x^2-32x+36}{x^3-3x^2+4}, \ x<2 \\
\end{cases}
$

Defina $f(2)$ de modo que $f$ seja contínua nos reais.
\item Seja $h(x)=g(f(x))$, calcule $h^\prime(3)$ onde $f(3)=5$,  $f(6)= 4, \ f^\prime(3)=7,  \ f^\prime(6)=10$, $g(3)=6, \ g(5)=11$, $g^\prime(3)=10$ e $g^\prime(5)=-1$. 
\end{enumerate}

\paragraph{$3^a$ Questão:} (2,0 pontos)

A resistência total de duas resistências conectadas em paralelo é dada pela fórmula 
$$R_t= \dfrac{R_1R_2}{R_1+R_2} $$
onde $R_t, R_1$ e $R_2$ são medidas em ohm, $R_1$ E $R_2$  estão aumentando a uma taxa de $1$ e $1,5 \ \Omega\textrm{/s}$, respectivamente. 

Qual a taxa de variação de $R_t$ quando $R_1=50\Omega$ E $R_2=75\Omega$? 

\paragraph{$4^a$ Questão:} (3,0 pontos)

Seja $P(x)$ uma polinomial real de grau não superior a $2$ tal que $\vert P(x) \vert \leq 1$ em $x\in [-1,1]$. Sabendo que $\vert P^\prime(x) \vert \leq S $  em $x \in (-1,1)$,  determine o menor  valor de $S$ que atende a todos os polinômios que satisfaçam as condições anteriores. Dê o exemplo de uma $P(x)$ no caso limite.

\noindent \textbf{Lembre-se: \\ -O uso da Regra de L'Hopital, aproximações polinomiais, números complexos e/ou integração não é permitido em qualquer questão da prova. \\ -Nem todos os dados da prova precisam ser usados nas questões da prova. \\ -Teoremas que não foram ministrados no curso devem ser provados. \\ -Todas as questões devem ser justificadas matematicamente.}
 \newpage
\subsubsection{2015 \original{https://drive.google.com/file/d/1vmdYSMDPtf2bYDcNAfgM1HinpLRuzqn_/view?usp=sharing}}


\paragraph{Questão 1: }(3,0 pontos - 1,5 cada)\\
Calcule ou mostre que não existe:
\begin{tasks}(2)
\task $\lim\limits_{x\to \frac{\pi}{6}} \dfrac{\mathrm{sen}\left(x+\dfrac{5\pi}{6}\right)}{\mathrm{cotg}^3(x) - 3\mathrm{cotg}(x)}$
\task $\lim\limits_{x\to 0} \dfrac{7^x - 7^{-x}}{7x}$
\end{tasks}

\paragraph{Questão 2: }(2,0 pontos - 1,0 cada)\\
Dê o que se pede:
\begin{tasks}(1)
\task Seja $f(x) = xg(x)$, onde $g(x)$ é uma função contínua e não diferenciável em $x=0$. Determine se $f$ é diferenciável em $x=0$.
\task Use o teorema do valor intermediário para determinar se existe pelo menos uma solução $x$ da equação $\frac{(x^3 - 2x)}{x+3} = -3$ com $-2\le x \le 2$.
\end{tasks}

\paragraph{Questão 3: }(2,0 pontos)\\
O ângulo formado pelos lados de mesmo comprimento $S$ de um triângulo isósceles é denotado $\theta$. Se $\theta$ está crescendo a uma taxa de 0,5 rad/min e $S$ é fixo e mede 3 cm, calcule a taxa de variação da área quando $\theta = (\pi/6)$\textdegree.

\paragraph{Questão 4: }(3,0 pontos)\\
Seja $f$ e $g$ duas funções reais de uma variável real, não constantes e diferenciáveis em $\R$. Além disso, suponha que:
$$f(x+y) = f(x)f(y) - g(x)g(y)$$
$$g(x+y) = f(x)g(y) + g(x)f(y)$$
Válido para todos os números $x$ e $y$ reais.
Se $f^{\prime}(0) = 0$, prove que a soma $\left(f(x)\right)^2 + \left(g(x)\right)^2$ é constante $\forall x \in \R$ e determine o seu valor.

\newpage

\subsubsection{2013  \original{https://drive.google.com/file/d/1CqLJEQ23GQSyvgVnIj5N87Fysyl9ZtSt/view?usp=sharing}}

\paragraph{Questão 1: }(3,0 pontos - 1,5 cada)\\
Calcule ou mostre que não existe:
\begin{tasks}(2)
    \task $\lim\limits_{x\to \frac{\pi}{4}}^{-} \left(\tanx\right)^{(\tan2x)}$
    \task $\lim\limits_{x\to 0} \dfrac{\tanh^2{x}}{\cosh x - 1}$
\end{tasks}

\paragraph{Questão 2: }(2,0 pontos - 1,0 cada)\\
Dê o que se pede:
\begin{tasks}(1)
\task Seja $f(x)= x^2$ se $x$ for racional, e $f(x) = 0$ se $x$ for irracional. Verifique se $f$ é diferenciável em $x=0$.
\task Seja$g: \R \to \R$ uma função diferenciável tal que $g(1) = 2$ e $g^{\prime}(1) = 3$. Calcule $f^{\prime}(0)$, sendo $f$ dada por $f(x) = e^x g(3x+1)$.
\end{tasks}

\paragraph{Questão 3: }(2,0 pontos)\\
Acumula-se areia em um monte de forma cônica, a taxa de 10 $\mathrm{dm}^3$/min. Se a altura do monte é sempre igual ao dobro do raio da base, a que taxa cresce a altura do monte quando esta é igual a 7 dm?

\paragraph{Questão 4: }(3,0 pontos)\\
Seja $f$ uma função real de uma variável real tal que 
$$xf(y) + yf(x) = (x+y)f(x)f(y)$$
Válido para todos os números $x$ e $y$ reais.
Determine $f^{\prime}(0)$ (ou mostre que não existe) de todas as funções que satisfaçam a igualdade acima.

\newpage

\subsubsection{2012  \original{https://drive.google.com/file/d/1thorAtniQglGnXdZSaHm4jAWuZ0GTwfN/view?usp=sharing}}

\paragraph{Questão 1} (5,0 pts). Nas sentenças abaixo, preencha com V, caso seja verdadeira ou $F$, caso seja falsa, justificando cada caso. Nos casos em que a sentença seja falsa, basta apresentar um contra-exemplo. Sentenças sem justificativas não serão consideradas.

\begin{tasks}(1)
\task (2,0 pts) ( \quad  ) $\lim\limits_{x\to p} f(x) = 0$ se e somente se $\lim\limits_{x\to p}\vert f(x) \vert = 0$.
\task (1,5 pts) ( \quad  ) Se $f: \R \to \R$ for contínua em $p$, então existe $\delta > 0$, tal que $f$ é contínua $\forall x \in (p - \delta,p+\delta)$.
\task (1,5 pts) ( \quad  ) Se $a \in D_g$ e existem $\lim\limits_{x\to p}f(x) = a,\, \lim\limits_{u\to a}g(u) = b$ e $\lim\limits_{x\to p}g \circ f(x) = c$.
\end{tasks}

\paragraph{Questão 2} (2,0 pts). Calcule os limites a seguir:
\begin{tasks}(2)
\task (1,0 pts) $\lim\limits_{x\to 0} \dfrac{\mathrm{log}_2 \left[\cos^2\left(\dfrac{x}{\pi}\right)+1\right]}{x^{-3}}$
\task (1,0 pts) $\lim\limits_{x\to +\infty} \left[x^{\frac{1}{2x}} - \left(\mathrm{ln} 2x \right)^{\frac{1}{x}} \right]$
\end{tasks}

\paragraph{Questão 3} (2,0 pts). Determine os valores de $k,\, h \in \R$ para que a função abaixo seja contínua em $x=0$:

$$f(x) = \begin{cases}\dfrac{k - k\cos{x}}{x^2},\quad \text{ se }x<0 \\ 1, \qquad \text{ se } x=0 \\ \dfrac{h^x - h^{-x}}{4x},\quad \text{ se } x>0 \end{cases}$$

\paragraph{Questão 4} (1,0 pts). Seja $f: [0,a] \to \R$, tal que $0< f(x) < a,\, \forall x \in [0,a]$. Prove que existe pelo menos um ponto $y$ em $[0,a]$ para o qual $f(y) = y$.

\newpage

\subsubsection{2011  \original{https://drive.google.com/open?id=12Kp8xZQmZvnoQS_m-H4aGk_Uo2QkQlu3}}

\paragraph{Questão 1:} (3,0 pontos)\\
Calcule ou mostre que não existe:
\begin{tasks}(1)
\task $\lim\limits_{x\to 0} \dfrac{\sin \frac{1}{x}}{1+e^{\frac{1}{x}}}$
\task $\lim\limits_{x\to 1} \dfrac{\sin (x^2-1)}{x^2+2x-3}$
\task $\lim\limits_{x\to +\infty} (x^2-1)^{\mathrm{arcsec}\left(1+\frac{1}{x^2-1}\right)}$
\end{tasks}

\paragraph{Questão 2:} (2,0 pontos)\\
Seja $f: \R \to \R$ uma função tal que $lim_{x\to 0} f(x) = 0$ e $\lim\limits_{x\to 0}\dfrac{f(x) - f(ax)}{x} = 0$ sendo $a \in (0,1)$ uma constante. Prove que:

\begin{enumerate}[label=(\alph*)]
\item $\lim\limits_{x\to 0} \dfrac{f(x) - f(a^n x)}{x} = 0, \, \forall n \in \mathbb{Z}^+$.

\item $\lim\limits_{x\to 0} \dfrac{f(x)}{x} = 0$
\end{enumerate}

\paragraph{Questão 3:} (2,0 pontos)\\
Seja $f: \R \to \R$ uma função derivável satisfazendo $\lim\limits_{h\to 0} \dfrac{f(x+2h) - f(x+h)}{h} = 0,\, \forall x \in \R$. Prove que $f^{\prime}(x) = 0,\, \forall x \in \R$.

\paragraph{Questão 4:} (3,0 pontos)\\
Dada a função $f(x) = \sqrt{\dfrac{1-x^3}{1-x^4}}$ se $x\neq 1$ e $f(1)=c$, 

\begin{enumerate}[label=(\alph*)]
\item Determine o valor de $c$ de modo que $f$ seja contínua.
\item Mostre que $f$ é derivável no ponto $x=1$, sem utilizar regra da cadeia.
\end{enumerate}

\newpage
\subsubsection{2010  \original{https://drive.google.com/open?id=1vwrcdzuQ7-nUAPLrX78B0JDDdOgpcEIB}}

\paragraph{Questão 1:} (2,5 pontos)\\
Seja $f(x) = x^2 \mathrm{sen} \left(\dfrac{1}{x}\right)$ se $x\neq 0$ e $f(0) = 0$. Mostre que $f$ é derivável, mas sua derivada não é contínua.

\paragraph{Questão 2:} (2,5 pontos)\\
Seja $y=f(x)$ uma função dada implicitamente pela equação $[g(x)]^2 + y^4 = 16$, onde $g(x)$ é uma função qualquer definida em $\R$. Calcule $\lim\limits_{x\to +\infty} \dfrac{f(x)+g(x)}{x}$. 

\paragraph{Questão 3:} (2,0 pontos)\\
Calcule $\lim\limits_{x\to \frac{\pi}{2}} \left(\mathrm{sen}x\right)^{\frac{1}{\cos x}}$ 

\paragraph{Questão 4:} (3,0 pontos)\\
Seja $f:\R \to \R$ uma função contínua, estritamente decrescente e não necessariamente derivável,

\begin{tasks}(1)
\task \textbf{(1,5 pontos)} Mostre que existe um ponto $x_0 \in \R$, tal que $f(x_0) = x_0$. Este ponto é denominado \textit{ponto fixo de }$f$. Mostre também que ele é único.
\task \textbf{(1,5 pontos)} Prove que o sistema de equações $x=f(y), \,y=f(z),\,z=f(x)$ tem solução única. (Sugestão: usar item a) )
\end{tasks}

\newpage

\subsubsection{2008  \original{https://drive.google.com/open?id=1lLryQY20lhsav0_b_KpqDdEhK8T-ZmI7}}

\paragraph{Questão 1 (1,5 pontos)}

Provar, por indução, que a igualdade abaixo é válida para qualquer $n$ inteiro positivo.$$\dfrac{1}{n+1} +\dfrac{1}{n+2} +\dots+\dfrac{1}{2n} = 1 - \dfrac{1}{2} + \dfrac{1}{3} - \dots + \dfrac{1}{2n-1}-\dfrac{1}{2n}$$

\paragraph{Questão 2 (5 pontos)}

\subparagraph{Letra a)} Calcule o limite $L = \lim\limits_{n\to \infty} \dfrac{1^p + 2^p + \dots + n^p}{n^{P+1}}$
\subparagraph{Letra b)} Mostre que $\sum\limits_{k=1}^{n} (k^2+1)k! = k(k+1)!$
\subparagraph{Letra c)} Calcule os limites abaixo.
    \begin{tasks}[counter-format={(tsk[r])}](3)
    \task $\lim\limits_{x \to \frac{\pi}{2}} (\mathrm{sen}x)^{\frac{1}{\cos x}}$
    \task $\lim\limits_{x \to 0} \dfrac{\mathrm{arctg}x}{x}$
    \task $\lim\limits_{x\to 0}(1+\mathrm{arctg x})^{\frac{1}{x}}$
    \end{tasks}


\paragraph{Questão 3 (1,5 pontos)}

Seja uma função $f$ contínua. Determinar todas as $f$ tal que $f: [0,1] \to \R$ e $\int\limits_{0}^{1} f(x)[x - f(x)] \mathrm{d}x = \dfrac{1}{12}$.

\paragraph{Questão 4 (1 ponto)}

São dadas as equações de duas curvas polares.
$$c_1: r= 3\mathrm{sen}\theta \text{ e }c_2: r = 2\sqrt{2} - \mathrm{sen}\theta$$

\noindent Determinar as áreas das regiões interior e exterior às curvas e esboçar o gráfico delas. \normalsize

\paragraph{Questão 5 (1 ponto)}

Uma função $f$ é estritamente crescente e contínua, com $f(0) = 0$. Mostrar que $$\int\limits_{0}^{a} f(x)\mathrm{d}x + \int\limits_{0}^{b} f^{-1}(x)\mathrm{d}x \ge ab$$

e que a igualdade ocorre quando $b=f(a)$.

\newpage

\subsubsection{2004  \original{https://drive.google.com/open?id=1rwT6guwYhYzv9RXt-GaY-DZw5l7VbjB5}}


\paragraph{Questão 1} (1,5 Pontos)\\
Calcule os limites, caso existam:

\begin{tasks}(2)
\task $\lim\limits_{x\to a} \dfrac{\cos x - \cos a}{x-a}$
\task $\lim\limits_{x\to 0^+} x\mathrm{ln}x$
\task $\lim\limits_{x\to 0} \dfrac{\mathrm{sen}(1/x)}{1+e^x}$
\end{tasks}

\paragraph{Questão 2} (1,5 Pontos)\\
Calcule os limites, caso existam:

\begin{tasks}(2)
\task $\lim\limits_{x\to 0^+} x^{(x^4-1)}$
\task $\lim\limits_{n\to \infty} n[(a+\dfrac{1}{n})^4 - a^4]$
\task $\lim\limits_{n\to \infty} \dfrac{a^n - b^n}{a^n + b^n}$
\end{tasks}

\paragraph{Questão 3} (1,5 Pontos)\\
Seja a função $f$ definida pela expressão:$$f(x) = x\mathrm{sen}(1/x),\, x\in \R.$$

\begin{tasks}(1)
\task Prove que $\lim\limits_{x\to 0} f(x) = 0$.
\task A função $f$ é contínua em $x=0$? Caso não seja, redefina $f$ de modo a torná-la contínua.
\end{tasks}

\paragraph{Questão 4 (1,5 Pontos)}

Calcule as derivadas:

\begin{tasks}(2)
\task $f(x) = \sqrt{x+\sqrt{x+\sqrt{x}}}$
\task $f(x) = \mathrm{arcsen}(1/x)$
\task $f(x) = \dfrac{x^3(3-x)^{\frac{1}{3}}}{(1-x)(3+x)^{\frac{2}{3}}}$
\end{tasks}

\paragraph{Questão 5 (1,5 Pontos)}

Uma função é definida como segue: $$f(x) = \dfrac{1}{\modu{x}} \text{ se }\modu{x}>0 \text{ e } f(x) = a + bx^2 \text{ se }\modu{x} \le c$$

Encontre valores de $a$ e $b$ em termos de $c$, tais que $f^{\prime}(c)$ exista.

\paragraph{Questão 6 (1,5 Pontos)}

Calcule a integral: $$I = \int\limits_{0}^{\frac{n}{2}} \vert \mathrm{sen} x - \cos x \vert \mathrm{d}x.$$

Esboce o gráfico do integrando e dê uma interpretação geométrica para o resultado.

\paragraph{Questão 7 (1,5 Pontos)}

Um sólido tem base circular de raio $2$. Cada seção transversal cortada por um plano perpendicular a um diâmetro fixo é um triângulo equilátero. Calcule, por integração, o volume do sólido.

\newpage

\subsubsection{2003 \original{https://drive.google.com/open?id=1HM9q4TJG9IX1hN_5smiDRbjzq2JZucgM}}


1) Calcule os limites:
\begin{enumerate}[label=\alph*)]
\item $\displaystyle{\lim_{x \to 0^+}} \dfrac{\cos{(a\sqrt{x})}-\cos{(b\sqrt{x})}}{\sqrt{x}}$
\item $\displaystyle{\lim_{x \to 0}}\dfrac{\sqrt[3]{x^2}-tg(\sqrt[3]{x})}{\sen{\sqrt[3]{x}}}$
\item $\displaystyle{\lim_{x \to \infty}}(2n\textrm{ln}(n))^{\frac{1}{2n}}
$ \\
\end{enumerate}

2) Calcule a derivada das seguintes funções:

\begin{enumerate}[label=\alph*)]
\item $f(x)=\sqrt[3]{\dfrac{1-x}{1+x}}$
\item $g(x)=\textrm{arcsen}(\sen{(x)}-\cos{x})$
\item $h(x)=x^{\textrm{ln}x}$
\end{enumerate}

3) Seja $c\neq 0$ e a função: 
$$
\begin{cases}
f(x)=1/x^2\textrm{, se }x\geq c\\
f(x)=a+bx^2\textrm{, se }x\leq c
\end{cases}
$$
Calcule $a$ e $b$ para que $f'(c)$ exista.
\\

4) Calcule a área abaixo do gráfico de $f(x)=|senx-cosx|$ de $x=0$ até $x=\pi$.
\\

6) Seja a sequência $(x_n)$ que satisfaz $\dfrac{1
}{x_{n+2}}=\dfrac{1}{x_{n+1}}+\dfrac{1}{x_n} \ ; \ x_0=1, \ x_1=1$. Calcule $\displaystyle{\lim_{x\to \infty}}\dfrac{x_{n+1}}{x_n}$.

\newpage

\subsubsection{2002  \original{https://drive.google.com/open?id=1__NA61VldSD-MI9_8ujzrrdRjg_uC8ZS}}


%aqui


\paragraph{Questão 1} (1,5 Pontos)\\
Calcule os limites:

\begin{tasks}(2)
\task $\lim\limits_{x\to 0} \dfrac{e^{a\sqrt{x}} - e^{b\sqrt{x}}}{\sqrt{x}}$
\task $\lim\limits_{x\to 0} \dfrac{1-\cos (\sqrt[3]{x^2})}{\sqrt[3]{x}\mathrm{sen}(\sqrt[3]{x})}$
\end{tasks}

\paragraph{Questão 2} (1,0 Ponto)\\
Calcule as derivadas:

\begin{tasks}(3)
\task $f(x) = \sqrt{\dfrac{x}{1+x^2}}$
\task $f(x) = \mathrm{arctg}(\tan^2 x)$
\task $f(x) = x^{\cos x}$
\end{tasks}

\paragraph{Questão 3} (1,0 Ponto)\\
Considere as funções $f(x) = x - x^2$ e $g(x) = ax$.

\begin{tasks}
\task Determine o valor de $a$ de modo que a área da região $R$ limitada pelos gráficos das duas funções seja igual a $9/2$.
\task Faça um esboço da região $R$ obtida.
\end{tasks}

\paragraph{Questão 4} (1,0 Ponto)\\
Os meteorologistas têm interesse na expansão adiabática de grandes massas de ar, em que as temperaturas podem variar, mas nenhum calor é adicionado ou retirado. A lei de transformação adiabática para o ar é $pV^{\frac{7}{5}} = c$, onde $p$ é a pressão, $V$ é o volume e $c$ é uma constante. O volume de uma certa câmara de ar isolada está decrescendo uniformemente a uma taxa de $2,96 \cdot 10^{-2} \mathrm{m^3/s}$. Determine a taxa de variação da pressão, em pascal por segundo, no instante em que a pressão vale $45\, \mathrm{N/cm^2}$ e o volume é de $37\cdot 10^{-2} \mathrm{ m^3}$.

\paragraph{Questão 5} (1,0 Ponto)\\
Considere a sequência ${x_n}$ definida por $0<x_1<1$ e $x_{n+1} = 1 - \sqrt{1-x_{n}},\, n\ge 1$.

\begin{tasks}(1)
\task Sabendo que toda sequência monótona (i.e., crescente ou decrescente) e limitada converge, mostre que ${x_n}$ converge.
\task Calcule $\lim\limits_{n\to \infty} x_n$.
\task Calcule o limite $\lim\limits_{n\to \infty} \dfrac{x_{n+1}}{x_n}$.
\end{tasks}

%aqui

\newpage

\subsubsection{1999 \original{https://drive.google.com/open?id=1cQN-_Ws-5FsVpCYxmamlsEXPs_JlqTXu}}

\vspace{1cm}

\paragraph{Questão 1:}
Dada a função $y= \tan \left(
2 \mathrm{arccot \left({\dfrac{x}{2}
}\right)}\right)$, mostre que
$\dfrac{\mathrm{d} y}{\mathrm{d} x}= \dfrac{4(1+y^{2})}{4+x^{2}}.$

\vspace{1.5cm}

\paragraph{Questão 2:}
Seja $f(x)=(\sen x^2)^{3x}$ calcule $f'(x)$.

\vspace{1.5cm}

\paragraph{Questão 3:}
Se $f(x)=\Ln\begin{pmatrix}\dfrac{\sen x}{x}\end{pmatrix}$ calcule $f'\begin{pmatrix}
\dfrac{\pi }{4}
\end{pmatrix}$.

\vspace{1.5cm}

\paragraph{Questão 4:}

Calcule a área da região compreendida entre $y_{1}=\dfrac{x^{2}}{3}$ e $y_{2}=4-\dfrac{2x^2}{3}$.

\vspace{1.5cm}

\paragraph{Questão 5:}

Calcule o valor de $c$ sabendo que $\displaystyle\int\limits_{c}^{0}\left \vert x(1-x) \right \vert\,\dd x=0$.


\newpage


\subsection{VC}

\subsubsection{2019 \original{}}

\paragraph{1ª QUESTÃO)} (3,0 pontos)

Um determinado sistema tem sua capacidade $C(x)$ variando com um parâmetro $x$ através de dois modelos definidos dependentes do valor de $x$. Quando $x$ é não-negativo sabe-se que $C^3(x) = (x-4)(x-1)^2$, e quando $x$ é negativo sabe-se que $[C(x) - 3 ]x^2 = 2x+1$. Você é o engenheiro responsável e precisa entender o total funcionamento do sistema com a variação do parâmetro $x$. Com isso:
\begin{enumerate}[label = (\alph*)]
    \item Faça o esboço do modelo completo da capacidade do sistema para a variação de $x$ em todo $\mathbb{R}$. (Obs.: Devem ser apresentados todos os cálculos antes do desenho do esboço, englobando todos os passos vistos em teoria em sala).
    \item Os pontos críticos obtidos na capacidade do sistema quando $x$ varia em todo $\mathbb{R}$.
    \item Os pontos de inflexão obtidos na capacidade do sistema quando $x$ varia em todo $\mathbb{R}$.
    \item Sabendo que o sistema funcionará apenas para $x \leq - \left(\dfrac{1}{2}\right)$ e $0 \leq  x \leq 3$, determine, caso existam, os máximos e mínimos globais para cada um dos intervalos de funcionamento.
\end{enumerate}

\paragraph{2ª QUESTÃO)} (2,0 pontos)
Uma determinada máquina deve operar em velocidade constante de rotação. Sua jornada de trabalho é limitada entre $64 < v < 128$ rotações por minuto. O custo energético da máquina, medido por $kW$, é de R\$ 2,00 por minuto e é consumido em uma razão de $(64 + \dfrac{v^2}{16})kW$ por minuto. O operador da máquina é pago a um custo de $H$ reais por minuto. Considere que a velocidade de rotação constante é obtida logo que a máquina entra em funcionamento. Determine para cada valor de $H$, a velocidade de rotação que se obtém o custo mínimo consumido pela máquina em uma jornada de trabalho, sabendo que $0 \leq H \leq 2000$.

\newpage

\paragraph{3ª QUESTÃO)} (1,0 ponto)

Seja $f:[a,b] \rightarrow \mathbb{R}$ derivável em $(a,b)$. Suponha que $f(a)=f(b)=0$. Prove que existe um ponto no intervalo $[a,b]$ tal que $f'(x)=kf(x), k \in \mathbb{R}$.\\
\textbf{Dica:} Use um função auxiliar $p(x)=f(x)e^{-kx}$.

\paragraph{4ª QUESTÃO)} (1,5 ponto)

Seja $f(x) = \ln{1 + \dfrac{1}{x}} - \left( \dfrac{1}{1+x} \right); x > 0$.
Prove (usando derivação) que $f(x) > 0; \forall x>0$ ou prove (usando teoremas de continuidade) que existe um número $c>0$ tal que $f(c) \leq 0$.

\paragraph{5ª QUESTÃO)} (2,5 pontos - a)1,2; b)1,3 )

Seja $f$ derivável até segunda ordem em $\mathbb{R}$ tal que $|x| \geq |f'(x)|$.
Seja também $|x^{1/5}| f''(x) + f'(x) -6x=0; \forall x \in \mathbb{R}$.
Dê o que se pede:
\begin{enumerate}[label = (\alph*)]
    \item Determine $f''(0)$, se possível, de modo que $f''$ seja contínua no ponto $x=0$.
    \item Determine se o ponto $x=0$ é o ponto mínimo, máximo, inflexão ou nenhum dos casos.
\end{enumerate}

\textbf{Lembre-se: \\
- Teoremas que não foram ministrados no curso devem ser provados. \\
- Todas as questões devem ser justificadas matematicamente.}

\newpage

\subsubsection{2018 \original{}}
\normalsize{\textcolor{red}{Nessa prova, não foi permitido utilizar derivadas, ou seja, nada de L'Hôpital! Também não valeu usar complexos, nem aproximação de Taylor.}}

\paragraph{Questão 1:} (3,0 pontos - a) 0,2; b) 0,5; c) 0,8; d) 0,5; e) 0,5; f) 0,5)\\
Dada a função $y = f(x)$ definida por $f(x) = \dfrac{x}{\sqrt[7]{x^{2} - 1}}$, pede-se:
\begin{enumerate}[label = (\alph*)]
    \item O domínio e a imagem de $f(x)$;
    \item $f'(x)$ e $f''(x)$;
    \item regiões de concavidade;
    \item pontos de máximo, mínimo e inflexão, caso existam;
    \item assíntotas horizontais e verticais, caso existam; e
    \item esboço do gráfico.
\end{enumerate}

\paragraph{Questão 2:} (1,0 ponto)
Analise a aplicabilidade do Teorema do Valor Médio para a função $f(x) = x^{2/3}$ no intervalo $\left[-a, a\right]$, $a > 0$.

\paragraph{Questão 3:} (2,0 pontos) Seja $r$ uma reta que passa pelo ponto $\left(1, 2\right)$ e intercepta os eixos nos pontos $A = \left(a, 0\right)$ e
$B = \left(0, b\right)$, com $a > 0$ e $b > 0$. Determine $r$ de modo que a distância de $A$ e $B$ seja a menor possível.

\paragraph{Questão 4:} (2,0 pontos) É correto afirmar que existe ao menos uma linha reta
normal (perpendicular) ao gráfico de $y = \cosh{x}$ no ponto $\left(a, \cosh{a}\right)$ e que também seja normal ao gráfico de $y = \sinh{x}$ no ponto $\left(b, \sinh{b}\right)$?

\paragraph{Questão 5:} (2,0 pontos - a) 0,7; b) 0,7; c) 0,6) \\
Seja $f(x) = x^{2} \sen{\dfrac{1}{x}}$; $x \neq 0$ e $f(0) = 0$.
\begin{enumerate}
    \item Determine $f'(0)$, ou prove que não existe.
    \item Determine $f''(0)$, ou prove que não existe.
    \item Determine se o ponto $x = 0$ é ponto de máximo, de mínimo, de inflexão ou
    nenhum dos casos.
\end{enumerate}

\newpage

\subsubsection{2017 \original{https://drive.google.com/file/d/1ZAImwda_-AjUhh0gd20wXibnRYYiKFVV/view?usp=sharing}}
\normalsize{
\textcolor{red}{Nessa prova, não foi perrmitido utilizar derivadas, ou seja, nada de L'Hôpital! Também não valeu usar complexos, nem aproximação de Taylor.}

\paragraph{Questão 1: }(3,5 pontos) \\
Esboce o gráfico da função $f(x)$ descrita abaixo, e determine, caso existam:

\begin{tasks}(2)
\task O domínio e a imagem.
\task Pontos críticos.
\task Pontos de máximos e mínimos locais.
\task Pontos de inflexão.
\task Interseções com os eixos.
\task Intervalos de crescimento.
\task Intervalos de convexidade.
\task Assíntotas.
\end{tasks}

$$f(x) = \dfrac{x^2 + x - 1}{x^2} + g(x)\cdot x$$

onde $g(x) = \begin{cases}0,\textrm{ para }x\ge 0 \\ -1,\textrm{ para }x < 0\end{cases}$

\paragraph{Questão 2: }(2,0 pontos) \\
Um veículo de levantamento topográfico, movido a combustível, tem seu consumo dado por $\frac{4800+v^2}{2400}$ litros/hora, onde $v$ representa a sua velocidade em km/h. O litro do combustível custa $R\$ 0,30$. Para executar o levantamento de forma correta, o veículo deve percorrer o trecho em uma velocidade constante e respeitando o valor entre $60$ km/h $\le v \le 120$ km/h.
Um engenheiro deve planejar o levantamento de um trecho de $300$ km. Sabendo que o custo do motorista é medido a $m$ reais por hora, qual deve ser a velocidade utilizada pelo veículo para se minimizar o custo do levantamento topográfico?
Considere como custo o gasto com combustível e com o motorista, sendo que $M\ge0$.

\paragraph{Questão 3: }(1,0 ponto) \\
Use o Teorema de Rolle para provar que se $\dfrac{a_0}{1} + \dfrac{a_1}{2} + \dots + \dfrac{a_n}{n+1} = 0$ então $a_0 + a_1 x + \dots + a_nx^n = 0$ tem pelo menos uma raiz no intervalo $(0,1)$.
Considere todos os coeficientes reais.

\paragraph{Questão 4: }(1,5 pontos) \\
Suponha que $y=f(x)$ seja uma função derivável dada implicitamente pela equação:

$$\mathrm{arccossec}(x+2) + e^{xy} = \cos(x) + \mathrm{arctg}(x+y)$$Determine se a reta tangente ao gráfico de $f$ no ponto $x=0$ é crescente, decrescente ou horizontal.

\paragraph{Questão 5: }(2,0 pontos)\\
Seja $g: \R \rightarrow \R$ uma função duas vezes diferenciável, tal que sua segunda derivada seja sempre positiva nos reais. Seja $g(0) = 0$ e seja $f(x) = \dfrac{g(x)}{x}$ uma função definida nos reais positivos.
Determine se $f$ é invertível no seu domínio (reais positivos).
}
\newpage

\subsubsection{2016 \original{https://drive.google.com/open?id=1XmbmqmXBqoPviTSGrnkN7tFxL5eOqz8m}}

\paragraph{Questão 1} (1,0 ponto)\\
Seja $P(x)$ uma função polinomial não constante. Prove que, entre dois zeros consecutivos de $P^{\prime}(x)$ (isto é, dois valores de $x$ que anulam a derivada e tal que entre eles não exista outro valor que anule a derivada), existe no máximo uma raiz de $P(x)$.

\paragraph{Questão 2} (1,5 ponto)\\
Seja $y=f(x)$. Use derivação implícita para determinar os pontos da lemniscata $(x^2+y^2)^2 = x^2-y^2$ em que as retas tangentes são horizontais. Desconsidere a origem.

\paragraph{Questão 3} (2,0 pontos)\\
Seja $f:\R \to \R$ uma função duas vezes diferenciável, com segunda derivada positiva. Seja $g(x) = f(x+f^{\prime}(x))-f(x)$.
Prove que $g(x) $ é não negativa ou dê um contraexemplo.

\paragraph{Questão 4} (2,0 pontos)\\
Achar as dimensões do cone circular reto de volume mínimo que pode ser circunscrito a uma esfera de raio 8 unidades.

\paragraph{Questão 5} (3,5 pontos - a)0,2; b)0,8; c)1,0; d)0,5; e)0,5; f)0,5)\\
Dada a função $y=f(x)$ definida por $f(x) = \dfrac{x}{\sqrt[3]{x^2-1}}$, pede-se:
\begin{tasks}(1)
\task O domínio e a imagem de $f(x)$;
\task $f^{\prime}(x)$ e $f^{\prime\prime}(x)$;
\task Regiões de concavidade
\task Pontos de máximo, mínimo e inflexão, caso existam;
\task Assíntotas horizontais e verticais, caso existam;
\task Esboço do gráfico.
\end{tasks}

\newpage

\subsubsection{2015 \original{https://drive.google.com/open?id=1mdAzrMdfrA7_gXcNztzUJuo4iLVIb7V_}}

\paragraph{Questão 1} (3,5 pontos - a) 0,2; b)1,0; c)0,5; d)0,8; e)0,5; f)0,5)\\
Dada a função $f$ definida por $f(x) = \dfrac{4x+x^2}{x^2-1}$, pede-se:

\begin{tasks}(2)
\task O domínio e a imagem de $f(x)$;
\task $f^{\prime}(x)$ e $f^{\prime\prime}(x)$;
\task Pontos de máximo, mínimo e inflexão, caso existam;
\task Regiões de concavidade;
\task Assíntotas;
\task Esboço do gráfico.
\end{tasks}

\paragraph{Questão 2} (1,5 ponto)\\
Qual a distância mínima de um ponto da curva $y=\sqrt{x}$ ao ponto $\left(\dfrac{3}{2}, 0 \right)$?

\paragraph{Questão 3} (1,0 ponto)\\
Use o Teorema do Valor Médio para provar a desigualdade abaixo: $$\sqrt{2020} - \sqrt{2019} < \dfrac{1}{2\sqrt{2019}}$$

\paragraph{Questão 4} (2,0 pontos - a)0,5; b)1,5)\\
Seja $f(x)$ uma função real de uma variável real de domínio $(-\infty, +\infty)$. Sabe-se que $f(0) = 1$ e $f^{\prime}(x) =\dfrac{\mathrm{sec}(\mathrm{tgh}(x))}{3-\cos(x)}$.

\begin{tasks}(1)
\task Calcule (ou mostre que não existe) $\lim_{x\to +\infty} f^{\prime}(x)$.
\task Determine se $f(x)$ é invertível e, caso seja, obtenha $(f^{-1})^{\prime}(1)$.
\end{tasks}

\paragraph{Questão 5} (2,0 pontos)\\
Seja $f: \R \to [-1,1]$ uma função diferenciáveç até 2ª ordem e seja $$\left(f(0)\right)^2 + \left(f^{\prime}(0)\right)^2 = 4.$$
É correto afirmar que existe um número $x=c$ tal que $f(c) + f^{\prime\prime}(c) = 0$?
Em caso afirmativo, prove. Em caso negativo, dê um contraexemplo.


\newpage

\subsubsection{2014 \original{https://drive.google.com/open?id=1dIUE3zysDeVc0egdvl3kDJ79QsjEO6VM}}

\paragraph{Questão 1}
Faça o estudo da função $y=\dfrac{x}{\sqrt[3]{x^2-1}}$, determinando o seguinte:
\begin{tasks}(2)
\task O domínio da função;
\task Os pontos de extremos e de descontinuidades;
\task Os intervalos de monotonicidade;
\task Pontos de inflexão e direção da concavidade;
\task Assíntotas;
\task Esboço do gráfico.
\end{tasks}

\paragraph{Questão 2}
Dada uma folha circular, qual deve ser o ângulo do setor a ser cortado que, quando enrolado, obtém-se o funil de capacidade máxima.

\paragraph{Questão 3}
Suponha que $y=f(x)$ seja uma função derivável dada implicitamente pela equação $xy^2 + \mathrm{arctg}(\mathrm{arcsenh}y) + x = c$. Sabendo que $f(1) = 0$, determine o valor numérrico de $f^{\prime}(c)$, caso exista.

\paragraph{Questão 4}
Seja $f$ uma função real de variável real, não nula e continuamente diferenciável. Sabendo que $f$ obedece a igualdade $$f(x) = \lim\limits_{h\to 0}\dfrac{f^{\prime}(x+h) - f^{\prime}(x-h)}{2h}$$

e que $f(0)=\pi$. Determine $f(x)$ na sua forma explícita.

\paragraph{Questão 5}
Suponha $g: \R \to \R^+$ uma função diferenciável tal que $g^{\prime}(x) = g(g(x))$, válida para todo $x \in \R$.
Pergunta: existe alguma função real que atenda todos os esquisitos acima? Em caso afirmativo, dê um exemplo. Em caso negativo, prove.

\newpage

\subsubsection{2011 \original{https://drive.google.com/open?id=1vnhOh0HXSND4CspT8ZsXAwfaBkc_133y}}

\paragraph{Questão 1 (2,0 pontos)}

Analise algebricamente o comportamento da função $f(t) = e^{-\modu{t}}\cos t$ e faça um esboço de seu gráfico, o mais completo possível.

\paragraph{Questão 2 (2,0 pontos)}

Os pontos $A$ e $B$ são opostos um ao outro nas margens de um rio reto de largura $r$. $C$ é um terceiro ponto que dista $s$, $(s \ge r)$, rio abaixo na mesma margem de $B$. Uma companhia deseja estender um cabo de $A$ até $C$. Se o custo do cabo é $25\%$ mais caro sobre a água do que sobre a terra, qual a configuração do cabo de menor custo? Resolva pelo teste da primeira derivada e desenhe no caderno de respostas esta configuração.

\begin{figure}[h]
\centering
\includegraphics[width=8cm]{temp.png}
\end{figure}

\paragraph{Questão 3 (1,5 pontos))}

Seja $y = f(x)$ uma função definida implicitamente pela equação $xy^2 + \mathrm{arcsec } y + x = 7\pi$. Obtenha a equação da reta tangente ao gráfico de $f$ no ponto $(a, f(a))$ sabendo que $f(a) = 2$.

\paragraph{Questão 4 (2,5 pontos)}

Seja $f$ derivável até segunda orgem em $\R$ de modo que $\modu{f^{\prime}(x)} \le \modu{x}$ e $\modu{x^{1/3}}f^{\prime \prime}(x) + f^{\prime}(x) = 4x,\, \forall \in \R$.

\begin{enumerate}[label=(\alph*)]
\item (0,5) Mostre que $f^{\prime\prime}$ é contínua em todo $x\neq 0$.
\item (1,0) Determine $f^{\prime\prime}(0)$ de modo que $f^{\prime\prime}$ seja contínua no ponto $x=0$.
\item (1,0) Mostre que $f$ tem um ponto de inflexão horizontal no ponto $(0, f(0))$.
\end{enumerate}

\paragraph{Questão 5 (2,0 pontos)}

Suponha que $f(0) = 0$ e que $f^{\prime}$ é crescente em $\R$. Prove que a função $g(x) = \dfrac{f(x)}{x}$ é crescente em $(0, +\infty)$.


\newpage
\subsubsection{2010 \original{https://drive.google.com/open?id=1Nbrgno_q7fJQ-hH_qV9SnLmD_8t9RIOx}}

\paragraph{Questão 1 (2,5 pontos)}

Seja $f: \R \to \R$ dada por $$f(x) = \begin{cases}x-[x],\text{ se }0\le x-[x]< \frac{1}{2} \\ 1 - (x-[x]), \text{ se }\frac{1}{2} \le x - [x] < 1 \end{cases}$$

\begin{tasks}(1)
\task (1,0 pt) Determine os pontos de continuidade e descontinuidade da função;
\task (1,5 pt) Determine e caracterize os pontos críticos da função.
\end{tasks}

\paragraph{Questão 2 (2,0 pontos)}

Utilizando o Teorema do Valor Médio para derivadas, mostre que $(x+1)\cos \dfrac{\pi}{x+1} - x\cos \dfrac{\pi}{x} > 1,\, \forall x \ge 2$. (Sugestão: utilize a função $f(x) = x\cos \dfrac{\pi}{x})$.

\paragraph{Questão 3 (2,5 pontos)}

Considere a curva definida pela equação $(x^2 + y^2 + y)^2 = x^2 + y^2$.

\begin{tasks}(1)
\task (1,5 pt) Usando derivação implícita, determine os pontos da curva que possuem tangentes horizontais;
\task (1,0 pt) Seja $f:[-1,1] \to \R,\, y=f(x)$ definida implicitamente pela equação acima, tal que $f(x) \ge 0$. Determine os pontos críticos de $f$.
\end{tasks}

\paragraph{Questão 4 (3,0 pontos)}

Sejam $f, g: [a,b] \to \R$ contínuas e deriváveis em $(a,b)$, tais que $f(a) = g(a)$.

\begin{tasks}(1)
\task (1,0 pt) Mostre que se $f^{\prime}(x) > g^{\prime}(x)$ em $(a,b)$, então $f(x)>g(x)$ em $(a,b)$;
\task (1,0 pt) Justificando através de um contra-exemplo, mostre que não vale a recíproca da afirmação acima;
\task (1,0 pt) Utilizando o item (a), mostre que $x-\dfrac{1}{2}x^2< \mathrm{ln} (x+1) < x$ e conclua que $\lim\limits_{x\to 0} \dfrac{\mathrm{ln}(x+1)}{x} = 1$.
\end{tasks}

\newpage
\subsubsection{2003 \original{https://drive.google.com/open?id=1RAoouERZSL2CNv6vJkks5gfHAwLBubP7}}
\footnotesize{
\begin{multicols}{2}\setlength{\columnsep}{1.5cm}
\setlength{\columnseprule}{0.2pt}
\paragraph{Questão 1} (1,5 pontos)\\
Sejam os números de Fibonacci:
$$\begin{cases}F_0 = 0,\, F_1 = 1 \\ F_{n+1}=F_n + F_{n-1},\, \forall n \ge 1 \end{cases}$$

Use indução para mostrar que todo inteiro positivo pode ser escrito como a soma de dois ou mais números de Fibonacci distintos.

\paragraph{Questão 2} (1,5 pontos)\\
Considere uma função não-nula $f: \R \to \R$ que satisfaz 
$$\begin{cases}
f(x+y) = f(x)f(y),\,\forall x,y \in \R
\\f^{\prime}(0) = a
\end{cases}$$

\begin{enumerate}[label=(\roman*)]
\item Mostre que:
\begin{tasks}(1)
\task $f(0) = 1$;
\task $f(-x) = f(x)^{-1}$;
\task $f(x-y) = f(x)/f(y)$.
\end{tasks} 
\item Calcule $f^{\prime}(x)$.
\end{enumerate}

\paragraph{Questão 3} (1,5 pontos)\\
Sabendo que a área do círculo unitário vale $\pi$, mostre que a área da elipse $\dfrac{x^2}{a^2}+\dfrac{y^2}{b^2} = 1$ vale $\pi a b$.

\paragraph{Questão 4} (1,5 pontos)\\
Calcule o volume do sólido obtido girando a curva $x^{2/3}+y^{2/3} = a^{2/3}$ em torno do eixo-x.

\paragraph{Questão 5} (1,5 pontos)\\
Calcule a área da região limitada pela curva $x^{2/3}+y^{2/3} = a^{2/3}\quad (a>0)$.

Sugestão: a curva acima pode ser reescrita pelas equações paramétricas$$\begin{cases}x=a\cos ^3 \theta \\ y = a\sen ^3 \theta,\, 0 \le \theta \le 2\pi \end{cases}$$\paragraph{Questão 6} (1,5 pontos)\\
Uma escada de 5m de comprimento está apoiada numa parede de 3m de altura. A base da escada, apoiada no piso, está se afastando da parede com velocidade constante de 1,5 m/min. Determine a taxa de variação com respeito ao tempo do ângulo $\alpha$ que a escada forma com o piso no instante em que a extremidade superior da escada encontra-se a 4m de altura.

\paragraph{Questão 7} (1,5 pontos)
\begin{tasks}(1)
\task Seja $n$ um inteiro positivo, mostre que $$(n-1)! \le n^n e^{-n} e \le n!$$Sugestão: Observe que $F(x) = x\Ln (x) - x$ é uma antiderivada de $f(x) = \Ln (x)$. Compare a integral $\int\limits_1^n \Ln (x) \dd x$ com a integral de funções degrau acima e abaixo de $\Ln (x)$ com respeito à partição ${1, 2, \dots, n}$ do intervalo $[1, n]$.
\task Use o item $a)$ para provar que $\lim\limits_{n\to \infty} \dfrac{(n!)^{1/n}}{n} = \dfrac{1}{e}$.
\task Calcule $\lim\limits_{n\to \infty}\left[ \dfrac{(3n)!}{n!n^{2n}} \right]^{1/n}$
\end{tasks}

\end{multicols}
}
\normalsize
\newpage
\subsubsection{2002 \original{https://drive.google.com/open?id=13bYttzTWI9Dzl9JcHPNK3RO4Ekr8OiHa}}

\paragraph{Questão 1:}(1,5 pontos)\\
Use indução para provar que é possível pagar, sem receber troco, qualquer quantia inteira de reais, maior do que 7, com notas de 3 reais e 5 reais.

\paragraph{Questão 2:}(1,5 pontos)\\
Considere a região $R$ limitada pelas curvas $x^2+y^2=3x$ e $x^2+y^2=x+\sqrt{x^2+y^2}$, com $x \geq 0$.

\begin{enumerate}[label=(\alph*)]
\item Faça um esboço da região $R$.

\item Calcule a área de $R$.

\end{enumerate}

\paragraph{Questão 3:}(1,5 pontos)\\
Dois pontos $A$ e $B$ estão deslocando-se sobre os semi-eixos positivos $Ox$ e $Oy$ de forma que a distância da reta por $A$ e $B$ até a origem $O$ permanece constante e igual a $\sqrt{2}\,$m. Se o ponto $A$ está se afastando de $O$ com velocidade constante de $4\sqrt{2}\,$m/s, determine a velocidade do ponto $B$ no instante em que $OA=3\sqrt{2}\,$m.

\paragraph{Questão 4:}(1,5 pontos)\\
Seja $\left \lfloor x \right \rfloor$ o maior inteiro menor ou igual a $x$. Considere a função $f:\left [  0,\infty  \right )  \to \mathbb{R}$ definida por$f(x)=\left \lfloor \sqrt{x} \right \rfloor$.

\begin{enumerate}[label=(\alph*)]
\item Esboce o gráfico de $f$ restrita ao intervalo $\left [ 0,16 \right ]$.

\item Mostre que $\displaystyle\int\limits_{0}^{n^2}\left \lfloor \sqrt{x} \right \rfloor\,dx=\dfrac{n(n-1)(4n+1)}{6}$.

\end{enumerate}

\paragraph{Questão 5:}(1,5 pontos)\\
Considere uma função não-nula $f:\left [  0,\infty  \right )  \to \mathbb{R}$ que satisfaz
\begin{enumerate}
\item $f(xy)=f(x)+f(y)$ para todo $x,y>0$.
\item $f'(1)=a$.

\begin{enumerate}
\item Mostre que:
\begin{enumerate}
\item $f(1)=0$;
\item $f\left ( \dfrac{1}{x} \right )=-f(x)$;
\item $f\left ( \dfrac{x}{y} \right )=f(x)-f(y)$. 

\end{enumerate}

\item Calcule $f'(x)$.

\end{enumerate}
\end{enumerate}

\paragraph{Questão 6:}(1,5 pontos)\\
Seja $f$ uma função periódica de período $p>0$ e integrável em $\left [ 0,p \right ]$. Mostre que
$$\int\limits_{a}^{a+p}f(x)\,dx=\int\limits_{0}^{p}f(x)\,dx\;\;\; \text{ para todo }a \in \mathbb{R}.$$

\paragraph{Questão 7:}(1,5 pontos)\\
Seja $f:\left [ a,b \right ] \to \left [ a,b \right ]$ uma função contínua. Mostre que existe $c \in \left [ a,b \right ]$ tal que $f(c)=c$ (c é chamado de um ponto fixo de $f$). Interprete este resultado graficamente.

\newpage
\subsubsection{2001 \original{https://drive.google.com/open?id=18VeC0cOijh8UPlM8ONahJHsSAem3YCcs}}

Prova que vale \textbf{10,5} é o \textbf{bizu}!!!

\paragraph{Questão 1}(1,5 pontos)\\
Seja $F(x,a)=\displaystyle\int\limits_{0}^{x}\dfrac{t^p}{(t^2+a^2)^q}\,dt$, onde $a>0$, e
 p e q são inteiros positivos. Mostre que $F(x,a)=a^{p+1-2q}F\begin{pmatrix}
\frac{x}{a},1
\end{pmatrix}.$


\paragraph{Questão 2:}(1,5 pontos)\\
Se n é um número inteiro, mostre que $\displaystyle\int\limits_{\sqrt{nx}}^{\sqrt{(n+1)x}}sen(t^2)dt=\dfrac{(-1)^n}{c}$
, onde $\sqrt{n\pi }\leq c\leq \sqrt{(n+1) \pi}.$


\paragraph{Questão 3:}(1,5 pontos)\\
Esboce um gráfico da função: $f(x)=\dfrac{1}{(x-1)(x-3)}.$


\paragraph{Questão 4:}(1,5 pontos)\\

\begin{enumerate}[label=(\alph*)]
\item Se $I_{n}(x)=\displaystyle\int\limits_{0}^{x}t^n(t^2+a^2)^{-\frac{1}{2}}dt$, use a integração por partes para mostrar 
que: $nI_{n}(x)=x^{n-1}\sqrt{x^2+a^2}-(n-1)a^2I_{n-2}(x)$, se $n\geq 2.$
\item Use o resultado de (a) para calcular: $\displaystyle\int\limits_{0}^{2}x^5(x^2+5)^{-\frac{1}{2}}dx.$
\end{enumerate}


\paragraph{Questão 5:}(1,5 pontos)\\
Uma função $f$ é definida pela expressão: $f(x)=\sqrt{\dfrac{4x+2}{x(x+1)(x+2)}}$, se $x>0.$

\begin{enumerate}[label=(\alph*)]
\item Determine a declividade do gráfico de $f$ no ponto $x=1$;
\item A região sobre o gráfico e acima do intervalo $[1,4]$ é rotacionada em torno do eixo x, gerando um sólido de revolução. Escreva a integral e calcule-a.
\end{enumerate}


\paragraph{Questão 6:}(2,0 pontos)\\
Calcule:

\begin{tasks}(2)
\task $\underset{x \to +\infty }{\displaystyle\lim}\dfrac{\displaystyle\int\limits_{0}^{x}e^t(t^2-t+5)dt}{\displaystyle\int\limits_{0}^{x}e^t(3t^2+7t+1)dt}$
\task $\underset{x \to 0^{+} }{\displaystyle\lim}\begin{bmatrix}
\dfrac{ln\,x}{(1+x)^2}-ln\begin{pmatrix}
\dfrac{x}{x+1}
\end{pmatrix}
\end{bmatrix}$
\task $\displaystyle\int\dfrac{x\,dx}{\sqrt{4x^2+8x+5}}$
\task $\displaystyle\int\dfrac{ax+b}{cx+d},\; c\neq 0$
\end{tasks}


\paragraph{Questão 7:}(1,0 ponto)\\
Se A reais são investidos e A+B reais são recuperados em n períodos iguais
observados, então a taxa I de juros por período, incluída no balanço corrente satisfaz a equação:
$$(A+B)[1-(1+I)^{-n}]=AnI$$
Usando o polinômio de Taylor do segundo grau para as aproximações necessárias, encontre
uma solução aproximada da equação acima para I em termos de A, B e n, considerando que I é pequeno.

\newpage

\subsubsection{2000 \original{https://drive.google.com/open?id=1Slz2QBfIxnZuLY1xZT6YeNYisANdpUUG}}


\paragraph{Questão 1:}(1,5 pontos)

\begin{enumerate}[label=(\alph*)]

\item Use indução para provar que $(1+x)^n\geq 1+nx$, para $n\geq 1$ e $x>-1$.

\item Se $n>1$, determine para que valores de $x>-1$ vale a igualdade $(1+x)^n=1+nx$.

\end{enumerate}

\paragraph{Questão 2:}(1,5 pontos)\\
Sejam $f(x)=x-x^2$ e $g(x)=ax\,$. Determine o valor de $a$ de modo que a região limitada pelos gráficos da $f$ e da $g$ tenha área igual a $\dfrac{9}{2}$.

\paragraph{Questão 3:}(1,5 pontos)\\
Esboce o gráfico da curva $(x^2+y^2)^2=x^2-y^2\,$, $y^2\leq x^2\,$, e calcule a área da região limitada por ela.

\paragraph{Questão 4:}(1,5 pontos)\\
Considere o sólido $S$ obtido girando-se a região $Q=\left \{(x,y)\vert \,0\leq x\leq \pi, 0\leq y \leq \sen x+\cos x \right \}$ em torno do eixo-x. Faça o esboço deste sólido e calcule seu volume.

\paragraph{Questão 5:}(1,5 pontos)\\
Dada $f:\mathbb{R}\rightarrow \mathbb{R}$ função periódica de período $2$. Sendo $f$ integrável, seja $g(x)=\displaystyle\int\limits_{0}^{x}f(t)\,dt$.

\begin{enumerate}[label=(\alph*)]

\item Prove que $g:\mathbb{R}\rightarrow \mathbb{R}$ é função par e periódica de período $2$.

\item Calcule $g(2n)$ para todo inteiro $n$.

\end{enumerate}

\paragraph{Questão 6:}(1,5 pontos)\\
Dada $f:\mathbb{R}\rightarrow \mathbb{R}$ com $A=\displaystyle\int\limits_{-1}^{1}f(t)\,dt$, seja $g:\mathbb{R}\rightarrow \mathbb{R}$ definida por $g(x)=k\,f\left ( \dfrac{x}{k} \right )$, onde $k>0$. Calcule $\displaystyle\int\limits_{-k}^{k}g(t)\,dt$.

\paragraph{Questão 7:}(1,5 pontos)

\begin{enumerate}[label=(\alph*)]

\item Use o teorema binomial para mostrar que $\left ( 1 + \dfrac{1}{n} \right )^n=1+\displaystyle\sum\limits_{k=1}^{n}\left [ \dfrac{1}{k!}\prod\limits_{r=0}^{k-1}\left ( 1-\dfrac{r}{n} \right ) \right ]$, $n>1$.

\item Sendo $e=\lim\limits_{n \to \infty }\left ( 1+\dfrac{1}{n} \right )^n$, prove que $2<e<3$.

\end{enumerate}

\newpage

\subsubsection{1999 \original{https://drive.google.com/open?id=1snMUbdOrwCUuMMJlK3MbSkC5GsHhvhdz}}

\paragraph{Questão 1} (1,5 pontos)\\
A sequência dos \textit{números de Fibonacci} é definida pela seguinte regra de recorrência:$$F_0 = 0;\, F_1 = 1$$$$ F_{n+2}=F_{n+1}+F_n, \text{ para }n\ge 0$$Denotando por $a$ e $\hat{a}$ as soluções da equação $x^2=x+1$, com $\hat{a}<0<a$, mostre por indução que $F_n = \dfrac{a^n-\hat{a}^n}{\sqrt{5}}$ para $n\ge 0$.

\paragraph{Questão 2} (1,5 pontos)\\

Esboce o gráfico da curva $(x^2+y^2)^3=x^2$ e calcule a área da região limitada por ela.

\paragraph{Questão 3} (1,5 pontos)\\

Mostre que a área da região limitada pela elipse $(x/a)^2 + (y/b)^2 = 1$ é igual a $\pi ab$.

\paragraph{Questão 4} (1,5 pontos)\\

Considere o sólido $S$ obtido girando-se a região $Q=\left\{ (x,y) \vert 0\le x \le 2,\, \frac{1}{4}x^2 \le y \le 1  \right\}$ em torno da reta vertical que passa pelo ponto $(2,0)$. Faça um esboço deste sólido e calcule seu volume.

\paragraph{Questão 5} (1,5 pontos)\\

Considere a função $f(x) = \begin{cases}2\cos x, \text{ se }x\le \pi /4 \\ ax^2 + b, \text{ se } x> \pi /4 \end{cases}$

Determine os valores de $a$ e $b$ de modo que a $f$ seja derivável em todos os pontos.


\paragraph{Questão 6} (1,5 pontos)\\

\begin{tasks}
\task Enuncie o Teorema do Valor Médio para Integrais.
\task Seja $f$ uma função contínua no intervalo $[a,b]$ com $\displaystyle\int\limits_{a}^{b} f(x) \dd x = 0$. Mostre que existe pelo menos um número $c$ em $[a,b]$ tal que $f(c) = 0$.
\end{tasks}

\paragraph{Questão 7} (1,5 pontos)\\

\begin{tasks}
\task Se $f: \R \to \R$ é uma função ímpar, mostre que $\displaystyle\int\limits_{-a}^a f(x) \dd x = 0$ para todo $a>0$.
\task Definindo $\pi = \displaystyle\int\limits_{-1}^1 \sqrt{1-x^2} \dd x$, calcule a integral $\displaystyle\int\limits_{-2}^{2}(\sen x - 3) \sqrt{4-x^2}\dd x$
\end{tasks}

\newpage

\subsubsection{1998 \original{https://drive.google.com/open?id=1UCHV2gm9jw99ERNvxjgy709G47crLp4T}}

\paragraph{Questão 1} (1,5 pontos)\\
Considere a função$$f(x) = \begin{cases}1-e^{-\frac{1}{x^2}},\, x\neq 0 \\ 1, \, x=0 \end{cases}$$\begin{tasks}(1)
\task Existe $f^{\prime}(0)$? Justifique.
\task Esboce o gráfico da $f$ (usando os sete passos).
\task Obtenha o polinômio de Taylor da $f$ em torno de $a=0$.
\end{tasks}

\paragraph{Questão 2} (1 ponto)\\
Calcule os limites
\begin{tasks}(2)
\task $\lim\limits_{x \to 0^+} \left(\Ln \,\dfrac{1}{x} \right)^x$
\task $\lim\limits_{x\to 0^+} \dfrac{x^{x^x}}{x}$
\end{tasks}

\paragraph{Questão 3} (1 ponto)\\
Calcule o comprimento de arco da catenária centrada no eixo-y para $-b \le x \le b$.

\paragraph{Questão 4} (1 ponto)\begin{tasks}(1)
\task Seja $f: [a,b] \to [a,b]$ uma função contínua. Mostre que existe $c\in [a,b]$ tal que $f(c) = c$. O ponto $c$ é chamado de um \textit{ponto fixo} da $f$.
\task Determine os pontos fixos de $f(x) = x^2-1$ e também os de $g(x) = (f\circ f)(x)$.
\end{tasks}

\paragraph{Questão 5} (1 ponto)\\
Mostre que $(x^b+y^b)^{1/b} < (x^a+y^a)^{1/a}$ sempre que $x,y>0$ e $0<a<b$.

\paragraph{Questão 6} (1 ponto)\\
Calcule a área limitada pelo eixo-x e por um ciclo da cicloide.

\paragraph{Questão 7} (1 ponto)\\
Considere a hipérbole $H: x^2-y^2 = 1$.

\begin{tasks}(1)
\task Mostre que $\mathrm{cosh}^2t - \mathrm{senh}^2t = 1$ e conclua que o ponto $P = (\mathrm{cosh}t, \mathrm{senh}t)$ pertence a essa hipérbole.
\task Se $t>0$, mostre que a área da região $OAP$ é igual a $t/2$.
\end{tasks}

\begin{figure}[ht]
\centering
\includegraphics[width=8cm]{vc1998q7.png}
\end{figure}

\paragraph{Questão 8} (1 ponto)\\
Seja $f(x) = \displaystyle\int\limits_0^{\pi} (1+t^3)^{-1/2} \dd t,\, x\ge 0$. Determine se a imagem da $f$ é conjunto limitado ou não.

\paragraph{Questão 9} (1 ponto)\\
Uma partícula movimenta-se no semieixo-x positivo sob a ação da força $F(x) = 5x^2 + Ax^{-3}$.
\begin{tasks}(1)
\task Essa força é conservativa? Justifique.
\task Determine o menor valor de $A$ de modo que $F(x) \ge 14,\, \forall x >0$.
\end{tasks}

\paragraph{Questão 10} (1 ponto)\\
Seja $I_n = \displaystyle \int \limits_0^{\pi /2}\sen ^n x\, \dd x,\, n\in \mathbb{N}$.
\begin{tasks}(1)
\task Mostre que $I_n = \dfrac{n-1}{n} I_{n-2}$.
\task Calcule $\lim\limits_{n\to \infty} \dfrac{I_{2n}}{I_{2n+1}}$.
\task Obtenha a Fórmula de Wallis:$$\dfrac{\pi}{2} = \lim\limits_{n\to \infty} \left( \frac{2}{1} \cdot \dfrac{2}{3} \cdot \dfrac{4}{3}\cdot \dfrac{4}{5} \cdot \frac{6}{5} \dots \dfrac{2n-2}{2n-1} \cdot \dfrac{2n}{2n-1}\cdot \dfrac{2n}{2n+1} \right).$$
\end{tasks}

\newpage

\subsubsection{1996 \original{https://drive.google.com/open?id=1UvvYZLaMSFs7J6LTwt3of2MU1M2tmELR}}

\paragraph{Questão 1} (1,5 pontos)\\
Seja $A(n)$ a afirmação: $1+2+\dots + n = \frac{1}{3}(2n+1)^2$.
\begin{enumerate}[label=(\alph*)]
\item Prove que se $A(k)$ é verdade então $A(k+1)$ também é
\item Critique a afirmação: "Por indução segue que $A(n)$ é verdade para todo $n$.
\item Modifique a igualdade da afirmação $A(n)$ para uma desigualdade de modo que ela se torne verdadeira para todo inteiro $n$
\end{enumerate}

\paragraph{Questão 2} (1,5 pontos)\\
Mostre que a área da região limitada pela elipse $(x/a)^2 + (y/b)^2 = 1$ é igual a $\pi ab$.

\paragraph{Questão 3} (1,0 ponto)\\
Considere a curva $(x^2 + y^2)^2 = y^2-x^2$ com $x^2 \le y^2$.

\begin{enumerate}[label=(\alph*)]
\item Esboce o gráfico desta curva.
\item Calcule a área da região limitada pela curva.
\end{enumerate}

\paragraph{Questão 4} (1,0 ponto)\\
Considere o sólido $S$ obtido girando-se a região $Q = \left\{ (x,y)\vert 0\le x \le 2,\, \frac{1}{4}x^2\le y \le 1 \right\}$ em torno da reta vertical que passa pelo ponto $(2,0)$. Faça um esboço deste sólido e calcule seu volume.

\paragraph{Questão 5} (1,5 pontos)\\
Seja $f$ uma função periódica de período $p$ e integrável em $[0,p]$. Mostre que $\displaystyle\int\limits_0^p f(x) \dd x = \displaystyle\int\limits_{a}^{a+p} f(x) \dd x$ para todo $a$.

\paragraph{Questão 6} (1,0 pontos)\\
Seja $f$ uma função definida por:$$f(x) = \begin{cases} x^2\quad \text{se }x\le \frac{3}{2} \\ ax+b\quad \text{se }x>\frac{3}{2}\end{cases}$$onde $a$ e $b$ são constante. Encontre os valores de $a$ e $b$ de modo que exista $f^{\prime}(3/2)$.

\paragraph{Questão 7} (1,5 pontos)\\
Calcule os limites:

\begin{tasks}(3)
\task $\lim\limits_{x\to 0} x\sen \frac{1}{x}$
\task $\lim\limits_{x\to 0}\dfrac{1-\cos x}{x^2}$
\task $\lim\limits_{x\to 0^-} \dfrac{\sqrt{x^2}}{x}$
\end{tasks}

\paragraph{Questão 8} (1,0 ponto)\\
Sejam $f(x)=\dfrac{1}{1+1/x}$ se $x\neq 0$, e $g(x) = \dfrac{1}{1+1/f(x)}$. Calcule $f^{\prime}(x)$ e $g^{\prime}(x)$.

\paragraph{Questão 9} (1,0 ponto)\\
Mostre que se uma função $f$ é derivável em $a$, então a $f$ é contínua em $a$. A recíproca é verdadeira? Justifique.

\newpage

\subsubsection{1992 \original{https://drive.google.com/open?id=1fhdKNLt2nU57LRMyrsUmHaRHOpavO-rI}}

\paragraph{Questão 1} (2,0 pontos)\\
Suponha que $\lim\limits_{x\to p} f(x) = L$, com $L>0$ e $f: \R \to \R$. Prove que se existe $\delta >0$ tal que, $\forall x \in D_f,\, p-\delta < x < p+\delta, \, x\neq p$ então $f(x) > 0$.

\paragraph{Questão 2} (2,0 pontos)\\
Seja $f$ a função definida por:$$f(x) = \begin{cases}3x-1 \text{ se } x\le 0 \\ \\ \dfrac{\sqrt[3]{7-x} - \sqrt[3]{7}}{x} \text{ se } 0<x\le 1\\ \\ \dfrac{2x^2-17}{4} \text{ se }1<x\le 3 \\ \\ \dfrac{x^3-6x^2 + 11x - 6}{x^2 + 2x - 15} \text{ se }x>3\end{cases}$$\begin{tasks}(1)
\task $f$ é contínua em $x=0$?\newline $f$ é contínua em $x=3$?
\task $f$ é contínua em $[0,3]$?\newline $f$ é derivável em $x=0$? \newline $f$ é derivável em $x=3$?
\end{tasks}
\underline{Justificar cada resposta}.

\paragraph{Questão 3} (1,5 pontos)\\
A função diferenciável $y=f(x)$ é tal que, para todo $x$ no domínio de $f$, o ponto $\left(x, f(x)\right)$ é solução da equação $xy^3 + 2xy^2 + x = 4$. Sabe-se que $f(1) = 1$. Calcule $f^{\prime}(1)$.

\paragraph{Questão 4} (2,0 pontos)\\
Use derivada, para demonstrar que a equação $4x^5 +3x^3 + 3x - 2 = 0$ tem exatamente uma raiz real no intervalo $(0,1)$.

\paragraph{Questão 5} (2,5 pontos)\begin{tasks}(1)
\task Derivar e simplificar ao máximo a função:$$f(x) = \Ln \dfrac{1+\cos 2x}{1-cos 2x}$$\task Encontrar os limites que seguem:

$b_1. \quad \lim\limits_{x\to +\infty} \sqrt{x^2 - 5x + 6} - x$

$b_2. \quad \lim\limits_{x \to 0} \dfrac{\tan \, x - \sen \, x}{x^3}$
\end{tasks}

Obs.: Não usar a regra de L'Hôspital.
\newpage

\subsection{VE 2}

\subsubsection{2019}

\paragraph{Questão 1 (3,0 pontos - 1,0 cada)}

Resolva, desenvolvendo detalhadamente, as integrais abaixo:

(Obs.: Não é permitido propor primitivas e derivá-las. As integrais devem ser resolvidas!)

\begin{tasks}(2)
    \task $\displaystyle\int \dfrac{2x^3 + x +2x^5 + 1}{x + x^3} \dd x$
    \task $\displaystyle\int\limits_{1}^{2} \sqrt{\dfrac{t + 1}{t}} \dd t$
    \task $\displaystyle\int\dfrac{(1 +\sqrt{\sen{u}})}{\sec^{3}u}$

\end{tasks}


\paragraph{Questão 2 (2,0 pontos - 1,0 cada)}

\begin{enumerate}[label=(\alph*)]
    \item Determine o valor de $k$ real que satisfaz a equação abaixo:
    $$ \displaystyle\int\limits_{0}^{\pi} x f(\sen{x}) \dd x = k \displaystyle\int\limits_{0}^{\pi} f(\sen{x}) \dd x$$

    \item Utilize o resultado acima e determine o valor da integral abaixo: 
    $$ \displaystyle\int\limits_{0}^{\pi} \dfrac{2x \sen{x}}{1 + \cos^2{x}} \dd x$$    
\end{enumerate}

\paragraph{Questão 3 (2,0 pontos)}

Seja $n \in \mathbb{Q^{+}}$; $b \in \mathbb{R}$; $b^2 > x^2$. Prove a fórmula de recorrência abaixo:

$$\displaystyle\int (b^2 - x^2)^n \dd x = \left ( \dfrac{x}{2n + 1} \right )(b^2 - x^2)^n + \dfrac{b^2n}{n + (1/2)} \displaystyle\int(b^2 - x^2)^{n-1} \dd x $$


\paragraph{Questão 4 (3,0 pontos - a)2,0; b)1,0 ) }


Seja $f:\mathbb{R^+} \rightarrow I$; $I \subset \mathbb{R}$, definida abaixo:

$$f(x) = \displaystyle\int\limits_{x^2}^{1/x} \cos{(\sen{(t)})} \dd t$$


\begin{enumerate}[label=(\alph*)]
    \item Determine se $f$ é inversível em seu domínio.
    \item Determine quantas raízes $f$ possui.

\end{enumerate}

\noindent \textbf{Lembre-se: \\ -O uso da Regra de L'Hopital, aproximações polinomiais, números complexos e/ou integração não é permitido em qualquer questão da prova. \\ -Nem todos os dados da prova precisam ser usados nas questões da prova. \\ -Teoremas que não foram ministrados no curso devem ser provados. \\ -Todas as questões devem ser justificadas matematicamente.}
 \newpage


\subsubsection{2017 \original{https://drive.google.com/open?id=0ByGXcWkPZfiiR0x4U0VVU09tX1REdDFybTQxNVlKd3pEb01V}}

Obs.: Todas as integrais, exceto as integrais imediatas, devem ser provadas.

\paragraph{Questão 1 (1,0 ponto)}

Calcule a integral $\int \mathrm{sen} ^5 x \mathrm{d} x$, em função de $\cos^n x$, com $n$ inteiro.

\paragraph{Questão 2 (1,0 ponto)}

Calcule a integral $\displaystyle\int\limits_{-2}^2 \dfrac{\sen u \, \modu{u^3}}{\sqrt[4]{1+4u^4}}\dd u$.

\paragraph{Questão 3 (1,0 ponto)}

Determine $g(e/2)$ sabendo que $$g(x) = \int\limits_1^e (x+1)\Ln x\, \dd x$$.

\paragraph{Questão 4 (1,5 pontos)}

Calcule a integral $\displaystyle\int\limits_1^3 \dfrac{\dd x}{\sqrt{(x-1)(3-x)}}$.

\paragraph{Questão 5 (1,5 pontos)}

Determine $g(x)$ sabendo que $g(x) = \displaystyle\int 4\dfrac{1+\sen x}{\sen x}\dd x$, para todo $x$ real diferente de $n\pi$ ($n$ inteiro), e $g(\pi/2) = 2\pi$.

\paragraph{Questão 6 (2,0 pontos)}

Calcule a integral $\displaystyle\int \dfrac{t}{\sqrt{1+t^2+\sqrt{(1+t^2)^3}}}\dd t$.

\paragraph{Questão 7 (2,0 pontos)}

Determine $g(x)$ sabendo que $g(x) = \displaystyle\int \dfrac{x^5+x}{x^5-x}\dd x$, para todo $0<x<1$.
\newpage

\subsubsection{2016 \original{https://drive.google.com/file/d/163O-1F9ZUoMfhYYjooaRlV5GRodccLi5/view?usp=sharing}}

\paragraph{Questão 1} (4,0 pontos - 1,0 cada)\\
Resolva, desenvolvendo detalhadamente, as integrais abaixo:\\
\noindent(Obs.: Não é permitido propor primitivas e derivá-las. As integrais devem ser resolvidas!)
\begin{tasks}(2)
\task $\displaystyle\int \dfrac{x^3 + x^2 + x + 2}{x^4 + 3x^2 + 2} \dd x$
\task $\displaystyle \int \dfrac{x^2}{\sqrt{2x-x^2}}\dd x$
\task $\displaystyle \int \dfrac{\sen(x) \cos (x)}{1+\sen ^4 (x)}\dd x$
\task $\displaystyle\int\limits_0^{\pi / 2}\Ln\left( \sen (x) \right) \dd x$
\end{tasks}

\paragraph{Questão 2} (2,5 pontos - a)1,0; b)1,5)\\
Seja $A$ a área interna da figura $r=2\sen(5\theta)$ tal que $r\le 1$.
\begin{tasks}(1)
\task Faça um esboço da figura no plano $xy$ e hachure a área $A$. 
\task Determine $A$.
\end{tasks}

\paragraph{Questão 3} (1,5 pontos)\\
Calcule a área da região entre os gráficos de $f(x) = 3x^3 - x^2 - 10x$ e $g(x) = -x^2 + 2x$ no intervalo $-2\le x\le 2$.


\paragraph{Questão 4} (2,0 pontos)\\
Considere que a variação de temperatura temporal de um corpo é dado por uma polinomial de grau não superior a 3. Prove que a temperatura média do corpo entre 6:00 e 12:00 pode ser obtida pela média aritmética de duas temperaturas tomadas em dois tempos fixos, $t_1$ e $t_2$, que independe de qual seja a polinomial.
Para provar, encontre $t_1$ e $t_2$.
Lembrete: A média de uma função $f(x)$ em um intervalo $a\le x \le b$ é definida por $$\frac{1}{b-a}\displaystyle\int\limits_a^b f(x)\dd x$$Dica: O horário de 9:00 pode ser visto como um ponto médio dos tempos.

\newpage
\subsubsection{2015 \original{https://drive.google.com/open?id=1dBXZ8g77vz10Pv-SOPZ6VNYKsHVNDqgd}}

\paragraph{Questão 1} (5,0 pontos - 1,0 cada)\\
Resolva, desenvolvendo detalhadamente, as integrais abaixo:
(Obs.: Não é permitido propor primitivas e derivá-las. As integrais devem ser resolvidas!)

\begin{tasks}(2)
\task $\displaystyle\int \dfrac{mu+n}{1+u^2}\mathrm{d}u$, onde $m$ e $n$ são constantes não nulas.
\task $\displaystyle\int \dfrac{3x^2 + 5x + 4}{x^3 + x^2 + x -3}\mathrm{d}x$
\task $\displaystyle\int \dfrac{\mathrm{d}x}{\sqrt{4+x^2}}$
\task $\displaystyle\int \mathrm{sen}(\sqrt[4]{x-1})\mathrm{d}x$
\task $\displaystyle\int (1+2x^2)e^{x^2}\mathrm{d}x$
\end{tasks}

\paragraph{Questão 2} (1,0 ponto)\\
Seja $n\in \R^+$ e $a\in\R$ tal que $a^2>x^2$. Use integração por partes para mostrar a fórmula abaixo:$$\int (a^2-x^2)^n\mathrm{d}x = \dfrac{x(a^2-x^2)^n}{2n+1} + \dfrac{2a^2n}{2n+1}\int (a^2-x^2)^{n-1}\mathrm{d}x + C$$ 

\paragraph{Questão 3} (2,0 pontos)\\
Determine a área do primeiro quadrante do plano real limitada pelos eixos coordenados, pela função $f(x) = 2x^3 - 1$ e por sua inversa $f^{-1}(x)$.

\paragraph{Questão 4} (2,0 pontos)\\
Seja $y=f(x)$ uma função com derivada contínua e definida implicitamente como função de $x$ pela equação abaixo. Determine, se possível, $f^{\prime}(f(1))$.$$\dfrac{y}{x^2-2x+2} + \int\limits_1^y \dfrac{e^{t^2}}{\mathrm{arctg}^2t + 1}\mathrm{d}t = x^4y$$

\newpage

\subsubsection{2014 \original{https://drive.google.com/open?id=1ayRiDMSVALdTBVXuDYoT8cCHNTlBfXAb}}

\paragraph{Questão 1:}(2,0 pontos)

Determinar uma função $f$ não identicamente nula e contínua para todo valor de $x$, tal que:
$$f^2(x)=\int\limits_0^xf(t)\dfrac{dt}{1+e^{2t}}$$

\paragraph{Questão 2:}(4,0 pontos - 1,0 cada)

Resolva, desenvolvendo detalhadamente, as integrais abaixo:

\begin{tasks}(2)
\task $\displaystyle\int \dfrac{\dd x}{x^4+1}$
\task $\displaystyle\int x( \arctg(x) )^2dx$
\task $\displaystyle\int\cossec^5(5x)dx$
\task $\displaystyle\int \dfrac{1-\sen(x)+\cos(x)}{1+\sen(x)-\cos(x)}dx$
\end{tasks}

\paragraph{Questão 3:}(2,0 pontos)

Determine a área da região limitada pela parábola $x^2=2ay$ e pela curva $x^2y=a^2(a-y)$, onde $a>0$.

\paragraph{Questão 4:}(2,0 pontos)

Determine a área de interseção das figuras $r=sen(3\theta)$ e $r=cos(3\theta)$. Faça o gráfico das figuras, marcando a área de interseção.

\newpage
\subsubsection{2011 \original{https://drive.google.com/open?id=1U9W12zCjTyu39bKnv9in6Eudj_hfruAa}}

\paragraph{Questão 1:} (3,0 pontos) 

Obtenha as integrais: 

\begin{tasks}(3)
\task (1,0) $\displaystyle\int\cos(\Ln x)dx$ 
\task (1,0) $\displaystyle\int \dfrac{1}{16x^4-1}dx$ 
\task (1,0) $\displaystyle\int \dfrac{cosh^2x}{sinhx}dx$
\end{tasks}

\paragraph{Questão 2:} (2,0 pontos)

Calcule a área da interseção das regiões $R_1$ e $R_2$ limitadas pelas curvas $r=cos2\theta$ e $r=2=\sin 2\theta$ (em coordenadas polares), respectivamente. Faça o gráfico das curvas identificando a região de intersecção.

\paragraph{Questão 3:} (3,0 pontos)

Seja $f(x)=\displaystyle\int\limits_{\pi/2}^{1/x}\cos(\cos t)dt,x>0$.

\begin{enumerate}[label=(\alph*)]
\item (2,0) Mostre que $f$ é invertível.
\item (1,0) Calcule $(f^{-1})^\prime(0)$.
\end{enumerate}

\paragraph{Questão 4:} (2,0 pontos) 

Seja $\phi$ uma função duas vezes diferenciável em $\mathbb{R}$ com $\phi(x)>0$ em $[a,b]$. Suponha que $\phi ^\prime$ é estritamente crescente e não se anula em $[a,b]$. Mostre que existe uma constante $c \in [a,b]$ tal que:

$$ \int\limits_{a}^{b} \dfrac{dx}{\phi ^2(x)} = \dfrac{ln\phi(c)-ln\phi(a)}{\phi(a)\phi^\prime(a)}+\dfrac{ln\phi(b)-ln\phi(c)}{\phi(b)\phi^\prime(b)}$$

\newpage
\subsubsection{2010 \original{https://drive.google.com/open?id=17aFbxZ6NOGf853zbG9APxFYc2Ix9EJNC}}

\paragraph{Questão 1} (6,0 pontos)\\
Esboce o gráfico da função $f(x) = \modu{x+2}e^{-\frac{1}{x}}$, determinando:
\scriptsize\begin{tasks}(3)
\task (0,5) Domínio da função
\task (0,5) Interseção com os eixos
\task (1,0) Pontos críticos
\task (1,0) Pontos de máximo e mínimo
\task (0,5) Invervalos de crescimento
\task (1,0) Pontos de inflexão
\task (0,5) Intervalos de convexidade
\task (1,0) Assíntotas e imagem
\end{tasks}
\normalsize
\paragraph{Questão 2} (2,0 pontos)\\
Na figura a seguir, dois vértices do retângulo estão no eixo $x$ e os outros dois estão no gráfico de $f: (0,+\infty) \to \R$, $y=f(x)=\dfrac{x}{1+x^2}$. Mostre que o volume do cilindro obtido pela rotação do retângulo em torno do eixo $x$ é dado por $V(x) = \dfrac{\pi x(1-x^2)}{(1+x^2)^2}$ e determine as coordenadas $(p_1,q_1),(p_2,q_2),(p_3,q_3),(p_4,q_4)$ dos vértices do retângulo, de modo que o volume $V(x)$ seja máximo.

\begin{figure}[h]
\centering
\includegraphics[width=8cm]{alglinve22010q2.png}
\end{figure}

\paragraph{Questão 3} (2,0 pontos)\\
Utilizando, obrigatoriamente, substituição trigonométrica, calcule a integral $\displaystyle \int \dfrac{\dd x}{(1-x^2)\sqrt{2x^2-1}}$, levando em conta os intervalos em que $2x^2-1>0$.

\newpage

\subsubsection{2009 \original{https://drive.google.com/open?id=0BzVrEvbC9NPfWV9acjRDTVRKVVk}}

\paragraph{1ª Questão: }(3,0 pontos)\\
Classifique cada item em V ou F, caso a sentença seja verdadeira ou falsa, respectivamente. Caso seja verdadeira, apresente uma prova. Caso seja falsa, prove ou dê um contra-exemplo.
\begin{tasks}
\task (\qquad) Existe uma função periódica, tal que sua primitiva não é periódica;
\task (\qquad) Se existe $A(x) = \intdef{0}{x} f(t)\dd t$ para todo $x$ real e se $f$ é contínua em $x_0$, então pelo Teorema Fundamental do Cálculo, a derivada de $A(x)$ é contínua em  $x_0$;
\task (\qquad) Existe uma função racional irredutível $\dfrac{P(x)}{Q(x)}$, tal que existe $\intdef{0}{2} \dfrac{P(x)}{Q(x)}\dd x$ e $Q(x)$ divisível por $(x-1)$
\end{tasks}


\paragraph{2ª Questão: }(3,4 pontos)\\
Seja $f(x) = x^n + a_{n-1}x^{n-1} + \dots + a_0$ uma função polinomial.

\begin{enumerate}[label=(\alph*)]
\item Suponha que $f$ tenha exatamente $k$ pontos críticos distintos e que a derivada segunda de $f$ em cada um deles é diferente de zero. Mostre que $n-k$ é ímpar.
\item Sejam $n, \, k_1, \, k_2$ três números inteiros positivos com $k_2 = k_1 + 1$ se $n$ for par e $k_2 = k_1$ se $n$ for ímpar. Mostre que existe uma função polinomial de grau $n$ com $k_1$ pontos de máximo local e $k_2$ pontos de mínimo local.
\end{enumerate}

\paragraph{3ª Questão: }(3,6 pontos)\\
Calcule
\begin{tasks}(2)
\task $\intin \dfrac{\dd x}{\sqrt{3}\cos x - \sen x}$
\task $\dfrac{\dd}{\dd x}\left( \intdef{\cos x}{\sen x} \dfrac{e^t}{\sqrt{1-t^2}} \dd t \right), \,0\le x \le \dfrac{\pi}{2}$
\task $\intdef{\pi}{arccos\frac{1}{3}} \dfrac{\sen x}{\sqrt{1+3\cos ^2 x}} \dd x$
\end{tasks}


\newpage

\subsubsection{2002 \original{https://drive.google.com/open?id=1e1hz898KCWAUG7D1JYyGIJyk3Alai4J6}}

\paragraph{Questão 1:} (1,5 pontos)

Esboce o gráfico da função $f(x)=x(ln\modu{x})^2$.

\paragraph{Questão 2:} (1,5 pontos)

Calcule as integrais:

\begin{tasks}(3)
\task $\displaystyle\int e^{\sqrt{x}}dx$ 
\task $\displaystyle\int arctg(\sqrt{x})dx$ 
\task $\displaystyle\int \left(\dfrac{a+x}{a-x}\right)^{1/2}dx$
\end{tasks}

\paragraph{Questão 3:} (1,0 pontos)

Considere os retângulos de base no semi-eixo x positivo cujos vértices da base oposta encontram-se sobre a curva $y=\dfrac{x}{x^2+1}$. Girando esses retângulos em torno do eixo x obtém-se cilindros retos.
\begin{enumerate}[label=(\alph*)]
\item Determine as dimensões e o volume do cilindro de volume máximo.
\item O que se pode dizer sobre a área total desses cilindros?
\end{enumerate}

\paragraph{Questão 4:} (1,0 pontos)

Considere a função $F(x)=\displaystyle\int\limits_1^x \dfrac{e^t}{t}dt,x>0$.
\begin{enumerate}[label=(\alph*)]
\item Para que valores de $x$ vale a desigualdade $ln\,x\leq F(x)$?
\item Prove que $\displaystyle\int e^{1/t}dt = xe^{1/x}-F(1/x)-e$.
\end{enumerate}
\newpage

\subsubsection{Simulado do Mocador do Bigode I}
\paragraph{$1^a$ Questão:} (4,0 pontos - 1,0 cada)

Resolva, desenvolvendo detalhadamente, as integrais abaixo:
\begin{tasks}(2)
\task $\displaystyle\int (x^6+x^3)\sqrt[3]{x^3+3} dx$ 
\task $\displaystyle\int \sqrt{\dfrac{e^x-1}{e^x+1}}dx, \ x>0$ 
\task $\displaystyle\int \dfrac{dx}{(1-x^2)\sqrt[4]{2x^2-1}}$
\task $\displaystyle\int \dfrac{\sen{x}}{\sen{x} + \cos{x}}dx$
\end{tasks}

\paragraph{$2^a$ Questão:} (2,0 pontos)

Encontre ou mostre que não existe uma função $f$ tal que:

$$ f(x)=a(1+x^2)(1+\displaystyle\int\limits_0^x \dfrac{f(t)}{1+t^2}dt) $$
\paragraph{$3^a$ Questão:} (2,0 pontos)

Encontre todas as funções reais tais que:

$$f(x)+\displaystyle\int\limits_0^x (x-t)f(t)dt=1$$
\paragraph{$4^a$ Questão:}

Determinar uma função $f$ não identicamente nula e contínua nos reais, tal que:

$$f^2(x)=\displaystyle\int\limits_0^x f(t)\dfrac{dt}{1+e^{2t}}$$

\newpage

\subsubsection{Simulado do Mocador do Bigode II}
\paragraph{$1^a$ Questão:} (1,0 ponto)

Prove que:

$$ \displaystyle\int \dfrac{1}{(x+\beta)^2+\alpha^2} \dd \alpha = \dfrac{1}{\alpha} \textrm{arctg}\left(\dfrac{x+\beta}{\alpha} \right) + C_1, \ \textrm{com } C_1 \in \mathbb{R}$$

\paragraph{$2^a$ Questão:} (2,0 ponto)

Sejam $f(x),g(x)$ e $h(x)$ funções diferenciáveis em $\mathbb{R}$ tais que para todo $x \in \mathbb{R}$, tem-se:

$$
\begin{cases}
g(x)=f^\prime(x) \\
f(x)+h(x)=-g^\prime(x) \\
g(x)=h^\prime (x) \\
\end{cases}
$$

Se $h(0)=g(0)=f(0)=1$, prove que:

$$f^2(x)+g^2(x)+h^2(x)=3$$

\paragraph{$3^a$ Questão:} (2,0 ponto)
Seja $f:\mathbb{R}\rightarrow \mathbb{R}$ derivável até a $2^a$ ordem tal que para todo $x$, $f^{\prime \prime} (x)=f(x)$.
\begin{enumerate}[label=\alph*)]
\item Prove que $g(x)=e^x[f^\prime (x)-f(x)]$ é constante.
\item Encontre um valor para $f(x)$ em função de constantes reais.
\end{enumerate}
\paragraph{$4^a$ Questão:} (5,0 ponto)

Calcule as seguintes integrais:

\begin{tasks}(2)
\task $\displaystyle\int x(\textrm{arctg}(x))^2 dx$ 
\task $\displaystyle\int \dfrac{1}{x} \dfrac{1}{\sqrt{1+x^2}}dx$ 
\task $\displaystyle\int \dfrac{x}{(1+x^4)}dx$
\task $\displaystyle\int \textrm{sen}^2x\textrm{cos}^2xdx $
\task $\displaystyle\int x^3\cos{x^2}dx $
\end{tasks}

\newpage

\subsubsection{Simulado do Mocador do Bigode III}

\paragraph{$1^a$ Questão:} (5,0 pontos)

\begin{tasks}(2)
\task $\displaystyle\int \sen{2x} \sqrt{1+3\textrm{cos}^2x}  \ dx$ 
\task $\displaystyle\int \dfrac{\sqrt{1+2x}}{x}dx$ 
\task $\displaystyle\int \dfrac{x^2+1}{(x^4-x^2+1)}dx$
\task $\displaystyle\int x \ \textrm{sen}^2(x)\textrm{\cos}^2(x) \ dx$
\task $\displaystyle\int \dfrac{\sen{x}}{\sen{x} + \cos{x}} \dd x $
\end{tasks}

\paragraph{$2^a$ Questão:} (1,0 ponto)

Sabendo que $f(0)=1$, $f(2)=3$ e $f^\prime(2)=5$, calcule:

$$\displaystyle\int\limits_0^1 x f^{\prime \prime}(2x) dx$$

\paragraph{$3^a$ Questão:} (2,0 pontos)

Prove que se $f$ é uma função não-negativa no intervalo $[0,1]$ e que 
$$ \displaystyle\int\limits_0^1 f(x)dx=0 $$
Então, $f(x)=0, \ \forall x \in [0,1]$.
\paragraph{$4^a$ Questão:} (1,0 ponto)

Seja $f$ uma função contínua real. Prove a existência ou mostre que não existe um $c \in [0,1]$ tal que:

$$\displaystyle\int\limits_0^1 f(x)x^2 dx = \dfrac{1}{3} f(c)$$

\newpage

\subsubsection{Simulado do Mocador do Bigode IV}

\paragraph{$1^a$ Questão:} (1,0 ponto)

Sendo $m$ e $n$ constantes não nulas, prove que:

$$ \displaystyle\int \dfrac{mu+n}{1+u^2}du = \dfrac{m}{2}\textrm{ln}(1+u^2) + n\textrm{arctg}(u) + K $$

\paragraph{$2^a$ Questão:} (2,0 pontos)

Seja $f$ derivável até a $2^a$ ordem e tal que para todo x, $f(x)+f^{\prime \prime}(x) = 0$. Se $\modu{f(0)}\leq 2$ e $\modu{f^\prime(0)}\geq 1$, prove que:
$$ [f(x)]^2+[f^\prime(x)]^2 \geq 1 $$

\paragraph{$3^a$ Questão:} (2,0 pontos)

Ache todas as funções $f:[0,1]\rightarrow \mathbb{R}$ satisfazendo:

$$\displaystyle\int\limits_0^1 f(x) dx = \dfrac{1}{3} + \displaystyle\int\limits_0^1 f^2(x) dx $$ 


\paragraph{$4^a$ Questão:} (5,0 pontos)

\begin{tasks}(2)
\task $\displaystyle\int \dfrac{2x-3}{1+4x^2} dx$ 
\task $\displaystyle\int \dfrac{2x+1}{x^2+4x+5}dx$ 
\task $\displaystyle\int x^3e^{x^2}dx$
\task $\displaystyle\int \textrm{cos}^3x(1+\sqrt(\sen{x}))dx$
\task $\displaystyle\int \textrm{arctg}(\sqrt{x})dx$
\end{tasks}


\newpage

\subsubsection{Simulado do Mocador do Bigode V}

\paragraph{$1^a$ Questão:} (5,0 pontos)

\begin{tasks}(2)
\task $\displaystyle\int (1+2x^2)e^{x^2} dx$ 
\task $\displaystyle\int \dfrac{x+\sen{x}-\cos{x}-1}{x+e^x+\sen{x}}dx$ 
\task $\displaystyle\int \dfrac{x^3+x+1}{x^2-4x+3} dx$
\task $\displaystyle\int \textrm{sec}^5(x) dx$
\task $\displaystyle\int \dfrac{2x+3}{\sqrt{1-4x^2}} dx$
\end{tasks}

\paragraph{$2^a$ Questão:} (1,0 ponto)
Sendo $n$ um natural não nulo, verifique:
\begin{tasks}(1)
\task $\displaystyle\int \textrm{sen}^n(x) dx = -\dfrac{1}{n}\textrm{sen}^{n-1}(x)\textrm{cos}(x)+\dfrac{n-1}{n}\displaystyle\int \textrm{sen}^{n-2}(x)dx$ 
\task $\displaystyle\int \textrm{cos}^n(x) dx = \dfrac{1}{n}\textrm{cos}^{n-1}(x)\textrm{sen}(x)+\dfrac{n-1}{n}\displaystyle\int \textrm{cos}^{n-2}(x)dx$
\end{tasks}

\paragraph{$3^a$ Questão:} (2,0 pontos)

Seja $f:\mathbb{R} \rightarrow \mathbb{R}$ uma função derivável até $2^a$ ordem com $f^{\prime \prime} + f(x)=0$. Prove que $f(x)$ pode ser
escrita em função de funções trigonométricas e constantes reais.

\paragraph{$4^a$ Questão:} (2,0 pontos)

Seja $f:[0,1]\rightarrow \mathbb{R}$ tal que:
$$\displaystyle\int\limits_0^1 f(x)dx=\displaystyle\int\limits_0^1xf(x)dx=1$$

Prove que:
$$\displaystyle\int\limits_0^1 f^2(x)dx \geq 4$$

\newpage

\subsubsection{Simulado do Mocador do Bigode VI}

\paragraph{$1^a$ Questão:} (4,0 pontos)

\begin{tasks}(2)
\task $\displaystyle\int \sqrt{1-e^{x}} dx$ 
\task $\displaystyle\int \dfrac{\sen{x} \cos{x}}{1+\textrm{sen}^4(x)}dx$ 
\task $\displaystyle\int \dfrac{\textrm{arctg}(e^x)}{e^x} dx$
\task $\displaystyle\int \dfrac{1}{2+\sen{x}} dx$
\end{tasks}

\paragraph{$2^a$ Questão:} (2,0 pontos)

Seja $0 \geq a \geq 1$. Encontre todas as funções $f$ em $[0,1]$ não-negativas que satisfaçam:
$$
\begin{cases}
\displaystyle\int\limits_0^0 f(x)dx = 1 \\
\displaystyle\int\limits_0^1 xf(x)dx = a \\
\displaystyle\int\limits_0^1 x^2f(x)dx = a^2 \\
\end{cases}
$$

\paragraph{$3^a$ Questão:} (2,0 pontos)

Determine uma função $f$ não identicamente nula e contínua tal que, para todo $x$:

$$f^{\prime \prime} = \displaystyle\int\limits_0^x f(t)\dfrac{\textrm{sen}(t)}{2+\textrm{cos}(t)}dt$$

\paragraph{$4^a$ Questão:} (2,0 pontos)

Seja $y=f(x)$ uma função com derivada contínua e definida implicitamente como função de $x$. Determine, se possível, $f^\prime(f(1))$.

$$\dfrac{y}{x^2-2x+2} + \displaystyle\int\limits_1^y \dfrac{e^{t^2}}{\textrm{arctg}^2(t)+1} = x^4y$$

\newpage
\subsection{VF}


\subsubsection{2019}

\paragraph{1ª Questão) (2,0 pontos)}


Determine a área de interseção das figuras $r = 1 + \cos{\theta}$ e $r = 2 - \cos{\theta}.$ Faça o gráfico das figuras marcando a área de interseção.\\

\paragraph{2ª Questão) (3,0 pontos - 1,0 cada)}


\begin{enumerate}[label=(\alph*)]
    \item A área A
    \item O volume de revolução gerado quando a área A rotaciona em torno do eixo y.
    \item O volume de revolução gerado quando a área A rotaciona em torno do eixo x.\\
    \textbf{Obs.:} No item c) não precisa resolver a(s) integral(is). Basta apenas deixá-la(s) indicada(s).


\end{enumerate}

\paragraph{3ª Questão) (2,0 pontos - 1,0 cada)}


Calcule ou mostre que não existe:\\
\textbf{Restrição:} O uso da Regra de L'Hôpital só poderá ser utilizado em apenas um dos itens.

\begin{tasks}(2)
    \task $\lim\limits_{x \to 0^{+}} x^{(x^x-1)}$
    \task $\lim\limits_{x \to 0} \dfrac{x^2 \arctan{x}}{x - \dfrac{x^2}{2} - \Ln(x+1)}$

\end{tasks}

\paragraph{4ª Questão) (1,0 ponto)}\\

Use Polinômio de Taylor com notação de o-pequeno para expandir até a 3ªordem a função $f(x) = x^{-1} \arcsinh{x}$, quando $x \rightarrow 0$.

\paragraph{5ª Questão) (2,0 pontos)}\\

Seja uma função $g:[-2, 2] \rightarrow \mathbb{R}$.\\
Sabe-se que $g(x)$ é contínua e não se anula em qualquer ponto do seu domínio.\\
Determine o valor máximo e o valor mínimo assumidos por $g(x)$ em seu domínio, sabendo que: 
$$g^2(x) + \displaystyle\int\limits_{x}^{0} \dfrac{u g(u)}{4 + \sqrt{4 - u^2} - u^2} \dd u = [\Ln(\sqrt{2} + 1)]^2$$

\newpage

\subsubsection{2017 \original{https://drive.google.com/file/d/1Gs5HNH5wUFftXuSzlIBlEpWkF8WfEL2X/view?usp=sharing}}

\paragraph{Questão 1: }

Calcule ou mostre que não existe.

\begin{enumerate}[label=(\alph*)]

\item $\lim\limits_{x\to 0} \dfrac{x^2\Ln(x+1)}{x-\mathrm{arctg}x}$

\item $\lim\limits_{x\to +\infty} (\Ln x)^{\mathrm{arccotg}x}$

\end{enumerate}

\paragraph{Questão 2: }

Seja $f: [0,1] \rightarrow \R$ uma função contínua tal que, para quaisquer dois pontos $u$ e $v$ do seu domínio: \hfill \scriptsize \hypertarget{calculovf2017q2volta}{* }\hyperlink{calculovf2017q2ida}{Solução} \normalsize

$$uf(v) + vf(u) \le 1$$

Prove a desigualdade abaixo ou dê um contraexemplo:

$$\int\limits_{0}^{1} f(t)dt \le \dfrac{4}{5}$$

\paragraph{Questão 3: }

Seja $A$ a área formada pelo conjunto de pontos $y\le 7-x^2$ tal que $x^2 \ge 1+y^2$.

\begin{enumerate}[label=(\alph*)]

\item Determine a área $A$.

\item Determine o volume $V$ obtido da revolução da área $A$ em torno do eixo dos $y$.

\end{enumerate}

\paragraph{Questão 4: }

Seja $S$ a área formada pelo conjunto de pontos que estão no interior da figura $r_1 = \mathrm{sen}(2\theta)$, porém no exerior da figura $r_2 = \cos(2\theta)$

\begin{enumerate}[label=(\alph*)]

\item Faça um esboço das figuras no plano $xy$, hachurando a área $S$.

\item Deermine a área $S$.


\end{enumerate}

\paragraph{Questão 5: }

Seja $f$ uma função definia e contínua para todo $x$ real, e que satisfaça uma equação da forma:

$$\int \limits_{0}^{x^2} f(t)dt = \int \limits_{x^2}^{1}t^2f(t)dt + \dfrac{x^{18}}{9} + \dfrac{x^{22}}{11} + k,\quad k \mathrm{ constante}$$

Encontre $f(x)$ e o valor da constante $k$

\newpage
\subsubsection{2016 \original{https://drive.google.com/open?id=1ESpK2hK5nsNpZ4kRFSCM46KtiSG1s3vX}}

\paragraph{Questão 1} (2,0 pontos - 1,0 cada)\\
Seja $f(x) = x^4 - 12x^3 + 48x^2 - 64x$

\begin{tasks}(1)
\task Esboce o gráfico de $f(x)$;determinando e classificando por derivação, todos os pontos de máximo, mínimo e/ou inflexão do interior do domínio.
\task Seja $A$ a área limitada por $f(x)$ e o eixo dos $x$. Determine o volume de revolução gerado pela rotação de $A$ em torno do eixo dos $y$.
\end{tasks}

\paragraph{Questão 2} (2,0 pontos - 1,0 cada)\\
Resolva, desenvolvendo detalhadamente, as integrais abaixo:

\begin{tasks}(2)
\task $\displaystyle\int \sqrt{\dfrac{1-x}{x}}\dd x,\, 0<x<1$
\task $\displaystyle\int \dfrac{x+x^5}{x+x^7}\dd x,\quad x>0$
\end{tasks}

\paragraph{Questão 3} (2,0 pontos - 1,0 cada)\\
Calcule ou mostre que não existe:

\textbf{Restrição:} O uso da Regra de L'Hôpital só poderá ser utilizado em apenas um dos itens.

\begin{tasks}(2)
\task $\lim\limits_{x\to 0} \dfrac{\Ln(\mathrm{cosh}(x)) - \left(\dfrac{x^2}{2}\right)}{1-\cos (x) - \left(\dfrac{x^2}{2}\right)}$
\task $\lim\limits_{x\to 0^+} x^{-2}\displaystyle\int\limits_0^{x} \mathrm{senh}(t) \sqrt{1-\dfrac{\sen ^2 (t)}{2}}\dd t$
\end{tasks}

\paragraph{Questão 4} (2,0 pontos - 1,0 cada)\\
Seja $S$ a área da região limitada pelas curvas $y=e^{-2x} - 1,\, y=e^{-x}+1$ e a reta $x=0$. Calcule o volume obtido pela rotação de $S$ em torno de:

\begin{tasks}(1)
\task eixo dos $x$.
\task eixo dos $y$.
\end{tasks}

\paragraph{Questão 5} (2,0 pontos)\\
Seja $f$ uma função real, integrável e definida no intervalo $[0,1]$ tal que:$$\int\limits_0^1 f^2(x)\dd x < \dfrac{1}{3}< \int\limits_0^1 xf(x) \dd x$$

Determine quantas (e quais, se for o caso) funções atendem a todos os requisitos anteriores.

\newpage
\subsubsection{2015 \original{https://drive.google.com/open?id=1RXPjZMk1N2RTC_4caGBFqunX45WucsHU}}

\paragraph{Questão 1:}(2,5 pontos - a)1,0; b)1,5)\\
Seja $f(x)=\sqrt{x^4-x^6};\;-1 \leq x \leq 1$.

\begin{enumerate}[label=(\alph*)]

\item Esboce o gráfico de $f(x)$; determinando e classificando, por derivação, todos os pontos de máximo, mínimo e/ou inflexão do interior do domínio.

\item Seja $A$ a área limitada por$f(x)$ e o eixo dos $x$. Determine o volume de revolução gerado pela rotação de $A$ em torno do eixo dos $y$

\end{enumerate}

\paragraph{Questão 2:}(2,0 pontos - 1,0 cada)\\
Resolva, desenvolvendo detalhadamente, as integrais abaixo:

\begin{tasks}(2)
\task $\displaystyle \int x(\textrm{arccotg} (x))^2 \,dx$
\task $\displaystyle \int \dfrac{dx}{x\sqrt{x^2+x+1}},\,x>0$.
\end{tasks}

\paragraph{Questão 3:}(1,5 ponto)\\
Seja $f(x);\,x\neq 0$, a função definida por integração abaixo. Determine as constantes $\textbf{a}$, $\textbf{b}$, $\textbf{c}$, $\textbf{d}$ de tal modo que:\\
$$f(x)=\displaystyle\int\limits_{0}^{x}\dfrac{t}{\textrm{arctg}(t)}\,dt=\textbf{a}+\textbf{b} x+\textbf{c} x^2+\textbf{d} x^3+o(x^3),\text{ quando }x \to 0$$

\paragraph{Questão 4:}(2,0 pontos)\\
Determine a área de interseção das figuras $r=\sen (2\theta)$ e $r=\sen (\theta)$. Faça o gráfico das figuras, marcando a área de interseção.

\paragraph{Questão 5:}(2,0 pontos - 1,0 cada)\\
Calcule ou mostre que não existe:

\begin{tasks}(2)
\task $\lim\limits_{x\to +\infty} \left ( \dfrac{7^x-1}{6x} \right )^{\dfrac{1}{x}}$
\task $\lim\limits_{x\to 0^+} \left ( \displaystyle\int\limits_{0}^{1}\left ( 3t+2\left (  1-t\right ) \right )^x \,dt \right )^{\dfrac{1}{x}}$
\end{tasks}
\newpage

\subsubsection{2013 \original{https://drive.google.com/open?id=1mwBzM5Y-JjrXpuzqZKSRQAmOa4J76RHq}}

\paragraph{Questão 1:}(2,0 pontos - 1,0 cada)\\
Calcule ou mostre que não existe:

\begin{enumerate}[label=(\alph*)]
\item $\lim\limits_{x\to \infty}\sqrt{x}\left(\mathrm{arctg}\left( \dfrac{3}{4}\sqrt{x} \right )-\mathrm{arctg}\left(\dfrac{1}{2}\sqrt{x} \right )  \right )$
\item $\lim\limits_{x\to 0}\dfrac{x^2\,\sen x}{x-\mathrm{arctg}x}$
\end{enumerate}

Obs: No item $b)$ use aproximação polinomial com notação de "o-pequeno".(Não use Regra de L'Hoptal)
\paragraph{Questão 2:}(2,0 pontos)\\
Seja $S$ a esfera de raio $R$. Determine o volume interno de uma calota esférica de $S$ cuja altura é $h\,(h<R)$, utilizando exclusivamente integração do volume de revolução. Desenvolva a questão detalhadamente.

\paragraph{Questão 3:}(2,0 pontos)\\
Determine a área de interseção das figuras $r=1+\cos\theta$ e $r=1-\cos\theta$. Faça o gráfico das figuras marcando a área de interseção.

\paragraph{Questão 4:}(2,0 pontos)\\
Considere $x \geq 0$, $y \geq 0$. A reta $y=c$ intercepta a curva $y=2x-3x^3$ em dois pontos distintos $P$ e $Q$, com abscissas $x=a$ e $x=b$ respectivamente,\,$(0<a<b)$.\\
Sejam $S_1$ e $S_2$ as áreas dadas pelos conjuntos de pontos:\\
$S_1=\left \{ (x,y) \in (\R-\R^-)^2 \vert\, y\leq c;\,y\geq 2x-3x^3;\,0\leq x \leq a\right \}$\\
$S_2=\left \{ (x,y) \in (\R-\R^-)^2 \vert\, y\geq c;\,y\leq 2x-3x^3;\,a\leq x \leq b\right \}$\\
Calcule o valornumérico de $c$ tal que $S_1=S_2$.

\paragraph{Questão 5:}(2,0 pontos)\\
Seja $f:\R^+ \to \R$ uma função diferenciável tal que
$$f(a)+f(b)=f(ab)$$
válida para quaisquer dois pontos $a$ e $b$ em seu domìnio.\\
Sabe-se funções do tipo $f(x)=C\,\Ln x$, $(C \in \R)$; satisfazem as condições acima.\\
Pergunta: existe algum outro tipo de função que também satisfaz tais condições?\\
Em caso afirmativo, dê um exemplo, em caso negativo, prove.

\newpage

\subsubsection{2011 \original{https://drive.google.com/open?id=1PTQE1whASIVeupcRjxuyAj5q5jsATNwX}}

\paragraph{Questão 1:}(2,0 pontos)\\
Uma bruxa de Agnesi de termo geral $0,5$ é cortada pelas retas $x=2$ e $x=-2$. A região $R$ limitada pela curva, por estas duas retas e o eixo $x$, revoluciona em torno de uma das retas que corta a curva, gerando um sólido. Faça um esboço da região $R$.(Não é necessário um esboço minucioso) e calcule o volume do sólido de revolução obtido.\\
\\
A bruxa de Agnesi é a curva $y=\dfrac{8a^3}{x^2+4a^2}$, onde $a$ é o termo geral.

\paragraph{Questão 2:}(2,0 pontos)\\
Um cone circular reto de altura $21\,$cm e raio $6\,$cm, é cortado por um plano paralelo à sua base. As que distância da base deve ser feito este corte para que o cone circular reto de base coincidente com a seção determinada e de vértice no centro da base do cone dado, tenha volume máximo?

\paragraph{Questão 3:}(2,0 pontos - 1,0 cada)\\
Seja $f$ uma função com função derivada contìnua e definida implicitamente como função de $x$, pela equação
$$y+\displaystyle\int\limits_{x}^{y}\left \vert \dfrac{\sen (\pi t)}{t-1}\right \vert \,dt=xy $$
\begin{tasks}(1)
\task Obtenha uma expressão para $f'(x)$.
\task Obtenha a equação da reta tangente ao gráfico de $f$ no ponto $(1,f(1))$.
\end{tasks}

\paragraph{Questão 4:}(2,4 pontos - 0,8 cada)
\begin{tasks}(1)
\task $\displaystyle\int\dfrac{dx}{x^3-1}$
\task $\lim\limits_{x \to 0^+}x^{\mathrm{tan}\,x^2}$
\task um polinômio $P(x)$ tal que $P(x)=\dfrac{x}{\mathrm{arcsin}\,x}=o\left( x^2 \right)$ quando $x \to 0$.
\end{tasks}

\paragraph{Questão 5:}(1,6 pontos)\\
Seja $f$ contínua em $[a,b]$ e duas vezes derivável em $(a,b)$. Seja $(c,f(c))$, com $c \in (a,b)$, um ponto de interseção do gráfico de $f$ com a reta unindo os pontos $(a,f(a))$ e $(b,f(b))$.\\
Mostre que a segunda derivada de $f$ se anula em pelo menos um ponto de $(a,b)$.

\newpage

\subsubsection{2010 \original{https://drive.google.com/open?id=1ty9bN1897Q0qARDPph3qTqZR-bZ3cO_m}}

\paragraph{Questao 1:}(3,0 pontos - 1,0 cada)\\
Calcule:
\begin{enumerate}[label=(\alph*)]
\item $\displaystyle\int\limits_{0}^{\dfrac{x}{2}}\dfrac{\sen ^nx}{\sen ^nx+\cos^nx}dx$ (sugestão: mudança de variável $x=\dfrac{\pi}{2}-u$ e somar a nova integral com a antiga). 
\item O valor de $g'(0)$, dado que $g(x)=\displaystyle\int\limits_{\sen x}^{1+x^2}\sqrt{t^3+t^2+1}\,dt$.
\item $\lim\limits_{x \to 0}\dfrac{x-\Ln (1+x)}{2x\,\Ln(1+x)}$.
\end{enumerate}
\paragraph{Questão 2:}(2,0 pontos)\\
Determine a área sombreada compreendida pelo laço menor da curva "limaçon" dada por $r(\theta)=2\cos(\theta)+1$, conforme o gráfico abaixo:

\begin{figure}[h]
\centering
\includegraphics[width=8cm]{calculovf2010q2.png}
\end{figure}

\paragraph{Questão 3:}(2,0 pontos)\\
Seja $f(x)=\sqrt{4-\dfrac{x^2}{2}}$ com $x \in [0,2\sqrt{2}]$ e seja $b \in [0,2]$. Calcule o valor de $b$ que maximiza e o valor de $b$ que minimiza o volume do sólido de revolução obtido pela rotação da região limitada pelo gráfico de $f$ e o eixo $y=b$ em torno deste mesmo eixo.

\paragraph{Questão 4:}(2,0 pontos)\\
Mostre que $\displaystyle\int\limits_0^x\dfrac{udu}{\Ln (1+u)}=x+\dfrac{x^2}{4}+o\left(x^2\right)$, quando $x \to 0$.

\paragraph{Questão 5:}(1,0 ponto)\\
Mostre que $F:\left[ 0,\dfrac{\sqrt{2}}{2}\right) \to \R$ dada por $F(x)=\displaystyle\int\limits_0^x\dfrac{1}{1-2u^2}du$ é inversível e determine sua inversa $F^{-1}$.

\newpage

\subsubsection{2008 \original{https://drive.google.com/open?id=1DOQEtUxVQ8cAYAtmib_EV9ZVScSDp58Z}}

\paragraph{Questão 1:}(1,0 pontos)\\
Calcule, utilizando aproximação polinomial por polinômio de Taylor:
$$\lim\limits_{x \to 1}\left(\dfrac{1}{log\,x}-\dfrac{1}{x^2-x}\right)$$

\paragraph{Questão 2:}(2,0 pontos)

\begin{tasks}(1)
\task Prove que:
$$\mathrm{sec}\,x=1+\dfrac{x^2}{2}-\dfrac{x^4}{4!}+o\left( x^2 \right)\; quando\; x \to 0$$.
\task Encontre o polinômio $P(x)$ de menor grau tal que $x\,\mathrm{sec}\,x=P(x)+o\left( x^3 \right)$.
\end{tasks}

\paragraph{Questão 3:}(3,0 pontos)\\
Dada a função
$$f(x)=\displaystyle\int\limits_{0}^{x}e^{-t}t^2\,dt$$

\begin{tasks}(1)
\task Mostre que:
$$f(x)=2!\,e^{-x}\left(e^x-1-x-\dfrac{x^2}{2!}\right)$$
\task Mostre que a função $f$ satisfaz a desigualdade:
$$\dfrac{x^3}{3!}<\dfrac{1}{2!}e^xf(x)<e^c\,\dfrac{x^3}{3!},\;\;\forall x >0,$$
onde $c$ é uma constante qualquer positiva.
\task Conclua que
$$f(x)=o\left( x^2 \right).$$
\end{tasks}

\paragraph{Questão 4:}(2,0 pontos)\\
Esboce, em 7(sete) passos, o gráfico da função:
$$f(x)=\dfrac{3x^2-8}{x^2-4}$$

\paragraph{Questão 5:}(2,0 pontos)\\
Calcule:

\begin{tasks}(1)
\task $\displaystyle\int\dfrac{\cos\,\theta}{\mathrm{sen}^2\,\theta+4\,\mathrm{sen}\,\theta-5}d\theta$
\task $\displaystyle\int\dfrac{2x^2-1}{(4x-1)(x^2+1)}dx$
\task $\displaystyle\int\limits_{\sqrt{2}}^{2}\dfrac{\sqrt{2x^2-4}}{x}dx$
\task $\lim\limits_{x \to \pi}\dfrac{\Ln \modu{\mathrm{sen}\,x}}{\Ln \modu{\mathrm{sen}\,2x}}$
\end{tasks}


\newpage

\subsubsection{2005 \original{https://drive.google.com/open?id=1t9ZBqOJ-T4vh2t153EDWb7GldmQsIGmC}}

\paragraph{Questão 1:}(1,0 ponto)\\
Seja $f(x)=\displaystyle\int(1+t^3)^{-1/2}dt\;,\;x\geq 0$. Determine se a imagem de $f$ é um conjunto limitado ou não.

\paragraph{Questão 2:}(1,5 pontos)\\
Seja $I_n=\displaystyle\int\limits_{0}^{\pi/2}\mathrm{sen}^nx\,dx\,$ , $n \in \mathbb{N}$. 

\begin{tasks}(1)
\task Mostre que $I_n=\dfrac{n-1}{n}I_{n-2}$ , se $n\geq 2$.
\task Calcule $\lim\limits_{n\to \infty}\dfrac{I_{2n}}{I_{2n+1}}.$
\task Obtenha a fórmula de Wallis:\\
$\dfrac{\pi}{2}=\lim\limits_{n\to \infty}\left(\dfrac{2}{1}\cdot\dfrac{2}{3}\cdot\dfrac{4}{3}\cdot\dfrac{4}{5}\cdot\dfrac{6}{5}\cdot\cdot\cdot\dfrac{2n-2}{2n-1}\cdot\dfrac{2n}{2n-1}\cdot\dfrac{2n}{2n+1}\right)$
\end{tasks}

\paragraph{Questão 3:}(3,5 pontos)\\
Calcule as integrais e os limites abaixo:

\begin{tasks}(1)
\task $\displaystyle\int \dfrac{1}{1-\cos\,x+\mathrm{sen}\,x}dx$
\task $\displaystyle\int \dfrac{xe^{\mathrm{arctg}x}}{(1+x^2)^{1/2}}dx$
\task $\displaystyle\int \dfrac{x}{\sqrt{3-2x-x^2}}dx$
\task $\lim\limits_{x\to \infty}e^x\left[e-\left(1+\dfrac{1}{x}\right)^{x}\right ]$ (o expoente da fração está rasurado)
\task Mostre que: $(1+x)^c=1+cx+o(x)$, quando $x\to 0$. Use isso para calcular:
$$\lim\limits_{x\to \infty}[(x^4+x^2)^{1/2}-x^2]$$

\end{tasks}

\paragraph{Questão 4:}(1,0 ponto)\\
Calcule a área da interseção das regiões limitadas pelas curvas, cujas equações em coordenadas polares são:
$$\rho=3\cos\,\theta\;\text{ e }\;\rho=1+\cos\,\theta$$

\paragraph{Questão 5:}(1,5 pontos)\\
Mostre que:\\
$(x^b+y^b)^{1/b}<(x^a+y^a)^{1/a}$, sempre que $x,y>0$  e  $0<a<b$.

\paragraph{Questão 6:}(1,5 pontos)\\
Uma função $f$ tem uma $3^a$ derivada contínua em todos os pontos e satisfaz a relação:
$$\lim\limits_{n\to 0}\left( 1+x+\dfrac{f(x)}{x}\right)^{1/x}=e^3\;.$$
Calcule: $f(0),f'(0),f''(0)$ e $\lim\limits_{x\to 0}\left(1+\dfrac{f(x)}{x}\right)^{1/x}$


\newpage

\subsubsection{1997 \original{https://drive.google.com/open?id=1-Ie2NamXz-5rITK6ZTzeSxAOnevdjILy}}

\paragraph{Questão 1:} (1,5 pontos)

Uma janela tem a forma de um retângulo com um semicírculo sobreposto, de modo que o diâmetro do semicírculo é a base superior do retângulo. A parte retangular vai receber um vidro transparente e a parte semicircular vai receber um vidro colorido que só permite passar um terço da luminosidade, por metro quadrado, que a do vidro transparente. O perímetro total da janela tem valor fixo $P$. Determine as dimensões da janela de modo que a luminosidade que atravessa a janela seja máxima.

\paragraph{Questão 2:} (1,5 pontos)

Considere uma cápsula esférica com $30$cm de espessura cujo volume seja o dobro do volume do espaço oco dentro dela. Use o método de Newton para calcular o raio externo da cápsula com precisão de três casas decimais.

\paragraph{Questão 3:} (1,5 pontos)

Esboce o gráfico de $f(x)=x(x-1)^{-2/3}$. Apresente todas as informações usuais.

\paragraph{Questão 4:} (1,5 pontos)

\begin{enumerate}[label=(\alph*)]
\item Mostre que para todo $x>0$ vale:
$$e^x>1+x \textrm{ e } e^{-x}>1-x.$$
\item Mostre que para todo $x>0$ vale
$$e^x>1+x+\dfrac{x^2}{2!} \textrm{ e } e^{-x}>1-x+\dfrac{x^2}{2!}.$$

\item Mostre que $2,5<e<3$.
\end{enumerate}

\paragraph{Questão 5:} (1,5 pontos)

Seja $f(x)=\displaystyle\int\limits_0^x(1+t^3)^{-1/2}dt,x>0$.

Mostre que a $f$ é estritamente crescente e que a segunda inversa satisfaz:
$$(f^{-1})^{\prime \prime}(x)=c\cdot (f(x))^2.$$

Determine a constante $c$.

\paragraph{Questão 6:}(1,5 pontos)

Seja $f$ a função definida por $f(x)=e^{-1/x^2},x\neq 0,$ e $f(0)=0$. Determine o polinômio de Taylor de grau $n$ para a $f$ em torno de $x=0$.

\paragraph{Questão 7} (1,5 pontos)

Mostre que $(1+x)^{c}=1+cx+o(x)$ quando $x\rightarrow 0$. Use isto para calcular o limite de $$(x^4+x^2)^{1/2}-x^2\textrm{ quando }x\rightarrow 0.$$

\newpage
\subsubsection{1996 \original{https://drive.google.com/open?id=1MVfM8xAr69dsGtkeIC8nlgWhaQpcBC4j}}

\paragraph{Questão 1:}(1,5 pontos)\\
Esboce o gráfico da função $f(x)=\dfrac{log\,|x|}{x}$.

\paragraph{Questão 2:}(1,5 pontos)\\
Mostre que a função
$$f(x)=\displaystyle\int\limits_{0}^{x}\dfrac{\mathrm{sen}\,t}{t+1}dt\;\text{ para todo }x\geq 0.$$
\paragraph{Questão 3:}(1,0 ponto)\\
Mostre que a equação $x^3+3x=8$ tem somente uma raiz real e calcule-a com precisão de três casas decimais.

\paragraph{Questão 4:}(1,0 ponto)\\
Considerando $f(x)=\mathrm{arctan}\,x-x+\dfrac{x^3}{3}$, mostre que
$$x-\dfrac{x^3}{3}<\mathrm{arctan}\,x\;\;se\;\;x>0.$$

\paragraph{Questão 5:}(1,5 pontos)

\begin{tasks}(1)
\task Um triângulo isósceles está inscrito num círculo de raio $r$. Se o ângulo do vértice(i.e., o ângulo contido pelos lados iguais), denotado por $2a$, estiver restrito entre $0$ e $\dfrac{\pi}{2}$, encontre o valor máximo e o valor mínimo para o perímetro do triângulo.
\task Qual é o raio do menor disco circular grande o suficiente para cobrir todo triângulo isósceles de um dado perímetro $L$?
\end{tasks}
\paragraph{Questão 6:}(2,5 pontos)

\begin{tasks}(3)
\task $\displaystyle\int\dfrac{\mathrm{arcsen}\,x}{x^2}dx$
\task $\displaystyle\int\mathrm{arctan}\sqrt{x}\,dx$
\task $\displaystyle\int \dfrac{dx}{x^4-2x^3}$
\task $\displaystyle\int\sqrt{9-x^2}\,dx$
\task $\displaystyle\int\sqrt{x^2-9}\,dx$
\end{tasks}

\paragraph{Questão 7:}(1,0 ponto)\\
Determine o valor de $c$ tal que
$$\lim\limits_{x\to +\infty}\left(\dfrac{x+c}{x-c}\right)^x=4.$$


\newpage

\subsubsection{1995 \original{https://drive.google.com/open?id=1EltLvOHXbvCQv6Pj4VsqRM5YeL6tuSCs}}
\paragraph{Questão 1:}(2,0 pontos)\\
Calcule a área da região limitada pela curva $\left( x^2+y^2 \right)^3=x^2$.

\paragraph{Questão 2:}(2,0 pontos)\\
A base de um sólido é a região limitada pelo gráfico de uma função não-negativa, $f$, e o eixo-$x$, ao longo do intervalo $\left [ 0,a\right ]$. Todas as seções transversais perpendiculares a esse intervalo são quadrados. O volume do sólido é
$$a^3-2a\,\cos\,a+\left ( 2-a^2 \right )\,\mathrm{sen}\,a\,\text{, para todo } a\geq 0$$
Supondo a $f$ contínua, calcule $f(a)$.

\paragraph{Questão 3:}(2,0pontos)\\
Seja A o valor da integral $\displaystyle\int\limits_{0}^{\pi}\dfrac{\cos\,x}{(x+2)^2}dx$. Calcule a seguinte integral em função de A:
$$\displaystyle\int\limits_{0}^{\pi/2}\dfrac{\mathrm{sen}x\,\cosx}{x+1}dx$$

\paragraph{Questão 4:}(4,0 pontos)\\
Calcule as integrais:

\begin{tasks}(2)
\task $\displaystyle\int\mathrm{cot}\,x\,dx$
\task $\displaystyle\int xlog\,x\,dx$
\task $\displaystyle\int\dfrac{dx}{x\,log\,x}$
\task $\displaystyle\int e^{\sqrt{x}}dx$
\task $\displaystyle\int x\,\mathrm{arctg}x\,dx$
\task $\displaystyle\int\dfrac{x+2}{x^2+x}dx$
\task $\displaystyle\int\dfrac{dx}{\left ( x^2-4x+4\right )\left ( x^2-4x+5\right )}$
\task $\displaystyle\int e^{ax}\cosbx\,dx$
\end{tasks}

\newpage

\subsubsection{1992 \original{https://drive.google.com/open?id=14iVhKKsen7PL486kJdSmH2OZ0S3nZMcr}}

\paragraph{Questão 1:} (2,0)

Prove usando o polinômio de Taylor de ordem $n$ que o número "e" é irracional.

\paragraph{Questão 2:} (2,0)

Encontrar as integrais indefinidas que seguem:

\begin{enumerate}[label=(\alph*)]
\item (1,0) $I_1=\displaystyle\int arctg \sqrt{x} dx$
\item (1,0) $I_2=\displaystyle\int \dfrac{e^{2x}dx}{4-e^{2x}+\sqrt{4-e^{2x}}}$
\end{enumerate}
\paragraph{Questão 3:} (2,0)

Uma função é sempre contínua e satisfaz a equação $$\displaystyle\int\limits_0^x t^2f(t)dt = -\dfrac{1}{2}+x^2+xsen2x+\dfrac{1}{2}cos2x$$ para todo $x$ no domínio de $f$. Calcule $f(\dfrac{\pi}{4})$ e $f^\prime(\dfrac{\pi}{4})$.

\paragraph{Questão 4:} (2,0)

A função logarítmica natural é definida por $lnx=\displaystyle\int\limits_1^x \dfrac{1}{t}dt$, $\forall x>1$.
\begin{enumerate}[label=(\alph*)]
\item (1,0) Interprete geometricamente essa definição. 
\item (1,0) Use a definição para mostrar que $ln(a\cdot b)=ln(a)+ln(b)$, $\forall a>0$ e $b>0$.
\end{enumerate}
\paragraph{Questão 5:} (2,0)

Encontre a área da região interior a $f(\theta)=3sen\theta$ e exterior à LIMAÇON $g(\theta)=2-sen\theta$. 

\newpage

\subsection{REC}
\subsubsection{2017 \original{https://drive.google.com/file/d/138iTg3BISE04jMocXAkEXTX0s5qOXOgQ/view?usp=sharing}}

\paragraph{Questão 1:} (1,0 ponto)

Seja a função $f$ derivável até $2^a$ ordem nos reais tal que $f(0)=0$. 

$$ g(x)=\displaystyle\int\limits_0^x x^{-1}f(t)dt, \  x\neq 0 \ ; \ g(0)=0 $$
A função $g$ é derivável no ponto $x=0$?

\paragraph{Questão 2:}  (2,0 pontos)
O limite abaixo é finito e não nulo:

$$ \lim\limits_{x \rightarrow \infty} \{(x^7+5x^6+2)^c - x\} $$
Determine o valor da constante c e o resultado do limite.

\paragraph{Questão 3:} (1,5 pontos)

Calcule a área da região delimitada pelas curvas $x^4-y^4=2xy, \ x^2 +y^2=1$ e pela reta $x=0$. Considere a região no $1^o$ quadrante dos reais.

\paragraph{Questão 4:} (2,0 pontos)

Calcule o volume obtido pela rotação da região delimitada pelas funções $y=\textrm{arcsen}x$, $y=\textrm{arccos}x$ e pela reta $x=0$  em torno da reta $x=1$. 

Considere que $-\pi/2 \leq \textrm{arcsen}x \leq \pi/2$ e $0 \leq \textrm{arccos}x \leq \pi$. 

\paragraph{Questão 5:} (3,5 pontos)

\begin{tasks}(2)
\task (0,5 ponto) $\displaystyle\int \dfrac{\Ln(tg^3(x))}{\textrm{sen}x \ \textrm{cos}x} \textrm{d}x$

\task (1,0 ponto) $\displaystyle\int \dfrac{\textrm{sen}x-\textrm{cos}x}{1-\textrm{cos}x+\textrm{sen}x}\textrm{d}x$

\task (1,0 ponto) $\displaystyle\int \dfrac{3x^2+4x+2}{x^3+2x^2+x}\textrm{d}x$

\task (1,0 ponto)$\displaystyle\int \dfrac{\textrm{sen}x}{\textrm{sen}(3-x)+\textrm{sen}x}\textrm{d}x$
\end{tasks}
\newpage

\subsubsection{2011 \original{https://drive.google.com/open?id=1ABSgKAa8fr_VMYe_ccdfpNNCbyAkGjUW}}

\paragraph{Questão 1:} (2,0 pontos)

Encontre o valor máximo de $f(x)=x^3-3x$ no conjunto dos números reais $x$ satisfazendo $x^4+36 \leq 13x^2$.

\paragraph{Questão 2:} (1,6 pontos)

Um cone circular reto é obtido girando-se um triângulo retângulo de hipotenusa constante e igual a $6$cm, em torno de um de seus catetos. Determine a taxa de variação do volume do cone no instante em que a altura do cone é $2\sqrt{5}$cm e está aumentando a razão de $2$cm/s.

\paragraph{Questão 3:} (2,0 pontos)

Analise a continuidade de $f(x)=\lim\limits_{t\rightarrow+\infty} \dfrac{ln(1+e^{xt})}{ln(1+e^t)}$ em $\mathbb{R}$.

\paragraph{Questão 4:} (2,4 pontos)

Calcule ou mostre que não existe:

\begin{enumerate}[label=(\alph*)]
\item (0,8) $\displaystyle\int\limits_0^1 \dfrac{dt}{\sqrt{1-t^2}(3\pi arcsent + arccos^2t)}$
\item (0,8) $\lim\limits_{x\rightarrow0} \dfrac{1}{x^2} \displaystyle\int\limits_x^{x^2} \sqrt{t^2+sent}dt$
\item (0,8) $\lim\limits_{x\rightarrow 1}\left(\dfrac{1}{lnx}-\dfrac{1}{x-1} \right)$, usando fórmula de Taylor.
\end{enumerate}
\paragraph{Questão 5:} (2,0 pontos)

Seja $f$ uma função contínua não constante em $\mathbb{R}$ com derivada satisfazendo 

$$(f^{\prime}(x))^2=\left | f(x)-\dfrac{1}{f^2(x)}\right |,\forall x\in \mathbb{R}\textrm{ tal que } f(x) \neq 0$$

\begin{enumerate}[label=(\alph*)]

\item Seja $J$ um intervalo onde $f(x)\geq 1,\forall x\in J$ ou $f(x) \leq 1, \forall x \in J$. Mostre que existe no máximo um ponto em $J$ onde $f^\prime$ se anula.
\item Se $x_0 \in J$ e $f^\prime(x_0)=0$ onde $J$ é o intervalo do tipo em (a), classifique-o como ponto de máximo local, ponto de mínimo local ou ponto de inflexão.

\end{enumerate}
\newpage
\subsubsection{2010 \original{https://drive.google.com/open?id=1DceZ0TLvSXztDyNk7W2EOf5cObPgKvZR}}

\paragraph{Questão 1 (3,0 pts)} Calcule:
\begin{tasks}(1)
\task (1,0 pt) $\displaystyle\int\limits_0^{\pi} \dfrac{x\sen x}{1+ \cos ^2 x}\dd x$. (Sugestão: u = $\pi - x$).
\task (1,0 pt) O valor de $f^{\prime}\left( \dfrac{\pi}{2} \right)$, onde $f(x) = \displaystyle \int \limits_0^{g(x)}\dfrac{1}{\sqrt{1+t^3}}\dd t$ e $g(x) = \displaystyle\int\limits_{0}^{\cos x}(1+ \sen t^2) \dd t$
\task (1,0 pt) A soma $a+b+c+d$, sabendo-se que $\lim\limits_{x\to 0} \dfrac{ax^2 + \sen bx + \sen cx + \sen dx}{3x^2 + 5x^4 + 7x^6} = 8$
\end{tasks}

\paragraph{Questão 2 (2,0 pts)} A figura abaixo representa o gráfico em coordenadas polares das funções $r(\theta) = 2 + \cos 2\theta$ e $r(\theta) = 2 + \sen \theta$. Calcule a área da região sombreada.

\begin{figure}[h]
\centering
\includegraphics[width=8cm]{calculorec2010q2.png}
\end{figure}

\paragraph{Questão 3 (2,0 pts)} Determine o volume da região de interseção de duas esferas de mesmo raio $r$, sendo que o centro de uma esfera encontra=se na superfície da outra.

\paragraph{Questão 4 (3,0 pts)} Seja $f$ uma função real derivável em todo o seu domínio, tal que $f(0) = a > 0$, $f(a) = 0$, $f^{\prime}(x) \neq 0, \, \forall x \in (0,a)$ e $f$ possui inversa contínua $f^{-1}$.

\begin{tasks}(1)
\task (2,0 pts) Prove que $\displaystyle\int_0^a f(x) \dd x = \displaystyle\int\limits_0^a f^{-1}(x)\dd x$.
\task (1,0 pt) Calcule $\displaystyle \int \limits_0^1 (\sqrt[7]{1-x^3} - \sqrt[3]{1-x^7})\dd x$
\end{tasks}



\newpage
\subsubsection{2004 \original{https://drive.google.com/open?id=1cueDlzP6kdOGKXl_dUKGMJ2rdUi4_zSb}}

\begin{multicols}{2}\setlength{\columnsep}{1.5cm}\setlength{\columnseprule}{0.2pt}

\paragraph{Questão 1} (4,0 pontos)\\
Calcule as integrais:
\begin{tasks}(1)
\task $\displaystyle \int \dfrac{x}{\sqrt{1+x^2+\sqrt{(1+x^2)^3}}}\dd x$
\task $\intin (x^2+1)^{-3/2}\dd x$
\task $\intin \dfrac{\sqrt{a+bx}}{x}\dd x$
\task $\intin \dfrac{x^4}{x^4+5x^2 + 4}\dd x$
\end{tasks}

\paragraph{Questão 2} (1,5 pontos)\\
Prove se forem verdadeiras as afirmações abaixo ou dê um contra-argumento se forem falsas:\\
\noindent Se $f(x+y) = f(x)f(y)$ para todo $x$ e $y$, e $f(x) = 1+xg(x)$, onde $g(x) \to 1$ quando $x \to 0$, então: (a)$f^{\prime}(x)$ existe para todo $x$ e, (b) $f(x)=e^x$

\paragraph{Questão 3} (1,5 pontos)\\
Que volume de material é removido de uma esfera de raio $2r$, perfurando-se um buraco de raio $r$ através do centro?

\paragraph{Questão 4} (1,5 pontos)\\
Independente do valor de $b$, existe no máximo um ponto $x$ no intervalo $-1\le x \le 1$ para o qual $x^3 -3x + b =0$. Utilize o Teorema de Rolle, argumentando por contradição, para provar a afirmação acima ou apresente um exemplo que a contrarie.

\paragraph{Questão 5} (1,5 pontos)\\
Calcule os limites:
\begin{tasks}(2)
\task $\lim\limits_{x\to 0} \dfrac{a^x-a^{\sen x}}{x^3}$
\task $\lim\limits_{x\to 0} \left[ \dfrac{(1+x)^{1/x}}{e}\right]^{1/x}$
\task $\lim\limits_{x\to 1^{+}}\dfrac{x^x - x}{1-x+\Ln x}$
\task $\lim\limits_{x\to \infty} \dfrac{\Ln(a+be^x)}{\sqrt{a+bx^2}},\, a,b>0$
\task $\lim\limits_{x\to 0^{+}}\left[\dfrac{\Ln x}{(1+x^2)} - \Ln \left(\dfrac{x}{1+x} \right) \right]$
\end{tasks}

\end{multicols}
\normalsize

\newpage

\subsubsection{2001 \original{https://drive.google.com/open?id=1conbWOg0WPp0NLASd0Nrjh41yYSqEoL5}}

\paragraph{$1^a$ Questão:} (2,0 pontos)

Calcule:

\begin{tasks}(2)
\task $\lim\limits_{x\rightarrow \infty} x(\sqrt{x^2+1}-x)$
\task $n$, tal que: $\lim\limits_{x\rightarrow \infty}\left(\dfrac{nx+1}{nx-1}\right)^x=9$
\task $\lim\limits_{x\rightarrow 0^{+}} \dfrac{x-\textrm{sen}x}{(x\textrm{sen}x)^{3/2}}$
\task $\lim\limits_{x\rightarrow 1}(2-x)^{\textrm{tg}\frac{\pi x}{2}}$
\end{tasks}

\paragraph{$2^a$ Questão:} (1,5 pontos)

Mostre que:

$$ \dfrac{1}{x+\frac{1}{2}} < \textrm{ln}(1+\dfrac{1}{x})< \dfrac{1}{x}, \ x>0. $$

\paragraph{$3^a$ Questão:} (1,5 pontos)

Esboce o gráfico da função: $x^2+y^2-y=\sqrt{x^2+y^2}$ e calcule a área limitada por ele.

\paragraph{$4^a$ Questão:} (1,5 pontos)

O raio de um cilindro circular reto aumenta a uma taxa constante. Sua altura é uma função linear do raio e aumenta 3 vezes
mais rápido que o raio. Quando o raio é $1$m, a altura é $6$m. Quando o raio é $6$m, o volume aumenta a uma taxa de $1\textrm{m}^3/s$.
Quando o raio é $36$m, o volume aumenta a uma taxa de $n\textrm{m}^3/s$, onde $n$ é um inteiro. Calcule $n$.

\paragraph{$5^a$ Questão} (1,5 pontos)

Dado que:
$$ \textrm{sen}x=\sum_{k=1}^n (-1)^{k-1} \dfrac{x^{2k-1}}{(2k-1)!} + \textrm{E}_{2n}(x),\textrm{ onde } \vert \textrm{E}_{2n}(x) \vert \leq \dfrac{\vert x \vert ^{2n+1}}{(2n+1!)};$$

\begin{enumerate}[label=(\alph*)]
\item Obtenha o número $r=\sqrt{15}-3$ como uma aproximação para a raiz não-nula da equação $x^2=\textrm{sen}(x)$ usando a aproximação polinomial
cúbica de Taylor para $\textrm{sen}(x)$.
\item Mostre que a aproximação de (a) satisfaz a desigualdade:
$$ \vert \textrm{sen}r-r^2 \vert < \dfrac{1}{200},\textrm{ dado que } \sqrt{15}-3<0,9.$$
\end{enumerate}

\paragraph{$6^a$ Questão} (1,0 ponto)

Calcule:

\begin{tasks}(2)
\task $\intin \dfrac{x\textrm{d}x}{\sqrt{1+x^2+\sqrt{(1+x^2)^3}}}$
\task  $\intin \dfrac{\Ln{\vert x \vert}}{x\sqrt{1+\Ln{\vert x \vert}}} \dd x$
\end{tasks}

\newpage

\subsubsection{1999 \original{https://drive.google.com/open?id=1c-X_Pb0kJu0iDxKeTBe8SgbmrPBg6ubY}}

\noindent Prova que vale \textbf{10,5} é o \textbf{bizu!!!}.

\paragraph{$1^a$ Questão:} (1,5 pontos) 

Considere a curva $(x^2+y^2)^2=y^2-x^2$ com $x^2 \leq y^2$.

\begin{enumerate}[label=(\alph*)]
\item Esboce o gráfico desta curva.
\item Calcule a área da região limitada pela curva.
\end{enumerate}

\paragraph{$2^a$ Questão:} (1,5 pontos)

Esboce o gráfico da função $f(x)=x^2-18\Ln{x}$. Indique os sete passos.

\paragraph{$3^a$ Questão:} (1,5 pontos)

Calcule a menor distância, na vertical, entre as curvas $y=x^2$ e $y=-\dfrac{1}{x^2}$.

\paragraph{$4^a$ Questão:} (1,5 pontos)

Sejam $g(x)=x^c e^{2x}$ e $f(x)= \displaystyle\int\limits_{0}^x e^{2t}(3t^2+1)^{\frac{1}{2}} \dd t$. Para um certo valor de $c$, o limite de $\dfrac{f^\prime(x)}{g^\prime(x)}$,
quando $x \rightarrow +\infty$, é finito e diferente de zero. Determine $c$ e o valor do limite.

\paragraph{$5^a$ Questão:} (1,5 pontos)

Sejam $n$ um inteiro positivo e $x>0$. Mostre que:
\begin{enumerate}[label=(\alph*)]
\item $\left( 1+ \dfrac{x}{n} \right) ^n < e^x$.
\item $e^x<\left( 1-\dfrac{x}{n} \right)^{-n}$, se $x<n$.
\item $2,5<e<2,99$. (Sugestão: use os itens acima.)
\end{enumerate}

\paragraph{$6^a$ Questão:} (1,5 pontos)

Seja $f$ uma função integrável e periódica de período $p>0$. Mostre que para todo $a \in \mathbb{R}$, 

$$ \displaystyle\int\limits_a^{a+p} f(x) \dd x =  \displaystyle\int\limits_0^p f(x) \dd x .$$

\paragraph{$7^a$ Questão:} (1,5 pontos)

\begin{enumerate}[label=(\alph*)]
\item Mostre que $(1+x)^c = 1+ cx + o(x)$, quando $x \rightarrow 0$.
\item Calcule o limite de $(x^4+x^7)^\frac{1}{2} - x^2$, quando $x \rightarrow + \infty$.
\end{enumerate}


\newpage

\subsubsection{Simulado 2011 \original{https://drive.google.com/open?id=1jlyBiHJz9APkO5SgN01Sl8AD3IlEF9cq}}

\begin{enumerate}
\item Calcule os limites abaixo:
\begin{enumerate}[label=(\alph*)]
\item $\lim\limits_{x\to 0^{+}} x^{x^x}$
\item $\lim\limits_{x\to +\infty} (\sqrt{x+\sqrt{x+\sqrt{x+\sqrt{x}}}} - \sqrt{x})$
\item $\lim\limits_{x\to 0^+}\sqrt{1/x + \sqrt{1/x + \sqrt{1/x}}}-\sqrt{1/x - \sqrt{1/x}}$
\end{enumerate}

\item Esboce o gráfico da função: $$f(x) = (2x^2-x^3)^{1/3}$$

\item Seja $f:\R^{+}\to \R$ uma função derivável em todo o seu domínio tal que: $$f(x,y) = f(x) + f(y)$$Mostre que: $$f(x)=c\cdot \Ln x,\, c\in \R$$

\item Use o TVM para derivadas para provar que para todo $x$ real tem-se que:

$$(a) \, e^x \ge x+1$$(Dica (a): aplique o TVM à função: $$f(x) = e^x$$no intervalo [0;x] para $x>0$ e [x;0] para $x<0$) $$(b) \, e^x \ge x^2/2 + x + 1$$(Dica (b): utilize o TVM de Cauchy e o resultado da letra (a). Pense bem para a escolha das funções para aplicar o teorema)

**\textbf{Desafio:} Pode-se provar por indução finita que: $$e^x \ge 1+\displaystyle\sum\limits_{k=1}^n x^k/k!,\, \forall n\, \text{natural}$$

\item Resolva a questão 4 - (a) fazendo o esboço do gráfico da função: $$g(x) = e^x - x - 1$$

\item (Questão do Cap Rocha\dots nosso Major Rocha S2)

\begin{enumerate}[label=(\alph*)]
\item Encontre o volume de um tronco de cone de altura $h$, raio menor $r_1$ e raio maior $r_2$.
\item Encontre o volume máximo, dado que: $r_1+r_2 = S$
\end{enumerate}

\end{enumerate}

\newpage

\section{Álgebra Linear I}

\subsection{VI}
\subsubsection{2017 \original{https://drive.google.com/file/d/1QADTZEY0qs7DyY9__lcWPd54k74lXG3r/view?usp=sharing}}
\paragraph{Questão Única:} (2,0 pontos)\\
Sejam os polinômios reais definidos em $\mathbb{R}$:
$$u_1(x)=1, \ \ \ u_2(x)=1+x, \ \ \ u_3(x)=1+x+x^2,\ . . . \ , u_k(x)=1+x+\dots+x^{k-1}$$
com $k$ um inteiro positivo. Demonstre que o conjunto $\{u_1,u_2,\dots,u_k\}$ é linearmente independente.
\newpage
\subsection{VE 1}

\subsubsection{2019 \original{https://drive.google.com/open?id=1hLRvBgNvdTu0GubhBeQP4kDxIlHchgam}}

\paragraph{Questão 1:} (1,0 ponto)\\
Seja $r \geq 2$ e seja $\{v_1, \dots, v_r\}$ um conjunto linearmente dependente (LD) de vetores de $\mathbb{R}^n$ tal que qualquer conjunto com $r-1$ vetores é linearmente independente (LI). Sejam $A = (\alpha_1, \dots, v_r)$ e $B= (\beta_1, \dots, \beta_r)$ dois vetores de coeficientes em $\mathbb{R}^r$ tais que 
\begin{equation*}
    \sum_{i=1}^{r} \alpha_iv_i = 0 \textnormal{     e    }  \sum_{i=1}^{r} \beta_iv_i = 0
\end{equation*} 
Prove que os dois vetores $A$ e $B$ são paralelos.

\vspace{2.0cm}


\paragraph{Questão 2:} (1,0 ponto)\\
Sejam $A = (2, -3, 0, \sqrt{3})$ e $B = (1, 0, \sqrt{2}, -1)$ dois vetores em $\mathbb{R}^4$. Encontre um terceiro vetor $C \in \mathbb{R}^4$ de comprimento unitário que faz ângulo de $\dfrac{\pi}{3}$ radianos com $A$ e $B$.


\newpage

\subsubsection{2018 \original{}}

\paragraph{Questão 1:} (1,5 pontos)

Enuncie e demonstre a desigualdade de Cauchy-Schwarz para dois vetores no espaço vetorial $V_n(\mathbb{C})$.

\paragraph{Questão 2:} (1,5 pontos)

Considere as retas $L(P; A)$ e $L(Q; B)$ em $V_n(\mathbb{R})$ não paralelas. Prove que a intersecção delas é um conjunto vazio ou consiste de um único ponto.

\paragraph{Questão 3:} (2,0 pontos)

Sejam $v_1, \dots, v_{r+1}$ vetores de $\mathbb{R}^n$ linearmente dependentes, tais que $v_1, \dots, v_r$ são linearmente independentes. Dados $\alpha_i \in \mathbb{R}$ com $i=1, \dots, r+1$, nem todos nulos, onde: 
\begin{equation*}
    \sum_{i=1}^{r+1} \alpha_i v_i = 0
\end{equation*}
Prove que se existem quaisquer outros escalares $\beta_i \in \mathbb{R}$, com $i = 1, \dots, r+1$, onde:
\begin{equation*}
    \sum_{i=1}^{r+1} \beta_i v_i = 0
\end{equation*}
devem satisfazer a igualdade:
\begin{align*}
    \beta_i = \lambda\alpha_i \textnormal{, com $i = 1, \dots, r+1$}
\end{align*}
para algum $\lambda \in \mathbb{R}$.


\newpage

\subsubsection{2017 \original{https://drive.google.com/file/d/1IOoOK7V1SkLl35WRNMCb5YPvpzosOWC7/view?usp=sharing}}

\paragraph{Questão 1:} (2,0 pontos)

Considere o conjunto $V=\mathbb{R}^2$ munido das seguintes operações: 
$$\begin{cases}
(x_1,y_1) \oplus (x_2,y_2)=(x_1+x_2,y_1+y_2-1)\\
\alpha \odot (x_1,y_1) = (\alpha x_1 + (1-\alpha)y_1,\alpha y_1) 
\end{cases}$$
onde $(x_1,y_1),(x_2,y_2) \in \mathbb{R}^2$ e $\alpha \in \mathbb{R}$. Verifique se V é um espaço vetorial. Caso negativo, identifique os axiomas que não se verificam.

\paragraph{Questão 2:} (2,0 pontos)

Para cada um dos itens a seguir, classifique as afirmações como verdadeiras ou falsas. Demonstre as afirmações verdadeiras e mostre um contra-exemplo para as falsas.
\begin{enumerate}[label=(\alph*)]
\item (1,0 ponto) Se $\{A, B, C\}$ é um conjunto linearmente dependente de vetores não nulos, então cada vetor pode ser obtido como combinação linear dos demais.

\item (1,0 ponto)

Seja $A$ uma matriz fixa, $A\in \mathbb{M}_{n \times n}(\mathbb{R})$. O conjunto $S=\{B \in \mathbb{M}_{n \times n}(\mathbb{R}) \ | \ AB \neq BA\}$ é um subespaço vetorial de $\mathbb{M}_{n \times n}(\mathbb{R})$. 
\end{enumerate}

\paragraph{Questão 3:} (2,0 pontos)

Sejam as funções reais definidas em $\mathbb{R}$:
$$u_1(x)=x, \ \ \ u_2(x)=e^x, \ \ \ u_3=sen(x), \ \ \ u_4=cos(x) $$

Classifique o conjunto $\{u_1,u_2,u_3,u_4\}$ como linearmente dependente (LD) ou linearmente independente (LI). Justifique sua resposta.

\paragraph{Questão 4:} (2,0 pontos)

Seja $E$ um espaço vetorial, e sejam $F,G,H$ subespaços de $E$ tais que $G \subset F$. Demonstre que:
$$F \cap (G+H) = G + (H \cap F) $$.

\newpage
\subsubsection{2016 \original{https://drive.google.com/file/d/1udj9wCi2iZC4ZcMNgXBmm9L8D53rQsTK/view?usp=sharing}}

\paragraph{Questão 1:} (1,5 pontos)

Sejam $u_1,\dots,u_r$ vetores linearmente independentes de um espaço vetorial $V$, e seja $v=\lambda_1u_1+\dots+\lambda_ru_r$, com $\lambda_1 \neq 0$. Prove que $\langle u_1,\dots,u_r \rangle = \langle v,u_2,\dots,u_r \rangle$.
\\ 

\textbf{Notação}: $\langle u_1,\dots,u_r \rangle = L\{u_1,\dots,u_r\}$

\paragraph{Questão 2:} (2,0 pontos)

Sejam $\pi$ um plano e $r$ uma reta no espaço $\mathbb{R}^3$. Prove que $\pi\cap r$ consta exatamente de um ponto se, e somente se, o vetor diretor de $r$ e dois vetores não paralelos do subespaço diretor de $\pi$ são linearmente independentes.

\paragraph{Questão 3:} (1,5 pontos)

Seja $r$ uma reta do $\mathbb{R}^4$ que passa pelo ponto $P=(1,2,-3,3)$ e possui vetor diretor $A=(1,2,-3,0)$. Seja $s$ uma reta que passa pelo ponto $Q=(-6,-2,-3,1)$ e possui vetor diretor $B=(5,1,2,-2)$. Determine a reta $t$ do $\mathbb{R}^4$ perpendicular comum às retas $r$ e $s$.

\newpage

\subsubsection{2015 \original{https://drive.google.com/file/d/1Y3kN_HtBe9hokdZkmQhndYtwYu_B7tkT/view?usp=sharing}}

\paragraph{Questão 1} (2,5 pontos)\\
Seja $\{g_1, g_2, g_3\}$ uma base ortonormal do $\R^3$. Para todo $u\in \R ^3$ definem-se os cossenos diretores de $u$ em relação à base dada por $\cos \alpha = \dfrac{u\cdot g_1}{\norm{u}}$, $\cos \beta = \dfrac{u\cdot g_2}{\norm{u}}$ e $\cos \gamma = \dfrac{u\cdot g_3}{\norm{u}}$. Prove que:
\begin{tasks}(1)
\task (1,5 pontos) $u = \norm{u}\left( (\cos \alpha)g_1 + (\cos \beta)g_2 + (\cos \gamma)g_3 \right)$;
\task (1,0 pontos) $\cos^2 \alpha + \cos ^2 \beta + \cos ^2 \gamma = 1$.
\end{tasks}

\paragraph{Questão 2} (2,5 pontos) (\textbf{Questão Repetida} - Questão 6 da VC de 2007) \\
No trapézio da figura abaixo, $C = B-\lambda A$, onde $\lambda > 0$. Prove que a razão entre $\norm{C}$ e $\norm{B-A}$ é igual a $\sqrt{\lambda}$.

\begin{figure}[h]
\centering
\includegraphics[width=8cm]{alglinve12015q2.png}
\end{figure}

\paragraph{Questão 3} (2,5 pontos)\\
Usando apenas conceitos da álgebra vetorial, sem recorrer às coordenadas, mostre que as três mediatrizes de um triângulo $ABC$ cortam-se em um ponto $O$, o qual é chamado de circuncentro do triângulo $ABC$, ou seja, é o centro da circunferência circunscrita ao triângulo $ABC$.

\paragraph{Questão 4} (2,5 pontos)\\
Dadas duas retas $L(P;A)$ e $L(Q;B)$ em $V_n$ não paralelas, provar que a interseção ou é vazia ou consta de um único ponto.

\newpage

\subsubsection{2014 \original{https://drive.google.com/file/d/1pmBk2jC-BNRFREGX4zO421-QUWgot3Hg/view?usp=sharing}}

\paragraph{Questão 1} (2,0 pontos)\\
Prove que se $v_1, v_2, \dots, v_r$ são vetores dois a dois ortogonais em $\R^n$, então $\norm{v_1+v_2+\dots+v_r}^2 = \norm{v_1}^2+\norm{v_2}^2 + \dots + \norm{v_r}^2$.

\paragraph{Questão 2} (3,0 pontos)\\
Seja $S$ a área de um triângulo $ABC$. Seja $S'$ a área do triângulo formado pelas medianas do triângulo $ABC$. Sabendo-se que o $O$ não pertence ao conjunto ${A, B, C}$, utilizando apenas conceitos da álgebra vetorial, sem recorrer às coordenadas dos vértices, mostre que $S'=\dfrac{3}{4} \,S$.

\paragraph{Questão 3} (5,0 pontos)\\
Classifique cada sentença abaixo com $V$, caso seja verdadeira, ou com $F$, caso seja falsa, sempre justificando, em qualquer caso, com uma prova ou com um contra-exemplo.

\begin{tasks}(1)
\task (1,0 ponto) A união de dois planos não paralelos é um subespaço do $\R ^n$ se, e somente se, ambos contém a origem;
\task (1,0 ponto) Seja $L(P;A)$ uma reta e $M=\{Q+sB+\lambda D\}$ um plano, tal que $A$ é paralelo a $B\times D$. Então $\#(L(P;A)\cap M) = 1$;
\task (1,0 ponto) Sejam $M_1 = \{P+xA+yB\}$ e $M_2=\{Q+zC+wD\}$ planos no $\R^4$, tais que $\{A, B, C, D\}$ é um conjunto $LI$. Então, $\#(M_1\cap M_2)=1$;
\task (1,0 ponto) Seja $S=\{A_1, \dots, A_k\}$ contido no $\R^n$ um conjunto $LI$, $V=L(S)$, $k<n-3$ e $\{F,G,H\}$ contido em $\R^n - V$. Então, a dimensão de $L(S\cup \{F, G, H\})$ vale $k+3$;
\task (1,0 ponto) Seja $S=\{A_1, \dots, A_k\}$ um subconjunto ortogonal do espaço das n-uplas. Então, $\dim L(S) \ge k-1$.
\end{tasks}
\newpage

\subsubsection{2013 \original{https://drive.google.com/file/d/1bwk1mAn79FwlUtRBnnPp72HRZ8yx1K2v/view?usp=sharing}}

\paragraph{Questão 1} (1,0 pontos)\\
Nas sentenças abaixo, preencha com $V$, caso seja verdadeira ou $F$, caso seja falsa. Não precisa justificar.
\begin{tasks}
\task (0,5 pt) (\quad) O cosseno do ângulo entre os planos $x+2y-4z = 5$ e $2x-y+3z=7$ é $\dfrac{-2}{7\sqrt{6}}$
\task (0,5 pt) (\quad) Se $A, B, C$ são três vetores linearmente independentes de $V_n$, então $(A-B),\, (B+C),\, (A+C)$ é LI.
\end{tasks}

\paragraph{Questão 2} (1,0 ponto)\\
Considere o conjunto $\alpha = \{ (-1, 3, 1), (1, -2, 4) \}$, determine:

\begin{tasks}(1)
\task (0,5 pt) O espaço gerado por $\alpha$;
\task (0,5 pt) O valor de $k\in \R$ para que $v=(5, k, 11)$ pertença ao espaço gerado por $\alpha$.
\end{tasks}

\paragraph{Questão 3} (1,0 pontos)\\
Ache um vetor do $\R^3$ que gere a interseção de $U$ e $W$, onde $U$ é gerado por $[ (a, b, 0) ]$ e $W$ gerado pelos vetores $(1,2,3)$ e $(1, -1, 1)$.

\paragraph{Questão 4} (1,0 pontos)\\
Qual o múltiplo de $a=(1,1,1)$ está mais próximo do ponto $b=(2,4,4)$?

\paragraph{Questão 5} (2,0 pontos)\\
Dados dois vetores não paralelos $A$ e $B$ e $V_n$, com $A\cdot B = 2,\, \norm{A} = \norm{B} = 4$. Seja $C = 2(A\times B) - 3B$. Calcule o valor do ângulo entre $B$ e $C$.

\paragraph{Questão 6} (2,0 pontos)\\
Considere um triângulo isósceles cujos lados congruentes estão sob as retas $y=1$ e $\sqrt{3}x - y + 1 = 0$. Sabendo que uma altura desse triângulo é [ilegivel] determine:

\begin{tasks}
\task (1,0 pt) O ponto médio da base;
\task (1,0 pt) A área do triângulo.
\end{tasks}

\paragraph{Questão 7} (2,0 pontos)\\
Sejam $A, B, C, D, E, F$ os vértices, em ordem, de um hexágono regular. Sendo $\overline{AD} = (\sqrt{2}, \sqrt{5})$, dê o valor de $\overline{AB}+ \overline{AC} + \overline{AD} + \overline{AE} + \overline{AF}$.

\newpage

\subsubsection{2012 \original{https://drive.google.com/file/d/1cPTDvUHajrI7FbYzNAGMJXli8peQZyDI/view?usp=sharing}}

\paragraph{Questão 1} (1,5 pontos)\\
Nas sentenças abaixo, preencha com $V$, caso seja verdadeira ou $F$, caso seja falsa, justificando cada caso. Nos casos em que a sentença seja falsa, basta apresentar um contra-exemplo. Sentença sem justificativas não serão consideradas.

\begin{tasks}(1)
\task (0,5 pt) (\quad) Se os espaços das combinações lineares de dois conjuntos são iguais, os conjuntos são iguais.
\task (0,5 pt) (\quad) Para quaisquer vetores $u,\,v$ e $w$, os vetores $u-v$, $v-2$ e $w-u$ formam um conjunto linearmente dependente.
\task (0,5 pt) (\quad) Dados três vetores não nulos $A$, $B$ e $C$ de $\R^n$. Suponha que o ângulo entre $A$ e $C$ é igual ao ângulo entre $B$ e $C$. O valor do produto escalar $C\cdot (\norm{B}A- \norm{A}B)$ é sempre positivo.
\end{tasks}

\paragraph{Questão 2} (1,0 ponto)\\
Determine o cosseno do menor ângulo que o plano que contém os pontos $P(1,0,2)$, $Q(-1,3,4)$ e $R(2,5,3)$, faz com a reta: $x=2+3t,\quad y=1-t,\quad z=4t$.

\paragraph{Questão 3} (1,0 ponto)\\
Utilize a desigualdade de Cauchy-Schwarz para provar a desigualdade triangular em $\R^n$.

\paragraph{Questão 4} (1,5 pontos)\\
Sejam $P$ e $Q$ dois pontos distintos de $\R^n$. Mostre que existe uma reta que passa por $P$ e $Q$ e que esta reta é única.

\newpage

\subsubsection{2011 \original{https://drive.google.com/file/d/1uCnndlKdxQAfkaMKY4NBQNDOhq1EG5uW/view?usp=sharing}}

\paragraph{Questão 1} (1,0 ponto)\\
Seja $S$ um conjunto de vetores de $\R^n$. Mostre que se $S$ gera de maneira única o vetor $0$, então $S$ gera todos os vetores de $L(S)$ de maneira única.

\paragraph{Questão 2} (1,5 ponto)\\
Sejam $A$, $B$ e $C$ vetores de $\R ^3$ tais que $B$ e $C$ são linearmente independentes. Sejam $P$ e $Q$ vetores de $\R^3$ tais que $(P-Q)$ não pertence ao subespaço gerado por $\{B, C\}$. Determine a intersecção da reta que passa por $P$ e $Q$ com o plano $M=\{A+sB+tC\}$. A resposta deve estar na forma vetorial em função de $A$, $B$, $C$, $P$ e $Q$. (Dica: Explore o fato do problema ser em $\R^3$.)

\paragraph{Questão 3} (1,5 ponto)\\
Determine o cosseno do menor ângulo que o plano que contém os pontos $P(1,0,2)$, $Q(-1,3,4)$ e $R(2,5,3)$, faz com a reta: $x=2+3t,\quad y=1-t,\quad z=4t$.

\paragraph{Questão 3} (1,0 ponto)\\
Utilize a desigualdade de Cauchy-Schwarz para provar a desigualdade triangular em $\R^n$.

\paragraph{Questão 4} (1,0 pontos)\\
Dados três vetores não nulos $A$, $B$ e $C$ de $\R^n$. Suponha que o ângulo entre $A$ e $C$ é igual ao ângulo entre $B$ e $C$. Calcule o valor do produto escalar $C\cdot (\norm{B}A- \norm{A}B)$.

\newpage

\subsubsection{2008 \original{https://drive.google.com/file/d/1eYsX_2o8P21uQGp9MoAtEEim-2_Syw0v/view?usp=sharing}}

\paragraph{Questão 1 (2,5 pts)} Através da Álgebra Vetorial e sem recorrer às coordenadas, prove que as diagonais de um paralelogramo se cruzam no ponto médio de cada uma delas.

\paragraph{Questão 2 (2,5 pts)} A figura abaixo representa um triângulo no $\R^n$, onde a origem $O$ não coincide com nenhum dos pontos $A$, $B$, $C$, $X$, $Y$ e $M$. Se $M$ é o baricentro do triângulo dado, prove que $M=\dfrac{A+B+C}{3}$ mediante uso de Álgebra Vetorial e sem recorrer às coordenadas.

\begin{figure}[h]
\centering
\includegraphics[width=8cm]{alglinve12008q2.png}
\end{figure}

Sugestão: os vetores $C-A$ e $B-A$ são linearmente independentes.

\paragraph{Questão 3 (2,5 pts)} Dados $A$ e $B\in \C ^n$:

\begin{tasks}(1)
\task Prove que $A\cdot B+B\cdot A$ é real e $-2 \le \dfrac{A\cdot B + B\cdot A}{\norm{A}\norm{B}} \le 2$;
\task Prove que se $\theta : = \mathrm{arccos} \dfrac{A\cdot B + B\cdot A}{2\norm{A}\norm{B}}$, então $\norm{A-B}^2 = \norm{A}^2 + \norm{B}^2 - 2\norm{A}\norm{B}\cos \theta$;
\task Considerando $\theta \in [0,\pi]$ definido anteriormente, calcule $\cos \theta$ para $A, B \in \C^3$ dados por $A=(1,0,-i)$ e $B=(1+i, 1-i, 1)$.
\end{tasks}

\paragraph{Questão 4 (2,5 pts)} Prove que, no $\R^n$, as retas $L(P;A)$ e $L(Q;B)$ se intersectam se, e somente se $P-Q$ pertence ao subespaço gerado por $\{A,B\}$. Utilizando este fato, determine se as retas do $\R^3$ $L_1=\{(1,1,-1)+t(-2,1,3) \}$ e $L_2 = \{(3,-4,1)+s(-1,5,2) \}$ se intersectam ou não.

\newpage

\subsubsection{2007 \original{https://drive.google.com/file/d/1GfMdrGdJLV1xukAoFEDlLR7_gPH-VDFW/view?usp=sharing}}

\paragraph{Questão 1} (3,0 pontos)\\
Usando os conceitos de Álgebra Vetorial, sem recorrer a coordenadas de vetores, provar:
\begin{tasks}(1)
\task (1,5 pontos) Sejam $A$, $B$ e $C$ os vértices de um triângulo inscrito em uma circunferência com centro em $O$, conforme a figura abaixo. Então esse triângulo é retângulo.
\task Sejam $A$, $B$ e $C$ os vértices de um triângulo qualquer. Então, as três alturas interceptam-se em um único ponto $H$, denominado ortocentro.
\end{tasks}

\begin{figure}[h]\centering\includegraphics[width=8cm]{alglinve12007q1.png}\end{figure}

\paragraph{Questão 2} (2,0 pontos)\\
Seja $S=\{x_1, x_2, \dots, x_n\}$ um conjunto de vetores LI.
\begin{tasks}(1)
\task (1,0 ponto) Dado um vetor $Y$ qualquer, tal que $\{Y\} \cup S$ seja LD, prove que $Y \in L(S)$.
\task (1,0 ponto) Dado $T = \{Y_1, Y_2, \dots, Y_n\} \subset L(S)$. Prove que $L(T) \subset L(S)$.
\end{tasks}

\paragraph{Questão 3} (2,0 pontos)\\
Seja $L=\{P+tA\}$ uma reta e $N=\left(P-\dfrac{P\cdot A}{A\cdot A}A  \right)$. Prove que $X\in L \iff \modu{X\cdot A} = \norm{X-N}\norm{A}$.

\paragraph{Questão 4} (3,0 pontos)\\
Seja $\Omega$ o subconjunto do $\R^n$ formado por todos os vetores $U$ de norma igual a 1 e seja $L=\{P+tA\}$ uma reta qualquer. A distância entre $\Omega$ e $L$, denotada por $d(L;\Omega)$, é dada pelo valor mínimo de $d(L,U)$, que é a distância entre a reta $L$ e um ponto $U\in \Omega$. Utilizando a Álgebra Vetorial e sem recorrer às coordenadas de vetores ou à Geometria Euclidiana Clássica, prove as seguintes afirmações:
\begin{tasks}(1)
\task (1,5 pontos) A interseção $L\cap \Omega$ é não-vazia, se e somente se, $(P\cdot N) \le 1$, onde $N=\left(P-\dfrac{P\cdot A}{A\cdot A}A  \right)$.
\task (1,5 pontos) Se a interseção $L\cap \Omega$ é vazia, então $d(L,\Omega) = (P\cdot U_N(-1$, onde $U_N$ é o vetor $N=\left(P-\dfrac{P\cdot A}{A\cdot A}A  \right)$ normalizado.
\end{tasks}

\newpage

\subsubsection{2005 \original{https://drive.google.com/file/d/1wDKOotxwp-mxfQW3jt-Kze9_wpE9ngzC/view?usp=sharing}}

\paragraph{Questão 1 (2,4 pts)} Classifique cada sentença abaixo com $V$, caso seja verdadeira, ou $F$, caso seja falsa, sempre justificando, em qualquer caso, com uma prova ou contra-exemplo.
\begin{tasks}
\task (0,8 pt) (\quad) Seja $S=\{A_1, \dots, A_k\}\subset \R^n$ um conjunto LI, $V=L(S)$, $k< n-3$ e $\{F,G,H\}\subset \R^n\\V$. Então a dimensão de $L\left( S \cup \{F,G,H\} \right)$ vale $k+3$;
\task (0,8 pt) (\quad) Seja $S:= \{A_1, \dots, A_k\}$ um subconjunto ortogonal do espaço das n-uplas. Então $\dim L(S) \ge k-1$;
\task (0,8 pt) (\quad) Sejam $M_1 := \{P+xA+yB\}$ e $M_2 := \{Q+zC+wD\}$ planos no $\R^n$, tais que $\modu{A\cdot C} = \norm{A}\norm{C}$ e $\{B,C,D\}$ é LD. Então $M_1$ é paralelo a $M_2$.
\end{tasks}

\paragraph{Questão 2 (2,6 pts)} A figura plana abaixo representa um triângulo no $\R^n$, onde a origem $O$ não coincide com nenhum dos pontos $A, B, C, X = \dfrac{A+C}{2}$ e $Y =\dfrac{B+C}{2}$. Se $\norm{\overrightarrow{AY}} = \norm{\overrightarrow{BX}}$, então prove que o triângulo dado é isósceles mediante uso de Álgebra Vetorial e sem recorrer às coordenadas.

\begin{figure}[h]\centering\includegraphics[width=8cm]{alglinve12005q2.png}\end{figure}

\paragraph{Questão 3 (2,5 pts)} Dados $A$ e $B\in \C ^n$:

\begin{tasks}(1)
\task (1,0 pt) Prove que $A\cdot B+B\cdot A$ é real e $-2 \le \dfrac{A\cdot B + B\cdot A}{\norm{A}\norm{B}} \le 2$;
\task (1,0 pt) Prove que se $\theta : = \mathrm{arccos} \dfrac{A\cdot B + B\cdot A}{2\norm{A}\norm{B}}$, então $\norm{A-B}^2 = \norm{A}^2 + \norm{B}^2 - 2\norm{A}\norm{B}\cos \theta$;
\task (0,5 pt) Considerando $\theta \in [0,\pi]$ definido anteriormente, calcule $\cos \theta$ para $A, B \in \C^3$ dados por $A=(1,0,-i)$ e $B=(1+i, 1-i, 1)$.
\end{tasks}

\paragraph{Questão 4 (2,5 pts)} Dado o plano $M=\{P+sA+tB\}$, onde $P=(2,3,1)$, $A=(a^2, a, 1)$, $B=(3,2,1)$ e $\Omega$ é outro plano de equação cartesiana $x-by+z=a$, determine:
\begin{tasks}(1)
\task (1,5 pts) O valor de $a$ e $b$ para que $M$ e $\Omega$ sejam paralelos;
\task (1,0 pts) A reta $L=\{G+xH\}$, interseção de $M$ com o plano $\Gamma$ de equação cartesiana $x+2y+z=4$, considerando o valor de $a$ obtido no item anterior.
\end{tasks}

\newpage

\subsubsection{2004 \original{https://drive.google.com/file/d/1y9u_e-tM-S_zBRItyYFeKysdHO2Qgqep/view?usp=sharing}}

\paragraph{Questão 1:} (2,0 pontos)

Dados os pontos $A=(1,-2,3), B=(-5,2,-1)$ e $C=(4,0,-1)$. Determine o ponto $D$ tal que $A,B,C$ e $D$ sejam vértices consecutivos de um paralelogramo.

\paragraph{Questão 2:} (2,0 pontos)

Considerando o conjunto de vetores $\{(2,0,2,0),(2,2,2,2),(0,2,0,2),(4,0,-2,2)\}$, verifique se o conjunto é linearmente independente ou linearmente dependente. Caso seja linearmente dependente, apresente o maior subconjunto linearmente independente do conjunto dado.

\paragraph{Questão 3:} (2,0 pontos)

\begin{enumerate}[label=(\alph*)]
\item Provar que para dois quaisquer vetores $A$ e $B$ de $V_n(\mathbb{C})$, a soma $\overline{A\cdot B}+A\cdot B$ é real. 
\item Se $A$ e $B$ são vetores não nulos em $V_n(\mathbb{C})$, provar que $-2 \leq \dfrac{A\cdot B + \overline{A\cdot B}}{\norm{A}\norm{B}} \leq 2$.
\end{enumerate}
\paragraph{Questão 4:} (2,0 pontos)

Sejam $A,B$ e $C$ pontos quaisquer com $A\neq B$. Prove que um ponto $X$ pertence à reta determinada por $A$ e $B$ se, e somente se, $\overrightarrow{CX}=\alpha \overrightarrow{CA}+\beta \overrightarrow{CB}$, com $\alpha+\beta=1$, onde $\alpha,\beta \in \mathbb{R}$.

\paragraph{Questão 5:} (2,0 pontos)

Dados os pontos $P(1,0,2),Q(-1,3,4)$ e $S(2,5,3)$ e sendo $\theta$ o ângulo formado pela reta $\begin{cases}
x=2+3t \\
y=1-t  \\
z=4t   \\
\end{cases}$
e o plano que contém os pontos $P,Q$ e $S$, determine $\cos\theta$.
\newpage
\footnotesize
\subsubsection{2003 \original{https://drive.google.com/file/d/138JIQV5mLv4zY-iHUJ274VbzcvAllqii/view?usp=sharing}}

\paragraph{Questão 1:} (2,0 pontos)

A identidade da geometria plana dada abaixo pode ser demonstrada vetorialmente. Considerando dois vetores não colineares $A$ e $B$, obtenha tal identidade e a demonstre usando a Álgebra Vetorial.

"A soma dos quadrados dos lados de um triângulo qualquer é igual ao triplo da soma dos quadrados das distâncias do baricentro desse triângulo a cada um de seus vértices".

\paragraph{Questão 2:} (2,0 pontos)

\begin{enumerate}[label=(\alph*)]
\item Dados dois vetores $A$ e $B$ não nulos em $V_n(\mathbb{C})$, defina o ângulo $\theta$ entre eles, justificando a unicidade do mesmo no intervalo $0 \leq \theta \leq \pi$; 
\item Mostre que $\norm{A-B}^2=\norm{A}^2+\norm{B}^2-2\norm{A}\norm{B}$;
\item Dados os vetores $A=(1,0,-i)$ e $B=(1+i,1-i,1)$ pertencentes a $V_3(\mathbb{C})$, sendo $\theta$ o ângulo entre eles, $0 \leq \theta \leq \pi$, calcule $cos\theta$.

\paragraph{Questão 3:} (2,0 pontos)

Sejam $P_0=(x_0,y_0,z_0)$ um ponto qualquer e a reta $r=\{x,y,z \ | \ x=x_1+ta,y=y_1+tb,z=z_1+tc, \forall t \in \mathbb{R} \}$;
\begin{enumerate}[label=(\alph*)]
\item Mostre que a distância de $P_0$ a $r$ é $\dfrac{\norm{P_1P_0 \times V}}{\norm{V}}$, onde $V=(a,b,c)$ é um vetor diretor de $r$ e $P_1=(x_1,y_1,z_1)$ é um ponto da reta $r$. 
\item Calcule a distância do ponto $P_0=(1,-1,2)$ à reta $r=\{(x,y,z) \ | \ x=1+2t,y=-t,z=2-3t, \forall t \in \mathbb{R}\}$.
\end{enumerate}

\paragraph{Questão 4:} (2,0 pontos)

Dada uma reta $L$ e um ponto não em $L$, provar que existe um e um só plano passando por $P$ contendo $L$. 

\paragraph{Questão 5:} (2,0 pontos)

(Trabalho em grupo)

\end{enumerate}
\normalsize
\newpage
\subsubsection{2002 \original{https://drive.google.com/file/d/1YJSKvgiseMvw9g3bdSrF_NCYUBYVadM-/view?usp=sharing}}

\paragraph{Questão 1:} (3,0 pontos)

O seguinte teorema da geometria plana sugere uma identidade vetorial usando dois vetores $A$ e $B$. Obtenha tal identidade e a demonstre usando álgebra vetorial:

"A soma dos quadrados de dois lados de um triângulo é igual ao dobro do quadrado da mediana relativa ao terceiro lado, mais a metade do quadrado deste lado".

\paragraph{Questão 2:} (3,0 pontos)

Enuncie e demonstre a desigualdade de Cauchy-Schwarz para o espaço vetorial complexo $V_n(\mathbb{C})$.

\paragraph{Questão 3:} (4,0 pontos)

Suponha que, ao invés de se definir a norma de um vetor $A=(a_1,a_2,\dots,a_n)$ pela fórmula $(A\cdot A)^{\frac{1}{2}}$, usou-se a seguinte definição: 
$$ \norm{A}=\sum_{i=1}^n \modu{a_i}$$

\begin{enumerate}[label=(\alph*)]
\item Prove que esta definição satisfaz as seguintes propriedades: 
\begin{enumerate}[label=(\roman*)]
\item $\norm{A}>0$, se $A \neq 0$;
\item $\norm{A}=0$, se $A=0$;
\item $\norm{cA}=\modu{c}\norm{A}$;
\item $\norm{A+B}\leq \norm{A}+\norm{B}$;
\end{enumerate}
\item Use esta definição em $V_2$ e descreva a figura formada pelo conjunto de todos os pontos $(x,y)$ com norma igual a $1$.
\item Verifique quais das quatro propriedades listadas no item (a) são válidas ao se utilizar a seguinte definição de norma:
$$ \norm{A} = \modu{\sum_{k=1}^n a_k}$$
\end{enumerate}
\newpage


\subsection{VC}


\subsubsection{2019}

\paragraph{1ª Questão (3,0 pontos)}

Sejam uma reta $L = \{P + tA | t \in \mathbb{R}\}$ e um plano $M = \{Q + rB + sC | r, s \in \mathbb{R}\}$ no $\R^n$.

\begin{enumerate}[label=\alph*)]
    \item (1,0 ponto) Fixando um ponto $P_f \in L$, demonstre que existe um único ponto $Q_f \in M$ tal que $P_f - Q_f \perp M$.

        Demonstre que $Q_f$ é tal que $\Arrowvert P_f - Q_f \Arrowvert \leq \Arrowvert P_f - X \Arrowvert$ para todo $X \in M$;
    

    \item (1,0 ponto) Fixando um ponto $Q_f \in M$, demonstre que existe um único ponto $P_f \in L$ tal que $P_f - Q_f \perp L$.

        Demonstre que $P_f$ é tal que $\Arrowvert P_f - Q_f \Arrowvert \leq \Arrowvert Y - Q_f \Arrowvert$ para todo $Y \in L$;
    \item (1,0 ponto) No $\R^4$, tome: 
    $$P = (2,0,2,2), A = (1,0,1,2), Q = (0,3,0,-1), B = (1,2,1,0), C = (1,1,1,1).$$

    Encontre um ponto $P_f \in L$ e um ponto $Q_f \in M$ tal que $\Arrowvert P_f - Q_f \Arrowvert$ seja a menor possível. Justifique a escolha dos pontos.
\end{enumerate}


\paragraph{2ª Questão (2,0 pontos)}

Seja $V$ o conjunto de vetores $(a, b, c, d)$ do $\mathbb{R^4}$ tais que: 

$$ \begin{cases}

        a + b + c + d = 0 \\
        a = c

    \end{cases} $$


Seja $W$ o conjunto de vetores do $\R^4$ que são ortogonais a todos os elementos de $V$, ou seja,

$$ W = \{ w \in \R^4 | w \cdot v = 0 , \forall v \in V   \} $$

\begin{enumerate}[label=\alph*)]
    \item (0,5 ponto) Demonstre que existe um conjunto A linearmente independentes (LI) tal que $V$ é o subespaço gerado por A;
    \item (0,5 ponto) Demonstre que existe um conjunto B linearmente independentes (LI) tal que $W$ é o subespaço gerado por B;
    \item (1,0 ponto) Seja $x = \{ 1,2,3,4 \}$, encontre vetores $v \in V$ e $w \in W$ tais que $ x = v + w$. Os vetores $v$ e $w$ encontrados são únicos? Justifique.
\end{enumerate}

\paragraph{3ª Questão (1,5 pontos)}

Sejam $r = \{P + tU | t \in \mathbb{R}\}$ e $s = \{Q + tV | t \in \mathbb{R}\}$ retas não-paralelas em $\R^3$. Prove que a distância $d(r, s)$ entre $r$ e $s$ é dada por:

$$ d(r,s) = \dfrac{\left\arrowvert U \times V \cdot \vec{PQ} \right\arrowvert}{\Arrowvert U \times V \Arrowvert} $$ 

\paragraph{4ª Questão (2,0 pontos)}

Seja $ABCD$ um tetraedro regular no $\R^3$ . Sejam $P$ e $Q$ os pontos médios das arestas $AB$ e $CD$, respectivamente. Demonstre que $\{ \overrightarrow{AB}, \overrightarrow{CD}, \overrightarrow{PQ} \}$ é um conjunto ortogonal de vetores.
 
\paragraph{5ª Questão (1,5 pontos)}

Sejam $u \neq 0, v, w$ vetores de $\R^3$. Prove que se 
$$ u \times v = w , u \times w = v
$$

então $v = 0, w = 0$.

\newpage


\subsubsection{2018 \original{https://drive.google.com/drive/folders/1XPn5NW2-31TxGcEpJcxMoFLFpvSTCHtt}}

\paragraph{Questão 1:} (3,0 pontos)

Considere um plano $\pi$ em $\mathbb{R}^n$, com $n \geq 3$. Seja $P \notin \pi$ um ponto de $\mathbb{R}^n$.
\begin{enumerate}[label = (\alph*)]
    \item (1,5 pontos) Suponha que existe um ponto $U \in \pi$ tal que, para todo $Q \in \pi$, cumpre-se a igualdade 
    \begin{equation*}
        (P - U) \cdot (Q - U) = 0        
    \end{equation*}
    Prove que tal ponto $U\in \pi$ é único.

    \item (1,5 pontos) Prove a existência do ponto $U \in \pi$.
    
\end{enumerate} 

\paragraph{Questão 2:} (2,0 pontos) Sejam $U, X, Y \in \mathbb{R}^3$ vetores, com $U \neq 0$. Prove que se 
\begin{equation*}
    U \times X = Y
\end{equation*}
\begin{equation*}
      U \times Y = X 
\end{equation*}

então $X = Y = 0$.

\paragraph{Questão 3:} (3,0 pontos) Considere as retas de $\mathbb{R}^n$:
\begin{equation*}
    L(P; A) = \{P + tA | t \in \mathbb{R}\}
\end{equation*}
    \begin{equation*}
        L(Q; B) = \{Q + sB | s \in \mathbb{R}\}
    \end{equation*}

onde $L(P; A) \cap L(Q; B) = \emptyset$.

\begin{enumerate}[label = (\alph*)]
    \item (2,0 pontos) Prove que existe uma perpendicular comum que intercepta $L(P; A)$ e $L(Q; B)$.
    \item (1,0 ponto) Considerando $n = 4$ e:
    \begin{align*}
        P = (1, 0, -1, 2), A = (2, 1, 3, 0), Q = (2, 4,-2, 0), B = (1, 1, 0,-1);
    \end{align*}
    determine a perpendicular comum que intercepta L(P; A) e L(Q; B).
\end{enumerate}

\paragraph{Questão 4:} (2,0 pontos) Seja $S = \{A_1, \dots, A_k\}$ um conjunto linearmente independente, $S \subset V_n = \mathbb{R}^n$. Prove que, se 
\begin{equation*}
    \{A_1 + B, \dots, A_k + B\}
\end{equation*}
é linearmente dependente, então $B \in L(S)$.

\newpage

\subsubsection{2017 \original{https://drive.google.com/file/d/1mv1XwBoZOp0ZJX1vPB-VQ5WNS_Xx_KLM/view?usp=sharing}}
\paragraph{Questão 1:} (3,0 pontos)

Sejam $u_1,v_1,u_2,v_2,w \in \mathbb{R}^5$ vetores linearmente independentes, e sejam $P_1,P_2,Q \in \mathbb{R}^5$ pontos. Sejam $\pi_1=\{P_1+su_1+tv_1 \ | \ s,t \in \mathbb{R}\}$, $\pi_2=\{P_2+su_2+tv_2 \ | \ s,t \in \mathbb{R}\}$ planos, e seja $r=\{ Q+tw \ | \ t \in \mathbb{R}\}$ uma reta. Prove que existe uma única reta $r^\prime $ paralela a r cortando $\pi_1,\pi_2$, i.e, tal que $r^\prime \cap \pi_1 \neq \emptyset,r^\prime \cap \pi_2 \neq \emptyset$

\paragraph{Questão 2:} (3,0 pontos)

No $\mathbb{R}^n$, considere um ponto $P$ e uma reta $r$ que não passa por $P$. Demonstre que existe um único plano $\pi$ que contém $P$ e $r$.

\paragraph{Questão 3:} (4,0 pontos)

Para cada um dos itens a seguir, classifique as afirmações como verdadeiras ou falsas. Demonstre as afirmações verdadeiras e mostre um contra-exemplo para as falsas.

\begin{enumerate}[label=(\alph*)]
\item (1,0 ponto) Sejam $u,v,w$ vetores do $\mathbb{R}^n, \ n > 2$, tais que $\{u,v\},\{u,w\},\{v,w\}$ são conjuntos linearmente independentes. Então o conjunto $\{u,v,w\}$ é linearmente independente.

\item (1,0 ponto) Seja $S=\{u_1,u_2,\dots,u_k\}$ um conjunto de vetores linearmente independentes do espaço $\mathbb{R}^n$, com $n>k$. Seja ainda $v \in \mathbb{R}^n$ um vetor que não é gerado por $S$. Então o conjunto $S^\prime = \{u_1,u_2,\dots,u_k,v\}$ é linearmente independente. 

\item (1,0 ponto) Sejam $u,v$ vetores do $\mathbb{R}^n$ tais que $\Vert u + tv \Vert \geq \Vert u \Vert$ para todo $t \in \mathbb{R}$. Então $u \cdot v=0$.

\item (1,0 ponto) Sejam $u,v,w$ vetores linearmente independentes do $\mathbb{R}^3$. Então existe um vetor $w^\prime$ do $\mathbb{R}^3$ tal que $w=w^\prime + (u\times v)$, onde $w^\prime \cdot (u \times v)=0$.
\end{enumerate}

\newpage
\subsubsection{2016 \original{https://drive.google.com/file/d/1hb_1RUzVND8VXIu_Akd0eUhl7D1AUVjK/view?usp=sharing}}

\paragraph{Questão 1:} (2,5 pontos) 

\begin{enumerate}[label=(\alph*)]
\item (1,0 ponto) Sejam $X_1=(0,3,4,-1), X_2=(2/3,1,2,-1/3)$ e $X_3=(3,0,3,0)$ vetores do $\mathbb{R}^4$. Verifique se é possível completar o conjunto $\{X_1,X_2,X_3 \}$ com algum vetor da base canônica para formar uma nova base para o $\mathbb{R}^4$. 
\item (1,5 ponto) Seja o conjunto $A=\{X_1,X_2,\dots,X_k\}$ de vetores linearmente independentes do espaço vetorial $\mathbb{R}^n$, onde $n>r$. Demonstre que é possível completar o conjunto $A$ com $n-r$ vetores da base canônica do $\mathbb{R}^n$ de modo a formar uma nova base do $\mathbb{R}^n$.
\end{enumerate}

\paragraph{Questão 2:} (2,5 pontos)

Seja $ABCD$ um tetraedro no espaço $\mathbb{R}^3$. 
\begin{enumerate}[label=(\alph*)]
\item (1,0 ponto) Suponha que $AB \perp CD$, $AC \perp BD$. Prove que $AD \perp BC$.
\item (1,5 ponto) Nas hipóteses de (a), prove que as quatro alturas relativas às quatro faces do tetraedro concorrem num único ponto.
\end{enumerate}

\paragraph{Questão 3:} (2,0 pontos)

Seja $ABC$ um triângulo no plano $\mathbb{R}^2$, e seja $P=aA+bB+cC$ um ponto do interior do triângulo, com $a,b,c>0$ e $a+b+c=1$. Prove que:
$$ a= \dfrac{S_{\triangle PBC }}{S_{\triangle ABC}},\ \ b=\dfrac{S_{\triangle APC }}{S_{\triangle ABC}},\ \ c=\dfrac{S_{\triangle ABP }}{S_{\triangle ABC}} $$
 onde $S_{\triangle PQR}$ indica a área do triângulo $PQR$. 
\paragraph{Questão 4:} (3,0 pontos)

\begin{enumerate}[label=(\alph*)]
\item (1,5 ponto) Seja $V$ um subespaço de $\mathbb{R}^n$. Defina o conjunto 
$$V^\perp = \{w \in \mathbb{R}^n \ | \ w \cdot v=0,\ \forall v \in V \} $$
Prove que $V^\perp$ é um subespaço vetorial.
\item (1,5 ponto) Sejam $u,v$ vetores linearmente independentes do espaço $\mathbb{R}^n$, com $n>2$. Demonstre que todo vetor $x \in \mathbb{R}^n$ pode ser escrito de maneira única como $x=\lambda u + \mu v +w$, onde $\lambda, \mu \in \mathbb{R}$ e $w$ é um vetor ortogonal a $u$ e a $v$.
\end{enumerate}

\newpage
\subsubsection{2015 \original{https://drive.google.com/file/d/1tqA1mBCGCLP3JHuWI8D-p2LwVpg7IjHz/view?usp=sharing}}

\paragraph{Questão 1:} (2,0 pontos) 

Seja $S$ a área de um triângulo $ABC$. Seja $S^\prime$ a área do triângulo formado pelas medianas do triângulo $ABC$. Sabendo-se que a origem $O$ do sistema de eixos não pertence ao conjunto $\{A,B,C\}$, utilizando apenas conceitos de álgebra vetorial, sem recorrer às coordenadas dos vértices, mostre que $S^\prime=3S/4$.

\paragraph{Questão 2:} (2,0 pontos) (\textbf{Questão repetida} - $4^a$ questão da VC de 2014) 

Sejam $M_1$ e $M_2$ dois planos pertencentes ao $\mathbb{R}^3$, tais que $N_1$ e $N_2$ sejam, respectivamente, seus vetores normais. Prove, utilizando apenas conceitos da álgebra vetorial, sem recorrer a coordenadas, que $M_1$ é paralelo a $M_2$ se, e somente se, $N_1$ é paralelo a $N_2$. 

\paragraph{Questão 3:} (2,0 pontos) (\textbf{Questão Repetida} - $5^a$ questão da VC de 2007)

Fixado $Q \in \mathbb{R}^n$, seja $M \subset \mathbb{R}^n$ um plano que satisfaz a condição $(X-U_0)(U_0-Q)=0$, qualquer que seja $X \in M$ e para algum $U_0 \in M$. Prove que $U_0$ é único. 

\paragraph{Questão 4:} (2,0 pontos)

Sejam $V,X$ e $Y \in\mathbb{R}^3$ vetores tais que $\overrightarrow{V} \neq \overrightarrow{0}$, $\overrightarrow{V} \times \overrightarrow{X} = \overrightarrow{Y}$ e $\overrightarrow{V} \times \overrightarrow{Y} = \overrightarrow{X}$. Prove que os vetores $\overrightarrow{X},\overrightarrow{Y}$ são nulos.

\paragraph{Questão 5:} (2,0 pontos)

Considere as retas
$$ r \equiv (1,1,0)+\alpha (2,1,0) $$
$$ s \equiv (1,0,0)+\beta  (0,1,2)$$

Calcule a reta perpendicular comum a $r$ e $s$ e encontre a distância entre as retas $r$ e $s$. 

\newpage
\subsubsection{2014 \original{https://drive.google.com/file/d/1eGsopDShFNA7CR_LAtOUedXkkwwMM06H/view?usp=sharing}}
\paragraph{Questão 1:} (2,5 pontos) 

O teorema de geometria enunciado a seguir sugere uma identidade vetorial relativa a três vetores $A,B,C$. Dizer qual é a identidade e provar que se verifica para vetores de $V_n$. Tal identidade proporciona uma demonstração do teorema por métodos vetoriais. 
\vspace{0.3cm}

\textbf{"A soma dos quadrados dos lados de um quadrilátero qualquer excede a soma dos quadrados das diagonais em quatro vezes o quadrado do comprimento do segmento de reta que une os pontos médios das diagonais.}

\paragraph{Questão 2:} (2,5 pontos) 

Seja $\{g_1,g_2,g_3\}$ uma base ortonormal do $\mathbb{R}^3$. Para todo $u \in \mathbb{R}^3$ definem-se os cossenos diretores de $u$ em relação à base dada por $cos\alpha=\dfrac{u \cdot g_1}{\Vert u \Vert}$, $cos\beta=\dfrac{u \cdot g_2}{\Vert u \Vert}$ e $cos\gamma = \dfrac{u \cdot g_3}{\Vert u \Vert}$. Prove que:
\begin{enumerate}[label=(\alph*)]
\item (1,0 ponto) $u=\Vert u \Vert ((cos\alpha)g_1+(cos\beta)g_2+(cos\gamma)g_3)$.
\item (1,5 pontos) $cos^2\alpha +cos^2\beta +\cos^2\gamma=1$
\end{enumerate}

\paragraph{Questão 3:} (2,5 pontos) 

Considere a seguinte proposição: "Seja $B=\{U_1,U_2,..,U_n\}$ um conjunto ortonormal do $\mathbb{R}^n$ e $S=\{V_1,V_2,\dots,V_n\} \subseteq \mathbb{R}^n$. Se $\{U_1+V_1,U_2+V_2,\dots,U_n+V_n\}$ for L.D, então $\sum_{i=1}^{n}\Vert V_i \Vert \geq 1$".
Utilizando esta proposição, dado $B=\{U_1,U_2,\dots,U_n\}$ ortonormal em $\mathbb{R}^n$, $S=\{V_1,V_2,..,V_n\} \subseteq \mathbb{R}^n$ e $\theta_i$ o ângulo entre $U_1 \textrm{ e } V_i$, prove que se $\sum_{i=1}^n\cos\theta_i > n-1/2$, então $S$ é L.I.

\paragraph{Questão 4:} (2,5 pontos) 

Sejam $M_1$ e $M_2$ dois planos pertencentes ao $\mathbb{R}^3$, tais que $N_1$ e $N_2$ sejam, respectivamente, seus vetores normais. Prove, utilizando apenas conceitos da álgebra vetorial, sem recorrer a coordenadas, que $M_1$ é paralelo a $M_2$ se, e somente se, $N_1$ é paralelo a $N_2$.

\newpage
\footnotesize
\subsubsection{2010 \original{https://drive.google.com/file/d/1iAPdKxS7dN4f7A9y3lARavMJEqBFKsgg/view?usp=sharing} }

\paragraph{Questão 1:} (2,4 pontos)

Suponha que a norma de um vetor $A=(a_1,a_2,\dots,a_n) \in \mathbb{R}^n$ seja definida pela fórmula $\Vert A \Vert = max\{\vert a_1 \vert, \vert a_2 \vert,\dots,\vert a_n \vert \}$ (máximo do conjunto). Prove que esta definição satisfaz às seguintes propriedades:

\begin{enumerate}[label=(\alph*)]
\item $\Vert A \Vert >0$, se $A \neq 0$.
\item $\Vert A \Vert =0$, se $A=0$.
\item $\Vert cA \Vert = \vert c \vert \Vert A \Vert$, $\forall c \in \mathbb{R}$.
\item $\Vert A+B \Vert \leq \Vert A \Vert + \Vert B \Vert$, $\forall A,B \in \mathbb{R}^n$.
\end{enumerate}

\paragraph{Questão 2:} (1,2 pontos) 

Dados $A,B$ vetores em $\mathbb{R}^n$, prove ou dê contra-exemplo para a afirmativa abaixo:

Se $A$ é ortogonal a $B$, então $\Vert A+xB \Vert \geq \Vert A \Vert$, $\forall x \in \mathbb{R}$.

\paragraph{Questão 3:} (2,4 pontos) 

Dadas duas retas não paralelas $L(P;A)$ e $L(Q;B)$ em $\mathbb{R}^n$, prove que $L(P;A) \cap L(Q;B)$ ou é um conjunto vazio ou consiste, exatamente, de um único ponto.

\paragraph{Questão 4:} (2,0 pontos)

Seja $\{A,B,C,D,E,F\}$ o conjunto dos vértices de um hexágono não-regular convexo, tal que seus lados opostos são paralelos. Usando produto vetorial, mostre que o triângulo de vértices $A,C,E$ tem a mesma área que o triângulo $B,D,F$.

\begin{figure}[h]\centering\includegraphics[width=6cm]{alglinvc2010q4.png}\end{figure}

\paragraph{Questão 5:} (2,0 pontos)

Dados os planos $M_1,M_2 \subseteq \mathbb{R}^3$. Prove que $M_1 \Vert M_2$ se e somente se, $\forall N_1$ e $N_2$ vetores normais a $M_1$ e $M_2$, respectivamente, tem-se $N_1 \Vert N_2$.

Obs: Um vetor $N$ é dito normal ao plano $M$ quando for normal a qualquer vetor do plano $M$.
\normalsize
\newpage
\subsubsection{2008 \original{https://drive.google.com/file/d/1ODHXkowN2YzVkhZNm0lSGx1qZDRCdHnl/view?usp=sharing}}

\paragraph{Questão 1:} (2,4 pontos) 

Classifique cada sentença abaixo com V, caso seja verdadeira, ou F, caso seja falsa, sempre justificando, em qualquer caso, com uma prova ou contra-exemplo.

\begin{enumerate}[label=(\alph*)]
\item  (0,8 pontos) \\
(\ \  ) A união de dois planos não paralelos é um subespaço do $\mathbb{R}^n$ se, e somente se, ambos contêm a origem. 
\item (0,8 pontos) \\
(\ \  ) Seja $L(P;A)$ uma reta e $M=\{Q+sB+\lambda D \}$ um plano, tal que $A$ é paralelo a $B\times D$. Então $\#(L(P;A)\cap M)=1$.
\item (0,8 pontos) \\
(\ \  ) Sejam $M_1 := \{P+xA+yB\}$ e $M_2 := \{Q+zC+wD\}$ planos no $\mathbb{R}^4$, tais que $\{A,B,C,D\}$é um conjunto LI. Então $\#(M_1 \cap M_2)=1$.
\end{enumerate}

\paragraph{Questão 2:} (2,0 pontos)

Seja $S$ a área de um triângulo $ABC$. Seja $S^\prime$ a área do triângulo formado pelas medianas do triângulo $ABC$. Sabendo-se que $O \notin \{A,B,C\}$, utilizando apenas conceitos de álgebra vetorial, sem recorrer às coordenadas dos vértices, mostre que $S^\prime = \dfrac{3}{4} S$.

\paragraph{Questão 3:} (2,6 pontos) 

Sejam $\{A,B\}$ um conjunto LI de vetores fixos considere a equação vetorial $Y + A\times Y = B$. Obtenha a solução $Y=\dfrac{B+(A\cdot B)A+B\times A}{1+A\cdot A}$ (não basta apenas substituir e verificar que satisfaz à equação) e mostre que ela é única para a equação dada.\\
Obs: Triplo produto vetorial: $F\times(G\times H) = (H\cdot F)G-(G\cdot F)H$

\paragraph{Questão 4:} (3,0 pontos) 

Dados os vetores ortonormais $A,B \in \mathbb{R}^3$, seja $P \in \mathbb{R}^3$, tal que $P \times B = A-P$, prove: 

\begin{enumerate}[label=(\alph*)]
\item (0,5 ponto) $P$ é ortogonal a $B$ e $\Vert P \Vert = \dfrac{\sqrt{2}}{2}$;
\item (0,5 ponto) $P,B,P\times B$ formam uma base do $\mathbb{R}^3$;
\item (1,0 ponto) $(P\times B)\times B = -P$;
\item (1,0 ponto) $P= \dfrac{1}{2}A - \dfrac{1}{2}(A\times B)$.
\end{enumerate}

\newpage
\subsubsection{2007 \original{https://drive.google.com/file/d/12NsjV4j1-xoKaBbTUO5W_yfG6VT9IGRH/view?usp=sharing}}

\paragraph{Questão 1:} (1,0 ponto)

Seja $M \subset \mathbb{R}^n$ um plano que satisfaz à seguinte condição: Se $P,Q \in M$, então $P+Q \in M$. Prove $M$ passa pela origem.

\paragraph{Questão 2:} (2,0 pontos)

Dados os planos $M_1$ e $M_2$ no $\mathbb{R}^3$ com $M_1 = \{P+sA+tB\}$, onde $P=(1,1,1)$, $A=(a,b,0)$, $B=(2,2,1)$ e $M_2$ tem por equação cartesiana $x+y-abz=2$, determine $a$ e $b$ de maneira que $M_1$ e $M_2$ sejam paralelos.

\paragraph{Questão 3:} (2,0 pontos) 

Dada uma reta $L=\{P+\lambda A\}$ e $Q \notin L$, utilizando os conceitos de álgebra vetorial, não recorrendo à representação por coordenadas, provar que a reta $L$ e o vetor $Q$ determinam um único plano.

\paragraph{Questão 4:} (1,0 ponto)

Seja $S=\{U_1,U_2,\dots,U_n\}$ um conjunto ortonormal de vetores do $\mathbb{R}^n$. Seja $X$ um vetor não nulo também do $\mathbb{R}^n$ e $\theta_i$ o ângulo entre $X$ e $U_i$, para $i=1,2,\dots,n$, Prove que 
$\sum_{i=1}^{n}\cos^2\theta_i =1$.

\paragraph{Questão 5:} (2,0 pontos) 

Fixado $Q \in \mathbb{R}^n$, seja $M \subset \mathbb{R}^n$ um plano que satisfaz à condição $(X-U_0)(U_0-Q)=0$, qualquer que seja $X \in M$ e para algum $U_0 \in M$. Prove que $U_0$ é único.

\paragraph{Questão 6:} (2,0 pontos) -

No trapézio da figura abaixo, $C-B=\lambda A$, onde $\lambda>0$. Prove que a razão entre $\Vert C \Vert$ e $\Vert B-A \Vert$ é igual a $\sqrt{\lambda}$.

\begin{figure}[ht]
\centering
\includegraphics[width=8cm]{alglinve12015q2.png}
\end{figure}

\newpage
\subsubsection{2003 \original{https://drive.google.com/file/d/1t1wd-XWCXO2ECK6n3p2MlNog4TR39LUJ/view?usp=sharing}}

\paragraph{Questão 1:} (2,5 pontos)

Considere quatro vetores $A,B,C,D \in V_3(\mathbb{R})$  que têm como origem um mesmo ponto V. Sabendo-se que: \\
- Os pontos extremos desses vetores são complanares;\\
- Os planos gerados por cada par de vetores são distintos entre si.\\

Determine em função de $A, B, C, D$ :
\begin{enumerate}[label=(\alph*)]
\item A distância de V ao plano a que pertencem os extremos dos vetores A, B, C, D;
\item A área do polígono determinado pelos extremos dos vetores $A, B, C, D$;
\item Identifique o sólido formado pelos vetores $A, B, C, D$ e o plano já mencionado e
obtenha o volume desse sólido.
\end{enumerate}

\paragraph{Questão 2:} (2,5 pontos)

Sejam os planos $M$ e $M^\prime$ em $V_3(\mathbb{R})$ dados por:
$$M=\{Q+sA+tB \ | \ s,t \in \mathbb{R}\}$$
$$M^\prime = \{R+sC+tD \ | \ s,t \in \mathbb{R} \}$$
, onde $ A \notin [C,D]$ e $ C \notin [A,B]$. Prove que:

\begin{enumerate}[label=(\alph*)]

\item $M\cap M^\prime=L$, onde $L$ é uma reta em $V_3(\mathbb{R})$.

\item Se existe a reta $L^\prime$ em $V_3(\mathbb{R})$, tal que $L^\prime \parallelsum M$ e $L^\prime \parallelsum M^\prime$, então $L^\prime  \parallelsum L$.

\end{enumerate}

\paragraph{Questão 3:} (2,5 pontos)

Supondo que a norma de um vetor $A=(a_1,a_2,\dots,a_n)$ seja definida pela fórmula $\Vert A \Vert = \max \vert a_k \vert$, $1\leq k \leq n$, onde o símbolo do segundo membro significa o máximo dos $n$ números $\vert a_1 \vert, \vert a_2 \vert,\dots,\vert a_n \vert$. Demonstre se essa definição de norma satisfaz as seguintes propriedades:
\begin{enumerate}
\item $\Vert A \Vert > 0$, se $ A \neq 0 $;
\item $\Vert A \Vert = 0 $, se $A = 0$
\item $\Vert cA \Vert = \vert c \vert \Vert A \Vert$;
\item $\Vert A + B\Vert \leq $ $\Vert A \Vert +\Vert B \Vert$;
\end{enumerate}

\paragraph{Questão 4:} (2,5 pontos)

Dados os pontos $A(2,1,1),B(-1,2,1)$ e $C(3,-2,4)$.
\begin{enumerate}
\item Determine no plano $2x-y+5z=2$ um ponto equidistante dos vértices do triângulo $ABC$.
\item Determine o circuncentro do triângulo $ABC$.
\end{enumerate}
\newpage
\subsubsection{2002 \original{https://drive.google.com/file/d/1jHizQEs33wQWcvFq_KMxZ1QMRuE5RRiW/view?usp=sharing}}

\paragraph{Questão 1:} (3,0 pontos) 

Considere os pontos $A=(3,4,2), B=(2,1,1)$ e $C=(5,2,3)$ e resolva cada uma das alternativas abaixo:
\begin{enumerate}[label=(\alph*)]
\item Encontre a equação do plano determinado pelos pontos $A, B$ e $C$;
\item Encontre o ponto $D$, sabendo que os pontos $A, B, C$ e $D$ formam um paralelogramo e $D$ é oposto ao vértice $B$;
\item Seja o ponto $G=(7,3,6)$, $CG$ é uma aresta lateral do paralelepípedo cuja umas das
faces é o paralelogramo ABCD. Determine a equação do plano que contém a face $EFGH$ que é oposta à face $ABCD$;
\item Dê a equação da reta suporte da diagonal do paralelepípedo que passa por $D$;
\item Calcule o ângulo da diagonal obtida no item anterior com a face $ABCD$;
\item Determine o volume do paralelepípedo $ABCDEFG$.
\end{enumerate}

\paragraph{Questão 2:} (2,0 pontos)

Seja $Q$ um ponto que não pertence à reta $L(P;A)$ em $V_n$. 
\begin{enumerate}[label=(\alph*)]
\item Seja $f(t)=\Vert Q-X(t) \Vert$, onde $X(t)=P+tA$. Prove que f(t) é um polinômio quadrático em $t$ e que esse polinômio assume o seu mínimo em apenas um único valor $t$, dito $t=t_0$.
\item Prove que $Q=X(t_0)$ é ortogonal a A.
\end{enumerate}

\paragraph{Questão 3:} (2,0 pontos) 

Sejam $L(P;A)$ uma reta e $M(Q;B,C)$ um plano em $V_n$. Se $L$ é paralela a $M$, prove que
a interseção entre eles é o conjunto vazio ou a própria reta $L$.

\paragraph{Questão 4:} (1,5 pontos)

Usar a desigualdade de Cauchy-Schwarz para mostrar que se $a_1,a_2,\dots,a_n$ forem números positivos, então:
$$(a_1+a_2+\dots+a_n)(\dfrac{1}{a_1}+\dfrac{1}{a_2}+\dots+\dfrac{1}{a_n})\geq n^2 $$

\paragraph{Questão 5:} (1,5 pontos)

Se três pontos $A, B$ e $C$ determinam um plano, prove que a distância de um ponto $Q$
até esse plano é:

$$\dfrac{\Vert (Q-A)\cdot (B-A)\times (C-A) \Vert}{\Vert (B-A)\times (C-A) \Vert} $$

\newpage
\subsection{VE 2}

\subsubsection{2019}

\paragraph{Questão 1} (1,0 ponto)

Seja $n \in \mathbb{N}$, e sejam $x_i \in \R, 0 \leq i \leq n$ pontos tais que $x_i \neq x_j, i \neq j$. Tome-se $y_0, ..., y_n \in \R$ arbitrários.

\begin{enumerate}[label=\alph*)]
    \item Prove que existe exatamente um polinômio $P(x)$ de grau $ \leq n$ tal que $P(x_i) = y_i$, se $0 \leq i \leq n.$

    \item Prove que tal polinômio é caracterizado pela condição
    $$det \begin{pmatrix}
    1       & x_0    & \cdots  & x^{n}_0 & y_0\\
    1       & x_1    & \cdots  & x^{n}_1 & y_1\\
    \vdots  & \vdots & \vdots  & \ddots  & \vdots\\
    1       & x_n    & \cdots  & x^{n}_n & y_n\\
    1       & x      & \cdots  & x^{n}   & P(x)

    \end{pmatrix} = 0, \textnormal{ para todo } x \in \R.$$

\end{enumerate}  

\paragraph{Questão 2} (1,0 ponto)
Seja $A$ uma matriz $m \times n$. Prove que $A$ tem posto $1$ se e somente existem $0 \neq u \in \R^m, 0 \neq v \in \R^n$ tais que $A = uv^{T}.$

\newpage


\subsubsection{2017}
Não houve VE2.
\newpage
\subsubsection{2016 \original{https://drive.google.com/file/d/1IoSmA359DYwP7JGRImxshns6t65fQ-NH/view?usp=sharing}}

\paragraph{Questão 1:} (2,5 pontos)

Seja $B=\{e_1,e_2,\dots,e_n\}$ a base canônica do espaço vetorial $\mathbb{R}^n$. Considere o conjunto $S=\{e_1-e_2,e_2+e_3,e_3-e_4,e_4+e_5,\dots,e_{n-1}+(-1)^{n-1}e_n,e_n+(-1)^{n}e_1\}$. Determine para quais valores de $n$ o conjunto $S$ é linearmente independente.

\paragraph{Questão 2:} (2,5 pontos)

Demonstre que o conjunto $F=\{A\in \mathbb{M}_{n \times n} \ | \ A^T=3A\}$ é um subespaço vetorial do espaço $\mathbb{M}_{n \times n}$ das matrizes quadradas $n\times n$ de coeficientes reais. Determine ainda a dimensão de $F$.


\newpage
\subsubsection{2015 \original{https://drive.google.com/file/d/1OAYL5U_CWi3ql44wBt8Ghes7WSzUt2XQ/view?usp=sharing}}

\paragraph{Questão 1:} (5,0 pontos)

Seja, para $x \in \mathbb{R}$, a matriz $A(x)$ dada por:

$$
A(x)=\left( 
\begin{array}{ccc}
1 & x & x^2 \\
0 & 1 & 2x  \\
0 & 0 & 1   \\
\end{array}
\right)
$$
\begin{enumerate}[label=(\alph*)]
\item (2,5 pontos) Mostre que $A(x+y)=A(x)A(y)$ para $x,y$ reais quaisquer.
\item (2,5 pontos) Calcular o subespaço $F$ de $M_{3\times 3} (\mathbb{R})$ gerado pelo subconjunto $\{A(x),X\in \mathbb{R}\}$. Pode explicitar $F$ dando as equações que descrevem $F$ ou um sistema de geradores.
\end{enumerate}
\paragraph{Questão 2:} (5,0 pontos)

Seja $OABC$ um triedro triretângulo, i.e., com $OA,OB,OC$ segmentos ortogonais entre eles e não nulos. Prove a seguinte fórmula que relaciona as áreas das quatro caras do tetraedro formado pelos vértices $O,A,B,C$ (aqui $\modu{}$ simboliza área):
$$ \modu{ABC}^2 = \modu{OAB}^2 + \modu{OBC}^2+\modu{OAC}^2  $$

\newpage
\subsubsection{2003 \original{https://drive.google.com/file/d/1kw8OtWkoGKkbaNkq593NvCbfYwr6HLoR/view?usp=sharing}}

\paragraph{Questão 1:} (3,0 pontos)

Dados os vetores $A,B$ e dada uma base ortonormal $\{U_1,U_2,U_3\}$, mostre que se $\{A,B,U_2\}$ for um conjunto linearmente dependente, então:

$$\norm{A\times B}^2 = (U_1\times A \cdot B)^2 + (U_3 \times A \cdot B)^2 $$

\paragraph{Questão 2:} (4,0 pontos)

Uma corda de uma seção cônica $\Omega$ é qualquer segmento $[X,Y]$, onde $\{X,Y\} \in \Omega$. Diz-se que a corda tem inclinação $m$, quando a reta que passa por $X$ e $Y$ tem coeficiente angular $m$. Dada uma elipse de equação $\dfrac{x^2}{a^2}+\dfrac{y^2}{b^2}=1$, determine o lugar geométrico formado pelos pontos médios de todas as cordas com inclinação $m$. 

\paragraph{Questão 3:} (3,0 pontos)

Classifique cada sentença abaixo com V, caso seja verdadeira, ou F, caso seja falsa. Se for verdadeira, apresente uma prova. Caso seja falsa, apresente um contra-exemplo \textbf{numérico}.

\begin{enumerate}[label=(\alph*)]
\item (1,0 ponto)  Sejam $A,B \in \mathbb{F}^{n\times n}$. Então $\textrm{posto}(AB)=\textrm{posto}(BA)$;
\item (1,0 ponto) Seja $\{A_1,A_2,\dots,A_{2n}\}$ um subconjunto ortogonal de $\mathbb{R}^m$. Defina $H=[A_1|\dots|A_n]$ e $F=[A_{n+1}|\dots|A_{2n}]$. Então posto$(H^TF)=0$.
\item (1,0 ponto) Seja $A\in \mathbb{F}^{n\times n}$, tal que as colunas de $A$ são a base canônica de $\mathbb{F}^{n}$ e $A^2=I$. Então $A=I$.
\end{enumerate}

\newpage
\subsection{VF}

\subsubsection{2019}

\paragraph{1ª Questão} (1,5 pontos)

Seja $S$ o conjunto de vetores $(x,y,z,w)$ do espaço vetorial $\mathbb{R}^4$ que obedecem ao sistema de equações:
$$\begin{cases}
-2x-y+2w=5 \\
3x+y-2z-2w=3 \\
-x-2y-6z+4w=34\\
-4x-y+2z+3w=12
\end{cases}$$
\begin{enumerate}[label=(\alph*)]
\item (1,0 ponto) Determine $S$.
\item (1,0 ponto) Determine o ponto $X \in S$ mais próximo da origem $O=(0,0,0,0)$.
\end{enumerate}

\paragraph{2ª Questão} (2,0 pontos)

Seja $\pi$ um plano em $\R^3$ que contenha os pontos não-colineares $A, B, C$. Seja $\pi_A$ o plano perpendicular ao segmento $BC$ que passa por $A$. Analogamente, define-se $\pi_B \perp AC$ que passa por $B$ e $\pi_C \perp AB$ que passa por $C$. Demonstre que $\pi_A \cap \pi_B \cap \pi_C$ é uma reta perpendicular a $\pi$.

\paragraph{3ª Questão} (2,5 pontos)

Sejam $A$ e $B$ matrizes $n \times n$ de coeficientes reais, sendo $A$ invertível. Consideramos a equação matricial 

$$ A^TX + X^TA = B $$

onde $X = (x_{ij})$ é a matriz das incógnitas.

\begin{enumerate}[label=\alph*)]
    
    \item (0,5 ponto) Se o sistema é possível, prove que $B$ é simétrica.
    \item (1,0 ponto) Calcule a dimensão do espaço solução do caso homogêneo.
    \item (1,0 ponto) Suponha que $B$ é simétrica arbitrária. É possível o sistema? Justifique sua resposta.

\end{enumerate}

\paragraph{4ª Questão} (1,0 ponto)

Classifique a afirmação abaixo como verdadeira ou falsa. Demonstre caso verdadeiro, e mostre um contra-exemplo caso falsa. A interpretação da afirmação é parte integrante da questão.
\begin{center}
    \textbf{(V ou F)} Sejam $A, B$ matrizes quadradas $n \times n$, tais que $AB = 0$. Então $BA = 0.$  
\end{center}


\paragraph{5ª Questão} (2,0 pontos)
Sejam as funções reais de variável real:
$$ u_1(t)=\sen{(t)}, u_2(t)=\sen{(t + 2\pi/5)}, u_3(t)=\sen{(t + 4\pi/5)},$$
$$u_4(t)=\sen{(t + 6\pi/5)}, u_5(t)=\sen{(t + 8\pi/5)}, v(t)=\sen{(2t)}.$$

\begin{enumerate}[label=\alph*)]
    
    \item (1,0 ponto) Determine a dimensão do subespaço gerado $S = L\{u_1, u_2, u_3, u_4, u_5 \}$
    \item (1,0 ponto) Demosntre que $v \notin S$.

\end{enumerate}

\paragraph{6ª Questão} (1,0 ponto)

Classifique a afirmação abaixo como verdadeira ou falsa. Demosntre caso verdadeira, e mostre um contra-exemplo caso falsa. A interpretação da afirmação é parte integrante da questão.

\begin{center} 
    \textbf{(V ou F)} Considere o espaço vetorial $C(-1,1)$ de funções reais contínuas no intervalo $[-1,1]$. O subespaço gerado pelas funções $f_1(x)=x, f_2(x)=|x|, f_3(x)=x^2$ possui dimensão 3.
\end{center}


\newpage

\subsubsection{2017 \original{https://drive.google.com/file/d/1OPWe4V1DsCrERUfZeAieW9IVA57QCQf-/view?usp=sharing}}
\paragraph{Questão 1:} (2,0 pontos)

Seja $S$ o conjunto de vetores $(x,y,z,w)$ do espaço vetorial $\mathbb{R}^4$ que obedecem ao sistema de equações:
$$\begin{cases}
-2x-y+2w=5 \\
3x+y-2z-2w=3 \\
-x-2y-6z+4w=34\\
-4x-y+2z+3w=12
\end{cases}$$
\begin{enumerate}[label=(\alph*)]
\item (1,0 ponto) Determine $S$.
\item (1,0 ponto) Determine o ponto $X \in S$ mais próximo da origem $O=(0,0,0,0)$.
\end{enumerate}
\paragraph{Questão 2:} (2,0 pontos)

No $\mathbb{R}^4$, considere as retas $r=\{(0,1,0,1)+t(1,0,0,1) \ | \ t \in \mathbb{R}\}$ e $s=\{(2,3,2,1)+t(0,1,1,1) \ | \ t \in \mathbb{R}\}$. Determine um plano $M$ que seja ortogonal às retas $r$ e $s$, de forma que $M \cap r \neq 0, \ M \cap s \neq 0$.
\paragraph{Questão 3:} (2,0 pontos)

Sejam $a_1,a_2,\dots,a_n,b$ números reais tais que $b>\sum_{i=1}^n a_i^2$. Demonstre que a matriz:
$$A=\left[ 
\begin{array}{cccccc}
1 & 0 & 0 & \dots & 0 & a_1 \\
0 & 1 & 0 & \dots & 0 & a_2 \\
0 & 0 & 1 & \dots & 0 & a_3 \\
\vdots & \vdots &\vdots & \vdots & \vdots \\
0 & 0 & 0 & \dots & 1 & a_n \\
a_1 & a_2 & a_3 & \dots & a_n & b \\
\end{array}
\right] $$
de tamanho $(n+1)\times(n+1)$, possui determinante diferente de 0.
\paragraph{Questão 4:} (2,0 pontos)\\
Sejam as funções reais definidas em $\mathbb{R}$:
$$ u_1(x)=1, \ \ \ u_2(x)=cos(x), \ \ \ u_3(x)=cos(2x), \ \ \ u_4(x)=sen^2(x),$$ $$u_5(x)=(1+cos(x))^2+(1+sen(x))^2, \ \ \ u_6(x)=sen(x+\pi/4)$$\\
Determine uma base para $L\{u_1,u_2,\dots,u_6\}$.
\paragraph{Questão 5:} (2,0 pontos)\\
Para cada um dos itens a seguir, classifique as afirmações como verdadeiras ou falsas. Demonstre as afirmativas verdadeiras, e argumente ou mostre um contra-exemplo para as falsas. A interpretação das questões é parte integrante da questão.

\begin{enumerate}[label=(\alph*)]
\item (1,0 ponto) Todo sistema de equações lineares que possui infinitas soluções admite uma base para o seu conjunto solução.
\item (1,0 ponto) Considere um espaço vetorial $V$ de dimensão infinita. Dados $S_1,S_2$ subespaçõs vetoriais de $V$ tais que $S_1 \oplus S_2 =V$. Então a dimensão de $S_1$ ou $S_2$ é infinita.
\end{enumerate}

\newpage
\subsubsection{2016 \original{https://drive.google.com/file/d/10W6mP98hydnaD-3MjBgYNoRxzbxX9n3q/view?usp=sharing}}

\paragraph{Questão 1:} (2,5 pontos)

Seja $M$ o conjunto de vetores $(x,y,z,w)$ do espaço vetorial $\mathbb{R}^4$ que obedecem ao sistema de equações

$$\begin{cases}
x+3y+z-w=2 \\
2x+6y+z-w=3 \\
\end{cases}$$

\begin{enumerate}[label=(\alph*)]
\item (1,0 ponto) Mostre que $M$ é um plano.
\item (1,5 pontos) Calcule a distância entre $M$ e a reta $r=\{(-3,10,5,6)+t(1,1,1,-1) \ | \ t \in \mathbb{R}\}$
\end{enumerate}

\paragraph{Questão 2:} (1,5 pontos)

Sejam $A$ uma matriz $n \times n$ de coeficientes reais e $B=\{v_1,v_2,\dots,v_n\}$ uma base para o espaço vetorial $\mathbb{R}^n$. Demostre que $\det (A) \neq 0$ se, e somente se, o conjunto $C=\{Av_1,Av_2,\dots,Av_n\}$ é uma base do $\mathbb{R}^n$.

Notação: Para os produtos $Av_i$ fazerem sentido, considere os vetores escritos como matrizes colunas.

\paragraph{Questão 3:} (2,0 pontos)

Sejam $A_1,A_2,\dots,A_{n-1}$ vetores do espaço vetorial $\mathbb{R}^n$.
\begin{enumerate}[label=(\alph*)]

\item (1,0 ponto) Suponha que $A_1,A_2,\dots,A_{n-1}$ são linearmente independentes. Prove que o subespaço gerado $H=\langle A_1,A_2,\dots,A_{n-1} \rangle$ coincide com o subespaço vetorial $G$ de vetores $X=[x_1,x_2,\dots,x_n]^T \in \mathbb{R}^n$ que satisfazem $\det (A_1,A_2,\dots,A_{n-1},X) = 0$. 

\item (1,0 ponto) Analise a relação entre os subespaços $G$ e $H$ para o caso em que os vetores $A_1,A_2,\dots,A_{n-1}$ são linearmente dependentes.

\end{enumerate}

\paragraph{Questão 4:} (2,0 pontos)

Sejam $u,v,w$ vetores do espaço vetorial $\mathbb{R}^3$. Prove que $\{u,v,w\}$ é linearmente independente se, e somente se, $\{u\times v, v\times w, w\times u\}$ é linearmente independente.

\paragraph{Questão 5:} (2,0 pontos)

Sejam $P_1,P_2,P_3,P_4$ pontos arbitrários do espaço vetorial $\mathbb{R}^3$. Prove que $P_1,P_2,P_3,P_4$ são não coplanares se, e somente se, o sistema $\norm{\overrightarrow{XP_1}}=\norm{\overrightarrow{XP_2}}=\norm{\overrightarrow{XP_3}}=\norm{\overrightarrow{XP_4}}$ têm solução única.

\newpage
\subsubsection{2011 \original{https://drive.google.com/file/d/1FP-ZBEWCjyx32t3fcMYN4H2FlOD6znPP/view?usp=sharing}}

\paragraph{Questão 1:} (1,0 ponto)

Seja $V$ um espaço euclidiano real. Sejam $x$ e $y$ elementos de $V$. Utilize a desigualdade de Cauchy-Schwarz para provar que: 

$$ \norm{x+y} \leq \norm{x} + \norm{y} $$

\paragraph{Questão 2:} (2,0 pontos)

Sejam $V$ e $W$ espaços vetoriais e $T:V \rightarrow W$ uma transformação linear.

\begin{enumerate}[label=(\alph*)]

\item (1,0 ponto) Prove que $T(V)$ é um subespaço vetorial de $W$.

\item (1,5 pontos) Prove que $N(T)$ é um subespaço vetorial de $V$.

\end{enumerate}

\paragraph{Questão 3:} (2,0 ponto)

Seja $V$ um espaço euclidiano real de dimensão finita. Prove que o produto interno de dois elementos de $V$ é igual ao produto escalar de suas componentes relativas a uma base \textbf{ortonormal}. Ou seja, se as componentes de dois elementos $x$ e $y$ de $V$ em uma base ortonormal forem $(x_1,\dots,x_n)$ e $(y_1,\dots,y_n)$, então:

$$ \langle x,y \rangle = \sum_{i=1}^n x_i y_i$$

(Dica: Se você utilizar a Fórmula de Parseval sem prová-la, a sua demonstração será considerada circular. Por outro lado, não se assuste caso você não se lembre da fórmula de Parseval apresentada no Apostol.)

\paragraph{Questão 4:} (2,5 ponto)

Seja $T$ a transformação linear no espaço dos polinômios reais de grau menor ou igual a 3 dada por:
$$ T(1)=1-t$$
$$ T(1+t)=t^3 $$
$$ T(t+t^2)=3-t^2$$
$$ T(t^2+t^3)=1+t^2$$
Determine:

\begin{enumerate}[label=(\alph*)]

\item (0,5 ponto) Base e dimensão do núcleo;
\item (0,5 ponto) Base e dimensão da imagem;
\item (1,0 ponto) T;
\item (0,5 ponto) O polinômio com imagem $(-7-t+8t^2+t^3)$.


\end{enumerate}

\paragraph{Questão 5:} (2,5 ponto)

Encontre a projeção ortogonal de $u=(5,3,1)$ sobre o subespaço gerado pelos vetores ortonormais, formado pelo complemento ortogonal a $t(-1,1,2),\ t \in \mathbb{R}$. 

\newpage
\subsubsection{2009 \original{https://drive.google.com/file/d/1zOltEr8nhcCT0R6_XoEeaoAjNzivXqgi/view?usp=sharing}}

\paragraph{Questão 1:} (2,0 pontos)

Prove as afirmações a seguir:
\begin{enumerate}[label=(\alph*)]
\item (1,0 ponto) Seja $A \in \mathbb{F}^{m \times n}$ tal que seu produto com sua transposta é comutativo. Se o posto de $A$ for igual ao número de linhas, então o sistema linear homogêneo $AX=0$ tem solução única.
\item (0,5 ponto) Considere a matriz aumentada $H=[A|B]$ do sistema linear $AX=B$. Se as colunas de $H$ formarem um conjunto LI, então o sistema dado não tem solução.
\item (0,5 ponto) Considere a cônica $\Lambda_1 = \{(x,y) \in \mathbb{R}^2 : \dfrac{x^2}{a^2}+\dfrac{y^2}{b^2}=1 \}$. Se $(u,v)=t(x,y),t \in \mathbb{R}$, então a cônica $\Lambda_2=\{(u,v)\in \mathbb{R}^2 : \dfrac{u^2}{a^2}+\dfrac{v^2}{b^2}=1 \}$ tem a mesma excentricidade de $\Lambda_1$.
\end{enumerate}
\paragraph{Questão 2:} (2,5 pontos)

Seja $\{U_1,\dots,U_n\}$ um conjunto ortonormal e $\{V_1,\dots,V_n\}$ subconjunto qualquer, ambos do espaço euclidiano $\mathbb{R}^n$, tais que $\{U_1+V_1,\dots,U_n+V_n\}$ é um conjunto LD. Mostre que $\sum_{j=1}^n \norm{V_j}^2 \geq 1$.

\paragraph{Questão 3:} (3,0 pontos)

Considere o sistema linear $AX=B$, onde $
A=\left[
\begin{array}{c|c}
7 & v \\
\hline
v^T & 1
\end{array}
\right]
\in \mathbb{R}^{(n+1)\times (n+1)}$,$v=(a,\dots,a)^T$, $
B=\left[
\begin{array}{cc}
v \\
\hline
b
\end{array}
\right]
$, sendo tanto $a$ quanto $b$ parâmetros reais.
\begin{enumerate}[label=(\alph*)]
\item (1,0 ponto) Encontre matrizes $G$ e $H$, na forma particionada, de modo que $GA$ esteja na forma escalonada e $HGA$ esteja na forma escalonada reduzida por linha;
\item (1,0 ponto) Encontre os parâmetros $a$ e $b$ de modo que $AX=B$ tenha solução única e determine o valor de X;
\item (1,0 ponto) Encontre os parâmetros $a$ e $b$ de modo que $AX=B$ tenha infinitas soluções e determine o conjunto solução.
\end{enumerate}
\paragraph{Questão 4:} (2,5 pontos)

Seja $A \in \mathbb{R}^{n\times n}$ nilpotente de índice $k$. Mostre que $I-A$ é invertível e determine essa inversa. 


\newpage
\subsubsection{2006 \original{https://drive.google.com/file/d/13f1ar5EX33BZ0PlEvSrW0Ce-etd_Yz31/view?usp=sharing}}
\textbf{Desfocada} - Atual matéria de Alg Lin 2

\paragraph{Questão 1} (1 ponto)

Utilizando o conceito de formas quadráticas. Determine o valor de C para que o gráfico da expressão $2xy - 4x + 2y + C = 0$ seja representado por um par de retas.

\paragraph{Questão 2} (2,0 pontos)

Diga se a afirmativa é verdadeira ou falsa, justificando sua resposta: $P$ e $Q$ são matrizes quadradas de ordem $n$ tais que $PQ=QP$ e $P^r=Q^s=0$ para algum inteiro $r$ e $s$ positivos. Então $(I+P+Q)$ e $(I-P-Q)$ possuem inversa.



\paragraph{Questão 3} (2,0 pontos)

Mostre que existe uma matriz real quadrada de ordem $n$ tal que $A^3=A + I$, e mostre que
isso deve satisfazer $det (A) >0$.

\paragraph{Questão 4} (1,0 ponto)

Sejam $A$ e $B$ matrizes complexas $n \times n$. Prove que $\modu{tr (AB*)}^2 < tr (AA*)\cdot tr (BB*)$.
\paragraph{Questão 5} (2,0 pontos)

Sejam $A$ e $B$ matrizes complexas unitárias $n \times n$. Prove que $\modu{det (A+B)} \leq 2^n$.

\paragraph{Questão 6} (2,0 pontos)

Suponha $A$ e $M$ são matrizes $n \times n$ definidas sobre o corpo dos complexos, $A$ é
invertível e $AMA^{-1} = M^2$ . Prove que os autovalores não nulos de $M$ são raízes da
unidade.

\newpage

\subsection{REC}

\subsubsection{2016 \original{https://drive.google.com/file/d/1P1qzC2gnhEmUvseWBoZeuXIoZxWfk0HX/view?usp=sharing}}

\paragraph{$1^a$ Questão:} (2,5 pontos)

\begin{enumerate}[label=\alph*)]
\item (1,5 pontos) Sejam $X_1 = (2,1,-5,12)$, $X_2=(-3,7,-1,-1)$, $X_3=(13,-11,-3,13)$ vetores do $\mathbb{R}^4$. Encontre escalares $\alpha,\beta \in \mathbb{R}$
tais que $X_3=\alpha X_1 + \beta X_2 + W$, onde $W$ é um vetor ortogonal a $X_1$ e $X_2$.
\item (1,0 ponto) É possível que em uma matriz $3x3$ o subespaço gerado pelos vetores-coluna seja distinto do subespaço gerado pelos vetores linha? 
\end{enumerate}

\paragraph{$2^a$ Questão:} (2,5 pontos)

Sejam $r,s,t$ retas no espaço $R^3$ cujos vetores diretores são linearmente independentes. Prove que existe uma única reta $t^\prime$  paralela a $t$
que intersecta $r$ e $s$.

\paragraph{$3^a$ Questão:} (2,5 pontos)

\begin{enumerate}[label=\alph*)]
\item (1,0 ponto) Defina o conjunto $S_B=\{X \in M_{2\times 2}\ | \ A^TX-X^TA=B \}$, onde $A$ e $B$ são matrizes $2 \times 2$ 
e $\det{A}\neq 0$. Determine as condições para a matriz $B$ de forma que $S_B$ seja um subespaço vetorial.
\item (1,5 pontos) Para 
$A=\left[
\begin{array}{cc}
1 & 0 \\
2 & 1 \\
\end{array}
\right]$ e 
$B=
\left[
\begin{array}{cc}
0 & 0 \\
0 & 0 \\
\end{array}
\right]$, determine uma base para $S_B$.
\end{enumerate}

\paragraph{$4^a$ Questão:} (2,5 pontos)

Sejam $P_i=(x_i,y_i,z_i), \ i=1,2,3,4$, quatro pontos do espaço $\mathbb{R}^3$. Prove que os vetores $P_i$ são coplanares se e somente se:

$$ \det{
\left(
\begin{array}{cccc}
x_1 & y_1 & z_1 & 1 \\
x_2 & y_2 & z_2 & 1 \\
x_3 & y_3 & z_3 & 1 \\
x_4 & y_4 & z_4 & 1 \\
\end{array}
\right)
}=0 $$

\newpage
\section{Física I}

\subsection{VE1}

\subsubsection{2019 \original{}}

\paragraph{Questão 1} (5,0) Um corpo de massa $m$ desloca-se por um trilho e sua trajetória é dada por $y = \alpha x^{\frac{3}{2}}$, onde $\alpha$ é uma constante real positiva e $x \geq 0$. Inicialmente o corpo encontra-se na origem do sistema de coordenadas com velocidade $v_0$. Em toda trajetória o motor do corpo provê uma força $F$ de módulo constante e direção tangencial à trajetória. O corpo está imerso em um fluido com força de resistência de módulo $kv$, onde $k$ é uma constante (o termo quadrático é desprezível). \\
Teorema: Considere uma função $f(x)$ tal que $f(x)$ e $f'(x)$ (derivada em relação a x) são contínuas em $\left[a, b\right]$, o comprimento $s$ de parte do gráfico de $f$ entre $x = a$ e $x = b$ é dado pela fórmula:

\begin{center}
\begin{align*}
    s = \int_{a}^{b} \sqrt{1 + [f'(x)]}dx. \\
\end{align*}
\end{center}

Aplicando este teorema, qual a posição do corpo no instante $t = 5m/k$?

\paragraph{Questão 2}(5,0) Consideremos uma partícula de massa $m$ e carga $q$ que entra em uma região de campo magnético uniforme $\vec{B} = B\hat{y}$. Inicialmente a partícula está em $\vec{r_{0}} = x_{0}\hat{x} + y_{0}\hat{y} + z_{0}\hat{z}$ com velocidade $\vec{v_{0}} = v_{0y}\hat{y} + v_{0z}\hat{z}$. Lembrando que a força magnética é dada por $q\vec{v} \times \vec{B}$, determine a posição da partícula em função do tempo.

\newpage

\subsubsection{2016 \original{https://drive.google.com/file/d/1J3Hzk_KPTWWYHTxj_ku4ffNkQAMRZs2Z/view?usp=sharing}}

\paragraph{Questão 1} (2,5 pontos)

Uma partícula de massa $m$ se move em uma circunferência de raio $R$, tal que sua energia cinética é dada por $T=A.s^2$, onde $A$ é uma constante $A>0$ e $s$ é o arco que a partícula descreve. Desconsidere os efeitos da força gravitacional e determine:

\begin{enumerate}[label=(\alph*)]

\item (0,5) O módulo da aceleração normal em termos do $s$.

\item (0,5) O módulo da aceleração tangencial em termos do $s$.

\item (1,5) O módulo da força total que atua sobre a partícula em termos do $s$.

\end{enumerate}

\paragraph{Questão 2} (2,5 pontos)

Uma bolinha de massa $m$ desliza, a partir do repouso e sem atrito, desde o alto de uma montanha de altura $h$, ao qual lhe segue um caminho circular de raio $R$. Considere a gravidade $g$.

\begin{enumerate}[label=(\alph*)]

\item (1,0) Qual a condição para que a bolinha consiga realizar um loop completo?

\item (1,0) Considere que a curva da montanha possa ser aproximada pela função: $y(x)=h\left(1-\dfrac x L\right)$, onde $L$ é a distância horizontal até a posição inferior do caminho circular. Calcule o tempo que leva a bolinha desde a posição $(0,h)$ até a posição $(L,0)$.

\item (0,5) Calcule o item (b) quando $h>>L$.

\end{enumerate}

\paragraph{Questão 3} (2,5 pontos)

Uma corrente uniforme de massa $m$ e comprimento $L$ é segurada por uma extremidade de modo que a extremidade oposta toca a superfície de uma balança. A corrente é então largada. Considere a gravidade $g$.

\begin{enumerate}[label=(\alph*)]

\item (1,0) Determine a máxima leitura na balança (é uma balança de físico, de modo que ele mede a força em Newtons!).

\item (1,5) Determine a força da balança em função do tempo $F(t)$.

\end{enumerate}

\paragraph{Questão 4} (2,5 pontos)

Uma conta desliza, sob a ação da gravidade e sem atrito, ao longo de um arame cuja forma é dada pela função $y(x)$. Considere que na posição $(0,0)$ o arame é vertical e que a conta passa por esse ponto com velocidade $v_0$ para baixo. Determine qual deve ser a forma do arame, ou seja $y(x)$, de modo que a componente vertical da velocidade permaneça $-v_o$ para qualquer instante de tempo.

\newpage

\subsubsection{2015 \original{https://drive.google.com/file/d/1rZyPa6pyJ_EWK4032TKOE4inEr9lm_6s/view?usp=sharing}}

\paragraph{Questão 1:}

Uma partícula de massa $m$ está sujeita a uma força dada por $F(t)=F_0 \ \sen (\Omega t)$.
Considerando $x_0=0$ e $v_0=0$, determine:

\begin{enumerate}[label=\alph*)]
\item A velocidade da partícula em função do tempo: $v(t)$
\item A posição da partícula em função do tempo: $x(t)$
\item Esboce o gráfico de $x(t)$
\end{enumerate}


\paragraph{Questão 2:}

Um corpo é arremessaado verticalmente para cima com velocidade $v_0$, ficando submetido ao campo gravitacional e a uma força de resistência 
proporcional à velocidade: $F_r=-\alpha v$. Oriente a trajetória para cima.

\begin{enumerate}[label=\alph*)]
\item Obtenha a velocidade da partícula em função do tempo: $v(t)$
\item Esboce o gráfico de $v(t)$
\item Calcule o tempo de subida do corpo
\end{enumerate}

\paragraph{Questão 3:}

Um cliente solicitou $1$kg de açúcar vendido a granel. O vendedor colocou o recipiente sobre uma balança
e começou a despejar o açúcar, a uma taxa de $250$ gramas por segundo, de uma altura de $1$m. No exato instante em que a balança
marcava $1$kg, o vendedor retirou o recipiente da balança. Determine a real quantidade de açúcar adquirida pelo cliente. Considere g=$9,8$m/s$^2$.


\paragraph{Questão 4:}

Uma bola é lançada com velocidade inicial $v_0$ a partir da base de um plano inclinado. O plano está inclinado de um ângulo
$\phi$ em relação à horizontal. A velocidade inicial da bola faz um ângulo $\theta$ em relação ao plano. Escolha o eixo \textbf{x}
paralelo ao plano inclinado e \textbf{y} perpendicular ao plano.

\begin{enumerate}[label=\alph*)]
\item Escreva a $2^a$ Lei de Newton para esses eixos e encontre o instante em que a bola colide com o plano (tempo de vôo).
\item Encontre o alcance da bola em relação ao ponto de lançamento.
\end{enumerate}

\newpage

\subsubsection{2008 \original{https://drive.google.com/file/d/1G-n4jHDRWxaKT9PkMD1ck6IEqnf9I-p9/view?usp=sharing}} 

\paragraph{Questão 1:}(2,0 pontos)\\
Um corpo parte da origem de um sistema de coordenadas cartesianas com velocidade nula e sob uma aceleração $a(t)=4u_x+(48t^3+16)u_y\;m/s^2$. Esboce o gráfico da função $y(x)$ a partir do instante inicial.

\paragraph{Questão 2:}(2,5 pontos)\\
Um corpo é lançado do solo com velocidade inicial $v_0$ na direção que faz com a horizontal um ângulo $\theta$. Sabendo-se que a aceleração da gravidade é $g$ e que sofre também uma aceleração horizontal $\vec{a}_x=g\,u_x$ $m/s^2$. Determine:

\begin {tasks}(1)
\task (1,5) a máxima altura \textbf{h} alcançada pelo corpo;
\task (1,0) o alcance \textbf{d} que o corpo atinge sobre o plano horizontal de lançamento.
\end {tasks}

\paragraph{Questão 3:}(2,5 pontos)\\
O gráfico abaixo mostra a variação da aceleração de um corpo com o tempo.

\begin {tasks}(1)
\task (1,2) Determine \textbf{através do gráfico a(t) x t}, o esboço de v(t) x t, sabendo que em $t=0$ o corpo tem velocidade igual a $2\,m/s$. Encontre a velocidade em $t=4$, $8$, $10$ e $12s$.
\task (0,8) Estude o coeficiente angular da tangente da função que você obteve no item $a)$ de $t=0$ a $t=12s$.
\task (0,5) Ache a velocidade média entre $t=0$ e $t=12s$.
\end {tasks}

\begin{figure}[ht]
\centering
\includegraphics[width=8cm]{fis1ve12008q3.png}
\end{figure}

\newpage
\paragraph{Questão 4:}(3,0 pontos)\\
Um corpo desloza sobre uma haste $X'$, sendo sua posição em relação à origem $O$ dada por $2\,A\,\coswt$, onde $A$ e $w$ são constantes.A direção da haste $X'$ em $t=0s$ coincide com o eixo $x$ e, durante todo o tempo observado, a haste está em rotação com velocidade angular constante $\vec{w}=w\,\hat{u}_z$. Determine para o corpo: 

\begin{tasks}(1)
\task (1,0) o vetor velocidade em coordenadas cartesianas e polares;
\task (0,8) à partir da expressão da velocidade em coordenadas polares, a velocidade na forma $\vec{v}=v\,\hat{u}_r$ letinindo $v$ e $\hat{u}_r$.
\task (1,2) à partir do item b, a componente da aceleração responsàvel pela variação do módulo da velocidade $v$ e responsável pela variação da direção da velocidade.
\end{tasks}
Obs: Descreva o vetor unitário na direção ( $ \hat{u}_r$ ) em função dos unitários das coordenadas polares.

\newpage

\begin{figure}[ht]
\centering
\includegraphics[width=8cm]{fis1ve12008q4.png}
\end{figure}


\newpage

\subsubsection{2006 \original{https://drive.google.com/file/d/1iLMci1L1dNk1WOM2tlaP4YxsDUP-H7J8/view?usp=sharing}}
\paragraph{$1^a$ Questão:} (2,5) 

Uma cunha de massa $M$ repousa sobre o topo horizontal de uma mesa sem atrito. Um bloco de massa $m$ é colocado sobre a cunha
conforme a figura abaixo. Não existe atrito entre o bloco e a cunha. O sistema é liberado a partir do repouso. Determine a aceleração
da cunha e do bloco em relação a mesa.

\begin{figure}[ht]
\centering
\includegraphics[width=12cm]{fis1ve12006q1.png}
\end{figure}

\paragraph{$2^a$ Questão:} (2,5) 

Um corpo descreve um movimento dado pela equação abaixo. Determine a aceleração no referencial tangencial ($u_t$ e $u_n$).
$$ \overrightarrow{r}(t) = (\omega R + R\omega \cos (\omega t))\overrightarrow{u}_x-(R\omega\sen(\omega t))\overrightarrow{u}_ym $$
\paragraph{$3^a$ Questão:} (2,5) 

Um corpo de massa $m=4$kg parte em $t=0$ da origem e tem velocidade dada pela equação abaixo:
$$ \overrightarrow{v}(t)=2t\overrightarrow{u}_x+3t^2\overrightarrow{u}_y-4\overrightarrow{u}_z$$

Determine em função do tempo para esta massa:

\begin{enumerate}[label=\alph*)]
\item (1,0) A força resultante;
\item (1,0) O momento angular;
\item (0,5) O torque resultante.
\end{enumerate}

\paragraph{$4^a$ Questão:} (2,5)

Um foguete com massa $2600$kg encontra-se com velocidade $1000$m/s expelindo um combustível com velocidade de escape $v_e=2700$m/s
pela sua região traseira. Quando sua massa for de $1300$kg, o primeiro reservatório de combustível fica vazio, sendo sua carcaça de $300$kg
desprendida do foguete, iniciando-se a queima do combustível do segundo reservatório. Quando a massa do foguete for $800$kg,
o segundo reservatório, também de $300$kg, é liberado. Despreze a atração gravitacional.

\begin{enumerate}[label=\alph*)]
\item (1,0) Mostre que $\dfrac{dp}{dt}=m\dfrac{dv}{dt} + v_e \dfrac{dm}{dt}$, onde $\dfrac{dm}{dt}$ é a variação de massa com o tempo do foguete.
\item (1,5) Determine a quantidade de movimento do foguete imediatamente após a liberação do segundo reservatório.
\end{enumerate}

\newpage

\subsubsection{2005 \original{https://drive.google.com/open?id=1pbDtSfy9FXzZvWNOPgZVUWwKYQeuL7Z2}}

\paragraph{Questão 1:}(2,5 pontos)\\
Um canhão encontra-se a uma distância horizontal $C$ e vertical $h$ do ponto mais alto de uma colina íngreme (veja figura).

\begin{tasks}(1)
\task Verifique a velocidade inicial do projétil lançado passando rente ao pico da colina é dada
$$v_0=\dfrac{C}{\cos\theta}\sqrt{\dfrac{g}{2(C\tan(\theta)-h)}}$$
$g$ é a aceleração da gravidade e $\theta$ é o ângulo de inclinação de $v_0$ com o plano horizontal.
\task Determine o instante em que o projétil atinge o solo, que encontra-se na mesma altura do ponto de lançamento.
\end{tasks}

\begin{figure}[ht]
\centering
\includegraphics[width=8cm]{fis1ve12005q1.png}
\end{figure}

\paragraph{Questão 2}(3,0 pontos)\\
Um foguete de massa $m$, partindo do repouso, é lançado verticalmente para cima com aceleração $a$ durante $t$ segundos. Desprezando a resistÊncia do ar, as dimensões do foguete e considerando-se a aceleração da gravidade constante e iguaç à $g$, determine:
\begin{tasks}(1)
\task A velocidade média no percurso de subida.
\task O intervalo de tempo entre o lançamento e o retorno do foguete no solo.
\end{tasks}

\paragraph{Questão 3:}(2,5 pontos)//
As coordenadas de um corpo em movimento são $x=A\mathrm{sen}(wt)$ e $y=B\mathrm{sen}(\omega t+\frac{\pi}{2})$ . Determine.

\begin{tasks}(1)
\task a equação cartesiana da trajetória. Desenhe e diga seu nome.
\task o valor da velocidade $\vec{v}$ e do seu módulo $\left | \vec{v} \right |=v$ em qualquer instante de tempo
\task o valor da aceleração $\vec{a}$ e de seu módulo $\left | \vec{a} \right |=a$ em qualquer instante de tempo.
\end{tasks}

\paragraph{Questão 4:}(2,0 pontos)\\

\begin{tasks}(1)
\task Qual é a expressão da conservação do momento linear da reção química de troca $A+BC \rightarrow AB+C$, no referencial em que a molécula $BC$ está inicialmente em repouso? Massas $m_A$, $m_{BC}$, $m_{AB}$, $m_C$.
\task Qual é o valor de $\vec{v}_C$?
\task Se $m_{AB}>>m_A$, qual é o valor aproximado de $\vec{v}_C$?
\task Desenhe o antes e o depois da reação química na condição do item anterior.
\end{tasks}

\newpage

\subsubsection{2003 \original{https://drive.google.com/open?id=14VEcfRhra-BzXweNOWDAuLFENl22v4Mk}}

\begin{enumerate}
\item 
\begin{tasks}
\task Resolva o problema 5.14 do Alonso $\&$ Finn;
\task Determine as expressões gerais da velocidade média e da aceleração média de $t_0$ até $t_0 + \Delta t$ e os seus valores entre $t=2s$ e $t=3s$;
\task Verifique se o valor da velocidade média é o mesmo da média de $v(t=2s)$ e $v(t=3s)$. Comente sua verificação.
\end{tasks}

\item A velocidade de uma partícula, em metros por segundo é dada por $v=6-4t$.
\begin{tasks}
\task Representar $v(t)$ contra $t$ e determinar a área entre a curva e o eixo dos $t$, de $t=2s$ até $t=5s$;
\task Determinar a função posição $x(t)$ pela integração e usar o resultado para determinar o deslocamento durante o intervalo $t=2s$ até $t=5s$;
\task Qual a velocidade média nesse intervalo?.
\end{tasks}
\item
\begin{tasks}
\task Resolva o problema 5.40 do Alonso $\&$ Finn;
\task À partir do vetor posição em coordenadas cartesiana, escreva o vetor posição da forma:
$$\vec{r}=r \ \hat{u_r}$$ onde $r$ é o módulo e $\hat{u_r}$ o vetor unitário;

\task Verifique que o vetor velocidade linear pode ser escrito por $$\vec{v}=v \ \hat{u_\theta}$$ onde $\hat{u_\theta} = -\sen{\theta} \ \hat{u_x} + \cos{\theta} \ \hat{u_y}$, sendo $\theta$ a posição angular.
\end{tasks}

\item 
\begin{tasks}
\task Resolva o problema 5.54 do Alonso $\&$ Finn;
\task Escreva os vetores posição, velocidade linear e aceleração linear em coordenadas cartesianas.\\
\end{tasks}

\item Um corpo de desloca pela linha helicoidal

$$\begin{cases} x=a \ \cos{kt} \\
y=a \ \sen{kt} \\
z=vt \end{cases}$$

onde $a$, $k$ e $v$ são constantes. 

\begin{tasks}
\task Determine os vetores velocidade e aceleração;
\task Verifique que é possível escrever os vetores posição da forma
$$\vec{r} = r \ \hat{u_r} + z \ \hat{u_z}$$
definindo $r$ e o vetor unitário $\hat{u_r}$.
\end{tasks}

\item Uma partícula realiza o movimento descrito pelas expressões $x = f(\theta)$, $y = g(\theta)$ e $z = 5t$, onde $t$ é o tempo e $\theta = h(t)$. Sabendo que $\theta(0) = 1, \dot{\theta}(0) = 1, \ddot{\theta}(0) = -1, f(1) = -2, f'(1) = 5, f''(1) = 4, g(1) = 2, g'(1) = 1/2, g''(1) = 3$, obtenha a posição, a velocidade e a aceleração da partícula ao iniciar o movimento.

\item Considerando o movimento unidimensional com aceleração constante, partindo das definições de velocidade e acelerações instantâneas,
obtenha a expressão conhecida como Equação de Torricieli, utilizando rigorosamente o Cálculo.

\item Dado o sistema cartesiano plano com vetores unitários $\textrm{û}_x$ e $\textrm{û}_y$, definida a posição na forma $\overrightarrow{r}=r\textrm{û}_r$,
onde $r$ é o módulo de $\overrightarrow{r}$ e $\textrm{û}_r$ é o vetor unitário na direção e sentido de $\overrightarrow{r}$, temos definido o ângulo $\theta$ entre
$\textrm{û}_x$ e $\textrm{û}_r$, medido no sentido trigonométrico. Definindo, agora, o vetor unitário $\textrm{û}_\theta$ normal ao vetor $\textrm{û}_r$ e com sentido trigonométrico,
obtenha os vetores $\textrm{û}_x$ e $\textrm{û}_y$ em função de $\theta,\textrm{û}_r$ e $\textrm{û}_\theta$.

\item Prove que uma partícula que se move sob aceleração $\overrightarrow{a}=k\textrm{û}\times \overrightarrow{v}$, onde $k$ é constante,
$\textrm{û}$ é um vetor unitário arbitrário e $\overrightarrow{v}$ é o vetor velocidade, tem um movimento circular com velocidade angular $\overrightarrow{w}=k\textrm{û}$ ou,
no caso mais geral, um movimento espiral paralelo a $\textrm{û}$.

\item Obtenha a velocidade em função do tempo, de uma partícula que parte do repouso com aceleração $a=b-cv^2$, onde $v$ é a velocidade e
$b$ e $c$ são constantes positivas.

\end{enumerate}


\newpage
\subsubsection{2002 \original{https://drive.google.com/open?id=15UxEYcIDLcGpL-iu4EZDOO-ymHzOVPLg}}

\begin{enumerate}
\item Uma partícula tem movimento unidimensional com a velocidade expressa por $$\vec{v} = (t^2 + 4t) \hat{u_x} \text{ m/s}$$ Determine:
\begin{tasks}
\task a aceleração média entre t=0 e $t=t_0$;
\task a aceleração num instante qualquer t;
\task a posição da partícula no instante t=2s, sabendo que ela estava em x=1m quando começou o movimento.
\task Faça um gráfico da aceleração em função do tempo e determine a área contida pela curva e o eixo do tempo para o intervalo de t=0 até t=2s. Interprete o resultado
\end{tasks}


\item Uma partícula está sujeita a um movimento definido pelas relações $$\begin{array}c r = 2(1+cos2\pi t) \\ \theta = 2 \pi t \end{array}$$ onde r é o módulo do vetor posição e o ângulo acima é o do vetor posição com o eixo x. Determine:

\begin{tasks}
\task os vetores posição, velocidade e aceleração em coordenadas polares para um instante qualquer t;
\task os vetores do item anterior para t=1/4 s.
\end{tasks}

\item Um corpo está no instante inicial t=0 na posição $$\vec{r}_0 = (2\hat{u}_x + 3 \hat{u}_y)m$$ e com velocidade $$\vec{v}_0 = (3\hat{u}_x) m/s$$ A aceleração a que está submetido é $$\vec{a} = (-10 \hat{u}_y + 2t \hat{u}_x) m/s^2$$ Determine:

\begin{tasks}
\task a velocidade e a posição do corpo em t=2s;
\task a velocidade média nos dois primeiros segundos
\end{tasks}

\item Uma partícula carregada descreve um movimento no interior de um acelerador, descrito por $$r=100t$$ $$\theta = 2\pi \times 10^4\, t$$ sendo t em segundos, r em metros e $\theta$ em rad, onde r mede a distância da partícula a origem do sistema de referência e o ângulo é o que o vetor posição faz com o eixo x do sistema de referência.

Sabendo que a partícula irá abandonar o acelerador quando $$r\ge 1000\, m$$ Determine:

\begin{tasks}
\task o instante em que a partícula abandona o acelerador
\task o vetor velocidade em coordenadas polares, no instante do abandono;
\task o vetor velocidade em coordenadas cartesianas no instante do abandono;
\task o módulo da velocidade no instante do abandono;
\task o ângulo que a velocidade de abandono faz com o eixo x, expresso por sua tangente.
\end{tasks}


\end{enumerate}

\newpage

\subsubsection{2001 \original{https://drive.google.com/file/d/1Qg7goiVlJvNilTg82hxx2ZEqBK4wA9eq/view?usp=sharing}}

\paragraph{$1^a$ Questão:} (2,5 pontos)

Um projétil é disparado sob um ângulo de $30^o$ da janela de um edifício a $20$m do solo.
\begin{enumerate}[label=\alph*)]
\item (1,0) Qual a mínima velocidade do projétil para que possa acertar aviões que passam a altitude de 1000m?
\item (1,0) A que distância horizontal da base do edifício ainda há risco de uma pessoa ser atingida utilizando utilizando o resultado anterior?
\item (0,5) Sob que ângulo o solo é atingido pelo projétil?
\end{enumerate}

Obs: Utilize $g=10$m/s$^2$.

\paragraph{$2^a$ Questão:} (2,5 pontos)

A posição de uma partícula é dada por:
$$
\begin{cases}
x(t)=(t-1)^2 \cos (3t^2+2t) \\
y(t)=(t-1)^2 \sen (3t^2+2t) \\
\end{cases}
$$
\begin{enumerate}[label=\alph*)]
\item (1,0) Obtenha a expressão geral da aceleração vetorial em coordenadas polares ($u_r$ e $u_\theta$) do movimento qualquer de uma partícula no plano.
\item (1,5) Determine o ângulo da aceleração resultante com sua componente na direção do vetor unitário $u_r$ no instante $t=2$s. Verifique o efeito desta componente
da aceleração na direção $u_r$ sobre a velocidade.
\end{enumerate}
\paragraph{$3^a$ Questão:} (2,5 pontos)

Uma partícula está em movimento retilínio, com a posição ao longo do eixo de movimento dada por $x(t)=2t-t^3$. Determine:
\begin{enumerate}[label=\alph*)]
\item (0,5) A expressão da velocidade e da aceleração médias entre $t=0$ e $t=t_0$;
\item (0,4) A expressão da velocidade e da aceleração em função do tempo;
\item (0,4) Em que instante a partícula alcança sua posição máxima em $x$;
\item (0,4) Qual o percurso total realizado pela partícula nos primeiros $2$s?
\item (0,4) Quando o movimento é acelerado e quando é retardado nos $2$ primeiros segundos;
\item (0,4) Faça um um gráfico a x t para o intervalo $t=0$s e $t=2$s e ache a área contida pela curva e o eixo do tempo. Interprete seu resultado.
\end{enumerate}

\paragraph{$4^a$ Questão:} (2,5 pontos)
Uma partícula está descrevendo um movimento circular de raio igual a $0,2$m. A observação do movimento começa quando o ângulo do vetor posição com o eixo horizontal
é igual a $\dfrac{\pi}{2}$ e a velocidade angular é 1 rad/s. Sabendo-se que a expressão da aceleração angular em função do tempo é $\alpha = \dfrac{6t}{\pi^2}$, determine:
\begin{enumerate}[label=\alph*)]
\item (0,7) As componentes cartesianas do vetor posição em função do tempo;
\item (0,6) O vetor velocidade média entre $t=0$ e $t=\pi$s;
\item (0,6) A velocidade angular média entre $t=0$ e $t=\pi$s;
\item (0,6) Verifique se existe uma relação entre o que foi pedido no item b e no item c. Explique.
\end{enumerate}
\newpage

\subsubsection{2000 \original{https://drive.google.com/open?id=1lD7Uc8_U7N-28f1SKEfbNLuzuwJVp0zs}}

\begin{enumerate}

\item A aceleração de um objeto, logo após uma grande explosão, é dada pela curva abaixo. Se o objeto está inicialmente em repouso,
determine:

\begin{enumerate}[label=\alph*)]
\item A velocidade em qualquer instante $t$;
\item A velocidade no instante $t=1$s, usando a expressão obtida no item a e à partir do gráfico;
\item O deslocamento total e a velocidade média em $t=2$s;
\end{enumerate}

\begin{figure}[ht]
\centering
\includegraphics[width=16cm]{fis1ve12000q1.png}
\end{figure}

\item A barra $AB$ gira num plano horizontal com velocidade angular constante de $2$rad/s. Um corpo movimenta-se na barra
devido a uma mola e o módulo do vetor posição varia de acordo com a relação $r=3\sen{t}$. Determine em coordenadas polares os vetores
posição, velocidade e aceleração.

Obs: Defina os vetores unitários em função dos dados.

\begin{figure}[ht]
\centering
\includegraphics[width=10cm]{fis1ve12000q2.png}
\end{figure}

\item Um motorista encontra-se em um veículo à velocidade de $30$m/s quando desliga o motor. Se a resistência do ar é de $F=-5$v, determine:
\begin{enumerate}[label=\alph*)]
\item A distância percorrida até que a velocidade seja reduzida à metade;
\item A aceleração centrípeta em $t=5$s, se a estrada é uma curva horizontal de raio $100$m.
\end{enumerate}

Dado: A massa do veículo com o motorista é de $500$kg.

\item Na figura abaixo, $m_1=5$kg e $m_2=15$kg e o coeficiente de atrito cinético é de $0,5$. As massas $m_1$ e $m_2$ estão sobre rampas inclinadas
de $30^o$ e $60^o$, respectivamente. A massa $m_2$ está a $3$m do solo sobre a rampa onde se encontra. Considerando-se que os corpos estão inicialmente
em repouso, determine:
\begin{enumerate}[label=\alph*)]
\item os vetores aceleração das massas $m_1$ e $m_2$ em relação aos eixos cartesianos da figura;
\item o momento linear do sistema formado pelas massas $m_1$ e $m_2$ imediatamente antes de $m_2$ atingir o solo.
\end{enumerate}

\begin{figure}[ht]
\centering
\includegraphics[width=16cm]{fis1ve12000q4.png}
\end{figure}

\item Um missil de massa $m_1=1$kg é disparado com velocidade de $400$m/s com um ângulo de $60^o$ em relação ao plano horizontal. Se no mesmo instante do lançamento outro missil,
no mesmo plano, de massa $m=2$kg é disparado com mesmo ângulo de lançamento e velocidade, determine:
\begin{enumerate}[label=\alph*)]
\item a distância entre os pontos de lançamento para que haja colisão no ponto mais afastado do solo;
\item o momento linear da massa $m_2$ imediatamente após o impacto, se no choque os mísseis mantêm sua integridade, e a massa $m_1$
passa a ter velocidade de $-400 u_x$m/s, onde a direção $x$ é tomada do ponto de lançamento de $m_1$ para o de $m_2$;
\item quais as forças que atuam nos corpos $m_1$ e $m_2$ durante os vôos dos mísseis. Discuta se há conservação do momento linear 
na solução das letras a) e b) do problema.
\end{enumerate}

\end{enumerate}

\newpage
\subsection{VC}
\subsubsection{2016 \original{https://drive.google.com/file/d/1CWmNZs__Zg8e2EggOGfoaTazIz2uXSIU/view?usp=sharing}}

\paragraph{Questão 1} (2,5 pontos)

Uma partícula se move numa trajetória plana $y(x)$ com uma velocidade $v$ de módulo constante. Considere que a trajetória tem a forma elipsoidal descrita pela relação:

$$\left(\frac x m\right)^2+\left(\frac y n\right)^2$$

Onde $m>0$ e $n>0$. Calcule para $x=0$:

\begin{enumerate}[label=(\alph*)]

\item (1,5) O módulo da aceleração total da partícula.

\item (1,0) O raio de curvatura da trajetória.

\end{enumerate}

\paragraph{Questão 2} (2,5 pontos)

Uma partícula de massa $m$ se move em uma dimensão e sobre ela atua uma força que depende da posição tal que $F(x)=-\dfrac{k}{x^2}$, onde $k$ é uma constante ($k>0$). Considere que no instante inicial $t=0$ a posição inicial era $x_0>0$ e a velocidade $v_0>0$. Se a partícula chega no infinito ($x\to \infty$) com velocidade nula, determine:

\begin{enumerate}[label=(\alph*)]

\item (1,5) A função $x(t)$.

\item (1,0) A função velocidade $v(t)$.

\end{enumerate}

\paragraph{Questão 3} (2,5 pontos)

Uma partícula de massa $m$ está submetida à força da gravidade e a uma força de resistência do ar do tipo $F_{res}=-bv$.

\begin{enumerate}[label=(\alph*)]

\item (1,5) Considerando que a partícula partiu da origem com velocidade inicial vertical $v_0$ e orientando o eixo $y$ para cima, determine sua posição em funçao do tempo $y(t)$.

\item (1,0) Considere que uma particula é solta ($v_0=0$) da origem em $t=0$ e que no mesmo instante outra partícula idêntica é arremessada para baixo com velocidade igual à velocidade terminal. Determine a maior distância vertical observada entre as partículas (há muito espaço vertical para o movimento).

\end{enumerate}

\paragraph{Questão 4} (2,5 pontos)

Uma partícula move-se no plano $XY$ tal que as componentes de sua velocidade são:

$$v_x=4t^3+4t,\qquad v_y=4t$$

No instante inicial, a posição da partícula era ($x_0,y_0$)=(1,2). Determine:

\begin{enumerate}[label=(\alph*)]

\item (1,5) A equação cartesiana da trajetória, $y(x)$.

\item (0,5) O módulo da aceleração tangencial no instante $t=1$ s.

\item (0,5) O módulo da aceleração centripeta no instante $t=1$ s.

\end{enumerate}
\newpage
\subsubsection{2015 \original{https://drive.google.com/file/d/1HRjfX3TmF9oLu2CmCn0ySnKPe9LkM5U7/view?usp=sharing}}

\paragraph{$1^a$ Questão}:

Um jogador de golfe arremessa uma bola com velocidade $v_0$ com um ângulo $\theta$ acima do solo que é horizontal. Considere que a resistência do ar
pode ser desprezada. 
\begin{enumerate}[label=\alph*)]
\item Determine a menor velocidade $v_0$ (mínima) para a qual a bola irá ultrapassar uma parede vertical de altura $h$, afastada uma distância $d$ da posição inicial do lançamento.
\item Considere agora: $g=10$m/s$^2$, $d=20$m, $h=15$m e $\theta = 45^o$. Determine o valor de $v_0$ (a velocidade inicial mínima) considerando os valores dados.
\item Determine, na situação do item b), se a bola ultrapassa a parede em movimento ascendente, descendente, ou quando está na altura máxima. Justifique.
\end{enumerate}

\paragraph{$2^a$ Questão}:

Considere uma bolinha de massa $m$ forçada a mover-se sobre o eixo $x$ e sujeita a uma força $F=-kx$, onde $k$ é uma constante positiva. A massa é largada a partir do repouso em $x=x_0$ no instante $t=t_0$.
\begin{enumerate}[label=\alph*)]
\item Determine a velocidade $v$ da massa como uma função da posição $x$.
\item Determine a posição $x$ em função do tempo $t$.
\end{enumerate}
\paragraph{$3^a$ Questão}:

Determine o trabalho realizado pela força $F$ definida abaixo, ao longo dos lados do quadrado cujos vértices são $(0,0),(1,0),(1,1),(0,1)$, percorridos no sentido anti-horário.

$$ \overrightarrow{F}= (5-xy-y^2)\overrightarrow{i}-(2xy-x^2)\overrightarrow{j}$$

\paragraph{$4^a$ Questão}:

Considere novamente um lançamento oblíquo com velocidade inicial $v=v_0$ e ângulo $\theta$. Agora, leve em conta que a força de resistência do ar é dada por:

$$ \overrightarrow{F}=-b\overrightarrow{v} \textrm{ (força de resistência do ar} $$

\begin{enumerate}[label=\alph*)]
\item Determine as componentes da velocidade do projétil $v_x$ e $v_y$, em função do tempo.
\item Determine a equação da trajetória do projétil, $y(x)$.
\end{enumerate}
\newpage

\subsubsection{2003 \original{https://drive.google.com/file/d/1DHYVV8VAxas7WYdZjn7KfkgFQ6bv0rQq/view?usp=sharing}}

\paragraph{$1^a$ Questão:} (2,0 pontos)

Um satélite está a uma distância $r_1$ do centro da Terra, no ponto $A$ cujas coordenadas são $(r_1,\theta)$. O satélite
pode ir de $A$ até o ponto $B$, que tem coordenadas $(r_2,\theta)$, seguindo dois caminhos diferentes:
\begin{enumerate}[label=\roman*)]
\item Direto de $A$ para $B$ seguindo a direção radial;
\item Seguindo de $A$ para $C$ paralelamente ao eixo $X$ e depois de $C$ para $B$ paralelamente ao eixo $Y$.  
\end{enumerate}
Qual é o trabalho, em cada caminho, da força gravitacional $\overrightarrow{F}=-\dfrac{k}{r^2}\overrightarrow{u}_r$, onde 
$k$ é constante, $r$ é a distância entre o satélite e o centro da Terra e $\overrightarrow{u}_r$ o vetor unitário na direção 
do vetor posição do satélite em relação ao referencial $XY$, cuja origem está no centro da Terra?

Observação: o triângulo $ABC$ está no plano $XY$.

\begin{figure}[ht]
\centering
\includegraphics[width=22cm]{fis1vc2003q1.png}
\end{figure}

\paragraph{$2^a$ Questão:} (2,0 pontos)

Uma partícula de massa $m$ se movimenta no plano segundo as relações:


$
\begin{cases}
r=2b \ \sen (\omega t)\\
\theta = wt \\
\end{cases}
$, onde $b$ e $\omega$ são constantes.
\vspace{0.5cm}
Determine:
\begin{enumerate}[label=\alph*)]
\item A velocidade e a aceleração da partícula num instante qualquer;
\item O momento angular e depois verifique a relação existente entre o módulo deste e a taxa de variação da coordenada $\theta$ com o tempo
\item Verifique que $\overrightarrow{\tau} = \dfrac{d\overrightarrow{L}}{dt} $
\end{enumerate}
\newpage

\subsubsection{1999 \original{https://drive.google.com/file/d/1k_gY1BbPEzPreiuaKeLreQ0r2AXyRHKL/view?usp=sharing}}

\paragraph{Questão 1:} (1,5 pontos)

Uma partícula move-se em linha reta com aceleração variável dada por:
$$ a = -0,3v^2 $$

Se em $t=0$ temos $x=0$ e $v_0=55$m/s, determine:
\begin{enumerate}[label=\alph*)]
\item A velocidade em função da posição;
\item O tempo necessário para a partícula alcançar a posição $x=10$m
\end{enumerate}

\paragraph{Questão 2:} (1,5 pontos)

Uma das forças que atuam sobre certa partícula depende da sua posição no plano $xy$. Esta força, expressa em Newtons, é dada por
$$ \overrightarrow{F}=xy \textrm{û}_x + xy \textrm{û}_y$$

Calcule o trabalho realizado por esta força quando a partícula se move do ponto $O$ ao ponto $C$ da figura, ao longo:

\begin{enumerate}[label=\alph*)]
\item do trajeto $OAC$, que consiste de 2 retas;
\item do trajeto $OBC$, que consiste de 2 retas;
\item da reta $OC$.
\item Esta força é conservativa? Justifique sua resposta.
\end{enumerate}

\begin{figure}[ht]
\centering
\includegraphics[width=9cm]{fis1vc1999q2.png}
\end{figure}

\paragraph{Questão 3:} (2,0 pontos)

O movimento de um corpo de massa $M$ é definido pelo vetor posição:

$$ \overrightarrow{r}= (R \ \sen \omega t)\textrm{û}_x +(R \ \cos\omega t) \textrm{û}_y, $$
onde $R$ e $\omega$ são constantes. Determine:

\begin{enumerate}[label=\alph*)]
\item a força e o torque relativo à origem;
\item a variação do momento linear e do momento angular entre $t=0$ e $t=2$s.
\item a variação da energia cinética entre $t=0$ e $t=2$s. Analise sua resposta.
\end{enumerate}

\paragraph{Questão 4:} (2,5 pontos)
\begin{enumerate}[label=\alph*)]
\item Estabeleça a relação entre o momento angular de um sistema de $2$ partículas relativo ao $CM$ e o momento angular relativo ao referencial laboratório;
\item A relação obtida no item a pode ser extendida para um sistema de $n$ partículas. Faça tal extensão.
\item O sistema de $n$ partículas do item b pode ser um corpo rígido. Considere uma bola que está rolando sem deslizar. 
Indique, na expressão obtida no item anterior, os termos correspondentes à translação e a rotação e, tomando o termo do momento angular relativo ao centro de massa, chegue que
$$ \overrightarrow{L}=I\overrightarrow{w},$$
onde $I$ é o momento de inércia e $w$ é a velocidade angular do corpo.
\end{enumerate}

\paragraph{Questão 5:} (2,5 pontos)

O sistema abaixo parte do repouso em $t=0$, estando o corpo de massa igual a $2$kg à $1$m do chão. 
Sabendo que a polia tem $M=4$kg e raio igual a $0,2$m, determine:

\begin{enumerate}[label=\alph*)]
\item A aceleração angular da polia;
\item A energia cinética e a energia potencial do corpo, no começo e quando atinge o chão.
\item Verifique se há conservação da energia mecânica do sistema polia+corpo.
\end{enumerate}

Dado: $I_{\textrm{polia}}=\frac{1}{2}MR^2$

\begin{figure}[ht]
\centering
\includegraphics[width=15cm]{fis1vc1999q5.png}
\end{figure}

\newpage
\subsection{VE 2}

\subsubsection{2008 \original{https://drive.google.com/open?id=1G6WmRYX6Ozo_Ib6cQp9Km8oPaZ9l8lRp}}

\begin{enumerate}
\item (2,0) Um disco de massa $M$ e raio $R$ cai sujeito ao efeito da gravidade. Além da queda, o disco está girando em torno do CM com velocidade angular constante $w$. Determine:

\begin{enumerate}[label=\alph*)]
\item o torque da gravidade relativo ao ponto $P$ da figura; 
\item o momento angular do corpo relativo ao ponto $P$ num instante qualquer, supondo que a velocidade inicial do CM é $v_0$;
\item qual é a velocidade do CM para que o momento angular relativo à $P$ seja nulo.
\end{enumerate}

Dado: $I=\dfrac{1}{2}MR^2$

\begin{figure}[ht]
\centering
\includegraphics[width=15cm]{fis1ve22008q1.png}
\end{figure}


\item (2,0) Dois prótons, inicialmente ($x=\infty$), movem-se em sentidos opostos com o módulo da velocidade igual a $V$, em relação a um referencial inercial $O$. Eles 
colidem frontalmente. A energia potencial elétrica varia com a distância $X$ entre as partículas da forma $V(X)=\dfrac{2,32\times 10^{-28}}{X}$J, sendo $X$ dado em metros. Determine:


\begin{enumerate}[label=\alph*)]
\item a energia cinética antes de interagirem ($x=\infty$) medida em um referencial inercial em que um dos prótons está parado antes de interagir (referencial $P$);
\item a energia cinética no referencial $P$ quando a distância entre os prótons é mínima;
\item as distâncias mínimas entre as partículas resolvendo o problema nos referenciais $O$ e $P$.
\end{enumerate}

Dado: $m_p=1,67\times 10^{-27}$kg

\item (2,0) Em um modelo unidimensional de um átomo vibrando em uma molécula, a função energia potencial é dada por:
$$U(x)=\dfrac{1}{2}\alpha x^4-\dfrac{1}{2}kx^2\textrm{ , onde } \alpha \textrm{ e } k \textrm{ são constantes.}$$

Determine a força $F(x)$. Faça um gráfico de $U(x)$, identifique os pontos de equilíbrio e analise as possíveis regiões de movimento.

\item (2,0) Um região do espaço encontra-se sob um campo de forças resultantes tal que $F(x,y,z)=-yz \textrm{û}_x - xz \textrm{û}_y - xy \textrm{û}_z$. Determine
para uma partícula de massa $m$ localizada no ponto de coordenadas $(x,x,0)$:

\begin{enumerate}[label=\alph*)]
\item o torque $\tau$ em relação ao ponto $O$, de origem do sistema cartesiano de coordenadas $(x,y,z)$.
\item a aceleração angular $\alpha$ da massa $m$ em relação ao ponto $O$.
\end{enumerate}

\item (2,0) Sobre uma barra rígida de massa desprezível e comprimento $L$, encontra-se uma partícula de massa $m$. No instante inicial $t_0$ a barra é abandonada
girando em um plano horizontal com velocidade angular $\omega_0$ em torno de uma das extremidades, tendo a massa $m$ na outra extremidade dirigindo-se com velocidade constante
$v$ na direção da extremidade em torno da qual a barra está girando.

\begin{enumerate}[label=\alph*)]
\item Determine a velocidade angular $\omega$ da barra em torno do eixo de rotação em função de $L$, $\omega_0$ e $r$, onde 
r é a distância do eixo de rotação à posição da massa $m$ sobre a barra em cada instante;
\item Determine a variação de energia cinética da massa $m$ entre a posição inicial e a posição da letra a;
\item Verifique que o torque $\tau=\dfrac{dL}{dt}$, onde $L=I\omega$ em qualquer posição da massa $m$ sobre a barra.
\end{enumerate}

\end{enumerate}

\newpage

\subsubsection{2006 \original{https://drive.google.com/file/d/17yHvLz6wr_5k-LVKXSLpFrtEx-3I3wvr/view?usp=sharing}}

\paragraph{$1^a$ Questão:} (2,5)

Um sistema formado pelas massas $m_1=3$kg e $m_2=2$kg tem as posições, velocidades e acelerações a seguir no instante inicial $t=0$.\\
$r_1=-5 u_x$m; $r_2=10 u_y$m; $v_1=10 u_y$m/s; $v_2=-5 u_x$m/s; $a_1=3 u_y$m/s$^2$ e $a_2=-8 u_y$m/s$^2$.

Determine:
\begin{enumerate}[label=\alph*)]
\item (0,5) A posição do centro de massa em um instante $t>0$;
\item (0,5) O momento de inércia em relação ao centro de massa do sistema em $t=0$;
\item (0,5) O momento angular em relação ao centro de massa em $t=0$;
\item (0,5) O torque das forças externas em relação ao referencial fixo $xy$ em $t=0$;
\item (0,5) A energia cinética no referencial do centro de massa em um instante $t>0$.
\end{enumerate}

\paragraph{$2^a$ Questão:} (2,5)

Uma roldana conforme a figura abaixo, tem massa $m=4M$ e raio $R$. Está presa por uma corda ideal às massas $m_1=M$ e $m_2=2M$.
A aceleração da gravidade é $g$. O coeficiente de atrito entre $m_1$ e o plano horizontal sobre o qual ela desliza é $\mu$.

Determine:

\begin{enumerate}[label=\alph*)]
\item (1,0) O momento de inércia do disco em relação a seu centro de massa.
\item (1,5) A aceleração da masas $m_2$.
\end{enumerate}

\begin{figure}[ht]
\centering
\includegraphics[width=8cm]{fis1ve22006q2.png}
\end{figure}

\paragraph{$3^a$ Questão:} (2,5)

Duas partículas de massa $m$ estão presas as extremidades de uma mola de massa desprezível, inicialmente com seu comprimento
relaxa $l_0$. A mola é esticada até o dobro desse comprimento e é solta depois de se comunicar velocidades iguais e opostas $(v_0,-v_0)$
às partículas, perpendiculares à direção da mola (figura). Seja $k$ a constante elástica da mola $(k=\dfrac{6mv^2_0)}{l^2_0})$,
calcule o vetor velocidade das partículas quando a mola volta a passar pelo seu comprimento relaxado.

\begin{figure}[ht]
\centering
\includegraphics[width=12cm]{fis1ve22006q3.png}
\end{figure}

\paragraph{$4^a$ Questão:} (2,5)

Calcule a magnitude da força horizontal $F$ que é preciso aplicara, em direção ao eixo $O$, para conseguir que um tambor cilíndrico, de massa $M$
e raio $R$, suba um degrau de altura $d<R$ (figura).

\begin{figure}[ht]
\centering
\includegraphics[width=12cm]{fis1ve22006q4.png}
\end{figure}

\newpage

\subsubsection{2003 \original{https://drive.google.com/open?id=122rY8tzBP2B_I6_SMNqJgFQrQlNd2R1E}}

\paragraph{1ª Questão} (2,5 pontos)\\
A energia potencial de uma partícula num campo conservativo unidimensional é dada pela expressão:$$U(x) = -2x^2 - 0,2x^3 + 0,5x^4$$onde $U(x)$ é dado em Joules.
\begin{tasks}(1)
\task Faça o gráfico de $U(x)$ no intervalo $-3 \le x \le 3$ e identifique os pontos de equilíbrio da partícula;
\task Quais são os movimentos possíveis da partícula quando a energia total é igual a -2J e quando é igual a +2J?
\end{tasks}

\paragraph{2ª Questão} (2,5 pontos)\\
Uma bolinha de massa $m_1$ desloca-se com velocidade $v$ em direção a 3 outras, alinhadas com ela, que possuem a mesma massa $m_2$ e estão inicialmente em repouso. Supondo $m_1 < m_2$ e que todas as colisões entre as partículas são elásticas, determine:
\begin{tasks}
\task as velocidades finais das bolinhas;
\task o que acontece se $m_1 \ge m_2$
\end{tasks}

\paragraph{3ª Questão} (2,5 pontos)\\
Um corpo de massa $m_1$ e velocidade $\overrightarrow{v_1}$ colide com outro corpo parado de massa $m_2$, ambos observados por um referencial $O$ inercial. Os dois se ligam formando um corpo de massa $m_1 + m_2$. Determine:
\begin{tasks}
\task as velocidades das partículas, após a colisão, relativas ao referencial inercial e ao referencial CM;
\task a energia cinética das partículas relativas ao referencial O e ao referencial CM;
\task A variação da energia cinética no referencial O e no referencial CM
\end{tasks}

\paragraph{4ª Questão} (2,5 pontos)\\
Definindo o Momento Linear $\overrightarrow{P}$ e o Momento Angular $\overrightarrow{L}$, de um sistema composto de $n$ partículas do sistema em um referencial inercial, mostre que $\overrightarrow{P} = 0$ e $\dfrac{\dd \overrightarrow{L}}{\dd t} = \overrightarrow{\tau _{Ext}}$, onde $\overrightarrow{P}$ é o Momento Linear do sistema medido no referencial do seu Centro de Massa e $\overrightarrow{\tau _{Ext}}$ é o Torque Resultante de todos os torques que têm origem em interações das partículas do sistema com partículas externas ao sistema (torques de origem externa). Defina por extenso todas as grandezas que forem necessárias à sua demonstração, deixando claro o referencial utilizado.
\newpage
\subsubsection{2003 - Lista de Exercícios \original{https://drive.google.com/open?id=1OqYX1h_cbI4DNim9YMLSBSbIpLETelhT}}

\paragraph{$1^a$ Questão:} (1,0 ponto)

Exercício $10.30$ do Alonso $\&$ Finn.

\paragraph{$2^a$ Questão:} (1,0 ponto)

Exercício $10.32$ do Alonso $\&$ Finn.

\paragraph{$3^a$ Questão:} (1,0 ponto)

Um sólido homogêneo de massa específica $\rho$ é gerado pela revolução, em torno do eixo $y$, da curva $y=-ax^2+b \ (a>0 \ \textrm{e} \ b>0)$, com $y \geq 0$. Obtenha
a massa deste sólido, a coordenada $y$ do seu centro de massa e seu raio de giração em torno do eixo $y$.

\paragraph{$4^a$ Questão:} (1,0 ponto)

Um corpo cilíndrico homogêneo de massa $m$, momento de inércia $I$ e raio $R$ repousa, com seu eixo horizontal, sobre uma prancha de massa $M$, também em repouso sobre uma mesa. A partir
do instante $t=0$, uma força $F$, horizontal e constante, passa a ser aplicada à prancha que adquire velocidade em direção perpendicular ao eixo do cilindro. Desprezando o atrito entre a prancha e a mesa e sabendo que o cilindro 
rola sobre a prancha sem deslizar, obtenha a velocidade do centro de massa do cilindro em função do tempo, em relação a mesa, tomada como referencial inercial.

\newpage

\subsubsection{2002 \original{https://drive.google.com/open?id=1tepTtUD4Wowc0J89mcTBJcg7TE7NjEm_}}

\paragraph{1) (2,0)} Uma partícula cuja massa vale 1,0 kg move-se ao longo do eixo x submetida a uma força do tipo $$F=5x - 3x^3$$
\begin{tasks}
\task Calcule a energia potencial V(x) da partícula, supondo que V(0) = 0. Faça um gráfico de V(x).
\task Suponha que a partícula tem velocidade v=5,0 m/s quando se encontra na posição x= -2,0 m. Qual a velocidade máxima atingida pela partícula e quais os pontos de retorno da partícula?
\end{tasks}

\paragraph{2) (1,0)} Exercício 6.11 do livro de Moyses Nussenzveig

\paragraph{3) (1,5)} Considere 1 partícula de massa m e velocidade $v_1$, quando observada pelo referencial laboratório. Ela colide elasticamente com uma prtícula idêntica que está em repouso. A partícula 1 sai da colisão fazendo um ângulo $\theta _1$ com a direção de incidência. Determine em relação ao referencial CM:
\begin{tasks}
\task as velocidades das duas partículas;
\task a energia cinética total;
\task as velocidades finais das partículas, usando a conservação da energia cinética;
\task os ângulos que as partículas saem.
\end{tasks}

\paragraph{4)} (1,5)\\
\begin{tasks}
\task Determine a expressão do momento linear de uma partícula de massa $m_1 = 2g$ e velocidade $V$, em função do seu ângulo de desvio $\theta _1$, após a colisão elástica com outra de massa $m_2 = 3g$, que estava inicialmente parada.
\task Suponha que a observação da partícula 1 do item anterior, ocorreu em $\theta _1 = 30$°, qual o módulo do momento final dela?
\end{tasks}

\newpage
\subsubsection{1998 \original{https://drive.google.com/open?id=1d8g4f80ocnCaqNcMROtdYcwdZtZz8GYj}}

\paragraph{1)} Uma partícula de massa 2 move-se em um campo de força variável no tempo e dado por $$\overrightarrow{F} = 24t^2 \,\hat{u}_x + (36t-16)\hat{u}_y - 12t \,\hat{u} _z$$ Supondo que em $t = 0$, a partícula esteja localizada em $\vec{r}_0 = 3 \hat{u}_x - \hat{u}_y + 4\hat{u}_z$ e que tenha velocidade $\vec{v}_0 = 6\hat{u}_x + 15\hat{u}_y - 8\hat{u}_z$, determine:
\begin{tasks}
\task a velocidade e a posição no tempo $t$;
\task o momento angular e o torque, em relação a origem, para a partícula no instante $t$.
\end{tasks}

\paragraph{2)} O dispositivo da figura gira em torno do eixo verticual com a velocidade angular $w$.

\imgh{fisicave21998q2}{4}

\begin{tasks}
\task Faça um esquema que mostre todas as forças que agem na bolinha;
\task Escreva a equação de movimento;
\task Qual deve ser o valor de $w$ para que o fio de comprimento $L$ faça um ângulo $\theta$ com a vertical?
\task Qual a tensão no fio nessa situação?
\end{tasks}



\paragraph{3)} Um corpo de massa $m$ cai através de um fluido. Além de sofrer a interação gravitacional, ele está sujeito a uma força proporcional ao quadrado da velocidade, isto é $fa = -kv^2$. Determine:
\begin{tasks}
\task a velocidade em função da distância percorrida, sabendo que a velocidade inicial é $b$;
\task a velocidade limite do corpo.
\end{tasks}

\newpage
\subsubsection{1998 - Prova 2  \original{https://drive.google.com/open?id=1V3W83_Yd0qJoUgovMUCDWM9U61O-a8lL}}


\paragraph{$1^a$ Questão:}

Uma partícula está descrevendo uma trajetória circular de raio $R$. Ela está inicialmente com velocidade $v_0$ quando é então
submetida a uma aceleração angular constante $\alpha$. Determine, para qualquer tempo $t$, os vetores: posição, velocidade e 
aceleração em coordenadas polares.

Observações:
\begin{enumerate}[label=\arabic*)]
\item Mostre todas as passagens envolvendo os vetores unitários;
\item Coloque sua resposta em função dos dados.
\end{enumerate}

\paragraph{$2^a$ Questão:}

Um bloco de massa $m$ move-se ao longo do eixo $x$ de acordo com a expressão $x=3\cos{2t}$. Considerando o tempo inicial $t=0$, determine:

\begin{enumerate}[label=\alph*)]
\item a posição e a velocidade iniciais;
\item a expressão da força que atua no bloco em função da sua posição;
\item o trabalho realizado pela força para ir da posição $x=3$ até $x=0$.
\item a quantidade de movimento do bloco quando passa por $x=0$ na primeira vez;
\item a variação da quantidade de movimento do bloco quando passa por $x=0$ pela segunda vez e pela primeira vez.
\end{enumerate}

\paragraph{$3^a$ Questão:}

Um fio de comprimento $L$ é preso a uma bola de massa $m$ e a um pino tendo um raio $a$. Se à bola é dada uma velocidade inicial
$\overrightarrow{v}_1$, perpendicular ao fio, determine:
\begin{enumerate}[label=\alph*)]
\item a velocidade da bola imediatamente apóes ela ter enrolado o fio quatro vezes em volta do pino;
\item o momento angular inicial e o depois do fio ter enrolado as 4 vezes. Analise seu resultado.
\end{enumerate}

\begin{figure}[ht]
\centering
\includegraphics[width=14cm]{fis1ve21998v2q3.png}
\end{figure}

\paragraph{$4^a$ Questão:}

Um bloco de peso $20$N percorre a pista $ABCDE$, situada em plano vertical. Desprezando o atrito e sabendo que o trecho $CDE$ é circular de $2$m de raio
e que o bloco parte do repouso em $A$($4$m acima do chão), determine:

\begin{enumerate}[label=\alph*)]
\item o diagrama das forças que atuam no corpo na posição $C$ e os seus módulos;
\item a quantidade de trabalho realizado por cada força para ir de $A$ até $C$; 
\item Se modificarmos a pista por uma superfície em que o atrito não possa ser desprezadao, a velocidade do bloco em $C$ 
seria $2/3$ do caso anterior. Qual seria o trabalho realizado por cada força, nesse caso?
\end{enumerate}

\begin{figure}[ht]
\centering
\includegraphics[width=14cm]{fis1ve21998v2q4.png}
\end{figure}


\newpage
\subsubsection{1995 \original{https://drive.google.com/open?id=1TLNTeFZ-ns7_3GVZ2P8ItPora6HQqNL6}}

\paragraph{1ª Questão:} Um sistema é composto de 2 partículas $m_1$ e $m_2$. A partícula $m_2$ permanece parada durante toda nossa observação na origem do sistema laboratório, enquanto $m_1$ que está inicialmente no ponto $\vec{r}_0(1 \hat{u}_x)m$ com $\vec{V}_{10} = (1\hat{u}_x)\, m/s$ sofre influência de uma força externa $\vec{F} = 2 \hat{u}_y$. Determine:
\begin{tasks}
\task O movimento do CM;
\task A velocidade das 2 partículas e o momento linear total relativos ao CM;
\task Qual o momento angular total relativo ao CM?
\end{tasks}
\vspace{0.7cm}
\paragraph{2ª Questão:}
\begin{tasks}
\task Estabeleça a relação entre a energia cinética de um sistema de 2 partículas relativa ao CM e a energia cinética relativa ao referencial laboratório;
\task A relação obtida em $a$ pode ser generalizada para um sistema de $n$ partículas. Faça tal generalização.
\task Considere, no item anterior, que as partículas formam um corpo rígido, uma bola, onde o momento de inércia é $2MR^2/5$, onde $R$ é o raio e $M$ a massa. Como fica a relação da energia cinética nos 2 referenciais?
\task Se a bola inicialmente rola sem deslizar sobre uma superfície horizontal com velocidade do CM igual a 20m/s e depois sobre um plano inclinado de 30°, qual a distância que ela percorreu ao subir o plano?
\end{tasks}
\newpage
\paragraph{3ª Questão:} Calcule a aceleração do sistema abaixo. COnsidere $R$ o raio da polia, $m$ a sua massa e que ela está girando. As massas dos corpos são: $m_1$ e $m_2$. Dado $I = mr^2/2$

\img{fisicave21995q3}{4}{ht}

\newpage

\subsubsection{Prova desconhecida 1  \original{https://drive.google.com/open?id=12EiWTRicXqmxhjXmdcgk7hYXJgmdMiVY}}

\paragraph{$1^a$ Questão:} 

Duas partículas de massa $1$ e $2$ kg movem-se sob a influência de um campo de força tal que seus vetores posição relativamente a um sistema de coordenadas fixo são respectivamente:

$$
\begin{cases}
\overrightarrow{r}_1=t^2\textrm{û}_y + \textrm{û}_z+2\textrm{û}_x \\
\overrightarrow{r}_2=2t\textrm{û}_x+t^2\textrm{û}_z. \\
\end{cases}
$$

Ache o momento angular total e o torque externo total:

\begin{enumerate}[label=\alph*)]
\item relativo ao CM;
\item verifique a relação $\dfrac{d\overrightarrow{L}_{CM}}{dt}=\overrightarrow{\tau}_{CM}$
\end{enumerate}

\paragraph{$2^a$ Questão:}

Determine o trabalho realizado por $1$ mol de um gás ideal quando o volume é triplicado mantendo a temperatura constante $T$. Calcule também o calor absorvido e a variação de energia interna.

\paragraph{$3^a$ Questão:}

Um fluido estacionário de densidade $\rho$ escola num tubo estreito. Numa seção transversal $A_1$ a velocidade e a pressão são $(v_1, P_1)$. Determine:

\begin{enumerate}[label=\alph*)]
\item a partir das equações de conservação da massa (continuidade) e de Bernoulli, numa outra região que tem seção transversal $A_2$, a velocidade e a pressão;
\item a quantidade de fluido que atravessa uma seção em $t$ segundos.
\end{enumerate}

\paragraph{$4^a$ Questão:}

\begin{enumerate}[label=\alph*)]
\item Demonstre que a energia cinética para um corpo rígido que rola sem deslizar pode ser escrita como:
$$E_k=\dfrac{1}{2}MV^2_{CM} + \dfrac{1}{2}I\omega^2,\textrm{ onde } I=\int R^2 dm;$$
\item Se o corpo rígido é uma esfera de massa $M$ e raio $R$, que parte do repouso do alto de um plano inclinado de altura $H$, qual o momento
linear total da esfera quando atinge o ponto mais baixo do plano inclinado? Dado: $I=\dfrac{2}{5}MR^2$
\end{enumerate}

\newpage

\subsubsection{Prova desconhecida 2 \original{https://drive.google.com/open?id=1IT7Nj9lLxmaPS4_iWg03e14R_fMooi5l}}

\begin{enumerate}
\item (2,5) Um tanque de água sofreu um tiro em sua parte posterior enquanto subia uma rampa onde o ângulo de inclinação $\alpha$
era tal que $\textrm{tg}(\alpha)=3/4$, o que provocou uma perda de $50$kg de água por segundo com uma velocidade de escape
de $20$m/s. O coeficiente de atrito cinético entre o caminhão e a rampa é de $0,8$. Se no instante inicial a velocidade é de 
$20$m/s e a massa $3000$kg, determine em função do tempo a aceleração e velocidade do caminhão.

\item (2,5) Um veículo de $1500$kg desloca-se para leste a $8$m/s e colide com outro de $500$kg que se deslocava para o norte a $18$m/s. 
Se imediatamente após o choque o veículo mais leve permanece em repouso, determine:
\begin{enumerate}[label=\alph*)]
\item A quantidade de energia cinética total dissipada na colisão;
\item Se o coeficiente de atrito cinético no local é $0,4$, qual a distância entre os veículos quando ambos estiverem parados?
\end{enumerate}

\item (2,5) Duas partículas, de massas $m_1=2$kg e $m_2=4$kg, interagem entre si com uma força de módulo constante igual a $2$N. A 
partícula $m_2$, além de possuir a interação com $m_1$, está sujeita a uma força $\overrightarrow{F}=3t\textrm{û}_y$.
Considerando as posições iniciais das partículas $\overrightarrow{r}_1(0)=\textrm{û}_x$ e $\overrightarrow{r}_2(0)=\textrm{û}_y$, 
e suas velocidades iniciais iguais a zero, relativamente à um observador inercial, determine:
\begin{enumerate}[label=\alph*)]
\item as coordenadas do CM como função do tempo;
\item a quantidade de movimento total como função do tempo quando visto pelo observador inercial e quando visto pelo observador 
no referencial CM.
\item Mostre, literalmente, que o torque total do sistema é apenas devido à força enterna. 
\end{enumerate}

\item (2,5) Uma bola de massa $m$ está em movimento numa curva circular de raio $R$. Além das forças peso e normal, existe 
uma força tangente ao círculo e de módulo constante, que faz um ângulo de $180^o$ com o deslocamento.

\begin{enumerate}[label=\alph*)]
\item Qual o trabalho das forças, quando a bola vai do ponto mais baixo ao ponto mais alto da trajetória no sentido anti-horário?
\item Se a velocidade da bolinha no ponto mais baixo é $v_0$, qual a velocidade no ponto mais alto?
\end{enumerate}

\end{enumerate}

\newpage

\subsubsection{Prova desconhecida 3 \original{https://drive.google.com/open?id=1mLobKXBUcGsyGsXz5k11rD2WTcIdK63o}}

\begin{enumerate}

\item Considere o movimento de uma partícula definida pelas relações $r=k \sen{\omega t}$ e $\theta=\omega t$, onde 
k e $\omega$ são constantes. Determine:

\begin{enumerate}[label=\alph*)]
\item o vetor velocidade e o vetor aceleração da partícula num instante qualquer;
\item o raio de curvatura da sua trajetória;
\item qual é a traketória que a partícula realiza.
\end{enumerate}

\vspace{1.2cm}

\item O movimento de uma partícula sobre uma superfície de um cilíndro circular reto, veja a figura, é definida pelas relações $r=A$, $\theta=2\pi t$ e
$z=B\sen{2\pi nt}$, onde $A$ e $B$ são constantes e $n$ é um inteiro. Determine os módulos da velocidade e da aceleração da partícula num instante qualquer.

\begin{figure}[ht]
\centering
\includegraphics[width=14cm]{fis1ve2descq2.png}
\end{figure}
\newpage
\item Um disco com raio $R$ gira com velocidade angular constate $\omega_2$ em relação ao braço $ABC$, que gira com velocidade angular constante $\omega_1$ em 
torno do eixo $x$. Determine:

\begin{enumerate}[label=\alph*)]
\item o vetor velocidade e o vetor aceleração do ponto $D$ da borda do disco;
\item a aceleração de Coriolis do ponto $D$;

\end{enumerate}

\begin{figure}[ht]
\centering
\includegraphics[width=14cm]{fis1ve2descq3.png}
\end{figure}
 
 \vspace{1.2cm}
 
\item Considere dois elétrons saindo de uma amostra radioativa com a mesma direção e sentidos opostos. A velocidade de cada elétron 
em relação à amostra é $0,67$c. Qual é a velocidade de um elétron em relação ao outro?

\end{enumerate}

\newpage
\subsection{VF}
\subsubsection{2017 \original{https://drive.google.com/file/d/1UEfbFkB3rR6FdH4rFbYCLeYstSFjfIuX/view?usp=sharing}}


\paragraph{Questão 1:} (1,7 pontos) 

Considere um projétil disparado com velocidade inicial $v_0$ em um campo gravitacional constante $g$ e força de arrasto $f = -bv$.\\
(a) (0,9 ponto) Calcule o tempo necessário para que o projétil alcance sua altura máxima. \\
(b) (0,8 ponto) Use a aproximação $ln(1+x) \cong x-x^2/2$ e encontre uma relação entre o resultado da parte (a) e o caso sem força de arrasto. Escreva os resultadaos em termos de $ \tau = m/b $ e $v_{lim} = mg/b$.

\paragraph{Questão 2:} (1,6 pontos) 

A velocidade de uma partícula de massa $m$ varia com a distância $x$ conforme $v(x) = \alpha x^{-n}$ tal que $\alpha > 0, n>0$. Considerando que no instante inicial $t_0 = 0$ a posição era $x_0 = 0$ e a velocidade $v_0 = 0$, calcule:\\
(a) (0,8 ponto) a posição em função do tempo $x(t)$. \\
(b) (0,8 ponto) a força $F(t)$ responsável pelo movimento em termos do tempo . 

\paragraph{Questão 3:} (1,7 pontos) 

Considere uma haste homogênea de comprimento $L$ e massa $M$, pendurada verticalmente em um de seus extremos. Um projétil de massa $m$ e velocidade $v$ atraveessa ortogonalmente o eixo extremo livre da haste, emergindo com velocidade $v/4$ e mantendo a mesma direção de incidência. Qual deve ser o valor de $v$ para que o pêndulo gire de uma volta completa? Observação: Despreze as perdas e a mudança na distribuição de massa tanto no projétil quanto na barra.\\

\paragraph{Questão 4:} (1,7 pontos) 

Considre uma máquina de Atwood em que a polia tem massa de $10kg$ e raio de $0,5m$ , inicialmente conectada a 2 blocos de massas iguais a $4kg$, na mesma altura e em repouso. Sabendo que um bloco é de gelo e começa a derreter na razão de $0,01kg/s$, perdendo massa na velocidade relativa ao bloco de $0,02m/s$, determine:\\
(a) (0,9 ponto) a aceleração linear do sistema;\\
(b) (0,8 ponto) a tensões no fio.

\paragraph{Questão 5:} (1,6 pontos) 

Uma partícula de massa $m$ move com velocidade $v_0$ em um círuclo de raio $r_0$ sobre uma mesa sem atrito. A partícula está presa a um círculo que passa por um furo através da mesa. O fio é lentamente puxado para baixo até que a partícula esteja a uma distância $r_f$ do furo e assim passa a se mover num círuclo de raio $r_f$. Qual o trabalho realizado sobre a partícula pela força de tensão?\\

\paragraph{Questão 6:} (1,7 pontos) 

Um disco horizontal uniforme de massa $M$ e raio $R$, gira em torno do eixo vertical que passa pelo seu centro de massa co velocidade angular $\omega$. O disco, girando, é largado sobre uma mesa horizontal com coeficiente de atrito cinético $\mu_c$ entre a mesa e o disco. Quanto tempo o disco leva para parar?\\

\newpage

\subsubsection{2011 \original{https://drive.google.com/file/d/1lGgUuQHGgyjriMwmD4zSKD_-Q-E9rIwp/view?usp=sharing}}

\paragraph{Questão 1:} (2,0 pontos)

A velocidade em função do tempo de uma dada partícula é dada pela expressão: $v_x=(10-5t)$m/s

Usando as técnicas do Cálculo:

\begin{enumerate}[label=(\alph*)]
\item Obtenha a expressao de $x$ contra $t$ (1 ponto);

\item Determine o instante de tempo $t$ e a correspondente posição $x$ no ponto de retorno (0,5 pontos);


\item Determine o instante em que a partícula chega a origem. Dado $x(t=0\text s)=30$ m (0,5 pontos).
\end{enumerate}

\paragraph{Questão 2:} (2,0 pontos)

Uma mola de constante elástica 2000 N/m é colocada entre um bloco de 1,0 Kg e outro de 2,0 Kg. Ambos os blocos estão sobre uma mesa sem atrito. Os blocos são empurrados um contra o outro, comprimindo a mola em 10 cm, e depois são liberados. Qual o valor da velocidade com que os blocos se afastam um do outro?

\paragraph{Questão 3:} (2,0 pontos)

Um disco homogeneo de massa $M$ e raio $R$ é abandonado sobre uma superfície plana horizontal, iniciando um rolamento sem deslizar, conforme a figura abaixo. A uma distância $\dfrac R 2$, um fio na borda do disco suspende uma massa $m_2$. Outro fio enrolado no disco a uma distância $\dfrac R 3$ está ligado a uma massa $m_1$ que encontra-se sobre uma superfície horizontal com coeficiente de atrito cinético $\mu$. Determine a aceleração da massa $m_2$ imediatamente após o início do movimento.
\begin{figure}[ht]
\centering
\includegraphics[width=8cm]{fis1vf2011q3.png}
\end{figure}


\paragraph{Questão 4:} (2,0 pontos)

Um corpo de massa $m$ encontra-se na posição $\vec{r}=x\vec{u_x}+x\vec{u_y}+x\vec{u_z}$, e sob energia potencial $Ep(x,y,z)=x\,y\,z$. Determine:

\begin{enumerate}[label=(\alph*)]
\item O vetor aceleração da massa $m$ se a força $\vec F$ conservativa associada a energia potencial $Ep$ for a força resultante (1 ponto);
\item O torque resultante sobre a massa $m$ em relação à origemdo sistema de coordenadas cartesianas, referência para a posição da massa $m$ (1 ponto).
\end{enumerate}

\paragraph{Questão 5:} (2,0 pontos)

Uma caixa vazia de massa $m$ encontra-se sobre o solo horizontal inicialmente em repouso. A caixa não possui tampa. Em um instante inicial $t_0=0$, começa a atuar na caixa uma força constante paralela ao chão de intensidade $F$. Nesse mesmo instante $t_0=0$, começa a chover. Os pingos de chuva caem na vertical. O coeficiente de atrito estático entre a caixa e o solo é igual ao cinético e vale $\mu$. O volume de água na caixa aumenta a uma razão constante e igual a $\dfrac{\lambda}{\rho}$, onde $\rho$ é a densidade da água. Considere conhecida a aceleração da gravidade $g$.

Admita que: $F>\mu \,m\,g$

\begin{figure}[ht]
\centering
\includegraphics[width=12cm]{fis1vf2011q5.png}
\end{figure}

Determine:
\begin{enumerate}[label=(\alph*)]
\item A expressão da velocidade em função do tempo (0,5 pontos);
\item o instante em que o bloco para e o valor da força de atrito nesse instante (0,5 pontos);
\item O instante em que a velocidade do bloco é máxima (0,5 pontos);
\item O valor da máxima velocidade do bloco (0,5 pontos).
\end{enumerate}

\newpage

\subsubsection{2010 \original{https://drive.google.com/file/d/1mAu-s7dG1emF7mEX3oMb9BTTRj0xdvb0/view?usp=sharing}}

\paragraph{Questão 1:}(2,0 pontos)\\
Um cubo sólido de lado \textbf{d} e densidade \textbf{a} flutua em um líquido de densidade \textbf{2a},  mantendo uma profundidade \textbf{h} submersa. Ao ser afundado de uma pequena profundidade \textbf{x} e abandonado, passa a oscilar em movimento vertical em MHS.  Determine o período (P) do movimento do cubo em função do lado \textbf{d},  e da aceleração da gravidade \textbf{g}. 

\begin{figure}[h]
\centering
\includegraphics[width=8cm]{fis1vf2010q1.png}
\end{figure}

\paragraph{Questão 2:}(1,5 pontos - 0,5 cada)\\
Um disco de raio R e massa M está sob uma força tangencial F de módulo constante,  conforme a figura abaixo. Ele é abandonado do repouso no instante inicial,  quando passa a girar em torno de um eixo que atravessa seu centro,  perpendicular a sua seção reta cilíndrica.  Determine para o disco em função do tempo:

\begin{tasks}(1)
\task a velocidade angular;
\task o momento angular;
\task a energia cinética;
\end{tasks}

\begin{figure}[h]
\centering
\includegraphics[width=8cm]{fis1vf2010q2.png}
\end{figure}

\paragraph{Questão 3:}(1,0 ponto)\\
Um aro de raio $\textbf{r}$ e massa $\textbf{m}$ parte do repouso do ponto mais alto de um semi-cilindro de raio $\textbf{R}$, que está fico no solo, e rola sem deslizar sobre a superfície do mesmo, conforme mostra a figura abaixo. Determine o ângulo que a linha que une os centros forma com a vertical no momento em que o aro perde contato com o semi-cilindro.\\
\\
Dado:\\
$I_{ARO}=mr^2$

\begin{figure}[h]
\centering
\includegraphics[width=8cm]{fis1vf2010q3.png}
\end{figure}


\paragraph{Questão 4:}(2,0 pontos)\\
Uma partícula de massa $\textbf{m}$ é abandonada a uma altura $\textbf{h}$ a partir do solo num local onde o módulo da aceleração da gravidade vale $\textbf{g}$. Além da força peso, atua sobre a partícula uma força de resistência proporcional ao quadrado da velocidade, ou seja, $F_R=-kv^2\,$, onde $\textbf{k}$ é uma constante de dimensões apropriadas e é conhecida.

\begin{tasks}(1)
\task $\textbf{(0,5)}$ Encontre uma equação diferencial envolvendo somenta a velocidade e o tempo;
\task $\textbf{(0,5)}$ Reescreva a equação diferencial acima separando as variáveis v e t;
\task $\textbf{(1,0)}$ Encontre o tempo em função da velocidade, ou seja, $t=t(v)$.
\end{tasks}

\paragraph{Questão 5:}(1,5 pontos)\\
Uma partícula de massa $2$ kg se move em um campo de força cuja dependência temporal (t) é dada por $\textbf{F}=(24t^2)\textbf{i}+(36t-16)\textbf{j}+(12t)\textbf{k}$, onde $\textbf{i}$, $\textbf{j}$ e $\textbf{k}$ são os versores (vetores unitários) ao longo das direções cartesianas. Assumindo que em t=0 a partícula está localizada em $\textbf{r}_0=3\textbf{i}-\textbf{j}+4\textbf{k}$, e tem velocidade $\textbf{v}_0=6\textbf{i}+15\textbf{j}-8\textbf{k}$, determine em qualquer instante t para a partícula:

\begin{tasks}(1)
\task $\textbf{(1,0)}$ a velocidade;
\task $\textbf{(0,5)}$ a posição.
\end{tasks}

\paragraph{Questão 6:}(2,0 pontos - 1,0 cada)\\
Seja um corpo que cai sob ação da gravidade após ter sido solto ($\textbf{v}_R=0$, para todo $t=0$, onde o índice \textbf{R} representa o referencial girante do problema) do ponto $(x_R^0,0,0)$ perto da superfície da Terra no Equador - veja figura. A origem do sistema $x_R$, $y_R$, $z_R$ está no centro da Terra. O eixo de rotação da Terra é o eixo $z$.

\begin{tasks}(1)
\task Calcule a coordenada $y_R$ na qual o corpo atinge a Terra, indicando qual é a direção (leste, oeste, norte e sul).
\task Indique se este deslocamento é difícil ou não de observação por meio de uma estimativa numérica. Utilize as aproximações adequadas ao problema sem perguntar ao professor. Use $g$ como módulo da aceleração devido à gravidade. Rotação da Terra $w=0,7\,x\,10^{-4}$ em unidades de $s^{-1}$. Apenas respostas devidamente justificadas serão aceitas.
\end{tasks}

\begin{figure}[ht]
\centering
\includegraphics[width=8cm]{fis1vf2010q6.png}
\end{figure}

\newpage
\subsubsection{1996 \original{https://drive.google.com/file/d/1wSj5zppODdEwHm49wiTNySOY2bTN64kp/view?usp=sharing}}

\paragraph{Questão 1:}
Um corpo de massa $\textbf{m}$ tem no seu movimento descrito por:
\\
$x=bt$\\
$y=t^2+c$\\
onde $b$ e $c$ são constantes. Determine as forças tangencial e normal em função das quantidades conhecidas.

\paragraph{Questão 2:}
A posição de um corpo, de massa $\textbf{M}$, relativa a um sistema de coordenadas $O$ é dado por:
$$\overrightarrow{r}=(t^2+3t)\,\widehat{U}_x+(t+4)\,\widehat{U}_y+(3t^2+5t)\,\widehat{U}_z$$
Determine:

\begin{tasks}(1)
\task A velocidade e a aceleração relativa de um sistema $\textbf{O'}$ em relação a $\textbf{O}$, se a posição do corpo para $\textbf{O'}$ é:\\
$$\overrightarrow{r}=(t^2+2)\,\widehat{U}_x+4\,\widehat{U}_y+(3t^2)\,\widehat{U}_z$$
\task O momento da força que atua no corpo, visto no referencial $\textbf{O}$;
\task A quantidade de movimento e o momento angular relativo ao referencial $\textbf{O}$;
\task Verifique, para este problema, que    $\overrightarrow{F}=\dfrac{\mathrm{d}\overrightarrow{P} }{\mathrm{d} t}$ e $=\dfrac{\mathrm{d}\overrightarrow{L} }{\mathrm{d} t}$
\end{tasks}

\paragraph{Questão 3:}
A energia potencial de uma partícula de massa 4 Kg é dada por $E_p=\dfrac{-200}{r}J$. Sabendo que em coordenadas polares o operador gradiente é dado por:   $\bigtriangledown =\dfrac{\partial }{\partial r}\widehat{U}_r+\dfrac{1}{r}\, \dfrac{\partial }{\partial \theta}\widehat{U}_\theta$\\
\\
Determine:\\

\begin{tasks}(1)
\task O campo de força a que a Partícula está submetida.
\task Se a partícula está em $r_0=1\,m$ com velocidade de $20m/s$, qual será a velocidade em $r=2m$?
\end{tasks}

\paragraph{Questão 4:}
Uma esfera maciça e uniforme rola sem deslizar sobre uma superfície horizontal com velocidade do centro de massa igual a $20m/s$. A esfera sobe então um plano inclinado, cuja inclinação é de $30$º com a horizontal. Supondo que a esfera rola o tempo todo sem deslizar, calcule:

\begin{tasks}(1)
\task A distância que a esfera percorre ao subir o plano;
\task O tempo decorrido até a esfera retorne à base do plano.
\end{tasks}

Dado: Momento de inércia da esfera$=\dfrac{2}{5}MR^2$, onde $R$ é o raio e $M$ a massa.

\paragraph{Questão 5:}
Um projétil de massa igual à $20g$ com velocidade de $10m/s$ penetra num corpo de massa igual à $180g$, que está pendurado por um fio de comprimento igual à $40cm$. O choque é totalmente inelástico e o sistema formado executa um MHS. Determine:

\begin{tasks}(1)
\task O período de oscilação;
\task A expressão do ângulo em função do tempo, explicitando a amplitude máxima;
\task A expressão da energia cinética e da energia potencial em função do tempo e à partir delas, mostre que a energia mecânica é constante.
\end{tasks}

\newpage
\subsubsection{1994 \original{https://drive.google.com/file/d/1PGEVNaCj3pDNdEnDzYftrAatgyTZyyOB/view?usp=sharing}}

\paragraph{$1^a$ Questão}:

Um corpo de $1$kg está inicialmente na posição $\overrightarrow{r}_0= (1\  \textrm{û}_y)$m com $\overrightarrow{v}_0=(2 \ \textrm{û}_y)$m/s. Ele então
é sujeito a uma força $\overrightarrow{F}= (t \ \textrm{û}_x)$N. Determine após 2 segundos:

\begin{enumerate}[label=\alph*)]
\item A variação da quantidade de movimento e a variação do momento angular.
\item O trabalho realizado pela força $\overrightarrow{F}$ para ir da posição inicial até a posição final.
\end{enumerate} 

\paragraph{$2^a$ Questão}:

A força de interação entre $2$ cargas $q_1$ e $q_2$ é dada por $\overrightarrow{F}=-\dfrac{1}{4\pi \epsilon} \dfrac{q_1 q_2}{r^2}\hat{\mu}_1$, onde $\epsilon$ é um fator constante devido ao meio que as cargas estão e $r$ a distância entre elas.

\begin{enumerate}[label=\alph*)]
\item Por que podemos chamar esta força de conservativa?
\item Determine a energia potencial associada a esta força.
\end{enumerate}

\paragraph{$3^a$ Questão}:

Obtenha a relação entre a energia cinética de um sistema de partículas medida no referencial laboratório e a energia cinética relativa ao centro de massa.
\vspace{0.1cm}

\paragraph{$4^a$ Questão}:

Determine o trabalho realizado por $1$ mol de um gás ideal quando seu volume é duplicado a uma temperatura constante $T$. 
Calcule também a variação da sua energia interna e o calor absorvido.
\vspace{0.1cm}

\paragraph{$5^a$ Questão}:

Calcule a aceleração do sistema abaixo, considere $R$ o raio da polia, $m$ a sua massa e que as massas dos corpos são $m_1$ e $m_2$.

\begin{figure}[ht]
\centering
\includegraphics[width=10cm]{fis1vf1994q5.png}
\end{figure}

\paragraph{$6^a$ Questão}:

\begin{enumerate}[label=\alph*)]
\item Determine a equação da trajetória do movimento resultante de $2$ MHS perpendiculares cujas equações são:
$$ 
\begin{cases}
x= 2A \sen (\omega t) \\
y=A \sen (\omega t + \pi) \\
\end{cases}
$$
e mostre que o movimento resultante é $1$ MHS.
\item Determine a energia mecânica do sistema em um instante qualquer t.
\end{enumerate}

\newpage
\subsubsection{1993 \original{https://drive.google.com/file/d/1fJB1r5D1oCB8G3Jm4nvZbKr295wHFdf0/view?usp=sharing}}

\paragraph{$1^a$ Questão:} (1,7 pontos)

O movimento de um corpo é descrito por
$$
\begin{cases}
x= b_1 t \\
y=t^2 \\
\end{cases}
$$
sendo $b_1$ constante. Determine:
\begin{enumerate}[label=\alph*)]
\item (0,3) O vetor velocidade num instante qualquer;
\item (1,4) As acelerações tangencial e normal.
\end{enumerate}

\paragraph{$2^a$ Questão:} (2,0 pontos)

Um corpo é lançado verticalmente para cima com velocidade $v_0$. Sendo $m$ a massa e sabendo que a resistência do ar ao movimento
do corpo é dada por $\overrightarrow{F}=-b\overrightarrow{v}$, onde $b$ é constante e $\overrightarrow{v}$ a velocidade, determine:
\begin{enumerate}[label=\alph*)]
\item (0,5) A equação de movimento;
\item (1,0) O tempo decorrido desde o lançamento até a altura máxima.
\end{enumerate}

\paragraph{$3^a$ Questão:} (2,0 pontos)

O movimento de um objeto é descrito por $\overrightarrow{x}=x_0 \cos (\omega t) \textrm{û}_x$, onde $x_0$ e $\omega$ são constantes. Determine:

\begin{enumerate}[label=\alph*)]
\item (0,6) A força e o momento da força que atua no corpo;
\item (0,6) O trabalho para ir da posição $x_0$ até $x_1$. E o trabalho para sair da $x_0$ e voltar a este mesmo ponto, qual é?
\item (0,8) É uma força central conservativa? Justifique e caso afirmativo dê a energia potencial associada.
\end{enumerate}

\paragraph{$4^a$ Questão:} (2,0 pontos)

Duas partículas $m_1$ e $m_2$ de velocidade $\overrightarrow{v}_1$ e $\overrightarrow{v}_2$ e posições $\overrightarrow{r}_1$ e
$\overrightarrow{r}_2$, respectivamente, formam um sistema isolaldo. Determine em relação ao CM:

\begin{enumerate}[label=\alph*)]
\item (0,8) A velocidade das partículas e momento linear total;
\item (0,6) O momento angular de cada partícula;
\item (0,6) Mostre que o momento angular total é o mesmo que de uma partícula de quantidade de movimento $\mu \overrightarrow{v}_{12}$ e vetor posição
$\overrightarrow{r}_{12}$, onde $\mu$ é a massa reduzida, $\overrightarrow{v}_{12}=\overrightarrow{v}_1-\overrightarrow{v}_2$ e $\overrightarrow{r}_{12}=\overrightarrow{r}_1-\overrightarrow{r}_2$.
\end{enumerate}

\paragraph{$5^a$ Questão:} (1,0 pontos)

Duas partículas iguais de massa $M$ estão montadas em hastes (de massa desprezível) ligadas a um mancal que gira com velocidade angular $\omega$, conforme figura.

\begin{enumerate}[label=\alph*)]
\item (0,5) Determine o momento angular total;
\item Mostre que a relação $\overrightarrow{L}=I\overrightarrow{w}$ é válida.
\end{enumerate}
\begin{figure}[ht]
\centering
\includegraphics[width=12cm]{fis1vf1993q5.png}
\end{figure}
\paragraph{$6^a$ Questão:} (1,0 pontos)

Suponha que a operação de uma máquina obedeça o diagrama PV abaixo. Sendo um gás ideal, calcule:

\begin{enumerate}[label=\alph*)]
\item (0,2) O volume máximo alcançado;
\item (0,8) A quantidade de calor trocado no processo DE;
\item (0,8) O trabalho total realizado.
\end{enumerate}
\begin{figure}[ht]
\centering
\includegraphics[width=12cm]{fis1vf1993q6.png}
\end{figure}
\newpage

\section{Física Experimental I}
Matéria introduzida em $\textbf{2017}$

\subsection{VC }
\subsubsection{2017 \original{https://drive.google.com/file/d/1HzrrIy45dWkC2uAnbkOWLmMVkTN1mn3W/view?usp=sharing}}



\paragraph{Questão 1:}  (1,6 pontos)

Observando os resultados de quatro atiradores abaixo  (a,b,c e d), preencha a tabela abaixo com sim ou não.

\begin{figure}[h]
\centering
\includegraphics[width=20cm]{fisexp1vc2017q1.png}
\end{figure}


\paragraph{Questão 2:} (2,4 pontos)

Consideremos duas medidas $ x = \langle x\rangle \pm \Delta x$  e  $ y = \langle y\rangle \pm \Delta y$. Utilizamos nas primeiras práticas a seguinte estimativa para o erro da soma $S = x + y : \Delta S = \Delta x + \Delta y$.
Por outro lado é  possível mostrar, com fundamentos estatísticos, que se as medições em x e y forem independentes e sujeitas apenas a incertezas aleatórias, a incerteza será dada por $\Delta S = \sqrt{(\Delta x)^2 +( \Delta y)^2} $. Utilizando apenas as linhas abaixo, explique sucintamente por que a expressão
$\Delta S = \Delta x + \Delta y$  superestima o erro.
\\

\rule{13cm}{0.4pt}

\rule{13cm}{0.4pt}

\rule{13cm}{0.4pt}

\rule{13cm}{0.4pt}

\paragraph{Questão 3:} (3,0 pontos)

Em determinado fenômeno físico a grandeza $A$ tem a seguinte dependência
temporal:
$$ A=A_0 e^{\dfrac{-t}{\tau}} $$
onde $A_0$, o valor inicial de $A$, e $\tau$ , uma constante de tempo característica do sistema, podem ser determinadas experimentalmente. Fez-se a regressão linear para um conjunto de pontos $\ln A$ $vs$ $t$.
O valor numérico do coeficiente angular obtido foi $m = -0, 200 \pm 0, 004$ e do coeficiente linear foi $b = 12, 3 \pm 0, 2$. Quais os valores de $A_0$, dado em $uA$ (unidade de A) e $\tau$ (dado em segundos) ?

\paragraph{Questão 4:} (3,0 pontos)

Uma esfera de raio $ r = 0, 2cm$ é constituída de um material de densidade
$\rho = 7, 2g/cm^3$. A esfera estava imersa em glicerina, a qual tem densidade $\rho_g = 1, 26g/cm^3$. A esfera partiu do repouso em movimento vertical de descida sujeita à força peso, ao empuxo e à força de atrito do fluido (apenas o termo linear, o termo quadrático é considerado desprezível).
Assuma $g = 980cm/s$.

\begin{enumerate}[label=(\alph*)]
\item  (0,5) Obtenha a velocidade da esfera em função do tempo. Mostre detalhadamente todos os
cálculos.
\item (0,5) Expresse o coeficiente de viscosidade da glicerina $\eta_g$ em termos de $\rho, \rho_g, r , g$ e a velocidade limite $V_l$. Mostre detalhadamente todos os cálculos.
\item (2,0) O experimento foi repetido muitas vezes e comparando os intervalos de tempo cronometrados em determinados trechos foi possível calcular (com tratamento estatístico de dados e propagação de erros) a velocidade limite: $V_l$ = $(6, 3 \pm 0, 2) cm/s$. Determine, propagando erros, o coeficiente de viscosidade.
\end{enumerate}
\newpage
\subsection{VF}
\subsubsection{2017 \original{https://drive.google.com/file/d/1Pp4yiMlwnpvGkK0ZCBaUQ2dwlppfBSf_/view?usp=sharing}}

Em relação aos experimentos de demonstração realizados no laboratório com plataforma giratória e roda de bicicleta, considere as 3 configurações abaixo e responda as questões $1$ e $2$.

Configuração 1) O voluntário sobre a plataforma giratória segura a roda de bicicleta em
giro com eixo na horizontal.

Configuração 2) O voluntário segura a roda em giro acima de sua cabeça com eixo na
vertical.

Configuração 3) O voluntário segura com as mãos para baixo a roda em giro com eixo na
vertical.

\paragraph{Questão 1:} (2,0 pontos)

O voluntário levanta a roda, passando da configuração 1 para 2, e a
plataforma atinge uma velocidade angular constante $\overrightarrow{\omega}$. Responda verdadeiro (V) ou falso(F).

( ) O momento angular do sistema na direção vertical se conserva.

( ) Torques internos atuaram no sistema.

\paragraph{Questão 2:} (3,0 pontos)

Após levantar a roda, passando da configuração 1 para 2, e atingir a
velocidade angular $\overrightarrow{\omega}$ da Questão $2$, o voluntário coloca a roda para baixo, passando à configuração
3, e passa a rotacionar com velocidade angular constante $\overrightarrow{\omega_f}$ . Há alguma relação entre $\overrightarrow{\omega_f}$ e $\overrightarrow{\omega}$?
Essa relação é exata ou aproximada? Explique utilizando apenas as linhas abaixo:
\\

\rule{13cm}{0.4pt}

\rule{13cm}{0.4pt}

\rule{13cm}{0.4pt}

\rule{13cm}{0.4pt}

\rule{13cm}{0.4pt}

\paragraph{Questão 3:} (3,0 pontos)

Consideremos um volante de massa M, em rotação com velocidade angular $\overrightarrow{\omega}$. No instante de tempo inicial o eixo do volante encontra-se no plano $yz$ fazendo um ângulo $\theta \pm \Delta\theta$ com a vertical. A distância do centro de massa do
volante ao suporte em que ele está apoiado é $l$, a gravidade é $g$ e o momento de inércia do volante
em relação ao seu eixo é $I$. Obtenha o vetor posição do centro de massa em função do tempo. Mostre detalhadamente todos os cálculos.
\\\\
\paragraph{Questão 4:} (2,0 pontos)

Consideremos um giroscópio montado utilizando a suspensão tipo Cardan (veja figura), que permite rotações livres em torno dos três eixos ortogonais $Ox,Oy, Oz,$ facultando ao sistema assumir qualquer orientação no espaço. Inicialmente colocamos o giroscópio na posição normal, mostrada na figura, com o volante em rotação rápida em torno de $Ox$. O eixo do volante está preso ao anel interno, que está inicialmente no plano horizontal $Oxy$. O anel interno está suspenso do anel externo, de forma a poder livremente em torno de $Oy$. Finalmente, o anel externo está encaixado no suporte $S$, de forma a poder girar livremente em torno de $Oz$, e está inicialmente no plano $Oyz$. Responda verdadeiro (V) ou falso (F):

\newpage

( \ \ \  ) Se pressionarmos o anel interno para baixo, exercendo um força vertical $F_v$ no ponto $A$
ocorre uma rotação do anel externo em torno de $Oz$.

\begin{figure}[h]
\centering
\includegraphics[width=20cm]{fisexp1vf2017q4.png}
\end{figure}


\newpage
\section{Química I}

\subsection{VE 1}

\subsubsection{2019 \original{https://drive.google.com/open?id=1TrwYScvgXoinEYdlh0igZyHAyiWV9k_E}}

\textbf{\large{Em cada questão, assinale a opção correta e justifique sua resposta no espaço logo abaixo do conjunto de opções. Somente serão consideradas opções assinaladas à caneta azul ou preta, bem como suas respectivas justificativas.}}


\paragraph{1ª Questão:} (2,0 pontos)



Se a Equação dos Gases Ideias reflete um princípio de conservação de energia, é correto afirmar que a EOS para líquidos expressa por 
\begin{equation*}
    V_{m}(T, p) = c_{1} + c_{2}T + c_{3} T^{2} - c_{4} p - c_{5}pT    
\end{equation*}

\begin{enumerate} [label = (\alph*)]
    \item descreve o volume ocupado por uma fase condensada em função da temperatura;
    \item descreve o volume ocupado por uma fase condensada em função da temperatura e pressão;
    \item é baseada no princípio de conservação de energia e descreve a relação entre funções termodinâmicas de estado;
    \item correlaciona variáveis termodinâmicas e prevê uma relação entre temperatura de fusão e ebulição; ou 
    \item associa constantes determinadas experimentalmente com o comportamento de fases.
\end{enumerate}

\newpage

\paragraph{2ª Questão:} (2,0 pontos)


A forma funcional $T = T \left(\langle \epsilon_{tr} \rangle\right)$, pode ser corretamente interpretada como
\begin{enumerate} [label = (\alph*)]
    \item uma relação entre a energia translacional média expressa em termos de temperatura;
    \item a definição de temperatura como uma função de energia translacional média;
    \item uma expressão de proporcionalidade entre a temperatura e energia translacional;
    \item uma relação entre a temperatura média de um gás e sua energia translacional; ou
    \item uma forma derivada do princípio de conservação de energia. 
\end{enumerate}



\paragraph{3ª Questão:} (2,0 pontos)


Sobre as Leis de difusão e efusão de Graham, as quais relacionam 
fluxo de partículas de um gás com a sua massa (molar ou molecular), 
é correto afirmar que 
\begin{enumerate}[label = (\alph*)]
    \item somente podem ser obtidas empiricamente, a partir de séries de experimentos com gases de massas diferentes;
    \item uma vez obtidas empiricamente ou a partir de TCG, apresentam formas matematicamente incompatíveis;
    \item somente podem ser aplicadas a sistemas cujo comportamento é ideal; 
    \item relacionam fluxo, coeficiente de difusão e massa molar dos gases, bem como suas velocidades médias; ou
    \item assumem que os gases são indistinguíveis entre si e os tratam como partículas da mesma natureza química.    
\end{enumerate}

\newpage

\paragraph{4ª Questão:} (2,0 pontos)


Observando a função de distribuição radial entre as distâncias interatômicas em água líquida, percebe-se que a intensidade de $g(r)$ é muito maior para as 
distâncias \chemfig{O-H} que para \chemfig{H-H} e \chemfig{O-O}, apresentando um máximo a $\SI{0,98}{ \angstrom}$. Esse comportamento é uma consequência

\begin{enumerate}[label = (\alph*)]
    \item da repulsão entre átomos do mesmo elemento;
    \item de flutuações de densidade eletrônica induzidas por momentos de dipolo transientes;
    \item da orientação das moléculas sob um campo eletromagnético externo;
    \item da direcionalidade das interações específicas entre moléculas de água no líquido; ou
    \item da auto-ionização da água, a qual gera íons hidrônio nas formas Eigen e Zundel.
    
\end{enumerate}

\paragraph{5ª Questão:} (2,0 pontos)

Sabendo que uma avaliação geométrica da função de distribuição de Maxwell-Boltzmann permite relacionar a taxa das colisões moleculares $\dfrac{dN_{w}}{dt}$ com uma parede de área $A$, por
\begin{equation*}
  \dfrac{1}{A} \dfrac{dN_{w}}{dt} = \dfrac{1}{4} \dfrac{N}{V} \langle v \rangle = \dfrac{1}{4} \dfrac{pN_{a}}{RT} \left( \dfrac{8RT}{\pi M} \right)^{1/2},
\end{equation*} é correto afirmar que:

\begin{enumerate} [label = (\alph*)]
    \item Um maior tempo de observação causa um aumento da taxa de colisões;
    \item Dois gases com a mesma massa molar apresentarão sempre a mesma taxa de colisões;
    \item O sentido físico da equação não é alterado substituindo-se $\langle v \rangle$ por $\langle v^{2} \rangle^{1/2}$ ;
    \item A geometria da parede impactada pelas moléculas de um certo gás não altera a relação entre as variáveis $p$ e $T$; ou
    \item Mantendo-se a taxa de colisões constante, um aumento na temperatura está relacionado inversamente a um aumento de pressão.
    
\end{enumerate}

\newpage


\subsection{VC}
\subsubsection{2017 \original{https://drive.google.com/file/d/1o8bV6XM_a2c-SkpdeQMM-RUzzHDn-P3k/view?usp=sharing}}

\paragraph{$1^a$ Questão} (1,5 ponto)

Dois experimentos independentes são realizados em um dispositivo constituído por dois balões "A" e "B", de $1$ (um) litro cada, ligados por
uma torneira de volume interno desprezível. No primeiro experimento, "A" e "B" são inicialmente evacuados e a torneira é fechada; a seguir, $35,0$ g
de clorofórmio (CHCl$_3$) são admitidos no balão "A" e $0,40$ g de acetona (CH$_3$COCH$_3$) no balão "B". No segundo experimento, "A" e "B" são inicialmente 
evacuados e a torneira é fechada; a seguir, $35,0$ g de clorofórmio (CHCl$_3$) são admitidos no balão "A" e $35,0$ g de acetona (CH$_3$COCH$_3$) no balão "B". Em
ambos os casos, os sistemas encontram-se a uma temperatura de $25^o$C. Responda:

\begin{enumerate}[label=\alph*]
\item No primeiro experimento, mantendo-se a torneira fechada, qual será a pressão em cada balão quando o sistema entrar em equilíbrio?
\item No segundo experimento, após ser aberta a torneira, qual seria a composição final do vapor em cada balão quando o equilíbrio for atingido, partindo-se do pressuposto de que a mistura tenha comportamento ideal?
\item Ainda com relação ao segundo experimento abordado no item anterior (item "b"), é razoável considerar que a mistura obtida tenha comportamento ideal? Em caso negativo, qual o desvio esperado em relação à Lei de Raoult? Haverá 
absorção ou liberação de calor? Por que?
\end{enumerate}

\paragraph{$2^a$ Questão:} (1,5 pontos)

Atenda aos seguintes pedidos:

Obs: Não é necessário apresentar justificativas para os itens i, ii, iii e iv.

\begin{enumerate}[label=\roman*]
\item Relacione todos os tipos de forças atrativas intermoleculares que estão presentes em cada uma das seguintes substâncias: ácido acético, sulfeto de hidrogênio, trióxido de enxofre, metilamina. 
\item Coloque os seguintes compostos em ordem crescente de solubilidade na água: oxigênio (O$_2$), cloreto de lítio (LiCl), bromo (Br$_2$) e metanol (CH$_3$OH).
\item Coloque as seguintes substâncias HCl, BaCl$_2$, Ar, HF e Ne em ordem crescente de pontos de ebulição.
\item Coloque os seguintes compostos em ordem crescente de velocidade molecular média, a $25^o$C: Ne, HBr, SO$_2$, NF$_3$, CO
\item Considerando que o flúor é mais eletronegativo que o oxigênio, explique por que a água líquida tem maior ponto de ebulição do que o fluoretode hidrogênio líquido.
\end{enumerate}

\paragraph{Questão 3} (1,0 ponto) \textbf{Repetida} - $2^a$ questão da VC de 2015

O ponto de fusão do potássio é 63,2 $^o$C. O potássio fundido tem uma pressão de vapor correspontente a 10,00 torr a 443 $^o$C. A 708 $^o$C a pressão de vapor é 400,00 torr. Determine o ponto de ebulição normal do potássio.


\paragraph{$4^a$ Questão:} (2,0 pontos)

Considere uma mistura líquida de acetona e acetonitrila. Admitindo que os componentes da mistura tenham (sob o ponto de vista molecular)
dimensões, forças iterativas e estruturas similares e que a variação de entalpia no processo de dissolução seja muito pequena,
atenda aos pedidos:

\begin{enumerate}[label=\roman*.]
\item Com as informações acima e as disponíveis em \textbf{DADOS COMPLEMENTARES}, esboce o diagrama P vs Composição da Mistura (fração molar),
ou seja, esboce graficamente o comportamento da variação da pressão de vapor da solução em relação à composição da mistura líquida, a $25^o$C.
Apresente no diagrama todas as informações disponíveis e mostre como a pressão de vapor de cada componente é abaixada pela presença do outro.
\item Prepara-se uma determinada quantidade desta mesma mistura líquida, sendo que a fração molar de acetona é igual a $0,405$. A seguir, coloca-se
a solução em um cilindro provido de êmbolo, a $45^o$C e $400$ mmHg (estado "A"). Em um processo a pressão constante, o sistema é aquecido lentamente, de modo
a manter-se o equilíbrio. Quando a temperatura atinge $50^o$C (estado "B") observa-se a primeira bolha de vapor e aos $55^o$C (estado "C"), observa-se a última gota 
de líquido do sistema. O aquecimento prossegue até que a temperatura chegue aos $60^o$C (estado "D"). Com os dados disponíveis:
\begin{enumerate}
\item Esboce um diagrama T ($^o$C) vs Composição da Mistura (fração molar). Neste diagrama, indique os estados A, B, C e D.
\item Determine a fração molar de acetona em cada fase dos estados A, B, C e D, inclusive da primeira bolha de vapor e da última gota de líquido.
\item Caracterize a fase (vapor saturado, líquido comprimido, líquido sub-resfriado, líquido super-resfriado, etc) em que se encontra o sistema, conforme os estados termodinâmicos caracterizados por cada etapa marcante do processo (A, B, C e D).
\end{enumerate}
\end{enumerate}

\paragraph{$5^a$ Questão:} (1,0 ponto) \textbf{Repetida} - $1^a$ questão da VC de 2015
 
 O espalhamento de odores pelo ar ocorre devido à difusão de moléculas de gás. Uma pessoa abriu um frasco contendo octanato de etila na estremidade norte de uma sala de 5,0 metros e, simultaneamente, outra pessoa abriu um frasco contendo p-anisaldeido na extremidade sul da sala (isto é, a 5,0 m de distância do outro). O octanoato de etila (C$_{10}$H$_{20}$O$_2$) tem odor semelhante ao de frutas, enquanto o p-anisaldeido (C$_8$H$_8$O$_2$) tem odor semelhante ao hortelã. A que distância (em metros) da extremidade norte da sala deve estar uma pessoa para sentir primeiro o cheiro de hortelã?
 
 \paragraph{$6^a$ Questão:} (0,5 ponto)

Considere uma amostra de $50,0$ g de gás criptônio a $54,85^o$C. Determine a capacidade calorífica da amostra de gás criptônio,
ou seja, a quantidade de energia necessária para elevar a temperatura da amostra em $1^o$C, sabendo-se que esta quantidade de energia
pode ser obtida pela variação da energia cinética das moléculas do gás.
\begin{enumerate}[label=\alph*)]
\item $12,47$ J/K
\item $623,47$ J/K
\item $59,8$ J/K
\item $7,44$ J/K
\item $35,9$ J/K
\end{enumerate}
 
 \paragraph{$7^a$ Questão:} (0,5 ponto)

Dissolvem-se $10,2$g de uma mistura de NaCl e sacarose (C$_{12}$H$_{22}$O$_{11}$) no volume de água suficiente para
preparar $250$mL de solução. A pressão osmótica da solução é $7,32$ atm a $23^o$C. A percentagem em massa de NaCl na mistura é:

\begin{enumerate}[label=\alph*)]
\item $14,2$\%
\item $18,3$\%
\item $31,4$\%
\item $68,6$\%
\item $85,8$\%
\end{enumerate} 
 
 
 \paragraph{$8^a$ Questão:} (0,5 ponto)
 
 A osmose ajuda as células biológicas a manter suas estruturas.  As membranas celulares são semipermeáveis e permitem a passagem de água, de pequenas moléculas e de íons hidratados, mas impedem a passagem dos biopolímeros sintetizados dentro da célula. A diferença de concentração de solutos dentro e fora da célula dá origem a uma pressão osmótica, permitindo a passagem de pequenas moléculas nutrientes, em um processo naturalmente seletivo.
 
 Uma das aplicações mais comuns da osmose é a osmometria, que permite a  medição de massas molares de proteínas e e polímeros sintéticos. Como essas enormes moléculas se dissolvem para produzir soluções que estão longe de serem ideais, podemos supor que a equação de Van't Hoff seja apenas o primeiro termo de uma expansão do tipo
 $$  \pi = [W]RT\{1+B[W]+C[W]^2+\dots\}$$
 onde $[W]$ é a concentração molar do soluto e os parâmetros empíricos B,C,\dots, são o $2^o$ coeficiente virial osmótico, $3^o$ coeficiente virial osmótico, respectivamente, e assim por diante.
 
 A estratégia é similar ao emprego da equação do Virial para gases reais. Considerando-se que os coeficientes do virial tendem a cair sequencialmente, de maneira significativa , utiliza-se, geralmente, apenas até o $2^o$ coeficiente. Uma sugestão para simplificar a resolução da equação acima é reescrevê-la, dividindo-se todos os termos pela concentração do soluto (C).  Neste caso, considera-se $y=\pi/C$ e $x=C$, obtendo-se, graficamente, uma reta. O ponto em que a reta cruza o eixo vertical pode ser obtido por extrapolação dos dados, ou por regressão linear ou até mesmo por programas matemáticos.
 
 Nas condições acima expostas, considere as pressões osmóticas de uma enzima em água, a $298$K, conforme dados a seguir:
 
 \begin{center}
  \begin{tabular}{| l | l | l | l | l | l | }
    \hline
    C(g/dm)$^3$ & 1,00 & 2,00 & 4,00 & 7,00 & 9,00 \\ \hline
    $\pi$ (Pa)  & 27 & 70 & 197 & 500 & 785 \\ \hline
  \end{tabular}
\end{center}
 
 Com base nos dados acima, estima-se que a massa molar da enzima seja próxima de:
 \begin{enumerate}[label=\alph*)]
 \item $99,3$ kg/mol
 \item $96,3$ kg/mol
 \item $57,5$ kg/mol
 \item $993,0$ kg/mol
 \item $127,1$ kg/mol
 \end{enumerate}
 \paragraph{$9^a$ Questão:} (1,0 ponto)

O líquido "A" é miscível em tetraidrofurano e sua pressão de vapor, a $25^o$C, é $201,3$ mmHg. Um sistema constituído por uma mistura
líquida contendo $60$\% de "A" e $40$\% de tetraidrofurano possui, a $25^o$C, pressão de vapor igual a $143,5$ mmHg. Prepara-se, em um frasco,
uma mistura em partes iguais os dois líquidos para submetê-los a uma destilação fracionada. Responda aos questionamentos a seguir:

\begin{enumerate}[label=\Roman*-]
\item Quanto à mistura, pode-se afirmar:
\begin{enumerate}[label=\alph*)]
\item A mistura forma uma solução ideal.
\item Espera-se a formação de azeótropo de máximo na destilação.
\item A mistura apresenta desvios positivos em relação à lei de Raoult.
\item As moléculas da solução encontram-se numa condição de mais alta energia em relação ao estado puro.
\item A solução não pode ser formada nas condições do problema.
\end{enumerate}
\item Esboce um gráfico e analise o desenvolvimento da destilação. Com base nessa análise criteriosa, leia atentamente 
todos os itens a seguir e escolha a resposta mais precisa.
\begin{enumerate}[label=\alph*)]
\item Espera-se obter tetraidrofurano puro no coletor (destilado) e a formação de um azeótropo como produto remanescente no frasco.
\item Espera-se obter o produto "A" puro no coletor (destilado) e a formação de um azeótropo como produto remanescente no frasco.
\item Espera-se obter azeótropo no coletor (destilado) e azeótropo remanescente no frasco.
\item As opções "a", "b" e "c" acima são possíveis.
\item Não há formação de azeótropo e o produto "A" poderá ser obtido no coletor (destilado).
\end{enumerate}
\end{enumerate}
 
 \paragraph{$10^a$ Questão:} (0,5 ponto)

O boro forma uma família de compostos binários com o hidrogênio, chamados boranos. Esses compostos incluem o diborano (D$_2$H$_6$) e
outros compostos mais complexos, como o decaborano (B$_{10}$H$_{14}$), entre outros. Mais de vinte hidretos de boro neutros com a fórmula geral 
B$_x$H$_y$ são conhecidos. O borano mais simples, o BH$_3$, é muito instável, sendo observado, muitas vezes, como um intermediário de vida curta em certas reações.
Como resultado, o BH$_3$ pode se unir a outra molécula BH$_3$ para formar o diborano, por meio de pontes de hidrogênio. O diborano pode, também, ser obtido pela reação do boro-hidreto de sódio com trifluoreto de boro, em solvente orgânico. Assim, é correto afirmar:
\begin{enumerate}[label=\alph*)]
\item O BH$_3$ interage com outra molécula BH$_3$ por meio de forças intermoleculares do tipo dipolo-dipolo para formar o diborano;
\item Os boranos são compostos deficientes de elétrons e a formação do diborano caracteriza uma limitação da regra do octeto e da estrutura de Lewis;
\item Duas moléculas BH$_3$ interagem por meio de ligações de hidrogênio para formar um produto mais estável, o diborano;
\item Devido à instabilidade do BH$_3$, átomos de hidrogênio e boro compartilham elétrons em ligações covalentes, conforme a regra do octeto, gerando, assim, um produto
mais estável, o diborano;
\item O diborano é formado por ligações de três centros, em uma estrutura de ressonância, satisfazendo, portanto, à regra do octeto e podendo ser precisamente representado
de acordo com a estrutura de Lewis.
\end{enumerate}
 \newpage
 \begin{center}
 \textbf{DADOS COMPLEMENTARES}
 \end{center}
 
 \begin{enumerate}[label=\Roman*.]
 \item \textbf{Constante universal dos gases perfeitos (R):} 
 
 \vspace{0.2cm}
 R=$0,0821 \dfrac{\textrm{atm.L}}{\textrm{K.mol}}=8,314 \dfrac{\textrm{J}}{\textrm{K.mol}}=62,3 \dfrac{\textrm{mmHg.L}}{\textrm{K.mol}}$
\vspace{0.3cm} 

\item \textbf{Pressão de Vapor:}

\begin{center}
\begin{tabular}{|l|l|l|l|}
\hline 
Substância & $25^o$C & $50^o$C & $55^o$C \\ \hline
Acetona & $222,0$ & $615,0$  & $733,8$ \\ \hline
Acetonitrilia & $100,0$ & $253,5$ & $307,0$ \\ \hline
Clorofórmio & $195,0$ & - & - \\ \hline
Tetraidrofurano & $159,5$ & - & - \\ \hline
\end{tabular}
\end{center}
 \end{enumerate}
\newpage

\subsubsection{2015 \original{https://drive.google.com/file/d/1pxQy_DI9YyjkB6-nHP5MF1te-C-QRj0H/view?usp=sharing}}

\paragraph{Questão 1} (1,0 pontos)

O espalhamento de odores pelo ar ocorre devido à difusão de moléculas de gás. Uma pessoa abriu um frasco contendo octanato de etila na estremidade norte de uma sala de 5,0 metros e, simultaneamente, outra pessoa abriu um frasco contendo p-anisaldeido na extremidade sul da sala (isto é, a 5,0 m de distância do outro). O octanoato de etila (C$_10$H$_20$O$_2$) tem odor semelhante ao de frutas, enquanto o p-anisaldeido (C$_8$H$_8$O$_2$) tem odor semelhante ao hortelã. A que distância (em metros) da extremidade norte da sala deve estar uma pessoa para sentir primeiro o cheiro de hortelã? Precisão esperada do resultado: aproximadamente $1,0\times 10^{-2}$

\paragraph{Questão 2} (1,0 pontos)

O ponto de fusão do potássio é 63,2 $^o$C. O potássio fundido tem uma pressão de vapor correspontente a 10,00 torr a 443 $^o$C. A 708 $^o$C a pressão de vapor é 400,00 torr. Determine o ponto de ebulição normal do potássio.

\paragraph{Questão 3} (1,0 pontos)

O chefe de um almoxarifado colocou 58,3 L de acetona em um tambor com capacidade de 94,6 L quando a temperatura era 25 $^o$C e a pressão atmosférica 750 mmHg. Após selar devidamente o tambor, um assitente deixou-o cair quando preparava para estocá-lo. O tambor foi danificado e seu volume interno diminui para 77,2 L. Determine a pressão no interior do tambor após o acidente.

\paragraph{Questão 4} (1,5 pontos)

Atenda aos seguintes pedidos:

Obs: Não é necessário apresentar justificativas para os itens i, ii, iii e iv.

\begin{enumerate}[label=\roman*.]
\item Relacione todos os tipos de forças atrativas intermoleculares que estão presentes em cada uma das seguintes substâncias: ácido acético, sulfeto de hidrogênio, trióxido de enxofre, metilamina. 
\item Coloque os seguintes compostos em ordem crescente de solubilidade na água: oxigênio (O$_2$), cloreto de lítio (LiCl), bromo (Br$_2$) e metanol (CH$_3$OH).
\item Coloque as seguintes substâncias HCl, BaCl$_2$, Ar, HF e Ne em ordem crescente de pontos de ebulição.
\item Coloque os seguintes compostos em ordem crescente de velocidade molecular média, a $25^o$C: Ne, HBr, SO$_2$, NF$_3$, CO
\item Considerando que o flúor é mais eletronegativo que o oxigênio, explique por que a água líquida tem maior ponto de ebulição do que o fluoreto de hidrogênio líquido.
\end{enumerate}

\paragraph{Questão 5} (1.0 pontos)

Uma mistura de cloreto de sódio sólido e glicose sólida (C$_{12}$H$_{22}$O$_{11}$) tem composição desconhecida. Quando 15,0 g da mistura são dissolvidos em água, perfazendo 500 mL de solução, constata-se que esta solução exibe uma pressão osmótica de 6,41 atm a 25 $^o$C. Determine a porcentagem em massa de cloreto de sódio na mistura.

\paragraph{Questao 6} (1,2 pontos)

Considere as misturas abaixo e escreva, para cada uma delas, o comportamento esperado em relação à lei de Raoult. Apresente as justificativas no cartão de soluções.

\begin{enumerate}[label=\roman*.]

\item metanol (CH$_3$OH) e etanol (CH$_3$CH$_2$OH)

\item clorofórmio (CHCl$_3$) e acetona (CH$_3$CH$_3$CO)

\item hexano (CH$_6$H$_{14}$) e H$_2$O

\item ciclo-pentano e ciclo-hexano

\end{enumerate}

\paragraph{Questão 7} (1,5 pontos)

A figura apresentada a seguir representa um dispositivo usado para medir velocidades atômicas e moleculares. Suponha que um feixe de átomos metálicos seja dirigido para um cilindro rotativo, instalado em uma camara sob vácuo. Uma pequena fenda no cilindro permite que os átomos se choquem contra um detector. Porém, considerando-se que o cilindro gira, os átomos, que se deslocam com velocidades diferentes, chocam-se com o detector em posições diferentes. Com o tempo, deposita-se uma camada de metal sobre o detector. Os pontos onde os átomos se chocam correspondem ás medidas de distribuição de velocidades. Em uma experiência, descobriu-se que, a 850 $^o$C, os átomos de bismuto (Bi) se chocavam em pontos afastados, em média, a 2,80 cm do centro do detector. Sabendo-se que o diametro do cilindro é de 15,0 cm e que ele gira a 130 rotações por segundo:

\begin{enumerate}[label=\roman*.]

\item Calcule a velocidade média dos átomos de Bi, a partir dos dados experimentais obtidos.

\item Com base na teoria cinético-molecular, calcule a raiz quadrada da velocidade quadrática média do Bi a 850 $^o$C.

\item Deduza (no caderno de soluções) a expressão da velocidade mais provável e determine seu valor para a temperatura de 850 $^o$C.

\end{enumerate}

\begin{figure}[h]
\centering
\includegraphics[width=12cm]{quimicavc2015q7.png}
\end{figure}

\paragraph{Questão 8} (1,0 pontos)

A 25 $^o$C a pressão de vapor do líquido "A" é 360 torr e a do líquido "B" é 300 torr. Para uma solução composta de igual quantidade de matéria das duas substâncias, verifica-se que a pressão de vapor é 250 torr a mesma temperatura. Assinale a única alternativa correta de cada item abaixo:

i.  O que se pode afirmar quanto ao comportamento dessa solução?

\begin{enumerate}[label=\alph*)]

\item A solução comporta-se como uma solução ideal e a variação de entalpia da dissolução é "zero".

\item A solução apresenta desvios positivos em relação à lei de Raoult e o processo de dissolução da mistura é exotérmico.

\item A solução apresenta desvios negativos em relação à lei de Raoult e o processo de dissolução da mistura é exotérmico.

\item A solução apresenta desvios positivos em relação à lei de Raoult e o processo de dissolução da mistura é endotérmico.

\item A solução apresenta desvios negativos em relação à lei de Raoult e o processo de dissolução da mistura é endotérmico.

\end{enumerate}

ii. Quanto à formação de azeotropos das duas substâncias, pode-se afirmar:

\begin{enumerate}[label=\alph*)]

\item Nas condições do problema, a solução constitui um azeotropo de minimo

\item Nas condições do problema, a solução constitui um azeotropo de máximo

\item A mistura das soluções não forma azeotropo

\item Em uma destilação fracionada espera-se obter um azeotropo de minimo

\item Em uma destilação fracionada espera-se obter um azeotropo de máximo

\end{enumerate}

\paragraph{Questão 9} (0,8 pontos)

O líquido "A' é miscivel e sua pressão de vapor, a 25 $^o$C, é 48,96 torr. Uma mistura de "A" em água, à mesma temperatura, tem uma pressão de vapor de 13,82 torr quando sua composição (quantidade de matéria de "A" em relação à de solução) é de 68,4\%. Após preparar uma mistura em partes iguais dos dois líquidos, em um frasco, e submetê-la a uma destilação fracionada, pode-se obter:

\begin{enumerate}[label=\alph*)]
\item água no coletor (destilado) e azeotropo no frasco
\item líquido "A" no coletor (destilado) e água no frasco
\item azeotropo no coletor (destilado) e água no frasco
\item azeotropo no coletor (destilado) e líquido "A" no frasco
\item azeotropos no coletor (destilado) e no frasco
\end{enumerate}

\newpage

\subsubsection{2014 \original{https://drive.google.com/file/d/1l8_Pa1kKV8Q-k-rtxUD8GLd-Z16cAdco/view?usp=sharing}}

\paragraph{1ª Questão} (1,5 pontos)\\
A figura apresentada a seguir representa um dispositivo usado para medir velocidades atômicas e moleculares. Suponha que um feixe de átomos metálicos seja dirigido para um cilindro rotativo, instalado em uma camara sob vácuo. Uma pequena fenda no cilindro permite que os átomos se choquem contra um detector. Porém, considerando-se que o cilindro gira, os átomos, que se deslocam com velocidades diferentes, chocam-se com o detector em posições diferentes. Com o tempo, deposita-se uma camada de metal sobre o detector. Os pontos onde os átomos se chocam correspondem ás medidas de distribuição de velocidades. Em uma experiência, descobriu-se que, a 850 $^o$C, os átomos de bismuto (Bi) se chocavam em pontos afastados, em média, a 2,80 cm do centro do detector. Sabendo-se que o diametro do cilindro é de 15,0 cm e que ele gira a 130 rotações por segundo:

\begin{enumerate}[label=\roman*.]

\item Calcule a velocidade média dos átomos de Bi, a partir dos dados experimentais obtidos.

\item Com base na teoria cinético-molecular, calcule a raiz quadrada da velocidade quadrática média do Bi a 850 $^o$C.

\item Determine o valor da velocidade mais provável para a temperatura de 850 ºC.

\end{enumerate}

\begin{figure}[h]
\centering
\includegraphics[width=12cm]{quimicavc2015q7.png}
\end{figure}

\paragraph{2ª Questão} (0,8 ponto)\\
A pressão de vapor do etanol é 43,9 mmHg a 20,0 ºC e 352,7 mmHg a 60 ºC. Determine o ponto de ebulição normal do etanol.


\paragraph{3ª Questão} (1,2 pontos)\\
Atenda aos seguintes pedidos:

\begin{enumerate}[label=\roman*.]
\item Coloque os seguintes compostos em ordem crescente de solubilidade na água: oxigênio (O$_2$), cloreto de lítio (LiCl), bromo (Br$_2$) e metanol (CH$_3$OH).
\item Coloque as substâncias HCl, BaCl$_2$, Ar, HF e Ne em ordem crescente de pontos de ebulição.
\item Coloque os seguintes gases em ordem crescente de velocidade molecular média, a $300$ K: $CO_2$, $N_2O$, HF, $F_2$, $H_2$
\item O Hidrogênio tem dois isótopos naturais, $^1H$ e $^2 H$. O cloro também tem dois isótopos naturais, $^{35} Cl$ e $^{37} Cl$. Assim, o gás cloreto de hidrogênio consiste em quatro tipos distintos de moléculas. Coloque essas quatro moléculas em ordem crescente de taxa de efusão.
\end{enumerate}

\paragraph{4ª Questão} (0,8 ponto)\\
Dissolvem-se 10,2 g de uma mistura de NaCl e sacarose ($\ch{C12H22O11}$) no volume de água suficiente para preparar 250 ml de solução. A pressão osmótica da solução é 7,32 atm a 23 ºC. Calcule a percentagem em massa de NaCl da mistura.

\paragraph{5ª Questão} (0,8 ponto)\\
Determine a pressão de vapor de uma solução aquosa 0,20 molal de $\ch{Fe(NO3)3}$, a 25 ºC.

\paragraph{6ª Questão} (1,2 pontos)\\
Considere as misturas abaixo e escreva, para cada uma delas, o comportamento esperado em relação à lei de Raoult. 

\begin{enumerate}[label=\roman*.]

\item metanol (CH$_3$OH) e etanol (CH$_3$CH$_2$OH)

\item clorofórmio (CHCl$_3$) e acetona (CH$_3$CH$_3$CO)

\item hexano (CH$_6$H$_{14}$) e H$_2$O

\item ciclo-pentano e ciclo-hexano

\end{enumerate}

\paragraph{7ª Questão} (0,5 pontos)\\
Um gás desconhecido, composto de moléculas diatômicas homonucleares, efunde-se a uma taxa correspondente a 0,355 vezes a taxa do oxigênio ($O_2$) à mesma temperatura. Qual a identidade do gás desconhecido?

\paragraph{8ª Questão} (0,8 pontos)\\
A 25ºC a pressão de vapor do líquido "A" é 360 torr e a do líquido "B" é 300 torr. Para uma solução composta de igual quantidade de matéria das duas substâncias, verifica-se que a pressão de vapor é 250 torr, à mesma temperatura. Assinale a única alternativa correta de cada item abaixo:

\begin{enumerate}[label=\roman*.]
\item O que se pode afirmar quanto ao comportamento dessa solução? 

\begin{tasks}(1)
\task A solução comporta-se como solução ideal e a variação da entalpia de dissolução é "zero";
\task A solução apresenta desvios positivos em reação à lei de Raoult e o processo de dissolução da mistura é exotérmico;
\task A solução apresenta desvios negativos em relação à lei de Raoult e o processo de dissolução da mistura é exotérmico;
\task A solução apresenta desvios positivos em relação à lei de Raoult e o processo de dissolução da mistura é endotérmico;
\task A solução apresenta desvios negativos em relação à lei de Raoult e o processo de dissolução da mistura é endotérmico
\end{tasks}

\item Quanto à formação de azeótropos entre as duas substâncias, pode-se afirmar:
\begin{tasks}(1)
\task Nas condições do problema, a solução constitui um azeótropo de mínimo; 
\task Nas condições do problema, a solução constitui um azeótropo de máximo;
\task A mistura das duas substâncias não forma azeótropo;
\task Em uma destilação fracionada espera-se obter um azeótropo de mínimo;
\task Em uma destilação fracionada espera-se obter um azeótropo de máximo;
\end{tasks}

\end{enumerate}

\paragraph{9ª Questão} (0,8 pontos)\\
O ponto de ebulição normal da mistura dos líquidos "A" e "B" é 75 ºC. Sabe-se que, a 75 ºC, a pressão de vapor do líquido puro "A" é 1,55 atm e a do líquido puro "B" é 0,65 atm. A composição do vapor em equilíbrio com o líquido no ponto de ebulição é dada pelas frações molares de "A" e "B" respectivamente:
\begin{tasks}(1)
\task 0,611 e 0,389
\task 0,389 e 0,611
\task 0,274 e 0,725
\task 0,387 e 0,603
\task 0,603 e 0,397
\end{tasks}

\paragraph{10ª Questão} (0,8 pontos)\\
O líquido "A" é miscivel em água e sua pressão de vapor, a 25 $^o$C, é 48,96 torr. Uma mistura de "A" em água, à mesma temperatura, tem uma pressão de vapor de 7,82 torr quando sua composição é de 68,4\%. Após preparar uma mistura em partes iguais dos dois líquidos, em um frasco, e submetê-la a uma destilação fracionada, pode-se obter:

\begin{enumerate}[label=\alph*)]
\item água no coletor (destilado) e azeótropo no frasco;
\item líquido "A" no coletor (destilado) e azeótropo no frasco;
\item azeótropo no coletor (destilado) e água no frasco;
\item azeótropo no coletor (destilado) e substância "A" no frasco;
\item azeótropos no coletor (destilado) e no frasco.
\end{enumerate}

\paragraph{11ª Questão} (0,8 ponto)\\
A 25ºC, a pressão osmótica de uma solução e glicose ($\ch{C6H12O6}$) em água é 10,50 atm. A massa específica da solução é 1,16 g/mL. O ponto de congelamento dessa solução é:
\begin{tasks}(5)
\task 0,737 ºC; \task 0,798 ºC \task -0,737 ºC \task -0,798 ºC \task -0,688 ºC
\end{tasks}




\newpage

\subsubsection{2013 \original{https://drive.google.com/file/d/1VC0vfRAONUSdPEqis5oB4u4HUN5vIE63/view?usp=sharing}}

\paragraph{$1^a$ Questão:} (1,0 pontos)

Em um experimento, uma amostra de $6,0$ litros de gás amônia, a $27^o\textrm{C}$ e $722 \ \textrm{mmHg}$ , foi mergulhada em $600 \ \textrm{mL}$ de solução $0,30 \textrm{ molar}$  de $\textrm{HCl}$. Admitindo-se que toda a amônia foi dissolvida e que o volume da solução manteve-se constante, determine o $\textrm{pH}$ da solução resultante. 


\paragraph{$2^a$ Questão:} (1,4 pontos)

Atenda aos pedidos abaixo:

\begin{enumerate}[label=(\alph*)]

\item  Coloque os seguintes compostos de acordo com a provável ordem crescente de ponto de ebulição: RbF, CO$_2$, CH$_3$OH, CH$_3$Br. Justifique.  (0,3 ponto)

\item  Explique as diferenças entre as forças que mantêm no estado sólido alguns sais preparados a partir de moléculas orgânicas, tais como o propionato de sódio, C$_2$H$_5$COOK,  e o metanolato de sódio,   CH$_3$ONa, em relação às forças que mantêm outros sais, tais como brometo de potássio e o cloreto de sódio.  Explique, ainda, sucintamente, as diferenças esperadas nas propriedades físicas desses compostos. (0,3 pontos)

\item Justifique:

\begin{enumerate}[label=(\roman*)]

\item  O metanol (CH$_3$OH) entra em ebulição a $65^o \textrm{C}$ , enquanto que o metanotiol (CH$_3$SH) entra em ebulição a $6^o \textrm{C}$; (0,2 ponto)

\item O Xe é líquido à pressão atmosférica e $120$K, enquanto que o Ar é um gás às mesmas condições; (0,2 ponto)

\item O Kr, peso atômico $84$, entra em ebulição a $120,9^o$C enquanto que o Cl$_2$, peso molecular $71$, entra em ebulição a $238$K; (0,2 ponto) 

\item A acetona entra em ebulição a $56^o$C, enquanto que o $2$-metilpropano entra em ebulição a $12^o$C. (0,2 ponto) 

\end{enumerate}

\end{enumerate}

\paragraph{$3^a$ Questão:} (1,6 pontos)

A dureza da água representa um sério problema para muitas atividades industriais. Além de interferir na eficiência de sabões e detergentes, a dureza é responsável pela deposição de sais em membranas e incrustações em caldeiras e tubulações, entre outros inconvenientes. Os principais íons causadores
de dureza são o cálcio e o magnésio, sob a forma de carbonatos, bicarbonatos e sulfatos. Dentro desse contexto, responda ao que se pede:

\begin{enumerate}[label=(\alph*)]
\item Uma empresa propõe abrandar a dureza da água a ser consumida, por meio de um processo que visa remover seletivamente os íos Ca$^{2+}$
pela adição controlada de carbonato de sódio. Sabendo-se que a solubilidade molar do CaCO$_2$ é $9,3 \times 10^{-5}$ M, qual é a sua solubilidade
molar em uma solução $0,050$ M de Na$_2$CO$_3$? O que se conclui diante do resultado? O processo é eficiente? Seria de se esperar esse resultado? Por que? (0,4 ponto)
\item Em um processo independente, os íons Mg$^{2+}$ são removidos como Mg(OH)$_2$ por adição de Ca(OH)$_2$ para produzir uma solução saturada. Qual o pH da solução saturada de 
Ca(OH)$_2$? (0,4 ponto)
\item Qual é a concentração de íons Mg$^{2+}$ para o pH calculado no item anterior? (0,4 ponto)
\item Por que os íons Mg$^{2+}$ não são removidos pelo primeiro processo (adição de carbonato de sódio)? (0,4 ponto)
\end{enumerate}

\paragraph{$4^a$ Questão:}

Uma amostra de $20,0$mL de solução 0,10 M de Ba(NO$_3$)$_2$ é adicionada a $50,0$ mL de solução $0,10$ M de Na$_2$CO$_3$. Haverá 
formação de um precipitado de BaCO$_3$? Justifique com base em cálculos, conceitos e regras de solubilidade.

\paragraph{$5^a$ Questão:} (1,2 pontos)

A $35^o$C a pressão de vapor da acetona, (CH$_3$)$_2$CO, é $360$ torr e a do clorofórmio, CHCl$_3$, é 300 torr. Para uma solução
composta de igual quantidade de matéria de acetona e de clorofórmio, verifica-se que a pressão de vapor é 250 torr, a $35^o$C. Pede-se que:

\begin{enumerate}[label=(\alph*)]
\item O que se pode afirmar quanto ao seu comportamento em relação à solução ideal? Constate, justifique e explique detalhadamente a existência
ou não de um possível desvio em relação ao comportamento ideal. (0,4 ponto)
\item Com base no comportamento da solução, espera-se que a dissolução da mistura de acetona e clorofórmio seja um processo exotérmico ou endotérmico? Justifique. (0,4 ponto)
\item Discuta a possível formação de azeótropo entre as duas substâncias, nas condições do problema ou em outras condições, discriminando o tipo, se for o caso. (0,4 ponto)

\end{enumerate}

\paragraph{$6^a$ Questão:} (1,0 ponto)

Determine a pressão de vapor do metanol, a $25^o$C.
\newpage


\subsubsection{2012 \original{https://drive.google.com/file/d/109v3NYZGyJRv-mjXadDJ39AyZmtd75M_/view?usp=sharing}}

\underline{\textbf{Recomendação Inicial:}}

\vspace{0.5cm}
A prova é composta de 2 (duas) partes, cada uma contendo 3 (três) questões, além de uma lista de informações úteis relacionadas
na última folha (DADOS COMPLEMENTARES). As soluções deverão ser apresentadas em 2 (dois) cadernos distintos, correspondentes a cada parte da prova.
\vspace{1.3cm}
\begin{center}
\underline{\textbf{1ª PARTE}}
\end{center}
\paragraph{$1^a$ Questão:} (2,0 pontos)

Dentre as moléculas abaixo indique quais apresentam atividade ótica e identifique nelas os carbonos quirais.
\begin{enumerate}[label=\alph*)]
\item $\ch{CH3CH2CCBr}$
\item 2-isopropil-propeno
\item Ácido 2,3-dihidróxi-6-en-decanóico
\item 2,3-octanodiol
\item 2-cloro-3,5-dimetilnonano
\end{enumerate}

\paragraph{$2^a$ Questão:} (2,0 pontos)

Desenhe as estruturas dos principais produtos da reação dos componentes aromáticos abaixo com $\ch{HBr}$ na presença de $\ch{FeBr3}$.

\begin{figure}[ht]
\centering
\includegraphics[width=8cm]{quimicavc2016q2.png}
\end{figure}


\paragraph{$3^a$ Questão:} (2,0 pontos)

Com base em seus conhecimentos sobre o metabolismo dos carboidratos, calcule o número de moléculas de ATP geradas pelo 
metabolismo completo de um polissacarídeo composto por 10 moléculas de glicose.

\vspace{3cm}

\begin{center}
\underline{\textbf{2ª PARTE}}
\end{center}

\paragraph{4ª Questão} (2,0 pontos) \ \ \ \ \textbf{Repetida} - VC 2014 e 2015

A figura apresentada a seguir representa um dispositivo usado para medir velocidades atômicas e moleculares. Suponha que um feixe de átomos metálicos seja dirigido para um cilindro rotativo, instalado em uma camara sob vácuo. Uma pequena fenda no cilindro permite que os átomos se choquem contra um detector. Porém, considerando-se que o cilindro gira, os átomos, que se deslocam com velocidades diferentes, chocam-se com o detector em posições diferentes. Com o tempo, deposita-se uma camada de metal sobre o detector. Os pontos onde os átomos se chocam correspondem ás medidas de distribuição de velocidades. Em uma experiência, descobriu-se que, a 850 $^o$C, os átomos de bismuto (Bi) se chocavam em pontos afastados, em média, a 2,80 cm do centro do detector. Sabendo-se que o diametro do cilindro é de 15,0 cm e que ele gira a 130 rotações por segundo:

\begin{enumerate}[label=\alph*)]

\item Calcule a velocidade média dos átomos de Bi, a partir dos dados experimentais obtidos.

\item Com base na teoria cinético-molecular, calcule a raiz quadrada da velocidade quadrática média do Bi a 850 $^o$C.

\item Deduza (apresente o desenvolvimento) a expresão que representa a velocidade mais provável e determine o seu valor para a temperatura de 850 ºC.

\item  Compare e discuta a coerência dos resultados, interprete o significado de cada valor obtido nos itens anteriores, informe
se seria de se esperar que os três resultados fossem iguais e justifique.

\end{enumerate}

\begin{figure}[h]
\centering
\includegraphics[width=12cm]{quimicavc2015q7.png}
\end{figure}


\paragraph{5ª Questão} (1,0 ponto)\\
Dissolvem-se $10,2$ g de uma mistura de NaCl e sacarose ($\ch{C12H22O11}$) no volume de água suficiente para preparar $250$ ml de solução. A pressão osmótica da solução é $7,32$ atm a $23$ ºC. Calcule a percentagem em massa de NaCl da mistura.

\paragraph{$6^a$ Questão:} (1,0 ponto)

Sabendo-se que, a $75^o$C, a pressão de vapor do líquido puro A é $1,55$ atm e a do líquido puro B é $0,65$ atm:
\begin{enumerate}[label=\alph*)]
\item Determine a composição da mistura dos líquidos, A e B, que ferve em $75^o$C e $1$ atm;
\item Determine a composição do vapor em equilíbrio com o líquido no ponto de ebulição.
\end{enumerate}

\newpage

 \begin{center}
 \textbf{DADOS COMPLEMENTARES}
 \end{center}
 
 \begin{enumerate}[label=\Roman*.]
 \item \textbf{Constante universal dos gases perfeitos (R):} 
 
 \vspace{0.2cm}
 R=$0,0821 \dfrac{\textrm{atm.L}}{\textrm{K.mol}}=8,314 \dfrac{\textrm{J}}{\textrm{K.mol}}=62,3 \dfrac{\textrm{mmHg.L}}{\textrm{K.mol}}$
\vspace{0.3cm} 

\item \textbf{Distribuição de velocidades de Maxwell-Boltzmann:}

$f(v)=4\pi \left( \dfrac{M}{2\pi RT} \right)^{\frac{3}{2}} v^2 e^{-\frac{Mv^2}{2RT}}$ 
\end{enumerate}
\newpage


\subsection{VE2}
\subsubsection{2019}

\paragraph{(Questão 1)}

Quais as condições de temperatura e pressão que favorecem a formação de produtos na reação descrita abaixo? Justifique sua resposta.
$$\ch{N2}_{(g)}     \rightleftharpoons 2N_{(g)}$$


\paragraph{(Questão 2)}

Considere uma reação $\ch{A}_{(g)}  \rightleftharpoons B_{(g)}$ a $298K$ e $1$ bar, com $ \Delta G^{\theta}_{r} = 0 $. Podemos afirmar que:

\begin{enumerate}[label=\alph*.]
    \item ( ) A reação é espontânea.
    \item ( ) A mistura reacional no equilíbrio possui $50\%$ de reagente e $50\%$ de produto, para qualquer temperatura.
    \item ( ) Um aumento na pressão parcial de A, deslocará o equilíbrio para a direita.
    \item ( ) Uma diminuição da temperatura da reação deslocará .
\end{enumerate}

Coloque S em caso de positivo, quando se pode afirmar a proposição, e N em caso negativo, quando não se pode afirmar. Justifique sua resposta.  

\paragraph{(Questão 3)}

As duas reações descritas nas Eq. 1 e Eq. 2 ocorrem paralelamente no processo industrial de reforma a vapor de metano. Assumindo um comportamento de gás ideal para os gases envolvidos nas reações, determine a composição de equilíbrio do gás que deixa o reator quando uma mistura equimolar de metano e de água é alimentada no reator, a $1000K$. As constantes de equilíbrio para as reações Eq. 1 e Eq. 2 a $1000K$ são $30$ e $1,5$, respectivamente.
$$ \ch{CH4}_{(g)} + \ch{H2O}_{(g)}  \rightleftharpoons \ch{CO}_{(g)} + 3\ch{H2} $$
$$ \ch{CO}_{(g)} + \ch{H2O}_{(g)}   \rightleftharpoons \ch{CO2}_{(g)} + \ch{H2} $$

\paragraph{(Questão 4)} As medidas das pressões de equilíbrio de $\ch{CO2}$ acima da mistura de $\ch{CaCO3}_{(s)}$ e $\ch{CaO}_{(s)}$, em diversas temperaturas, são dadas na tabela abaixo. 

\begin{center}
\begin{tabular}{l|l|l|l|l|l}
P ($torr$) & $23.0$ & $70$ & $183$ & $381$ & $716$   \\ \hline
T ($K$) & $974$ & $1021$ & $1073$ & $1025$ & $1167$   \\
\end{tabular}
\end{center}
 
 
Calcule a pressão de equilíbrio de $\ch{CO2}$ acima da mistura de $\ch{CaCO3}_{(s)}$ e $\ch{CaO}_{(s)}$ a $1000^o$C. 

(As atividades químicas de fases condensadas são iguais a 1)

\newpage

\subsection{VF}
\subsubsection{2016 \original{https://drive.google.com/file/d/1QhUJZqqem5bCbO-MFvVVdQRbjgOULufv/view?usp=sharing}}

\paragraph{1ª Questão} (1,5 ponto)\\
Uma amostra de 0,2140 g de um ácido monoprótico fraco, desconhecido, foi dissolvida em água, formando 25,0 mL de solução. A seguir, titulou-se a amostra com NaOH 0,0950 M, verificando-se que foram necessários 27,4 mL de base para atingir o ponto de equivalência. Sabendo-se que o pH era 6,50 após a adição de 15,0 mL de base na titulação, qual é a constante de acidez ($K_a$) do ácido desconhecido?

\paragraph{2ª Questão} (1,5 ponto)\\
Determine a solubilidade do cloreto de prata em uma solução aquosa de nitrato de prata $6,5 \times 10^{-3}\,M$. Interprete o resultado, comparando com a solubilidade do cloreto de prata em água. Justifique.

\paragraph{3ª Questão} (1,5 ponto)\\
Determine a variação de pH esperada, se 0,020 mol de HC$\ell$ for adicionado a uma solução tampão que foi preparada ao dissolver-se 0,120 mol de $\mathrm{NH_3}$ e 0,095 mol de $\mathrm{NH_4C}\ell$ em 250 mL de água. Admita que a variação de volume foi pequena.

\paragraph{4ª Questão} (1,5 ponto)\\
Quantos mols de AgBr podem ser dissolvidos em 1,0 L de uma solução aquosa de $\mathrm{NH}_3$ 1,0M?

\paragraph{5ª Questão} (1,5 ponto)\\
A dureza da água representa um sério problema para muitas atividades industriais. Além de interferir na eficiência de sabões e detergentes, a dureza é responsável pela deposição de sais em membranas e inscrustações em caldeiras e tubulações, entre outros inconvenientes. Os principais íons causadores de dureza são o cálcio e o magnésio, sob a forma de carbonatos, bicarbonatos e sulfatos. Assim, responda ao que se pede:

\begin{enumerate}[label=(\roman*)]
\item Uma empresa propõe abrandar a dureza da água a ser consumida, por meio de um processo que visa remover seletivamente os íons $\mathrm{Ca^{2+}}$ pela adição controlada de carbonato de sódio. Sabendo-se que a solubilidade molar do $\mathrm{CaCO_3}$ é $9,3 \times 10^{-5}$ M, qual é a sua solubilidade molar em uma solução 0,050 M de $\mathrm{Na_2CO_3}$? O que se conclui diante do resultado? O processo é eficiente? Justifique.\hfill (0,7 ponto)
\item Em um processo independente, os íons $Mg^{2+}$ são removidos como $Mg(OH)_2$ por adição de $Ca(OH)_2$ para produzir uma solução saturada. Qual o pH da solução saturada de $Ca(OH)_2$? \hfill (0,4 ponto)
\item Qual é a concentração de íons $Mg^{2+}$ para o pH calculado no item anterior? \hfill (0,4 ponto)
\end{enumerate}

\paragraph{6ª Questão} (1,5 ponto)\\
Dois líquidos voláteis, benzeno e tolueno, foram misturados a 25 ºC, formando uma solução em que a fração molar do benzeno é 0,33 e a do tolueno é 0,67. Após ser atingido o equilíbrio entre a solução e o seu vapor, coleta-se o vapor sobre a mistura. O vapor coletado é condensado e novamente aquecido a 25 ºC, de modo que um novo equilíbrio entre a mistura resultante e seu vapor seja atingido. Responda:
\begin{enumerate}[label=(\roman*)]
\item Qual será a composição desse novo vapor? \hfill (0,7 ponto)
\item O que se pode esperar quanto à possível liberação ou absorção de calor na formação da mistura? Justifique. \hfill (0,4 ponto)
\item O que se espera em relação à composição da mistura se o processo for repetido várias vezes? \hfill (0,4 ponto)
\end{enumerate}

\paragraph{7ª Questão} (1,0 ponto)\\
Em cada intem, assinale a alternativa correta.

\begin{enumerate}[label=(\roman*)]
\item Com base na Teoria das Colisões Moleculares, pode-se afirmar que:
\begin{tasks}(1)
\task Arrhenius observou que, para a maioria das reações, a constante de velocidade aumenta linearmente com a temperatura;
\task A frequência das colisões entre as moléculas está associada à seção de choque, à velocidade média relativa e à energia de ativação;
\task Processos com alta frequência de colisão e com alta energia de ativação tendem a ocorrer rapidamente;
\task Para reações complexas, em geral, a constante de velocidade está relacionada à combinação de 3 fatores: exigência estérica, frequência de colisão das moléculas e exigência energética;
\task Para que duas moléculas reagentes possam reagir, ao se colidirem, a condição necessária e suficiente é que estejam dotadas de uma energia cinética total igual ou superior à energia de ativação
\end{tasks}

\item Considere os 3 mecanismos propostos para a reação abaixo, cuja lei de velocidade é dada por $v = k[H_2][NO]^2$: $$2H_2\,(g)\, +\, 2NO \,(g)\, \to\, N_2\,(g)\, + \,2H_2O\,(g)$$\textbf{MECANISMO I:} $$H_2\, +\, NO\, \to\, H_2O\, + \,N\qquad (\text{lento})$$ $$N\,+\,NO\,\to \, N_2 \, + \, O\qquad (\text{rápido})$$ $$O\,+\,H_2\, \to\, H_2O\qquad (\text{rápido})$$\textbf{MECANISMO II:}$$H_2\,+\,NO\,\to\, H_2O\,+\,N\qquad \text{(lento)}$$ $$N_2O\,+\,H_2\,\to \, N_2\,+\, H_2O\qquad \text{(rápido)}$$\textbf{MECANISMO III:}$$NO\,+\,NO\, \leftrightarrows\, N_2O_2\qquad \text{(equilíbrio rápido)}$$ $$N_2O_2\, + \, H_2 \to N_2O\,+ \, H_2O\qquad \text{(lento)}$$ $$N_2O\, + \, H_2\, \to \, N_2 \, + \, H_2O\qquad \text{(rápido)} $$Com base nas informações acima, é correto afirmar:
\begin{tasks}(1)
\task Somente o mecanismo I não é compatível com a expressão da velocidade;
\task Somente o mecanismo II não é compatível com a expressão da velocidade;
\task Somente o mecanismo III não é compatível com a expressão da velocidade;
\task Os mecanismos I e III não são compatíveis com a expressão da velocidade;
\task Nenhum mecanismo é compatível com a expressão da velocidade.
\end{tasks}

\item Considere a decomposição do ozônio $\mathrm{(O_3)}$ na atmosfera. Sugere-se a seguinte lei de velocidade desta reação: $$v=k[O_3]^2\cdot [O_2]^{-1}$$Conclui-se que:
\begin{tasks}(1)
\task Esta lei de velocidade não está correta, pois a velocidade não pode depender das concentrações e reagentes e produtos simultaneamente;
\task Ao longo da reação, o aumento da concentração de oxigênio desacelera a reação, favorecendo a reação inversa;
\task A lei de velocidade não está correta, pois não está compatível com a estequiometria da reação;
\task A velocidade de desaparecimento do ozônio relaciona-se com a velocidade de aparecimento do oxigênio da seguinte forma: $v = -\dfrac{1}{2}\dfrac{\Delta(O_3)}{\Delta t} = \dfrac{\Delta (O_2)}{\Delta t}$;
\task A velocidade de desaparecimento do ozônio relaciona-se com a velocidade de aparecimento do oxigênio da seguinte forma: $v = -\dfrac{\Delta (O_3)}{\Delta t} = \dfrac{1}{2}\dfrac{\Delta (O_2)}{\Delta t}$;
\end{tasks}
\item Quanto à Teoria do Complexo Ativado, pode-se afirmar:
\begin{tasks}
\task A Teoria do Complexo Ativado desconsidera a questão da orientação espacial das moléculas durante as colisões;
\task A Teoria do Complexo Ativado aplica-se somente a reações em solução;
\task Energia de ativação é a energia necessária para levar os reagentes ao complexo ativado, que pode ser entendido como um intermediário da reação, possível de ser isolado e estudado como qualquer outra molécula, apesar de sua baixa estabilidade. O complexo ativado sempre levará à formação de produtos;
\task A Teoria do Complexo Ativado parte do princípio que as reações ocorrem somente quando duas moléculas dotadas da energia cinética mínima necessária, deslocando-se em linha reta, colidem ao longo de suas trajetórias de aproximação;
\task Pela Teoria do Complexo Ativado, quando as moléculas colidem, elas formam o complexo ativado (também designado por estado de transição), que é uma espécie transitória formada pelas moléculas de reagente como resultado da colisão antes da formação do produto;
\end{tasks}
\item Considere o seguinte mecanismo de formação de uma hélice dupla de DNA, a partir das fitas A e B:$$A\,+\, B \, \leftrightarrows \, \text{hélice instável (rápido, }k_1\,e\,k_1'\text{ são grandes)} $$ $$\text{hélice instável }\to \text{ hélice dupla estável (lento, }k_2 \text{ muito pequeno)} $$A equação da formação da hélice dupla estável é dada pela seguinte expressão:
\begin{tasks}(1)
\task $v = \dfrac{k_1k_2}{k_1'}[A][B]$
\task $v = \dfrac{k_2}{k_1k_1'}[A][B]$
\task $v = k_1k_1'[A][B]$
\task $v = k_2[A][B]$
\task $v = k_1[A][B]$
\end{tasks}
\end{enumerate}

\newpage

\subsubsection{2015 \original{https://drive.google.com/file/d/1RIAuxEyt3k5fJSYR7_shHYSQdcnjhRhJ/view?usp=sharing}}

\begin{center}
\underline{\textbf{1ª PARTE}}
\end{center}

\paragraph{$1^a$ Questão} (2,0 pontos)

Dentre as afirmações abaixo, preencha os parênteses com \textbf{V} para verdadeiro e \textbf{F} para falso.

( \  \ ) O ácido crômico é um oxidante forte capaz de oxidar qualquer álcool a cetona ou ácido carboxílico;

( \  \ ) Éteres são tão reativos que não podem ser usados como solventes em reações orgânicas;


( \  \ ) Polímeros com cadeias lineares são apropriados para a fabricação de materiais rígidos de alta densidade;


( \  \ ) Quanto maior o caráter "p" de um orbital híbrido maior será o otamanho da ligação química envolvendo esse orbita;


( \  \ ) A abertura de um epóxido sempre dará origem a um álcool.


\paragraph{$2^a$ Questão} (2,0 pontos)

Com base em seus conhecimentos sobre as reações dos álcoois, escolha, dentre as opções dadas, aquela que corresponde aos compostos 
indicados pelas letras de \textbf{A} a \textbf{E} do esquema abaixo. 
\begin{figure}[ht]
\centering
\includegraphics[width=17cm]{quimicavf2015q2.png}
\end{figure}

\begin{enumerate}[label=\alph*)]
\item \footnotesize{A=Alceno; B=Alcino; C=Alcano; D=Cetona/Aldeído; E=Haleto de alquila;}
\item \footnotesize{A=Alcino; B=Alceno; C=Alcano; D=Cetona/Aldeído; E=Haleto de Alquila; }
\item \footnotesize{A=Haleto de Alquila; B=Cetona/Aldeído; C=Alcano; D=Alcino; E=Alceno;}
\item \footnotesize{A=Cetona/Aldeído; B=Haleto de Alquila; C=Alcano; D=Alceno; E=Alcino;}
\item \footnotesize{A=Alceno; B=Cetona/Aldeído; C=Alcano; D=Alcino; E=Haleto de Alquila;}
\end{enumerate}

\begin{center}
\underline{\textbf{2ª PARTE}}
\end{center}

\paragraph{$3^a$ Questão} (1,0 ponto)

Em um experimento, uma amostra de $6,0$L de gás amônia, a $27^o$C e $722$mmHg, foi borbulhada em $600$mL de solução 
molar de HCl. Admitindo-se que toda a amônia foi dissolvida e que o volume da solução manteve-se constante, determine o pH da solução resultante.

\paragraph{$4^a$ Questão} (1,0 ponto)

Uma amostra de $20,0$mL de solução $0,10$M de Ba(NO$_3$)$_2$ é adicionada a $50,0$mL de solução $0,10$M de Na$_2$CO$_3$. Haverá 
formação de um precipitado de BaCO$_3$? Verifique numericamente e justifique com base em cálculos, conceitos e regras de solubilidade.

\paragraph{$5^a$ Questão} (1,0 ponto)

Quantos mols de AgBr podem ser dissolvidos em $1,0$L de uma solução aquosa de NH$_3$ $1,0$M? 

\paragraph{$6^a$ Questão} (1,0 ponto)

Uma amostra de $0,2140$g de ácido monoprótico fraco, desconhecido, foi dissolvida em água, formando $25,0$mL de solução. A seguir, 
titulou-se a amostra com NaOH $0,0950$M, verificando-se que foram necessários $27,4$mL de base para atingir o ponto de equivalência. Sabendo-se
que o pH era $6,50$ após a adição de $15,0$mL de base na titulação, qual é a constante de acidez (\textbf{K}$_a$) do ácido desconhecido? 

\paragraph{$7^a$ Questão} (0,6 ponto)

A $25^o$C, a pressão osmótica de uma solução de glicose (C$_6$H$_{12}$O$_6$) em água é $10,50$atm. A massa específica da 
solução é $1,16$g/mL. O ponto de congelamento dessa solução é:
\begin{tasks}(5)
\task $0,737^o$C
\task $0,798^o$C
\task $-0,737^o$C
\task $-0,798^o$C
\task $-0,688^o$C
\end{tasks}

\paragraph{$8^a$ Questão} (0,6 ponto)

Dois líquidos voláteis, benzeno e tolueno, foram misturados em um ambiente fechado, a $25^o$C, formando uma solução em que a 
fração molar do benzeno é $0,33$ e a do tolueno é $0,67$. Após ser atingido o equilíbrio entre a solução e o seu vapor, coleta-se 
o vapor sobre a mistura, transferindo-o para outro recipiente fechado. O vapor coletado é condensado e novamente aquecido a 
$25^o$C, de modo que um novo equilíbrio entre a mistura resultante e seu vapor seja atingido. É razoável afirmar que:

\begin{tasks}
\task A formação da solução inicial envolve uma forte troca de energia com a vizinhança do sistema e o vapor resultante (final) terá $85,3\%$ de benzeno 
e $14,7\%$ de tolueno;
\task A formação da solução inicial envolve uma forte troca de energia com a vizinhança do sistema e o vapor resultante (final) terá $63,0\%$ de benzeno e
$37,0\%$ de tolueno;
\task Não haverá significativa liberação ou absorção de calor na formação da solução inicial e o vapor resultante (final) terá $85,3\%$ de benzeno e 
$14,7\%$ de tolueno;
\task Não haverá significativa liberação ou absorção de calor na formação da solução inicial e o vapor resultante (final) terá $63,0\%$ de benzeno e
$37,0\%$ de tolueno;
\task A repetição contínua do processo levará a formação de um azeótropo, tornando as misturas subsequentes com composição invariável e baixa pureza de qualquer 
um dos componentes.
\end{tasks}

\paragraph{$9^a$ Questão} (0,8 ponto)

Em células galvânicas, a energia química é convertida em energia elétrica por meio de processos que envolvem reações de oxirredução. 
As reações que ocorrem nos eletrodos tendem a prosseguir espontaneamente, produzindo um fluxo de elétrons que é conduzido através de 
um condutor externo do anodo para o catodo. Com o tempo, a força eletromotriz (potencial da célula) cai progressivamente à medida que 
os reagentes são consumidos, até que ela deixe de ser funcional quando o seu potencial (E) cai a zero. Considere uma célula constituída por 
um eletrodo de zinco e um eletrodo padrão de hidrogênio (EPH), onde ocorre a reação global mostrada abaixo. 

$$ Zn_{(s)} \ + \ 2H^+_{(aq)} \rightarrow Zn^{2+}(1,0\textrm{M}) \ + \ H_2(1,0\textrm{M}) $$


Em um determinado momento, o potencial (E) da célula foi medido, obtendo-se o valor de $0,54$V, a $25^o$C. Determine o pH da solução, sabendo-se que o potencial 
padrão (E$^o$) da célula é $0,76$V e as variações da energia livre de Gibbs ($\Delta G$) podem ser obtidas pelas relações:

$$ 
\begin{cases}
\Delta G = -nFE_{celula} \\
\Delta G^0= -nFE^0_{celula} \\
\end{cases}\textrm{, onde n=2mol, F=96.500 C/mol}$$


\newpage
\begin{center}
\textbf{DADOS COMPLEMENTARES}
\end{center}
 
\begin{enumerate}[label=\Roman*.]
\item \textbf{Constante universal dos gases perfeitos (R):} 
 
\vspace{0.2cm}
R=$0,0821 \dfrac{\textrm{atm.L}}{\textrm{K.mol}}=8,314 \dfrac{\textrm{J}}{\textrm{K.mol}}=62,3 \dfrac{\textrm{mmHg.L}}{\textrm{K.mol}}$
\vspace{0.3cm} 

\item \textbf{Constantes de Dissociação Iônica a 25°C} 
\begin{center}
\begin{tabular}{|l|l|}
\hline 
Ácido & $K_a$ \\ \hline
$NH_4^+$ & $5,6\times 10^{-10}$ \\ \hline
\end{tabular}
\end{center}

\item \textbf{Constante de Diminuição (ou de abaixamento) do ponto de congelamento:}

Água (solvente): $1,86$K/m

\item \textbf{Produto de Solubilidade a 25°C:}

\begin{center}
\begin{tabular}{|l|l|}
\hline 
Composto & $K_{ps}$ \\ \hline
Brometo de Prata (AgBr) & $5,0\times 10^{-13}$ \\ \hline
Carbonato de Bário (BaCO$_3$) & $8,1\times 10^{-9}$ \\ \hline
\end{tabular}
\end{center}


\item \textbf{Constante de Formação ou Constante de Estabilidade:}

\begin{center}
\begin{tabular}{|l|l|}
\hline 
Composto & $K_{ps}$ \\ \hline
Diaminoprata: Ag(NH$_3$)$_2^+$ & $1,6\times 10^{7}$ \\ \hline
\end{tabular}
\end{center}


\item \textbf{Pressão de Vapor a 25°C:}

Benzeno: $75$ Torr

Tolueno: $22$ Torr
\end{enumerate}

\newpage
\subsubsection{2013 \original{https://drive.google.com/file/d/1xkOSkFuRNYj9ab0FD5wtkrv8BUNfQGjW/view?usp=sharing}}

\paragraph{1ª Questão} (1,5 ponto)\\
Sob condições-padrão, a seguinte reação é espontânea a 25 ºC $$\ch{O2 (g) + 4 H^+ (aq) + 4 Br^- (aq) -> 2 H2O (l) + 2 Br2 (l)}$$ A reação será espontânea se a concentração de $H^+$ for ajustada por meio de um tampão composto de 0,10 mol/L de ácido benzóico ($\ch{HC7H5O2}$) e 0,12 mol/L de benzoato de sódio ($\ch{NaC7H5O2}$)? Justifique.

\paragraph{2ª Questão} (1,0 ponto)\\
Determine o potencial-padrão do par $Ti^{4+}/Ti$.

\paragraph{3ª Questão} (1,5 ponto)\\
Uma haste de prata e o EPH são mergulhados em uma solução saturada de oxalato de prata, $\ch{Ag2C2O4}$ a 25 ºC. A diferença de potencial medida entre a haste e o EPH é de 0,589 V, sendo a haste positiva. Calcule a constante do produto de solubilidade do oxalato de prata.

\paragraph{4ª Questão} (2,0 pontos)\\
Desenhe as estruturas dos principais produtos das reações orgânicas descritas nos itens de "a" a "e" abaixo e mostre os mecanismos das reações em cada caso.
\begin{tasks}
\task Reação do metil ciclopenteno com água catalisada por ácido;
\task Reação do dinitro tolueno com ácido sulfúrico na presença de ácido nítrico;
\task Reação do 3,3-dimetil-1-buteno com HCl;
\task Reação do eteno com $\ch{CHCl3}$ na presença de KOH;
\task Reação do ciclohexeno com $Br_2$
\end{tasks}

\paragraph{5ª Questão} (2,0 pontos)\\
Calcule quantas moléculas de ATP seriam geradas pelo metabolismo completo de um ácido graxo contendo 22 átomos de carbono.

\paragraph{6ª Questão} (2,0 pontos)\\
Faça um esquema ilustrando o processo de síntese proteica e explique, de forma objetiva, o que ocorre em cada uma das etapas envolvidas (transcrição e tradução).

\hfill

Dados Complementares

\begin{enumerate}[label=\Alph*.]
\item Constante de Faraday (F)$$1F = 96500\, C/mol$$
\item Constante universal dos gases perfeitos (R) $$R = 0,0821 \frac{atm.L}{K.mol.} = 8,314 \frac{J}{K.mol} = 62,3 \frac{mmHg.L}{K.mol}$$
\item Constante de dissociação iônica de pares ácido-base conjugados a 25 ºC: 

\begin{tabular}{|c|c|c|c|}
Ácido & $K_a$ & Base & $K_b$ \\ \hline $\ch{HC_7H5O2}$ & $6,3 \times 10^{-8}$ & $\ch{C7H5O2}$ & $1,5 \times 10^{-10}$
\end{tabular}
\item Potenciais-padrão de redução em água a 25 ºC:

\begin{tabular}{||c|c||}
\hline REDUÇÃO & POTENCIAL (V) \\ \hline $\ch{Ag^+ (aq) / Ag (s)}$ & +0,80  \\ \hline $\ch{Br2 (l) / Br^- (aq)}$ & +1,065 \\ \hline $\ch{O2 (g), H^+ (aq) / H2O (l)}$ & +1,23 \\ \hline $\ch{Ti^{4+} (aq) / Ti^{3+} (aq)}$ & 0,00 \\ \hline $\ch{Ti^{3+} (aq) / Ti^{2+} (aq)}$ & -0,37 \\ \hline $\ch{Ti^{3+} (aq)/ Ti (s)}$ & -1,63 \\ \hline
\end{tabular}
\end{enumerate}

\newpage

\subsubsection{2012 \original{https://drive.google.com/file/d/1ipD7tl7dx75HRWbsFW_bt3JJQ8HBmqCv/view?usp=sharing}}

\paragraph{1ª Questão} (1,0 ponto)

Uma amostra de 0,2140 g de um ácido monoprótico fraco, desconhecido, foi dissolvida em água, formando 25,0 mL de solução. A seguir, titulou-se a amostra com NaOH 0,0950 M, verificando-se que foram necessários 27,4 mL de base para atingir o ponto de equivalência. Pede-se:

\begin{tasks}(1)
\task Qual é a massa molar do ácido?
\task Sabendo-se que o pH era 6,50 após a adição de 15,0 mL de base na titulação, qual é a constante de acidez $(K_a)$ do ácido desconhecido?
\end{tasks}

\paragraph{2ª Questão} (1,0 ponto)

Sob condições-padrão, a seguinte reação é espontânea a 25 ºC:$$\ch{O2 (g) + 4 H^+ (aq) + 4 Br^- (aq) -> 2 H2O (l) + 2 Br2 (l)}$$A reação será espontânea se a concentração de $\ch{H+}$ for ajustada por meio de um tampão composto de 0,10 mol/L de ácido benzóico $(\ch{HC7H6O2})$ e 0,12 mol/L de benzoato de sódio (\ch{NaC7H5O2})?

\paragraph{3ª Questão} (1,5 ponto) 

Uma célula galvânica é constituída por um eletrodo de prata merulhado em 318 mL de uma solução 0,100 M de $\ch{AgNO3}$ e um eletrodo de magnésio mergulhado em 266 mL de uma solução 0,100 M de $\ch{Mg(NO3)2}$.
\begin{tasks}(1)
\task Calcule a fem para a célula a 25 ºC;
\task Determine o trabalho máximo (por mol de magnésio) que a célula pode realizar;
\task A célula funciona até se depositarem 1,15 g de prata sobre o eletrodo de prata. Calcule a fem para a célula nessa fase de funcionamento.
\end{tasks}

\paragraph{4ª Questão} (1,5 ponto)

A reação seguinte foi descrita como a causa dos depósitos de enxofre formados nas regiões vulcânicas:$$\ch{SO2 + 2 H2S (g) <-> 3 S (s) + 2 H2O (g)}$$Com base na mesma reação, uma usina de energia sugeriu um método para a remoção de $\ch{SO2}$ dos gases de suas chaminés. Pergunta-se:
\begin{tasks}(1)
\task Qual é a constante de equilíbrio para a reação a 35 ºC?
\task Uma primeira avaliação permite considerar que o método para a remoção do $\ch{SO2}$, em princípio, é possível e eficaz? A reação é espontânea? Justifique.
\task Você espera que o processo seja mais ou menos efetivo a temperaturas mais altas? Justifique.
\end{tasks}

\paragraph{5ª Questão} (1,0 ponto)

Determine o potencial-padrão do par $\ch{Cu^+ / Cu}$.

\paragraph{6ª Questão} (1,2 ponto)

A dureza da água representa um sério problema para muitas atividades industriais. Além de interferir na eficiência de sabões e detergentes, a dureza é responsável pela deposição de sais em membranas e inscrustações em caldeiras e tubulações, entre outros inconvenientes. Os principais íons causadores de dureza são o cálcio e o magnésio, sob a forma de carbonatos, bicarbonatos e sulfatos. Assim, responda ao que se pede:

\begin{enumerate}[label=(\alph*)]
\item Uma empresa propõe abrandar a dureza da água a ser consumida, por meio de um processo que visa remover seletivamente os íons $\mathrm{Ca^{2+}}$ pela adição controlada de carbonato de sódio. Sabendo-se que a solubilidade molar do $\mathrm{CaCO_3}$ é $9,3 \times 10^{-5}$ M, qual é a sua solubilidade molar em uma solução 0,050 M de $\mathrm{Na_2CO_3}$? O que se conclui diante do resultado? O processo é eficiente? Justifique.
\item Em um processo independente, os íons $Mg^{2+}$ são removidos como $Mg(OH)_2$ por adição de $Ca(OH)_2$ para produzir uma solução saturada. Qual o pH da solução saturada de $Ca(OH)_2$?
\item Qual é a concentração de íons $Mg^{2+}$ para o pH calculado no item anterior? 
\item Por que os íons $Mg^{2+}$ não são removidos pelo primeiro processo (adição de carbonato de sódio)?
\end{enumerate}

\paragraph{7ª Questão} (1,5 ponto)

Um volume correspondente a 25,0 mL de HC$\ell$ 0,100 M foi titulado com uma solução 0,100 M de amônia adicionada por meio de uma bureta. Calcule os valores de pH da solução para cada uma as seguintes situações:
\begin{enumerate}[label=\alph* .]
\item Após a adição de 10,0 mL de solução de $\ch{NH3}$;
\item Após a adição de 25,0 mL de solução de $\ch{NH3}$;
\item Após a adição de 35,0 mL de solução de $\ch{NH3}$;
\end{enumerate}


\paragraph{8ª Questão} (1,3 ponto)

Leia os textos abaixo:

\textit{"As pilhas zinco-ar constituem a mais recente tecnologia comercializábel desenvolvida para o armazenamento de energia. Este tipo de bateria funciona extraindo o oxigênio existente no ar para reagir com o zinco e produzir eletricidade. Seu princípio de funcionamento é semelhante ao das baterias alcalinas, onde também o metal zinco existente em seu interior reage com o oxigênio para produzir energia. Porém, enquanto que nestas baterias o oxigênio é fornecido por um componente interno (dióxido de manganês), nas baterias do tipo zinco-ar ele provém da atmosfera - por isso a bateria possui vários orifícios ao seu redor.\\ Ao contrário de outros tipos, como NiCd, estas baterias não possuem metais pesados (como o cádmio) e portanto não agridem o meio ambiente. Em baterias do tipo zinco-ar o oxigênio faz o papel do cátodo, liberando mais espaço interno para o ânodo. Com isso, este é o tipo de bateria que fornece maior capacidade de energia armazenada no mesmo espaço em comparação com outros tipos.\\ Existem dois tipos de baterias zinco-ar: as que podem ser recarregadas e as descartáveis, de uso único. Baterias deste tipo recarregáveis são utilizadas em aplicações como veículos elétricos movidos à bateria e as descartáveis em aplicações como câmetras de vídeo, telefones celulares, palmtops etc. A grande vantagem deste tipo de bateria é sua durabilidade (tempo de descarga), bem maior do que a dos outros tipos até hoje existentes, o que a torna atrativa mesmo não podendo ser recarregada nos tipos de aplicações citados".}\\ \dotfill 


\textit{"As baterias de íons de Lítio melhoraram bastante, mas ainda são cercadas de muitos cuitados para evitar que explosões aconteçam. Zinco-ar é a nova combinação que promete mudar a realidade de nossos dispositivos móveis. No ano que vem, 2010, o possível substituto já começa a aparecer no mercado."}\\ \dotfill

\textit{"Uma das principais áreas de investigação em todo o mundo é a busca por novas tecnologias para aumentar a autonomia energética de computadores e outros dispositivos, sem esquecer da segurança.\\ Recalls acontecem em grande quantidade ao redor do globo devido a problemas de fabricação que levam ao superaquecimento das baterias, literalmente derretendo notebooks e por vezes explodindo aparelhos.\\ Uma das mais promissoras tecnologias que prometem substituir o lítio são as baterias zinco-ar. Essas futuras baterias possuem grandes vantagens, como um menor custo e uma capacidade significativamente maior para o armazenamento de energia, se comparando com a tecnologia empregada atualmente, além de serem bem mais estáveis do que as baterias de íon de lítio."}\\ \dotfill

\textit{"No interior da bateria um electrodo poroso de "ar" capta moléculas de oxigênio e redu-las para formar iões hidróxilos como o auxílio de catalisadores que se encontram na interface entre os electrólitos líquido e gasoso. Os iões movimentam-se através do electrólito até ao eléctrodo de zinco, oxidando-o e libertando electrões que geram corrente eléctrica".}\\ \dotfill

\textit{"Em uma bateria do tipo ar-zinco, um dos eletrodos é composto por uma mistura de zinco em pó e KOH, contida em uma cápsula metálica isolada eletricamente do outro eletrodo. Este último é composto por uma placa porosa de carvão que permite a passagem de $\ch{O2 (g)}$ e $\ch{H2O (g)}$. A capacidade da bateria é limitada pela massa de zinco que é consumida através da reação global $\ch{Zn (s) + 1/2 O2 -> ZnO (s)}$, processo este que envolve a formação e decomposição de hidróxido de zinco." }\\ \dotfill


A bateria zinco-ar já é uma realidade no mercado, sendo empregada, especialmente, em aparelhos auditivos. Um pequeno orifício no centro da bateria permite a entrada do oxigênio que, em contato com o elementro Zinco presente na bateria, gera a energia necessária para o aparelho auditivo funcionar. A tecnologia parece, ainda, muito promissora para os automóveis alimentados à eletricidade, por ser leve, podendo ser recarregável. As figuras, a seguir, representam esquematicamente o seu funcionamento:

\begin{figure}[h]
\centering
\includegraphics[width=12cm]{quimicavf2012q7.png}
\end{figure}

Com base nas informações acima e considerando uma bateria zinco-ar que opere a uma temperatura de 25 ºC, onde a pressão parcial de oxigênio é 0,21 atm:

\begin{enumerate}[label=\alph*.]
\item Escreva as semi-reações que ocorrem nos eletrodos zinco-ar e calcule a fem-padrão da bateria a 25 ºC; \hfill (0,4 pto)
\item Calcule a fem da bateria em condições operacionais quando, excepcionalmente, a pressão parcial do oxigênio for de 0,19 atm; \hfill (0,3 pto)
\item Qual é a densidade de energia (energia que pode ser obtida por meio de 1 kg de metal e medida em kilojoules) do eletrodo de zinco? \hfill (0,3 pto)
\item Qual é o volume de ar (em litros) que é preciso fornecer à bateria zinco-ar, por segundo, para se conseguir produzir uma corrente de $2,1 \times 10^5 A$? \hfill (0,3 pto)
\end{enumerate}

\newpage
\section{Química Experimental I}
Matéria introduzida em $\textbf{2017}$.

\subsection{VF}

\subsubsection{2017 \original{https://drive.google.com/file/d/1r3ypKFjMsOtZjC9vxksglDivUHRcPBmC/view?usp=sharing}}

\paragraph{$1^a$ Questão:} (1,0 ponto)

Tendo em mente as práticas realizadas no laboratório, assinale \textbf{F} (FALSO) ou \textbf{V} (VERDADEIRO) nas sentenças abaixo:

\begin{center}
\begin{tabular}{|l|l|}
\hline 
(\ ) & \footnotesize{Vidrarias como pipetas e buretas devem ser secas em estufa a $110^o$C}. \\ \hline
(\ ) & \footnotesize{Numa titulação de uma base forte por um ácido forte os produtos finais são o sal correspondente e água.} \\ \hline
(\ ) & \footnotesize{É correto transferir uma solução do frasco original direto para uma bureta.} \\ \hline
(\ ) & \footnotesize{A retirada de alíquotas de uma solução para titulação pode ser feita com provetas}. \\ \hline
(\ ) & \footnotesize{Padronizar uma solução significa determinar com precisão a concentração daquela solução.} \\ \hline
\end{tabular}
\end{center}

\paragraph{$2^a$ Questão:} (1,0 ponto)

Seguir as normas de segurança é fundamental para que o trabalho transcorra sem acidentes. Em relação às boas práticas de laboratório, 
podemos afirmar que:
\begin{enumerate}[label=\alph*)]
\item O uso de EPI não é necessário no laboratório.
\item Se não sabemos qual é a substância que está no frasco, podemos identifica-la pelo odor ou sabor.
\item A pressa aumenta a probabilidade de um acidente acontecer.
\item Quando alguma vidraria quebra, devemos recolher os cacos com a mão para que ninguém machuque os pés.
\item É permitido entrar no laboratório de chinelo quando a liberação do uso de sapato fechado foi publicado em BI.
\end{enumerate}

\paragraph{$3^a$ Questão:} (2,0 ponto)

A partir dos dados fornecidos na tabela abaixo, calcule o índice de acidez para as amostras de óleo. 

[KOH]=$0,1006$mol/L

MM$_{KOH}$=$56,1$g/mol

V$_{branco}$=$0,10$mL

\begin{center}
\begin{tabular}{|l|l|l|}
\hline 
Amostra & Massa / g & Volume de KOH / mL \\ \hline
Óleo Virgem 1 & $10,032$ & $0,44$ \\ \hline
Óleo Virgem 2 & $10,003$ & $0,42$ \\ \hline
Óleo Virgem 3 & $10,063$ & $0,46$ \\ \hline
Óleo Usado 1 & $10,011$ & $1,32$ \\ \hline
Óleo Usado 2 & $9,998$ & $1,30$ \\ \hline
Óleo Usado 3 & $10,016$ & $1,32$ \\ \hline
\end{tabular}
\end{center}

\begin{enumerate}[label=\Roman*]

\item O valor médio de índice de acidez encontrado para as amostras de óleo usado foi de:
\begin{enumerate}[label=\alph*)]
\item $0,19$ mg$_{KOH}$/g$_{amostra}$
\item $0,86$ mg$_{KOH}$/g$_{amostra}$
\item $0,68$ mg$_{KOH}$/g$_{amostra}$
\item $0,55$ mg$_{KOH}$/g$_{amostra}$
\end{enumerate}

\item O valor médio de índice de acidez encontrado para as amostras de óleo virgem foi de:
\begin{enumerate}[label=\alph*)]
\item $0,48$ mg$_{KOH}$/g$_{amostra}$
\item $0,19$ mg$_{KOH}$/g$_{amostra}$
\item $0,91$ mg$_{KOH}$/g$_{amostra}$
\item $0,10$ mg$_{KOH}$/g$_{amostra}$
\end{enumerate}
\end{enumerate}

\paragraph{$4^a$ Questão:} (2,0 ponto)

Uma solução de NaOH foi preparada no laboratório. O técnico esqueceu de anotar a concentração em mol/L no rótulo. Para se ter 
certeza da concentração, foi realizada uma titulação contra HCl. Usando os dados fornecidos na tabela abaixo, calcule:

V$_{NaOH}$=$1,00$L

MM$_{NaOH}=40$g/mol

\begin{center}
\begin{tabular}{|l|l|l|}
\hline 
[HCl] / mol L$^{-1}$ & V$_{HCl}$ / mL & V$_{NaOH}$ / mL \\ \hline
$0,1356$ & $14,00$ & $20,00$ \\ \hline
$0,1356$ & $14,04$ & $20,00$ \\ \hline
$0,1356$ & $13,98$ & $20,00$ \\ \hline
\end{tabular}
\end{center}


\begin{enumerate}[label=\Roman*]
\item O valor médio da concentração de hidróxido de sódio em mol/L é:
\begin{tasks}
\task $0,1155$mol/L
\task $0,1009$mol/L
\task $0,08679$mol/L
\task $0,09497$mol/L
\end{tasks}
\item A pureza do hidróxido de sódio usado no preparo da solução é de:
\begin{tasks}
\task $84,21\%$
\task $90,87\%$
\task $99,85\%$
\task $94,15\%$
\end{tasks}
\end{enumerate}

\paragraph{$5^a$ Questão:} (2,0 ponto)

Um estudante deseja titular uma amostra de óleo vegetal, com uma solução de KOH, previamente fatorada, para isso
precisa requisitar todo o material necessário ao almoxarifado. Dos materiais relacionados abaixo, assinale quais foram
requisitados.
\begin{tasks}
\task Tubo de ensaio;
\task Suporte para tubos de ensaio;
\task Suporte Universal;
\task Bureta;
\task Proveta;
\task Cápsula;
\task Placa de aquecimento;
\task Erlenmeyer;
\task Cadinho;
\task Bécher;
\end{tasks}

\paragraph{$6^a$ Questão:} (2,0 ponto)

Baseado nos conceitos de polaridade das substâncias, preencha o quadro abaixo:

\begin{center}
\begin{tabular}{|l|l|l|}
\hline 
Soluto/Solvente & Água & Heptano \\ \hline
NaOH &  &  \\ \hline
Óleo Vegetal &  &  \\ \hline
\end{tabular}
\end{center}

\newpage
\section{Introdução à Computação}

\subsection{VC}

\subsubsection{2017 \original{https://drive.google.com/open?id=1WrLTXpHAXz-GppO7XpR91qsiwrJVkvDO}}



\paragraph{1ª Questão:} (3,0 pontos)\\
Dado um natural $n$, definimos $$f(n,1) \begin{cases}0\text{ se }n=0 \\ 2^{\lfloor\mathrm{log}_2\,n\rfloor}\text{ se }n>0\end{cases}$$Ou seja, $f(n,1)$ é a maior potência de 2 menor ou igual a $n$.
Por exemplo, $f(18,1) = 16$, enquanto $f(1025,1) = 1024$.
Para $k>1$, definimos $f(n,k) = f(n-f(n,1), k-1)$.
Por exemplo $f(18,2) = f(18-16, 1) = 2, \,f(18,3) = f(18-16,2) = f(2-2,1)=0,\,f(1025, 2) = f(1025-1024, 1) = 1$.
Faça um programa (sem usar a função logaritmo) que leia os naturais $n$ e $k$ e imprima $f(n,k)$. O programa deve funcionar para $0\le n\le 2^{32} - 1$ e $1\le k \le 32$.

\paragraph{2ª Questão:} (4,0 pontos)\\
Duas peças de madeira, com as mesmas características, foram empregadas como vigas estruturais em duas edificaç~eos distintas. Em uma dessas edificações, a viga de madeira faz parte da sustentação de um escritório comercial, e na outra a peça de madeira ajuda a sustentar uma academia de ginástica.
Com o intuito de verificar se as vigas ainda estão em condições normais de uso, foi realizado um estudo de análise estrutural com as seguintes etapas:
\subparagraph{1. Elaboração do modelo computacional da viga, conforme a figura 1.}

Nesta etapa, foi criado um modelo de representação da viga com 02 (dois) nós. Em cada nó há 03 (três) graus de liberdade (deslocamento na direção horizontal, deslocamento na direção vertical e rotação em torno do nó) que foram numerados de 1 a 6 (conforme indicado na figura 1).

\begin{figure}[ht]
\centering
\includegraphics[width=8cm]{icvc2017q2f1.png}
\end{figure}

\subparagraph{2. Cálculo da matriz de rigidez (matriz [K])}

Nesta etapa, foi calculada a matriz de rigidez da viga, representada abaixo:$$[K] = \left[\begin{matrix}3,6 \times 10^5 & 0 & 0 & - 3,6 \times 10^5 & 0 & 0 \\
0 & 8,1 \times 10^3 & 1,62 \times 10^4 & 0 & -8,1 \times 10^3 & 1,62 \times 10^4 \\ 0 & 1,62 \times 10^4 & 4,32 \times 10^4 & 0 & -1,62 \times 10^4 & 21,6 \times 10^4 \\
-3,6 \times 10^5 & 0 & 0 & 3,6 \times 10^5 & 0 & 0 \\
0 & -8,1 \times 10^3 & -1,62\times 10^4 & 0 & 3,24\times 10^4 & -1,62\times 10^4 \\
0 & 1,62\times 10^4 & 21,6\times 10^4 & 0 & - 1,62 \times 10^4 & 4,32 \times 10^4
 \end{matrix}\right]_{6 \times 6}$$
 
 
\subparagraph{3. Obtenção dos deslocoamentos dos nós em cada grau de liberdade ( \{u\} )}

Os deslocamentos dos nós em cada grau de liberdade foram obtidos por uma equipe técnica através de instrumentos de medição instalados nas vigas. Os valores obtidos foram os seguintes:

\subparagraph{- Deslocamentos na viga que compõe o sistema de sustentação do escritório comercial:}

$$\{u\} = \begin{array}{c}
\left[ \begin{array}{c}0,0829 \\ 0,031 \\ 0,035 \\ 0,0815 \\ -0,031 \\ -0,0344\end{array}  \right] \begin{array}{c} (GL 1) \\ (GL 2) \\ (GL 3) \\ (GL 4) \\ (GL 5) \\ (GL 6) \end{array} \\ 1 \times 6
\end{array}$$

\subparagraph{- Deslocamentos na viga que compõe o sistema de sustentação da academia de ginástica:}

$$\{u\} = \begin{array}{c}
\left[ \begin{array}{c}0,1215 \\ -0,021 \\ -0,138 \\ 0,1214 \\ 0,0212 \\ -0,140\end{array}  \right] \begin{array}{c} (GL 1) \\ (GL 2) \\ (GL 3) \\ (GL 4) \\ (GL 5) \\ (GL 6) \end{array} \\ 1 \times 6
\end{array}$$

Você deverá auxiliar o engenheiro de fortificação e construção, responsável técnico pela análise estrutural das peças, a realizar as duas últimas etapas do estudo.
\textbf{Para tanto, será necessário elaborar um programa em linguagem C que atenda aos seguintes pedidos:}
\begin{enumerate}[label=(\alph*)]
\item (2,5 pontos) Calcular, e imprimir na tela, as forças atuantes em cada grau de liberdade (GL) para uma das vigas em estudo. A escolha de qual viga terá as forças atuantes calculadas (a viga que sustenta um escritório ou a viga que sustenta uma academia de ginástica) deverá ficar a critério do usuário do programa (engenheiro de fortificação e construção).

\textbf{Observação: }o vetor e forças atuantes nos graus de liberdade (\{F\}) é calculado através da seguinte expressão: $$[K].\{u\} = \{F\}$$

Logo, {F} será um vetor com 01 (uma) coluna e 06 (seis) linhas. O valor obtido em cada linha corresponderá à força atuante nos graus de liberdade do modelo da viga.

$$\{F\} = \begin{array}{c}
\left[ \begin{array}{c}F_1 \\ F_2 \\ F_3 \\ F_4 \\ F_5 \\ F_6\end{array}  \right] \begin{array}{c} (GL 1) \\ (GL 2) \\ (GL 3) \\ (GL 4) \\ (GL 5) \\ (GL 6) \end{array} \\ 1 \times 6
\end{array}$$

\item (1,0 ponto) A partir do valor máximo de força que a viga pode suportar, informação essa que deve ser solicitada ao usuário do programa, verificar se as forças atuantes nos graus de liberdade da peça ultrapassam esse limite. Caso todas as forças atuantes sejam menores ou iguais ao valor máximo, deve-se imprimir na ela a mensagem que \textit{"Todas as forças atuantes atendem ao projeto"}. Caso contrário, deve-se imprimir na tela em quais graus de liberdade (GL) as forças atuantes são superiores ao valor máximo. Exemplo: \textit{"As forças atuantes nos graus de liberdade 3 e 5 são superiores ao valor máximo permitido em projeto"}.

\item (0,5 ponto) O programa deve possibilitar ao usuário a realização de mais de um cálculo, sem a necessidade de inicialização do programa para cada cálculo a ser realizado. O encerramento da execução deve ficar a cargo do usuário, como uma opção de menu.


\end{enumerate}

Penalizações:

\begin{tasks}(1)
\task Programa não compila - perda integral da questão
\task Código não formatado - perda de 0,5 ponto
\end{tasks}

\paragraph{3ª Questão:} (3,0 pontos)\\
Escreva um programa em C que receba do usuário uma palavra (sequência de letras sem espaço) e faça a codificação dessa palavra com as seguintes operações (não necessariamente na ordem):

\begin{enumerate}[label=(\alph*)]
\item (0,8 ponto) Trocar maiúsculas por minúsculas (e vice-versa)
\item (0,8 ponto) Trocar cada letra pela seguinte no alfabeto (atenção para o Z, que deve ser codificado para o A)
\item (0,8 ponto) Em cada par de letras, inverter a ordem. Se a palavra tiver número ímpar de letras, a última deve permanecer nesta posição.

\textbf{Exemplo}

Palavra original: Telefone
Palavra codificada: FuFMPGFO

Outros itens de pontuação:

\item (0,3 ponto) Receber corretamente a palavra do usuário
\item (0,3 ponto) Imprimir corretamente a resposta


\end{enumerate}

\textbf{Penalizações}
\begin{tasks}
\task Programa não compila - perda integral da questão
\task Programa compila, mas trava quando o usuário tenta inserir os dados - perda integral da questão
\task Programa não exibe a resposta - perda de 0,5 ponto
\task Código não formatado - perda de 0,5 ponto
\end{tasks}

\newpage
\subsubsection{2016 \original{https://drive.google.com/file/d/1ODZowjTPq52PPWTXp1CpB_j-4R_SlLdA/view?usp=sharing}}

\paragraph{$1^a$ Questão:} (3,0 pontos)

Em álgebra abstrata e alguns outros ramos da matemática, o conceito de permutação de um conjunto é associado a funções bijetivas 
de um conjunto finito nele mesmo. De acordo com este conceito, por exemplo, as permutações dos números 1 a 9 seriam interpretadas 
como funções do conjunto $\{1,\dots,9\}$ nele mesmo. De forma mais simples, como está na enciclopédia Wolfram MathWorld, podemos 
definir uma permutação como um rearranjo dos elementos de uma lista ordenada. Computacionalmente, um inteiro x é uma permutação 
de um inteiro y se os dígitos de x podem formar uma permutação dos dígitos de y, conforme os conceitos anterioremente explicados. Por 
exemplo, 343455 é uma permutação de 553434, porém, 343454 não é uma permutação de 553434.

O número total de permutações de um conjunto de n dígitos é dada por $n!$. Por exemplo, o conjunto dos números de 1 a 3 ($n=3$) tem $3!$ 
permutações, ou seja, 6 permutações possíveis: $\{1,2,3\},\{1,3,2\},\{2,1,3\},\{2,3,1\},\{3,1,2\}$ e $\{3,2,1\}$.

Diante dos conceitos acima expostos, implemente:
\begin{tasks}
\task (1,0 ponto) Uma função contadordefrequenciadedigitos que recebe um inteiro $m$ e um inteiro $d$, $0<d\leq 9$ e retorna 
quantas vezes o dígito $d$ aparece em $m$. 
\task (1,0 ponto) Uma função totaldepermutacoes que recebe um inteiro $n$ e retorna o total de permutações possíveis dos dígitos 
desse inteiro $n$. Para simplificar, podem ser consideradas as repetições, por exemplo, $122$ pode ter $6$ permutações.
\task (1,0 ponto) Um programa que, usando as funções dos itens a) e b), recebe dois números inteiros positivos $x$ e $y$ e responde
se $x$ é permutação de $y$, além de responder quantas permutações são possíveis com a quantidade de dígitos dos números $x$ ou $y$.
\end{tasks}

\paragraph{$2^a$ Questão:} (4,0 pontos)

Um passageiro corre com velocidade constante, igual a $4,0$m/s, ao longo da plataforma onde o trem está parado. Quando ele 
se encontra a $25$m do último vagão, o trem arranca com aceleração constante de $0,50$m/s$^2$. Nestas condições, o passageiro 
não conseguirá alcançar o trem para embarcar. 

Escreva um programa em C que simule esta situação numericamente. Considere que a posição inicial do passageiro é a origem do 
eixo de movimento.

Seu programa deverá ter as seguintes funções:
\vspace{0.4cm}
\begin{enumerate}
\item (0,5 ponto) determinaPosicaoTrem

Parêmtro de entrada: número real tempo

Retorno: número real posicaoTrem, que é a posição do trem no eixo orientado.
\vspace{0.2cm}
\item (0,5 ponto) determinaPosicaoPassageiro

Parêmtro de entrada: número real tempo

Retorno: número real posicaoPassageiro, que é a posição do passageiro no eixo orientado.
\vspace{0.2cm}
\item (1,0 ponto) determinaDistancia

Parêmtro de entrada: número real tempo

Retorno: número real distancia, que é a distancia entre o trem e o passageiro no eixo orientado.
\vspace{0.2cm}
\item (1,5 pontos) main

Deverá chamar corretamente a função determinaDistancia, passando como parâmetro o tempo. Faça a variável tempo admitir 
valores no intervalo$[0.0 , 10.0]$ segundos, sendo incrementada em $0.1$s a cada iteração.

Obs: Esta função deverá chamar as funções $1$ e $2$ adequadamente.

(0,5 ponto) Também deverá ser impresso o valor da distância mínima e o instante em que ela ocorre.
\vspace{3cm}
\end{enumerate}

\textbf{Exemplo de Execução:}

\rule{12cm}{0.01cm}

\begin{tabular}{ll}
Tempo & Distância \\
0.00 & 25.00 \\
0.10 & 24.60 \\
0.20 & 24.21 \\ 
0.30 & 23.82 \\
0.40 & 23.44 \\
0.50 & 23.06 \\
0.60 & 22.69 \\
0.70 & 22.32 \\
0.80 & 21.96 \\
0.90 & 21.60 \\
1.00 & 21.25 \\
\end{tabular}\\

\dots

A distância mínima foi \rule{1cm}{0.01cm} metros e ocorreu no instante \rule{1cm}{0.01cm} segundos.

\rule{12cm}{0.01cm}

Obs: A última linha deve ser impressa exatamente com este texto, com os valores com duas casas decimais.

\paragraph{$3^a$ Questão:} (3,0 pontos)

Considere a seguinte lista de números:

$$ 3499,3709,3919,4129,4339,4549,4759,4969,5179 $$

Esses nove números são primos, e estão em uma progressão aritmética de razão $210$.

Por séculos, matemáticos se dedicaram ao problema de determinar qual seria a maior lista de números primos, que poderiam ser 
colocados em progressão aritmética. Até que em $2004$, foi demonstrado que essas listas podem ser arbitrariamente grandes (Green and Tao, $2004$).

O foco de nossa questão são essas listas. Para isso, resolva os seguintes itens:

\begin{enumerate}[label=\alph*)]
\item (0,5 ponto) Escreva uma função 

int eh\_primo(int n)

que retorna $1$ se $n$ é primo e $0$ caso contrário. ( Você pode assumir que a função é chamada com $n>1$)
 
\item (0,5 ponto) Escreva uma função

void exibe\_pa(int a, int k, int n)

que exibe a lista de $n$ elementos em progressão aritmética de razão $k$, cujo menor elemento é $a$. Por exemplo, sendo chamada
exibe\_pa(4,3,5) deve ser exibida na tela: 
$$4\ 7\ 10\ 13\ 16$$
\item (1,0 ponto) Escreva uma função

int max\_lista(int a, int k)

que retorna o maior número de primos que podem ser colocados em uma progressão aritmética de razãi k, cujo menor elemento é a. Assumiremos 
que listas de 1 ou 2 elementos estão por definição em P.A. 

Por exemplo:

max\_lista(3,1) retorna 1, pois 3 é primo, mas 4 não o é

max\_lista(4,2) retorna 0, por 4 não é primo

max\_lista(5,6) retorna 5, pois 5,11,17,23 e 29 são primos, mas 35 não o é.

\item (1,0 ponto) Escreva uma função main, que utiliza as funções anteriores para exibir todas as listas de exatamente 6 
números primos, que podem ser colocadas em progressão aritmética, cujo menor elemento é menor do que 100, e cuja razão da P.A. é menor do que 1000. 
Ou seja, a saída deve ser algo da forma:

7 37 67 97 127 157

7 937 1867 2797 3727 4657

11 71 131 191 251 311

11 491 971 1451 1931 2411

13 103 193 283 373 463

\dots

97 607 1117 1627 2137 2647

97 937 1777 2616 3457 4297
\end{enumerate}

\newpage
\subsubsection{2015 \original{https://drive.google.com/file/d/1kst8BroXERfp2S3HeuKB6Ollk7w_jhcO/view?usp=sharing}}

\paragraph{1ª Questão:} (4,0 pontos) Para desenvolver a n-ésima potência de um binômio $x+y$, pode-se usar a expressão conhecida como Binômio de Newton. A expansão de $(x+y)^n$ é uma suma de termos $a\cdot x^b\cdot y^c$, onde os expoentes $b$ e $c$ são números naturais, com $b+c=n$, e o coeficiente $a$ de cada termo é um inteiro que depende de $n$ e $b$. $$(x+y)^4 = x^4 + 4x^3y + 6x^2y^2 + 4xy^3 + y^4$$ Por exemplo:

Os coeficientes da expansão da n-ésima potência do binômio x + y, dão origem a um interessante fenômeno numérico conhecido como Triângulo de Pascal, descoberto pelos chineses há quase mil anos,
mas cujas principais propriedades foram demonstradas pelo matemático francês Blaise Pascal, no qual a n-ésima linha do citado triângulo é formada pelo cálculo de uma combinação simples, com p variando de 0 a n para calcular cada coeficiente na n-ésima linha.

O Triângulo de Pascal também pode ser construído numericamente, preenchendo com 1s os lados do triângulo a partir do vértice superior e para obter os números em cada linha, somando-se os dois
números logo acima dele na linha superior, por exemplo: 2=1+1, ou seja, o número 2 da terceira linha é igual à soma de 1+1, os dois números logo acima dele na segunda linha, assim 3=1+2, 6=3+3, 10=4+6, etc. A figura abaixo mostra o triângulo de Pascal até a sexta linha:

\imgh{icvc2015q1fig1}{4}

A partir deste último conceito do Triângulo de Pascal, escreva um programa em C que leia um inteiro não negativo n e imprima até a n-ésima linha do triângulo de Pascal, de forma que funcione até
n = 20. Por exemplo: Ao ler 4, a saída deve ser:

\large
$$\begin{array}{ccccc}
\mathbf{1} &   &   &   &   \\
\mathbf{1} & \mathbf{1} &   &   &   \\
\mathbf{1} & \mathbf{2} & \mathbf{1} &   &   \\
\mathbf{1} & \mathbf{3} & \mathbf{3} & \mathbf{1} &   \\
\mathbf{1} & \mathbf{4} & \mathbf{6} & \mathbf{4} & \mathbf{1} \\
\end{array}$$
\normalsize

\paragraph{2ª Questão:} (3,0 pontos) Sistema linear é um conjunto de $n$ equações lineares com n incógnitas relacionadas entre si. A solução de um sistema linear pode ser obtida de várias maneiras. Uma das formas mais comuns de resolução é a regra de Cramer.

Todo sistema linear pode ser associado a uma matriz envolvendo os coeficientes numéricos e a parte literal. Por exemplo, considere o seguinte sistema linear: $$\begin{cases}3x+y+2z = 3 \\ x+y-z = 4 \\ x+ y+3z= 0 \end{cases}$$

A representação matricial dos coeficientes é: $$\left[\begin{matrix}3 & 1 & 2 \\ 1 & 1 & -1 \\ 1 & 1 & 3 \end{matrix}\right]$$

A representação matricial completa do sistema é: $$\left[\begin{matrix}3 & 1 & 2 \\ 1 & 1 & -1 \\ 1 & 1 & 3 \end{matrix}\right] \left[\begin{matrix}x \\ y \\ z \end{matrix} \right] = \left[ \begin{matrix}3 \\ 4 \\ 0 \end{matrix} \right]$$

A regra de Cramer diz que: os valores das incógnitas de um sistema linear são dados por frações cujo denominador é o determinante da matriz dos coeficientes das incógnitas (chamamos de $\Delta _p$) e o numerador é o determinante da matriz dos coeficientes das incógnitas após a substituição de cada coluna pela coluna que representa os termos independentes do sistema (chamamos de $\Delta _x,\, \Delta _y $ e $\Delta_z$).

Assim, no exemplo citado, temos: $$\Delta_p = \left| \begin{matrix}3 & 1 & 2 \\ 1 & 1 & -1 \\ 1 & 1 & 3 \end{matrix}\right|,\, \Delta_x = \left| \begin{matrix}3 & 1 & 2 \\ 4 & 1 & -1 \\ 0 & 1 & 3 \end{matrix}\right|,\,\Delta_y = \left| \begin{matrix}3 & 3 & 2 \\ 1 & 4 & -1 \\ 1 & 0 & 3 \end{matrix}\right|,\,\Delta_z = \left| \begin{matrix}3 & 1 & 3 \\ 1 & 1 & 4 \\ 1 & 1 & 0 \end{matrix}\right| $$

A classificação do sistema é dada pelas seguintes condições:
\begin{itemize}
\item Possível e determinado: $\Delta _p \neq 0$
\item Possível e indeterminado: $\Delta_p,\, \Delta_x,\, \Delta_y,\,$ e $\Delta_z = 0$
\item Impossível: $\Delta_p = 0$ e ($\Delta_x$ ou $\Delta_y$ ou $\Delta_z \neq 0$)
\end{itemize}

Escreva um programa em C, de acordo com os seguintes pedidos:
\begin{tasks}
\task (0,5 ponto) Pedir e receber do usuário os coeficientes reais de um sistema 3x3;
\task (0,5 ponto) Pedir e receber do usuário os termos independentes reais do mesmo sistema;
\task (1,0 ponto) Declarar e implementar a função calculaDet, que recebe como parâmetro os termos reais de uma matriz 3x3 e retorna o seu determinante;
\task (0,5 ponto) Declarar e implementar a função classificaSistema, que recebe como parâmetros as matrizes de coeficientes e de termos independentes e, chamando a função calculaDet quando necessário, retorna um inteiro que indica a classificação do sistema;
\task (0,5 ponto) Implementar corretamente a função main( ), para que ela chame adequadamente a função do item (c) e responda ao usuário a classificação do sistema.
\end{tasks}

\paragraph{3ª Questão:} (3,0 pontos)\\
Se k é um inteiro positivo, definimos f(k) da seguinte forma:
f(k) = k/2, se k é par e f(k)= 3k+1, se k é ímpar. Uma conjectura antiga afirma que para todo o k existe um n tal que $f^n (k) = 1$. Isto é, f aplicada n vezes a k, f(f(\dots.f(k)\dots))=1.

Faça um programa que para cada k entre 2 e 100 imprima k e o menor n tal que $f^n (k) = 1$.

As duas primeiras linhas da saída do programa devem ser:

“k=2, n=1”
“k=3, n=7”

\newpage


\subsubsection{2014 \original{https://drive.google.com/file/d/1_x_LB8heEhzLLyMHgnHf0508rarUicIz/view?usp=sharing}}

\paragraph{1ª Questão:} (valor 3,0 pontos) A construção de um edifício residencial é realizada através de vários projetos de engenharia que possibilitarão, aos futuros moradores, todas as funcionalidades básicas de uma unidade habitacional (proteção contra as intempéries do tempo, fornecimento de energia elétrica,
abastecimento de água potável, saneamento básico, entre outras).

O projeto de instalações hidráulicas é responsável por dimensionar todo o sistema predial de água potável de uma edificação de forma que os moradores tenham fornecimento ininterrupto de água. Para
que isso ocorra, é essencial ter conhecimento do consumo diário total de água (CD). O CD corresponde ao volume máximo previsto para consumo da edificação durante 24 horas e é obtido através da soma dos consumos de todas as instalações existentes no prédio, conforme a Tabela 1 abaixo:
$$\text{Tabela 1 - Consumo diário das instalações}$$
\noindent 
\begin{center}
\begin{tabular}{ |c|c|c| } \hline
\textbf{Discriminação} &  \textbf{Consumo (l/dia)} &\textbf{Unidade}\\ \hline
apartamentos & 200 & \textit{per capita}\\ \hline jardins & 1,50 & $m^2$ \\ \hline garagens & 50 & por automóvel \\ \hline
\end{tabular}
\end{center}
\normalsize

\vspace{3mm}

A taxa de ocupação dos apartamentos é de duas pessoas para cada quarto.

A partir do consumo diário de água, pode-se determinar o diâmetro do ramal predial. O ramal
predial é a tubulação compreendida entre a rede pública de abastecimento de água e o hidrômetro do
edifício (Figura 1). A obtenção do diâmetro do ramal predial é feita com o auxílio da Tabela 2:

$$\text{Tabela 2 - Diâmetro de ramal predial}$$
\noindent 
\begin{center}
\begin{tabular}{ |c|c| } \hline
\textbf{Consumo ($m^3$/dia)} &  \textbf{Diâmetro do ramal predial (mm)} \\ \hline
5 & 15 \\ \hline
10 & 20 \\ \hline
22 & 25 \\ \hline
60 & 40 \\ \hline
140 & 50 \\ \hline
300 & 75 \\ \hline
$>$ 300 & 100 \\ \hline
\end{tabular}
\end{center}
\imgh{icvc2014q1}{8}

Você, como engenheiro militar, recebeu a missão de elaborar um programa que calcule o diâmetro
do ramal predial de edificações a serem onstruídas pelo Sistema de Obras Militares. O programa deve solicitar ao usuário todas as informações referentes às características físicas do edifício a ser projetado, bem como possibilitar ao mesmo a realização de mais de um cálculo, sem a necessidade de inicialização do programa para cada projeto a ser realizado.

O programa será avaliado por intermédio do seguinte barema:

\begin{tasks}
\task Obtenção das características físicas do edifício através do usuário (0,2 ponto);
\task Impressão na tela de todas as características fornecidas pelo usuário (0,2 ponto);
\task Elaboração e emprego corretos de funções para cálculo dos consumos diários de cada tipo de instalação (Tabela 1) do edifício (0,6 ponto);
\task Impressão na tela do valor correto do consumo diário de cada tipo de instalação (Tabela 1) do edifício (0,3 ponto);
\task Impressão na tela do valor correto do consumo diário total de água (CD) do edifício (0,5 ponto);
\task Impressão na tela do valor correto do diâmetro do ramal predial do edifício (0,5 ponto);
\task Possibilitar ao usuário a realização de mais de um cálculo, sem a necessidade de inicialização do programa para cada projeto a ser dimensionado (0,5 ponto);
\task Identação correta do código-fonte (0,2 ponto).
\end{tasks}

\paragraph{2ª Questão:} (valor 3,0 pontos) Crie um programa em C que calcule o seno ou o cosseno de um ângulo, dando ao usuário essa opção através de um menu. Deverão ser seguidas as restrições abaixo:

\begin{enumerate}[label=\alph*)]
\item Parte sem a biblioteca $<math.h>$

\begin{enumerate}[label=a.\arabic*)]

\item Solicitar ao usuário que insira o ângulo em graus. Em seguida deve ser impresso o ângulo em radianos. Use $\pi$ = 3.14159265359 (0,5 ponto).
\item Construir menu que dá ao usuário a opção de calcular o seno ou cosseno (0,5 ponto).
\item Cálculo do seno e do cosseno utilizando a expansão de Taylor: (1,0 ponto) $$\sen x = \displaystyle\sum\limits_{n=0}^{\infty} \dfrac{(-1)^n}{(2n+1)!}x^{2n+1}$$ $$\cos x = \displaystyle\sum\limits_{n=0}^{\infty} \dfrac{(-1)^n}{(2n)!}x^{2n}$$ Obs: a expansão dos termos para cálculo do seno e do cosseno deverá ser truncada no 4° termo. O valor
“x” deve estar em radianos.
\end{enumerate}

\item Parte com a biblioteca $<math.h>$
\begin{enumerate}[label=b.\arabic*)]
\item Compare os valores para o $\sen\, 60$° com dois procedimentos: i) expansão de Taylor (sua implementação), ii) função “sin(x)” (math.h). Esta comparação deve ser feita imprimindo em cada uma das k linhas o valor de “sin($\pi$/3)” e da expansão de Taylor (com 10 casas decimais) até o k-ésimo termo da expansão. Exemplo de impressão:

\noindent\scriptsize
linha 1: \#math.h\# $\sin (\pi/3)$ = $0,866025$, \#Meu algoritmo\# $\sen (\pi/3) $ = $\langle$Valor numérico para $\sen \, x $ = $ \sum_{n=0}^{1}\frac{(-1)^n}{(2n+1)!}x^{2n+1}\, \rangle$
\normalsize

\noindent\scriptsize
linha 2: \#math.h\# $\sin (\pi/3)$ = $0,866025$, \#Meu algoritmo\# $\sen (\pi/3) $ = $\langle$Valor numérico para $\sen \, x $ = $ \sum_{n=0}^{2}\frac{(-1)^n}{(2n+1)!}x^{2n+1}\, \rangle$
\normalsize

\noindent\scriptsize
linha 3: \#math.h\# $\sin (\pi/3)$ = $0,866025$, \#Meu algoritmo\# $\sen (\pi/3) $ = $\langle$Valor numérico para $\sen \, x $ = $ \sum_{n=0}^{3}\frac{(-1)^n}{(2n+1)!}x^{2n+1}\, \rangle$
\normalsize

\dots

\noindent\scriptsize
linha k: \#math.h\# $\sin (\pi/3)$ = $0,866025$, \#Meu algoritmo\# $\sen (\pi/3) $ = $\langle$Valor numérico para $\sen \, x $ = $ \sum_{n=0}^{k}\frac{(-1)^n}{(2n+1)!}x^{2n+1}\, \rangle$
\normalsize

Faça isso para k = 1..10 ou seja, imprima 10 linhas (1,0 ponto).

Objetivo do item: perceber que ao aumentar o número de termos da expansão, o valor de sen($\pi$/3) do algoritmo gerado se aproxima do valor “real” calculado pela função da biblioteca.

\end{enumerate}

\end{enumerate}

Obs:

i) Coloque todo o código em um mesmo arquivo *.c

ii) A pontuação será dada somente para códigos que compilem e imprimam cada um dos itens.

\paragraph{3ª Questão:} (valor 4,0 pontos) O sistema de radar é capaz de plotar a coordenada do ponto mais alto alcançado pelo projétil lançado por peças de artilharia inimiga. Sabe-se que a artilharia inimiga está posicionada na abcissa cujo valor é igual a 2000 metros. A trajetória dos projetis inimigos pode ser modelada como uma curva do segundo grau. Desprezando-se o efeito das forças dissipativas, e sabendo que um projétil da artilharia inimiga atingiu o alvo localizado na coordenada (0,0), medido em metro, elabore um programa que implemente as seguintes funcionalidades:

\vspace{4mm}

a) Entre com a coordenada (em metro) do ponto mais alto alcançado pelo projétil inimigo, como
no exemplo abaixo (0,6 ponto):

\vspace{4mm}

// Exemplo:

\vspace{4mm}

//Entre com a abcissa do ponto mais alto alcançado pelo projétil inimigo em metro(s): 13.3;

\vspace{4mm}

//Entre com a ordenada do ponto mais alto alcançado pelo projétil inimigo em metro(s): 334

\vspace{4mm}

b) Imprima a coordenada do ponto mais alto alcançado pelo projétil inimigo, com três casas
decimais, como no exemplo abaixo (0,6 ponto):

\vspace{4mm}

// A coordenada do ponto mais alto alcançado pelo projétil foi de (10.300, 17.720).

\vspace{4mm}

c) Calcule e imprima a função do segundo grau que descreve a trajetória do projétil. As constantes
a, b e c da função do segundo grau devem apresentar 3 casas decimais, conforme exemplo abaixo (1,1 ponto):

\vspace{4mm}

//"A equação de segundo grau que descreve a trajetória do projetil inimigo é dada por $f(x) =
34.567 x^2 + 2.000 x + 5.100$ "

\vspace{4mm}

d) Imprima a coordenada completa da peça de artilharia inimiga com 3 casas decimais, conforme
exemplo abaixo (1,1 ponto):

\vspace{4mm}

// A peça de artilharia inimiga está na ordenada (2000.000, 13.000)

\vspace{4mm}

// "A ordenada da artilharia inimiga é (432.123, 234.200)";

\vspace{4mm}

e) A identação do programa (0,6 ponto).

\newpage
\subsubsection{2009 \original{https://drive.google.com/file/d/1TIAZpqabsvT8Dh59R7izADa1MDxonnuy/view?usp=sharing}}

\paragraph{Questão 1:}(0,8 ponto)\\
Realize a conversão de base conforme solicitado abaixo:
\begin{tasks}(1)
\task Da base $10$ para a base $2$:\\
$248=\_\_\_\_\_\_\_\_\_\_$\\
$169=\_\_\_\_\_\_\_\_\_\_$
\task Da base 2 para a base 10:\\
$10101110=\_\_\_\_\_\_\_\_\_\_$\\
$00110011=\_\_\_\_\_\_\_\_\_\_$
\task Da base 10 para a base 16:\\
$122=\_\_\_\_\_\_\_\_\_\_$\\
$2604=\_\_\_\_\_\_\_\_\_\_$
\task Da base 16 para a base 10:\\
$AB=\_\_\_\_\_\_\_\_\_\_$\\
$3A5B=\_\_\_\_\_\_\_\_\_\_$
\end{tasks}

\paragraph{Questão 2:}(0,5 ponto)\\
Declare as variáveis de acordo com o seu uso nos comandos abaixo, utilizando a menor quantidade de bytes necessária e sem perder conteúdo:\\
\_\_\_\_\_\_\_\_\_\_ i;\\
\_\_\_\_\_\_\_\_\_\_ x;\\
\_\_\_\_\_\_\_\_\_\_ s;\\
\_\_\_\_\_\_\_\_\_\_ c;\\
\_\_\_\_\_\_\_\_\_\_ d;\\\\
$i = 377$;\\
$x = i / 2.0$;\\ 
$s = 3 * i \;\%\; 5 $;\\
$c = \text{'}a\text{'}$;\\ 
$d = 23679.278690645$;

\paragraph{Questão 3:}(2,0 pontos)\\
Qual o resultado das expressões abaixo, considerando os seguintes valores e tipos das variáveis (diga o resultado ou assinale o que equivale a expressão lógica):
int i=2, j=10;
float r=9.9, s=1.0;
char c= 'a';
\begin{tasks}(1)
\task ( j \;\%\; i * 5 + i ) = \_\_\_\_\_\_\_\_\_\_
\task ( j \;\%\; ( i * 5 ) + i ) = \_\_\_\_\_\_\_\_\_\_
\task ( j \;\%\; ( i + 2 ) * 5 + i ) = \_\_\_\_\_\_\_\_\_\_
\task (s / 10 + r ) * 10.0 = \_\_\_\_\_\_\_\_\_\_
\task ( j / i / 5 / 2 ) = \_\_\_\_\_\_\_\_\_\_
\task ( j \& i ) = \_\_\_\_\_\_\_\_\_\_
\task ( j | i ) = \_\_\_\_\_\_\_\_\_\_
\task ( j $\leq$ i ) $\left [\;\;\; \right] $ Verdadeiro      $\left [\;\;\; \right]$ Falso
\task ( c == ( 'c' \&\& r $\geq$ s ) ) $\left [\;\;\; \right] $ Verdadeiro      $\left [\;\;\; \right]$ Falso
\task ( ( j = i ) || ( c == 'b' ) ) $\left [\;\;\; \right] $ Verdadeiro     $\left [\;\;\; \right]$ Falso
\end{tasks}

\paragraph{Questão 4:}(1,2 ponto)\\
De posse do trecho do código abaixo, complete a tela do computador (representada pelo quadriculado abaixo) com as saídas correspondentes. O primeiro “printf” já está parcialmente preenchido como exemplo (preencha cada caracter em um quadrado). 
Mostre onde o cursor vai ficar após o último “printf” ser executado. Não se esqueça de deixar os espaços em branco quando for o caso. Note também que existe um ponto final após cada número e está sem espaço para o \%f.\\
\\
float numero;\\
numero = 64.9822;\\
printf("SAIDA=\%7.3f .$\backslash$n", numero);\\
printf("SAIDA=\%-5.3f .$\backslash$n", numero);\\
printf("SAIDA=\%10.3f .$\backslash$n", numero $+$ 100.0);\\
printf("SAIDA=\%-8.3f .$\backslash$n", numero);\\
printf("SAIDA=\%d .$\backslash$n", (int) numero);\\
\\Resposta:


\begin{figure}[ht]
\centering
\includegraphics[width=16cm]{icvc2009q4.png}
\end{figure}



\paragraph{Questão 5:}(1,5 ponto)\\
Escreva um programa que leia um número inteiro qualquer e imprima na tela se o número digitado é par ou ímpar.

\paragraph{Questão 6:}(2,0 pontos)\\
Escreva um programa que calcule a área de triângulos retângulos, com as seguintes condicionates:

\begin{tasks}(1)
\task O programa, ao ser executado, solicitará a entrada dos catetos (um de cada vez) e da hipotenusa;
\task Em seguida o programa deverá verificar se os dados introduzidos formam realmente um triângulo retângulo. Caso os dados não forneçam um triângulo retângulo, deverá ser impressa uma mensagem na tela avisando o usuário do erro. Caso contrário deverá ser impressa uma mensagem conforme o exemplo abaixo:
\end{tasks}

$$\text{“O triangulo de lados 3.00, 4.00 e 5.00 eh retangulo e tem area 6.00.”}$$

\paragraph{Questão 7:}(2,0 pontos)\\
Faça um programa que leia uma data (dia, mês e ano) entre os anos de 1900 e 1999 (inclusive), e forneça como saída a data por extenso, contendo o dia da semana correspondente e a data fornecida, conforme o exemplo de execução abaixo:
\\
\\ENTRADA:       Forneca uma data (dd mm aa): 23 4 96\\
SAÍDA:              Data fornecida: Terca, 23 - 04 - 1996
\\
\\
\textbf{OBS: Utilize apenas os comandos “printf”, “scanf” e “if”.}
\\
\\Dicas:
\begin{itemize}
\item O dia 01/01/1900 "caiu" exatamente em um domingo! 
\item Calcule o número total de dias que se passaram desde essa data até a data fornecida! 
\item Para realizar esse cálculo, lembre-se de considerar os anos bissextos. 
\end{itemize}

\newpage


\subsection{VE}

\subsubsection{2014 \original{https://drive.google.com/file/d/1tsuTm669zW9XvWFLxBP6JY1pdDvHgpvz/view?usp=sharing}}



\paragraph{Questão 1:}(5,0 pontos)\\
Você, futuro engenheiro militar, recebeu do Cmt do IME a incumbência \textbf{UU} de desenvolver um programa em “C” para cálculo balístico que será usando pelo Cmt da 1ª Bateria de Artilharia Anti-Aérea - 1ª Bia AAAe, ao final desta VE.\\
 Sabe-se que as aeronaves inimigas sobrevoam continuamente no nosso espaço áereo, portanto este programa deve ser flexível de modo que ao receber do atirador a velocidade desenvolvida pela munição e a posição futura do alvo calcule o ângulo de inclinação do tubo do canhão e o tempo para se atingir esta posição afim que se possa abater a aeronave inimiga. A figura abaixo ilustra o problema: 

\begin{figure}[ht]
\centering
\includegraphics[width=16cm]{icve2014q1.png}
\end{figure}

A solução matemática encontrada pelos professores do IME, foi usar das equações de movimento de um projétil, sob aceleração da gravidade, suposta constante e isento da ação do ar para determinar com que ângulo de tiro $\theta$ (theta) pode-se acertar um alvo fixo, donde se concluiu que:
$$\dfrac{g.x_a^2}{2.v_0^2}.\mathrm{tan}^2\theta-x_a.\mathrm{tan}\theta+\dfrac{g.x_a^2}{2.v_0^2}+y_a=0; \;\;\;\;\;\;\;t=\dfrac{x_a}{v_0}\mathrm{sen}\theta$$
Essa equação do segundo grau apresenta duas soluções, portanto, dois possíveis e distintos ângulos de tiro (tenso e elevado) que permitirão ao projétil atingir o alvo. De modo que, o Cmt da 1ª Bia AAAe irá informar as coordenadas futuras da posição do alvo A$(x_a,y_a)$ e a velocidade inicial do projétil $v_0$, restando a você o desenvolvimento de um programa lhe forneça o(s) ângulo(s) $(\theta)$ para canhão em graus, minutos e segundos e o tempo a ser gasto pela munição para atingir a posição estimada da aeronave. Considere para efeitos de cálculos $\pi=3.1415$, $g=9,8m/s^2$ e use a função “atan(theta)” pré-definida em “C”. Como também é esperado resultados com números complexos imprima estes também.\\
Utlize para teste  A(500,860) m e $v_0= 160m/s$ e obtenha $\theta=660\, 52^{\prime}\, 6^{\prime\prime}$ com tempo de $2.87s$ para um tiro tenso e  $\theta=820 \,57^{\prime}\, 45^{\prime\prime}$ com tempo de $3.10s$ para um tiro elevado.

\paragraph{Questão 2:}(5,0 pontos)\\
Um aluno do primeiro ano do IME resolveu desenvolver um programa que criptografa números de oito dígitos e os recupera. Assim, dois alunos poderiam trocar números pela internet, e recuperá-los com segurança.\\
 Para isto, ele planejou a seguinte estratégia: \\
- considerar o \textbf{número} de oito dígitos como uma \textbf{matriz 2 x 4}. Assim, 12345678 se transforma em:
$$N
=
\begin{bmatrix}
n_{11} & n_{12} & n_{13} &n_{14} \\ 
a_{21} & a_{22} & a_{23} & a_{24}
\end{bmatrix}
=
\begin{bmatrix}
1 & 2 & 3 & 4\\ 
5 & 6 & 7 & 8
\end{bmatrix}$$
- considerar a \textbf{chave de criptografia} como uma \textbf{matriz quadrada de ordem 2 inversível}, ou seja com determinante diferente de zero. Por exemplo:
$$A=
\begin{bmatrix}
a_{11} &a_{12} \\ 
a_{21} & a_{22}
\end{bmatrix}
=
\begin{bmatrix}
1 & -2\\ 
3 & 0
\end{bmatrix}$$
Para realizar a criptografia, ele planejou o seguinte:\\
 - considerar o \textbf{número criptografado} como o resultado do produto da matriz correspondente a chave pela matriz correspondente ao número (Chave * Número = Número Criptografado).\\
Para restaurar o número criptografado, ele planejou a seguinte estratégia:\\
- multiplicar \textbf{a matriz inversa da chave} (que é uma matriz quadrada de ordem 2) pela matriz correspondente ao número criptografado (Inversa da Chave * Número Criptografado = Número).\\
Para obter a \textbf{matriz inversa da chave} (que é uma matriz quadrada de ordem 2), ele planejou utilizar determinantes e cofatores, de acordo com os seguintes passos:
\begin{enumerate}[label=(\arabic*)]
\item Obter o determinante da matriz quadrada de ordem 2, chamada A. O determinante dessa matriz é dado por: $det\,A=a_{22}*a_{11}-a_{21}*a_{12}$, onde $a_{ij}$ é o elemento i,j da matriz $A$
\item Obter a matriz $A^{\prime}$ (matriz dos co-fatores), substituindo cada elemento de A pelo respectivo cofator. Um cofator de $A$ é definido como $a_{ij}=(-1)^{i+j}.D_{ij}$, onde $a_{ij}$ é o elemento i,j da matriz $A$ e $D_{ij}$ é o determinante obtido da matriz derivada de $A$, excluindo os elementos da linha $i$ e da coluna $j$. Por exemplo, $D_{11}=(-1)^2\cdot det\,(q_{22})$.
\item Obter a matriz $\bar{A}$ (matriz adjunta), sendo que $\bar{A}$ é a matriz transposta de $A^{\prime}$, ou seja, a matriz obtida trocando as linhas pelas colunas de $A^{\prime}$
\item Obter a inversa de $A$, fazendo a multiplicação de $\bar{A}$ por $\dfrac{1}{det\,A}$(ou seja, multiplica-se a matriz adjunta pelo inverso do determinan te de $A$).
\end{enumerate}
Crie um programa em C que implemente o que foi planejado pelo aluno do primeiro ano para criptografia e recuperação do número criptografado. A interface do programa, apresentada a seguir, ilustra o que foi pedido. Serão considerados na correção:
\begin{enumerate}[label=(\arabic*)]
\item  Execução correta do programa obedecendo a interface dada (valor 1,0);
\item função que carrega as matrizes com dados fornecidos pelo usuário: (valor 1,0),\\
função que imprime uma matriz (valor 0,5);\\
função que faz o produto de duas matrizes (valor 1,0);\\
função que calcula a matriz dos co-fatores (valor 0,5);\\
função que calcula a matriz transposta (valor 0,5); e\\
função que calcula a matriz inversa (valor 0,5).
\end{enumerate}


\newpage
\subsubsection{2009 \original{https://drive.google.com/file/d/1zxrGHAACXl1NuK9TfKUsTgyYb0Nk4oKc/view?usp=sharing}}

\paragraph{Questão 1:}(5,0 pontos)\\
Elabore um programa que encontre o número primo mais próximo de 1.000.000 (um milhão). O programa deverá conter uma função, chamada “primo”, que informará se um número (recebido como parâmetro) é primo ou não. 
\begin{tasks}(1)
\task Resposta = \_\_\_\_\_\_\_\_\_\_ (valor 1,0) 
\task Execução do programa (valor 1,5)
\task Código fonte – main e função (valor 2,5)
\end{tasks}

\paragraph{Questão 2:}(5,0 pontos)\\
Elabore um programa que permita realizar as quatro operações (soma, subtração, divisão e produto) entre polinômios. 
Inicialmente o programa deve solicitar a entrada dos coeficientes de cada um dos polinômios. O primeiro polinômio terá grau 6 e o segundo grau 2. Em seguida deve aparecer um menu onde possa ser escolhida uma das 4 operações. No código fonte, estas operações devem chamar funções que realizam a operação selecionada entre o primeiro e o segundo polinômios e imprimir na tela o resultado correto: “Os coeficientes da SOMA sao: 1 * 2 * 0 * -18 * 1 * -3 * 1". O programa deve permitir uma nova operação com os mesmos polinômios até que seja digitada uma tecla que no menu indique o fim do programa.
\begin{tasks}(1)
\task Execução do programa – entrada de dados e menu (valor 0,5)
\task Execução da soma (valor 0,5)
\task Execução da subtração (valor 0,5)
\task Execução da divisão (valor 0,5)
\task Execução do produto (valor 0,5)
\task Código fonte – main e funções (valor 2,5)
\end{tasks}

\newpage
\subsection{VF}

\subsubsection{2016 \original{https://drive.google.com/file/d/10pS7YST8LVVbgfYnIcE-asuEZ8NFXkkc/view?usp=sharing}}

\paragraph{1ª Questão} (3,5 pontos)\\
O triângulo de Pascal é um triângulo numérico infinito podendo ter a lei de formação definida por números binomiais $\left( \begin{matrix}n \\ k \end{matrix}\right)$, onde $n$ representa o número da linha e $k$ representa o número da coluna, iniciando a contagem a partir do zero.

Definição de $\left( \begin{matrix}n \\ k \end{matrix}\right) \longrightarrow C_k^n =\left( \begin{matrix}n \\ k \end{matrix}\right) = \dfrac{n!}{k!(n-k)!}$
\\
Representação piramidal do Triângulo de Pascal:

\imgh{pascal}{8}


\noindent Formação do Triângulo de Pascal utilizando combinação simples representado por linha e coluna:

\imgh{icvf2016q1}{8}

Escreva um programa em C no qual o usuário entre com a quantidade de linhas do Triângulo de Pascal a serem impressas, utilizando para isso OBRIGATORIAMENTE uma função que calcule a combinação simples para auxiliar na construção e imprime o Triângulo de Pascal completo para esta quantidade de linhas (considerar a opção até a oitava linha).

\paragraph{2ª Questão} (3,5 pontos)\\
Uma das equações mais usadas durante o Ensino Médio e o Ensino Superior em cadeiras de ciências exatas é a "equaçãodo 2º grau". Considere a equação do 2º grau no formato: 
$$ax^2 +bx + c = 0$$ É comum ser necessário analisar se a equação tem raízes reais ou não. Isso pode ser feito observando o sinal do discriminante ( $\Delta = b^2 - 4ac$ ), de acordo com o critério a seguir:
\begin{itemize}
\item $\Delta > 0$: A equação tem 2 raízes reais;
\item $\Delta = 0$: A equação tem 1 raiz real;
\item $\Delta < 0$: A equação não tem raízes reais.
\end{itemize}

Sua tarefa nesta questão é escrever um programa em C que registre análises do tipo explicado acima. Você receberá um arquivo chamado "entradas.txt" (fornecido via EAD) que tem linhas no seguinte formato: $$1\quad 1 \quad -4 \quad 3$$ Cada linha tem 4 alores inteiros, que são respectivamente: (identificador, a, b, c). Cada linha dessas representa uma equação do 2º grau, com seus coeficientes. Ou seja, a linha acima é: $$\text{Equação 1: }x^2 - 4x + 3 = 0$$ \underline{Seu programa} deve ler o arquivo "entradas.txt" e, a partir da leitura dos dados, classificar as equações. Deverá ser gerado um arquivo "resultados.txt", em que são registrados os resultados das análises. \underline{Tudo o que for impresso no arquivo deverá} \underline{ser impresso na tela também}.

Em cada linha, deve ser gravado o identificador da equação e o resultado da análise. Ao final, deverá ser colocado um resumo com as quantidades de cada tipo de equação.

O arquivo de saída deve estar \underline{exatamente} com o seguinte formato:\\

\begin{tabular}{c}
\hline \\ \hline \\ \underline{Exemplo de arquivo de saída} \\ Identificador 1 - Número de raízes reais: 2\\ Idenfiticador 2 - Número de raízes reais: 0 \\ \dots \\ Total de equações com 2 raízes reais: 3 \\ Total de equações com 1 raiz real: 5 \\ Total de equações sem raízes reais:2 \\ \hline \\ \hline \\
\end{tabular}

A correção seguirá os critérios abaixo:
\begin{enumerate}[label=\alph*.]
\item Manipulação correta dos arquivos (0,5 ponto);
\item Cálculo correto dos discriminantes (0,5 ponto);
\item Totalizações corretas dos tipos de equação (1,0 ponto);
\item Arquivo de saída com valores corretos e no formato correto (1,0 ponto);
\item Impressão na tela idêntica ao arquivo (0,5 ponto).
\end{enumerate}

P1. Identação incorreta do código-fonte (perda de 0,5 ponto);

P2. Código-fonte que não compila: perda da pontuação integral da questão. 

\newpage
\subsubsection{2016 - CG \original{https://drive.google.com/open?id=134wK-upxqmRjF2w9kVJd2NJH53YIaB6c}}

\paragraph{1ª Questão:} (3,0 pontos)\\
\footnotesize{\textbf{ATENÇÃO: Em todos os itens é terminantemente proibida a utilização de variáveis que sejam do tipo float, double ou long double, ou funções que operem com esses tipos. Será atribuído grau 0.0 a itens que desrespeitem essa restrição.}}
\normalsize
Considere a seguinte definição de tipo:

\noindent \texttt{typedef structRacional\{ }

\texttt{long int n; /* numerador */}

\texttt{long int d; /* denominador */ \\ \} Racional;  }

\vspace{4mm}

Esse tipo representa um número racional cujo numerador é $n$ e o denominador é $d$.

Observe que essa é uma péssima representação para racionais, pois o mesmo número pode ser escrito de diversas maneiras diferentes. Por exemplo: O racional cujos campos valem $n=1$ e $d=2$ corresponde ao mesmo racional cujos campos valem $n=2$ e $d=4$.

Considere também a definição das seguintes funções para criar e exibir racionais:

\imgh{icvf2016cgq1}{12}

Pede-se implemetar as seguintes funções:

\begin{enumerate}[label=\alph*)]
\item (1,0 ponto) Uma função cujo protótipo é:

void \texttt{racional\_reduz(Racional *r);}

Essa função deve reduzir $r$. Onde reduzir $r$ significa obter um racional equivalente a $r$ cujo numerador e denomidanor não possuem nenhum divisor comum maior que 1. Exemplos:

A redução de 40/8 é 5/1
A redução de 30/9 é 10/3

\item (1,0 ponto) Uma função cujo protótipo é:

\texttt{Racional racional\_soma (Racional r1, Racional r2);}

Essa função deve retornar um racional que é a soma dos racionais $r1$ e $r2$. O resultado deverá estar na forma reduzida.

\item (1,0 ponto) Uma função \textbf{main} que deve escrever na tela uma aproximação racional para $\pi$, utilizando a seguinte igualdade: $$\pi = 4 - \dfrac{4}{3} + \dfrac{4}{5} - \dfrac{4}{7} + \dfrac{4}{9} - \dots$$
\end{enumerate}

\paragraph{Questão 2:}(3,5 pontos)\\
Os meses de junho e julho historicamente concentram atividades esportivas, como Copa do Mundo, Jogos Olímpicos de verão, Jogos Pan-americanos e Olímpiadas do IME. Suponha que você foi contratado por um organizador de competições esportivas para desenvolver um programa em C que leia um arquivo com informações sobre jogos de vôlei e concentre informações relevantes, como classificação, sets vencidos, sets perdidos, etc..\\
\\
O arquivo "resultados.txt", fornecido para a realização da prova, contém informações sobre os jogos realizados na última edição do evento, que contou com 8 times. O formato das informações no arquivo é o seguinte:
$$A\; 1\; B\; 3$$
$$C\; 3\; D\; 0$$
A primeira linha informa que no confronto entre os times A e B, o resultados foi de 1 set para o time A e 3 sets para o time B. Ou seja, o time B venceu o time A por 3x1. A segunda linha informa que o resultado do jogo entre os times C e D foi de 3x0 para o time C.\\
\\
A pontuação dos times em uma partida é dada da seguinte maneira:
\begin{itemize}
\item Caso a vitória de um dos times seja por 3x0 ou 3x1, o vencedor recebe 3 pontos, e o perdedor não recebe ponto algum.
\item Caso a vitória de um dos times seja por 3x2, o vencedor recebe 2 pontos, e o perdedor recebe 1 ponto.
\end{itemize}
A tabela de classificação é montada com os seguintes critérios, em ordem decrescente de relevância:
\begin{itemize}
\item Pontos conquitados;
\item Números de vitórias; e
\item Sets average(razão entre o total de sets vencidos e o total de sets perdidos).
\end{itemize}
O programa que você deve implementar tem os seguintes requisitos:
\begin{enumerate}[label=(\alph*)]
\item (0,5 ponto) Imprimir na tela o nome do time campeão, com sua pontuação final do torneio;
\item Imprimir na tela a tabela completa de classificação, com os seguintes campos:
\begin{enumerate}[label=(b.\arabic*)]
\item (0,75 ponto) Pontuação correta de todos os times;
\item (0,5 ponto) Número correto de vitórias de todos os times;
\item (0,5 ponto) Sets average correto de todos os times; e
\item (0,75 ponto) Classificação correta de todos os times.
\end{enumerate}
\item (0,5 ponto) Repetir os campos do item (b), porém, com a impressão em vez de ser feita na tela, feita em um arquivo chamado "tabela.txt".
\end{enumerate}

O programa não deve solicitar qualquer informação ao usuário, somente imprimir (em tela ou arquivo) os itens pedidos.\\
\\
Será atribuído grau $0,0$ à questão caso o aluno:
\begin{itemize}
\item Forneça como resposta código-fonte que não compile; OU
\item Forneça como resposta código-fonte que tenha funções adicionais não utilizadas na solução.
\end{itemize}

\paragraph{Questão 3:}(3,5 pontos)\\
Escreva um programa em C que peça ao usuário para inserir uma matriz 3x3 de valores inteiros e a partir desta entrada executa as seguintes operações:

\begin{tasks}(1)
\task Imprime a matriz 3x3 inserida pelo usuário (0,5 ponto).
\task Troca a primeira linha e imprime a nova matriz (0,75 ponto).
\task Com a matriz do item b) troca a primeira coluna com a terceira coluna e imprime a nova matriz (0,75 ponto).
\task Com a matriz do item c), calcula e imprime seu determinante (1,0 ponto).
\task Com a matriz do item c) calcula e imprime sua matriz transposta (0,5 ponto).
\end{tasks}

\newpage


\subsubsection{2014 \original{https://drive.google.com/file/d/18J773gOPZI43GO21XQ-v4pyE1Q_6gso2/view?usp=sharing}}

\footnotesize{\textbf{Esta prova possui 4 questões. As duas primeiras (mais fáceis) são obrigatórias. Para as duas
últimas o aluno deverá escolher SOMENTE UMA (só a 3a questão ou só a 4a questão), resolver e
enviar normalmente seguindo as orientações convencionais.}}
\normalsize
\paragraph{$1^a$ Questão:} (3,5 pontos)

Escreva um programa em C que simule o que acontece em uma conta
bancária. Este programa deve permitir ao usuário o registro das seguintes operações:\\

Depósito: código D

Saque: código S

Consulta ao saldo: código C\\

O programa deve criar uma estrutura “Conta”, que tenha os campos NumeroConta (inteiro) e Saldo
(float). Todas as contas devem começar com saldo igual a \$ 100.00. Todas as operações devem ser
registradas em um arquivo chamado “LogConta.txt”, que deve ter o seguinte formato:

\begin{center}
\begin{tabular}{lll}
D & 11223 & 30.60 \\
D & 12365 & 50.00 \\
S & 12365 & 20.48 \\
C & 11223 & 130.60 \\
C & 12365 & 129.52 \\
\end{tabular}
\end{center}

Este arquivo mostra, sucessivamente, os seguintes passos:\\

Depósito na conta 11223, no valor de \$ 30.60

Depósito na conta 12365, no valor de \$ 50.00

Saque na conta 12365, no valor de \$ 20.48

Consulta de saldo na conta 11223, cuja resposta foi \$ 130.60

Consulta de saldo na conta 12365, cuja resposta foi \$ 129.52\\

O programa deve permitir que o usuário execute seguidas operações, até que escolha a opção de sair
(deve estar presente no menu). Uma sugestão básica de menu é:\\

1 – Consulta saldo

2 – Depósito

3 – Saque

4 – Sair\\

A pontuação será dada de acordo com os seguintes itens:

P1) Declaração da estrutura da forma correta (1,0 ponto)

P2) Criação correta do menu de operação (0,5 ponto)

P3) Registro correto no arquivo das operações realizadas (2,0 pontos)\\

Haverá descontos em caso de ocorrer algum dos erros abaixo:

D1) Código-fonte não endentado (perda de 0,3 ponto)

D2) Ausência de verificação de erro na abertura do arquivo (perda de 0,3 ponto)

D3) Não fechar o arquivo ao final das operações (perda de 0,3 ponto)

D4) Arquivo que não compila (perda da pontuação integral da questão)

\paragraph{$2^a$ Questão:} (3,5 pontos)

Considere as seguintes matrizes:

$$ [A]=\left[
\begin{array}{cccc}
1 & 0 & 0 & 2 \\
1 & 2 & 0 & -1 \\
2 & 0 & -1 & 2 \\
\end{array}
\right], \ [B]= \left[ 
\begin{array}{ccc}
1 & 3 & 4 \\
1 & 2,25 & -3,25 \\
3 & 7 & 3 \\
1 & 1,5 & -1,5 \\
\end{array}
\right]
$$

Faça um programa que:
\begin{enumerate}[label=\alph*.]
\item Leia as matrizes A e B a partir de um arquivo (matrizes.txt);
\item Realize a multiplicação das matrizes;
\item Calcule o determinante da matriz resultante;
\item Imprima na tela a matriz resultante e o seu determinante, de maneira organizada.
\end{enumerate}
Para tanto, deve-se cumprir os seguintes requisitos:
\begin{enumerate}[label=\alph*.]
\item As matrizes A e B devem ser armazenadas na memória principal, bem como a matriz
resultante da multiplicação A x B (não é necessário utilizar alocação dinâmica de memória);
\item O cálculo do determinante deve ser realizado através do método da eliminação de Gauss. Esse
método pode ser descrito da seguinte maneira:

Dada uma matriz quadrada [C], com n linhas e n colunas, executa-se as seguintes operações
com linhas:

\textbf{- Variando i de 1 até n, para cada valor de i:}
\begin{enumerate}[label=\arabic*.]
\item Armazena-se o valor do elemento da diagonal da linha L$_i$, a$_{ii}$;
\item Divide-se todos os elementos da linha L$_i$ pelo seu elemento da diagonal (a$_{ii}$), inclusive o
próprio, tornando-o igual a um (a$_{ii}=1$):
\begin{center}
$L_i \leftarrow L_i/a_{ii}$
\end{center}
\item Variando k de i+1 até n, para cada valor de k, subtrair de cada elemento a$_{kj}$ da linha L$_k$, o
produto do seu elemento a$_{ki}$ (da coluna do elemento da diagonal da linha L$_i$) pelo elemento da
mesma coluna j da linha i, ou seja, a$_{ij}$. Dessa forma, o elemento a$_{ki}$, da coluna do elemento da
diagonal a$_{ii}$, tornar-se-á zero:
\begin{center}
$L_k \leftarrow L_k - a_{ki} . L_i$
\end{center}
\item Multiplicar todos os elementos da diagonal armazenados no item 1, obtendo-se o valor do
determinante.

\end{enumerate}
\newpage
\begin{figure}[ht]
\centering
\includegraphics[width=13cm]{icvf2014q2.png}
\end{figure}

\item A pontuação será dada de acordo com os seguintes itens:
\begin{enumerate}[label=\roman*)]
\item Cálculo correto e impressão na tela, de maneira organizada, da matriz resultante do produto
A x B (2,5 pontos);
\item Cálculo correto e impressão na tela do determinante da matriz resultante (1,0 ponto).
\end{enumerate}
\item Haverá descontos na pontuação caso ocorra algum dos erros abaixo:
\begin{enumerate}[label=\roman*)]
\item Código-fonte não endentado (perda de 0,2 ponto);
\item Ausência de verificação de erro na abertura do arquivo (perda de 0,2 ponto);
\item Não fechar o arquivo (perda de 0,2 ponto);
\item Não calcular a matriz resultante através do código-fonte (perda de 1,0 ponto);
\item Não imprimir na tela a matriz resultante (perda de 1,0 ponto);
\item Imprimir na tela a matriz resultante de maneira desorganizada (perda de 0,1 ponto);
\item Não calcular o determinante da matriz resultante através do código-fonte (perda de 0,8
ponto);
\item Não imprimir na tela o determinante da matriz resultante (perda de 0,8 ponto);
\item Código-fonte que não compila (perda da pontuação integral da questão).
\end{enumerate}
\end{enumerate}
\paragraph{$3^a$ Questão:} (3,0 pontos)

Estão sendo fornecidos para os alunos dois arquivos de texto (yin.txt e
yang.txt) que juntos comporão uma pergunta que precisa ser respondida. A mensagem buscada
precisará sofrer decriptografia em duas fases para ser lida. A primeira etapa é a decriptografia usando o
algoritmo proposto no “Trabalho Especial para a VF” (chave: ferias). Na segunda etapa, cada aluno
deverá criar um código que deve ler cada um destes arquivos, armazenar os conteúdos em dois vetores
específicos, chamar uma função que pegue caractere por caractere destes dois vetores e grave em um
terceiro vetor o caractere decodificado (esta função deve receber como parâmetros os dois caracteres e
retornar o caractere decodificado), imprimir o vetor decodificado na tela e finalmente responder à
pergunta. A decodificação é feita somando o número 3 à soma dos caracteres de mesmo índice dos
vetores lidos. A pergunta é formada por 124 caracteres. A figura abaixo ilustra o que deve ser feito:

\begin{figure}[ht]
\centering
\includegraphics[width=13cm]{icvf2014q3.png}
\end{figure}

A resposta é: \rule{5cm}{0.01cm}

Você irá entregar dois códigos para esta questão que serão nomeados da seguinte forma:\\

$1^o$) fase1\_$<$turma$>$\_$<$nomedeguerra$>$.c\\ Exemplo: fase1\_B\_passos.c (do aluno Passos).

$2^o$) fase2\_$<$turma$>$\_$<$nomedeguerra$>$.c\\ Exemplo: fase2\_B\_passos.c (do aluno Passos).\\


Para a $1^a$ Etapa você deve preparar um código de decriptografia conforme orientação do
trabalho especial (execução através da linha de comando etc).

Você poderá testar o seu código da $2^a$ Etapa separadamente. Para isso, use os arquivos de
teste “yinteste.txt” e “yangteste.txt”. Eles fornecerão a mensagem “Funcionou”. O código da 2a Etapa
também deve ser executado através da linha de comando da seguinte forma.
\begin{center}
fase2\_$<$turma$>$\_$<$nomedeguerra$>$ yin\_pósfase1.txt yang\_pósfase1.txt
\end{center}

\noindent Critérios a serem avaliados:
\begin{tasks}
\task Compilação do código (se não compilar não pontua)
\task Execução correta da 1a Etapa (decriptografia conforme o trabalho) (1,5 pontos)
\task Resposta da pergunta (0,5 ponto)
\task Execução correta da $2^a$ Etapa (1,0 ponto)\\
\end{tasks}

Observação: Lembrar que, para evitar problemas, arquivos gerados como texto devem ser lidos como
texto e arquivos gerados como binário devem ser lidos como binário.

\paragraph{$4^a$ Questão:} (3,0 pontos)

Faça um programa que resolva um sistema de equações do primeiro grau nas seguintes condições:
\begin{enumerate}[label=\arabic*.]
\item O programa deve ler duas equações do primeiro grau.
\item O programa deve ler cada equação do primeiro grau como uma string. Cada string poderá conter os
seguintes símbolos.\\

x $>>$ variável literal ;

y $>>$ variável literal ;

+ $>>$ operação de adição ;

- $>>$ operação de subtração ;

* $>>$ operação de multiplicação ;

/ $>>$ operação de divisão ;\\

\item Os números devem apresentar duas casas decimais .
\item O programa deve solicitar ao usuário que entre com a primeira equação do primeiro grau. (0,25
ponto)

Exemplo :

ENTRE COM A PRIMEIRA EQUAÇÃO DO PRIMEIRO GRAU :

5.00 + x / 2.00 + 3.00 = 2.00 * 4.00 + y * 2.00 + 4.00


\item O programa deve solicitar ao usuário que entre com a segunda equação do primeiro grau. (0,25
ponto)

Exemplo:

ENTRE COM A SEGUNDA EQUAÇÃO DO PRIMEIRO GRAU :

8.00 - y * 4.00 = 16.00 / 2.00 + x * 3.00

\rule{16cm}{0.01cm}

\item O programa deve mostrar na tela as strings lidas nos itens anteriores sem espaço conforme exemplo
abaixo: (0,5 ponto)

Exemplo:

5.00+x/2.00+3.00=2.00*4.00+y*2.00+4.00

8.00-y*4.00=16.00/2.00+x*3.00

\rule{16cm}{0.01cm}

\item O programa deve mostrar na tela o sistema de equações lido no seguinte formato: (0,5 ponto)

a*x + b*y = m

c*x + d*y = n

Exemplo:

1.00*x-4.00*y=8.00

3.00*x+4.00*y=0.00

\rule{16cm}{0.01cm}

\item O programa deve mostrar na tela se as equações são linearmente independentes ou não. (0,5 ponto)

Exemplo:

As equações são linearmente independentes.

\rule{16cm}{0.01cm}

\item O programa deve mostrar na tela a solução do sistema, se houver: (1,0 ponto)

Exemplo:

x = 2.00

y = -1.50

\end{enumerate}

\newpage
\subsubsection{2009 \original{https://drive.google.com/file/d/18u7tsv8qBYGjGvuOF-52TcH-SkMpRu2q/view?usp=sharing}}


\paragraph{Questão 1:}(2,0 pontos)\\
Durante um exercício de combate simulado as forças oponentes foram divididas em Exército Azul e Exército Vermelho. O IME ficou encarregado de apoiar o Exército Azul em suas operações. Os alunos do Básico foram destacados para integrar uma Companhia de Guerra Eletrônica, a qual estava com a missão de interceptar as comunicações inimigas. Durante o exercício, os alunos verificaram que o inimigo estava trocando muitos programas em C, e que uma função, chamada “fff”, abaixo, estava sendo muito utilizada em tais programas.  Assim sendo, o Comandante do Exército Azul solicitou aos alunos que:
\begin{tasks}(1)
\task Explicassem o que essa função faz. (Sua resposta deve ter no máximo três linhas.)
\task Determinassem a saída desta função para valores de x e y iguais a 111 e 216, respectivamente.
\end{tasks}

int fff(int x, int y)\\
\{\\
if $(x==0)$ return y;\\
$\;\;\;\;\;\;\;$else return fff $( (x\&y)<<1, (x\;\hat{}\;y) )$;\\
\}

\paragraph{Questão 2:}(1,0 ponto – 0,2 cada)\\
Complete a tabela escrevendo o que será impresso pelo programa abaixo caso você digite os valores mostrados na coluna esquerda desta tabela.
\texttt{\#include\,$<$stdio.h$>$}

\texttt{int main(void)}

\texttt{\{}

\texttt{\qquad int m;}

\texttt{\qquad float rr=123.3456789;}

\texttt{\qquad scanf("\%d", \&m);}

\texttt{\qquad switch (m)}

\texttt{\qquad \{}
      
\texttt{\qquad \qquad case 1: case 3: case 5: case 7: case 8: case 10: case 12:}

\texttt{\qquad \qquad \qquad printf ("31:\%10.3f\textbackslash n", rr);}

\texttt{\qquad \qquad case 2:}

\texttt{\qquad \qquad \qquad printf ("28ou29:\%1.5f \textbackslash n", rr);}

\texttt{\qquad \qquad \qquad break;}

\texttt{\qquad \qquad case 4: case 6: case 9: case 11:}

\texttt{\qquad \qquad \qquad printf ("30:\%1.5f \textbackslash n", rr);}

\texttt{\qquad \qquad default:}

\texttt{\qquad \qquad \qquad printf (" invalido\textbackslash n");}

\texttt{\qquad return 0;}

\imgh{icvf2009q2}{14}


\paragraph{Questão 3:}(1,0 ponto – 0,5 cada erro encontrado e corrigido)\\
Em uma prova de IC I, um aluno do primeiro ano do IME deveria criar uma função para verificar se um número é primo (retornando 1) ou não (retornando 0). Entretanto, na pressa, ele cometeu dois pequenos erros que invalidaram a função. Sua missão é: assinalar as duas linhas onde esses erros se encontram e escrever ao lado a linha correta. Para auxiliar sua busca pelos erros, esta função está sempre informando que qualquer número, passado como parâmetro é primo, e a correção do segundo erro, apenas, vai fazer que a função diga que qualquer número passado como parâmetro não é primo.\\

\{\\
(\;\;)\;int verificador=0;\;\;\; \_\_\_\_\_\_\_\_\_\_\\
(\;\;)\;int resto;\;\;\; \_\_\_\_\_\_\_\_\_\_\\
(\;\;)\;int divisor=2;\;\;\; \_\_\_\_\_\_\_\_\_\_\\
(\;\;)\;float raiz;\;\;\; \_\_\_\_\_\_\_\_\_\_ \\
\\
(\;\;)\;while(verificador\neq0 \&\& divisor$<$=raiz);\;\;\; \_\_\_\_\_\_\_\_\_\_ \\
\{\\
(\;\;)\;\qquad resto=num\%divisor;\;\;\; \_\_\_\_\_\_\_\_\_\_\\
(\;\;)\;\qquad if(resto==0)\;\;\; \_\_\_\_\_\_\_\_\_\_\\
(\;\;)\;\qquad \qquad verificador=0;\;\;\; \_\_\_\_\_\_\_\_\_\_\\
(\;\;)\;\qquad else\;\;\; \_\_\_\_\_\_\_\_\_\_ \\
(\;\;)\;\qquad \qquad verificador=1;\;\;\; \_\_\_\_\_\_\_\_\_\_ \\
(\;\;)\;\qquad divisor+=1;\;\;\; \_\_\_\_\_\_\_\_\_\_ \\
\}\\
(\;\;)\;if(verificador=1)\;\;\; \_\_\_\_\_\_\_\_\_\_ \\
(\;\;)\;\qquad return 1;\;\;\; \_\_\_\_\_\_\_\_\_\_ \\
(\;\;)\;else\;\;\; \_\_\_\_\_\_\_\_\_\_ \\
(\;\;)\;\qquad return 0;\;\;\; \_\_\_\_\_\_\_\_\_\_ \\
\}

\paragraph{Questão 4:}(3,0 pontos)

Os alunos do primeiro ano, durante as aulas de IC, não podem usar a internet para acessar e-mail, MSN, Orkut e outros sites do gênero para poder prestar atenção na aula. Isto permitiu que os irmãos A e B, adquirissem conhecimentos suficientes até para se divertirem nos tempos vagos com a linguagem C. Como exemplo, eles desenvolveram um programa para jogar a tradicional forca. O professor gostou da idéia e passou, como trabalho, implementar seu próprio jogo de forca.//
Assim sendo, sua missão é implementar este jogo seguindo as seguintes condicionantes:

\begin{enumerate}[label=\arabic*]
\item A palavra a ser digitada por um dos jogadores deve ter no máximo 20 caracteres.
\item A palavra a ser digitada por um dos jogadores deve ter no máximo 20 caracteres.
\item Após digitada esta palavra, a tela deve ser limpa para que o outro jogador possa assumir o computador e tentar adivinhar a palavra digitada pelo primeiro jogador.
\item Supondo que o primeiro jogador tenha entrado com a palavra “ENGENHARIA”, a tela que o segundo jogador verá será a seguinte:\\
\noindent\fbox{%
    \parbox{\textwidth}{%
****
Erros permitidos: 5
Letras Digitadas:
Entre com uma letra: \_    }%
}
\item Como visto acima, o segundo jogador só poderá errar até 5 letras antes de ser “enforcado”.
\item Quando ele digitar uma letra e pressionar o “enter”, a tela anterior deverá ser apagada e impressa uma nova tela como a seguinte (supondo que ele digitou a letra “O”):\\
\noindent\fbox{%
    \parbox{\textwidth}{%
****
Erros permitidos: 4
Letras Digitadas: O – 
Entre com uma letra: \_}%
}
\item Quando for digitada uma letra correta (supondo a letra “E”):\\
\noindent\fbox{%
    \parbox{\textwidth}{%
E*E***
Erros permitidos: 4
Letras Digitadas: O – E –
Entre com uma letra: \_}%
}
\item Supondo que o aluno descubra a palavra:\\
\noindent\fbox{%
    \parbox{\textwidth}{%
ENGENHAR*A
Erros permitidos: 4
Letras Digitadas: O – E – N – G – H – A – R – 
Entre com uma letra: I
PARABENS. VOCE VENCEU!!!!
Fim de jogo. \_}%
}
\item Supondo que ele não saiba qual é a palavra:\\
\noindent\fbox{%
    \parbox{\textwidth}{%
E*E***
Erros permitidos: 1
Letras Digitadas: O – E – Y – W – T – 
Entre com uma letra: X
VOCE FOI ENFORCADO!!!!
Fim de jogo. \_}%
}
\end{enumerate}
Barema da Questão:
\begin{tasks}(1)
\task Entrada da palavra com a verificação: 0,5
\task Trocando os asteriscos pela letra correta: 0,5
\task Término do programa de forma adequada: 0,5
\end{tasks}


Código fonte: Valor 1,5 sendo
\begin{tasks}(1)
\task Identação e Organização: 0,5
\task Teste para prosseguir no programa se a palavra possui até 20 caracteres ou se é a mesma que foi digitada: 0,5
\task Gerenciamento da palavra que está sendo formada pelo segundo jogador e contagem dos erros: 0,5
\end{tasks}
\paragraph{Questão 5:}(3,0 pontos)\\
Escreva um programa que leia as coordenadas x e y de dois pontos e com estes pontos possa gerar as coordenadas de outros pontos de tal forma que todos os pontos informados pelo programa, juntamente com os dois iniciais, representem os vértices de determinadas figuras geométricas que foram selecionadas em um menu conforme descrito abaixo nas condicionantes:
\begin{enumerate}[label=\arabic*]
\item Entrada de dados:\\
\noindent\fbox{%
    \parbox{\textwidth}{%
Entre com as coordenadas x e y do primeiro ponto:  23  47
Entre com as coordenadas x e y do segundo ponto:  29  47. \_}%
}
Considere estas variáveis como float.
\item Após a entrada de dados, a tela deve ser apagada e aparecer o seguinte menu:
\noindent\fbox{%
    \parbox{\textwidth}{%
1 – Triangulo Equilátero
2 – Quadrado 
3 – Hexágono
4 – Fim do Programa
Escolha uma opção: \_}%
}
\item O programa deverá considerar que os pontos se sucedem no sentido horário e que os pontos 1 e 2 (primeiro e segundo pontos respectivamente) são paralelos ao eixo x.
\item Deverá ser utilizada uma struct COORDENADA com as coordenadas de cada ponto e cada figura geométrica deverá ter uma struct contendo as “n” struct COORDENADA necessárias para construir a figura. Ex: Hexágono: struct HEXAGONO, com 6 variáveis do tipo struct COORDENADA.
\item Declare todas as variáveis do tipo struct como globais.
\item Crie funções para cada um dos itens do menu, exceto para o item 4.
\item Use “switch – case” para implementar o menu.
\item Não esqueça que antes de sair do programa, eu posso acessar os itens do menu quantas vezes eu quiser, ou seja, o programa só deverá ser encerrado quando for escolhida a opção 4.
\item A saída dos dados em cada item do menu tem que ser da seguinte forma (não esqueçam de limpar a tela antes de exibir as coordenadas e quando voltar ao menu):\\
\\
As coordenadas do triangulo são:\\
Ponto 1: 1.00 , 5.00\\
Ponto 2: 5.00 , 5.00\\
Ponto 3: 3,0 , 1.54\\
\\
Barema da Questão:\\
Execução do programa: Valor 1,5 sendo 0,5 para cada forma geométrica.\\
\\
Código fonte: Valor 1,5 sendo
\begin{tasks}(1)
\task Identação e Organização: 0,5
\task Confecção do menu chamando as funções: 0,5
\task Utilização das struct: 0,5
\end{tasks}
\end{enumerate}
























































\newpage
\section{Conservando o IME Compendium: Introdução ao \LaTeX}

\subsection{Por que usar o \LaTeX?}

Usar \LaTeX \, é uma excelente maneira de organizar e formatar seu texto, utilizando comandos de alto nível (fáceis de utilizar). Por possibilitar a estruturação do texto de forma mais lógica, apresenta muitas vantagens em relação aos editores de texto mais tradicionais, como o Word e o LibreOffice.

\subparagraph{Beleza:} Os textos escritos em \LaTeX \, são mais bonitos (na opinião de muitos dos nossos professores, inclusive), sendo facilmente reconhecidos e distinguidos dos feitos em Word.

\subparagraph{Organização:} Os textos escritos em editores de texto comuns tem como inconveniente a \underline{incompatibilidade}: arquivos em Word ficam mal formatados em outros editores, como LibreOffice, ou no visualizador do Google Drive.

Já \LaTeX \, tem a vantagem de ser \underline{sempre igual}, dependendo apenas do código que é preparado. Além disso, é possível escrever em \LaTeX \, tanto no computador (com programas como o TexStudio) quanto na nuvem (como o Overleaf e o ShareLaTeX).

\subparagraph{Praticidade:} É muito mais fácil adaptar um texto em \LaTeX: basta adicionar ou excluir algumas linhas de código. Além disso, é possível deixar "comentários" (linhas de código fantasma, caso queira testar novas coisas).

\subparagraph{Outras funcionalidades:} Nesse arquivo, foi possível fazer muitas coisas que o Word seria incapaz de fazer:

\begin{itemize}
\item Fazer um sumário clicável, que facilita a navegação pelo documento;
\item Adicionar equações complexas, o que não é muito prático de se fazer no Word ou em outros editores;
\item Conectar diferentes partes do texto, através de âncoras.
\end{itemize}

\subsection{Estrutura do texto}

O texto em \LaTeX \, é estruturado de forma semelhante a HTML, então é possível fazer uma analogia. Isso porque ambos dividem seu texto em duas partes: um \textbf{preâmbulo} (semelhante ao \textit{header} em HTML), onde estão informações referentes à estruturação do texto, e o \textbf{corpo do texto} (semelhante ao \textit{body} em HTML), onde o texto a ser escrito é inserido de fato.

O texto fica escrito da seguinte forma (é possível alterar as opções, de acordo com a necessidade):

\vspace{5mm}
\begin{caixa}{Exemplo de código} 
\textbackslash documentclass[12pt, a4paper]\{article\}

\textbackslash usepackage[utf8]\{inputenc\}

\textbackslash usepackage\{amsfonts\}

\textbackslash usepackage\{amssymb\}

\textbackslash usepackage\{amsmath\}

\textbackslash newcommand\{\textbackslash comando1 \}[1]\{\textbackslash textbf\{\#1\} \}

\textbackslash newcommand\{\textbackslash comando2 \}\{\textbackslash LaTeX demais\}

\vspace{5mm}

\textbackslash begin\{document\}

Seu texto aqui: \textbackslash comando1\{essa parte em negrito\}. Pra escrever LaTeX eu uso \textbackslash comando2. Observe como eu escrevo uma integral do modo dificil: 

\$\$ \textbackslash displaystyle \textbackslash int \textbackslash limits \_\{-\textbackslash frac\{\textbackslash pi\}\{2\} \} \textasciicircum\{\textbackslash frac\{\textbackslash pi\}\{2\}\} f(x) \textbackslash mathrm\{d\}x  \$\$

\textbackslash end\{document\}
\end{caixa}

Observe que o \textbf{preâmbulo} começa com o \textbf{documentclass}, onde é inserido o tamanho da fonte, o tipo de papel, algumas outras informações (como  número de colunas, margem etc), e principalmente o tipo de documento (utilizamos \textit{article}, mas existem outros, como \textit{book}, por exemplo). Observe que nesse espaço colocamos as \textit{packages} utilizadas (à medida que se torna necessário, você pode inserir novas sem precisar editar o corpo do texto), bem como as \textit{macros} definidas por \textbackslash newcommand (é muito útil para encurtar seu texto, quando for utilizar comandos muito complexos).

O \textit{environment} limitado por \textbackslash begin\{document\} e \textbackslash end\{document\} define o \textbf{corpo do texto}. Aqui você pode inserir os comandos importados das \textit{packages} e das \textit{macros} criadas.



O texto acima, renderizado, fica assim:

\begin{caixa}{Texto renderizado}
Seu texto aqui: \textbf{essa parte em negrito}. Pra escrever LaTeX eu uso \LaTeX demais. Observe como eu escrevo uma integral do modo dificil:
$$\displaystyle\int\limits_{-\frac{\pi}{2}}^{\frac{\pi}{2}} f(x) \dd x$$
\end{caixa}

\newpage
\subsection{Packages e Comandos}

Ao longo da criação do texto, a demanda por certas funcionalidades exigiu a utilização de packages. Listaremos as utilizadas aqui, bem como os comandos principais de cada um.

\subsubsection{Tipos de \textit{header}}

Foram utilizados os comandos \texttt{\textbackslash section\{\}}, \texttt{\textbackslash subsection\{\}} e \texttt{\textbackslash subsubsection\{\}} para ordenar o texto hierarquicamente (aparecerão em \texttt{\textbackslash tableofcontents}).

É possível, ainda, utilizar \texttt{\textbackslash paragraph\{\}} e \texttt{\textbackslash subparagraph\{\}}, para enfatizar um parágrafo, ou iniciar um.

O comando \texttt{\textbackslash tableofcontents} é utilizado para gerar automaticamente o sumário, de acordo com as seções abertas. Com a utilização do \texttt{hyperref}, o sumário pode ser personalizado para ser clicável.

\subsubsection{Listagem de Itens}

\begin{enumerate}[label=\roman*.]
\item \texttt{\textbackslash begin\{itemize\}}: para listagem sem ordem definida.

Ex.: \begin{caixa2}
\textbackslash begin\{itemize\}

\textbackslash item Item1

\textbackslash item Item2

\textbackslash end\{itemize\} & \begin{itemize}
\item Item1
\item Item2
\end{itemize}
\end{caixa2}

\item \texttt{\textbackslash begin\{enumerate\}}: para listagem com ordem definida.

Ex.: 

\begin{caixa2}
\textbackslash begin\{enumerate\}

\textbackslash item Item1

\textbackslash item Item2

\textbackslash end\{enumerate\} & \begin{enumerate}[label=\arabic*.]
\item Item1
\item Item2
\end{enumerate}
\end{caixa2}

\begin{caixa2}
\textbackslash begin\{enumerate\}[label=..\textbackslash alph*)]

\textbackslash item Item1

\textbackslash item Item2

\textbackslash end\{enumerate\} & \begin{enumerate}[label=..\alph*)]
\item Item1
\item Item2
\end{enumerate}
\end{caixa2}

\subparagraph{\textit{Labels} permitem modificar o estilo de enumeração. Usamos comumente os seguintes:} 

\begin{tasks}(2)
\task \texttt{\textbackslash alph*} - (a, b, c \dots)
\task \texttt{\textbackslash Alph*} - (A, B, C \dots)
\task \texttt{\textbackslash arabic*} - (1, 2, 3 \dots)
\task \texttt{\textbackslash roman*} - (i, ii, iii \dots)
\end{tasks}

É possível personalizar a enumeração do \texttt{enumerate} utilizando o argumento \texttt{[label= algumacoisa ]}. Por exemplo, experimente por \texttt{[label= \textbackslash arabic*o item:]} e veja o resultado.

\item \texttt{\textbackslash begin\{tasks\}}: permite enumerar em grade.

Ex.: 

\begin{caixa2}
\textbackslash begin\{tasks\}

\textbackslash task Item1

\textbackslash task Item2

\textbackslash end\{tasks\} & \begin{enumerate}[label=\alph*)]
\item Item1
\item Item2
\end{enumerate}
\end{caixa2}

Se colocarmos \texttt{\textbackslash begin\{tasks\}(3)}, por exemplo, teremos os itens horizontalmente:

\begin{tasks}(3)
\task 1o
\task 2o
\task 3o
\end{tasks}

\end{enumerate}

\subsubsection{Packages utilizadas}

\begin{itemize}

\item \texttt{inputenc} - package base para \LaTeX. Sua explicação mais profunda pode ser encontrada \href{https://ctan.org/pkg/inputenc}{aqui}.
\item \texttt{amsfonts/amssymb} - necessárias para a utilização de caracteres especiais, tais como $\cap$ e $\cup$, setas diversas $\rightarrow$, $\Leftrightarrow$, enfim, uma lista desses símbolos pode ser vista \href{http://milde.users.sourceforge.net/LUCR/Math/mathpackages/amssymb-symbols.pdf}{aqui}.

\item \texttt{textcomp} - utilizado para símbolos extras, como o de graus \textdegree  com o comando \texttt{\textbackslash textdegree}.

\item \texttt{amsmath} - utilizado para ferramentas matemáticas em geral, como integrais, frações etc.
\item \texttt{mathtools} - utilizado para complementar a package acima. Um exemplo de comando utilizado foi o \texttt{\textbackslash ell} para o símbolo $\ell$.

\item \texttt{chemformula} - utilizado para símbolos de química. Por exemplo, temos $$\ch{O2 (g) + 4 H^+ (aq) + 4 Br^- (aq) -> 2 H2O (l) + 2 Br2 (l)}$$ escrito mais facilmente com o comando \$ \textbackslash ch\{O2 (g) + 4 H $\widehat{ }$ + (aq) + 4 Br $\widehat{ }$ - (aq) -$>$ 2 H2O (l) + 2 Br2 (l)\} \$ (facilita bastante, acredite\dots).

\item \texttt{setspace} - para regular o espaçamento entre linhas.

Utilizamos o comando \texttt{\textbackslash onehalfspacing} para espaçamento 1,5 entre linhas (padrão ABNT).

\item \texttt{indentfirst} - para lidar com indentação (parágrafos automáticos). Para evitar parágrafos automáticos, utilizamos o comando \texttt{\textbackslash noindent}.

\item \texttt{enumitem} - permite modificar o layout de listas. O melhor exemplo disso é a possibilidade de utilizar labels no enumerate.

\item \texttt{tasks} - para o uso do comando \texttt{\textbackslash tasks}.

\item \texttt{xcolor/afterpage} - utilizados para modificar a cor da página da capa. O \texttt{afterpage} permitiu modificar apenas uma página. O comando utilizado foi \texttt{\textbackslash pagecolor\{Cor\}\textbackslash afterpage\{\textbackslash nopagecolor\}}.

\item \texttt{graphicx} - necessário para inserção de imagens. O processo será explicado em uma seção à parte.

\item \texttt{\hyperlink{ops}{hyperref}} - permite adicionar hyperlinks e \hypertarget{botao}{âncoras}, para facilitar a navegação pelo PDF do documento. Será explicado em uma seção à parte.

\item \texttt{geometry} - utilizado para modificar o layout da página (por exemplo, a margem).

\item \texttt{multicol} - utilizado para possibilitar separação de partes do documento em multicolunas. Abaixo, exemplo de como foi utilizado em uma prova de cálculo, a partir do environment \texttt{multicols}:

\footnotesize \texttt{\textbackslash begin\{multicols\}\{2\}\textbackslash setlength\{\textbackslash columnsep\}\{1.5cm\}
\textbackslash setlength\{\textbackslash columnseprule\}\{0.2pt\}}
\normalsize

\end{itemize}

\subsubsection{\$Matemática\$}

Todos os códigos de matemática devem ser inseridos em ambiente matemático. Existem 2 tipos: \$ \texttt{equação na mesma linha} \$ e \$\$ \texttt{equação centralizada} \$\$. Por exemplo, podemos escrever a fórmula de Bháskara assim: $x = \dfrac{-b \pm \sqrt{b^2 - 4ac}}{2a}$ ou assim $$x = \dfrac{-b \pm \sqrt{b^2 - 4ac}}{2a}$$

O texto no ambiente matemático fica zoado. 

Tipo, $O texto no ambiente matematico fica zoado$ e $$O texto no ambiente matematico fica zoado$$ então é útil utilizarmos comandos que insiram texto. Temos \texttt{\textbackslash mathrm\{\}} que insere em roman, \texttt{\textbackslash mathit\{\}} que insere em \textit{itálico}, \texttt{\textbackslash mathbf\{\}} que insere em \textbf{bold}, \texttt{\textbackslash mathbb\{\}} que insere $\mathbb{BLACKBOARD\, BOLD\, CHARACTERS}$ etc.

Ficaria assim: $$\mathrm{roman , } \sqrt{3} \mathbf{ bold , }\left[\begin{matrix}1 & 2 & 3 \\ 1 & -3 & 7\end{matrix}\right]\mathit{italico, } x\in \mathbb{R}$$

Os comandos de matemática, como funções, integrais, frações etc, podem ser encontrados \href{https://en.wikibooks.org/wiki/LaTeX/Mathematics}{aqui}.

Alguns exemplos: $\sqrt[n]{x}$ (\$ \texttt{\textbackslash sqrt[n]\{x\}} \$), $\left[\dfrac{a}{b}\right]$ (\$ \textbackslash left[\textbackslash dfrac\{a\}\{b\} \textbackslash right] \$), $\langle u, v \rangle$ (\$ \textbackslash langle u, v \textbackslash rangle  \$). 

\subsubsection{Tabelas, matrizes, environments especiais}

Existem estruturas que requerem mais esforço para serem construídas. Mostraremos algumas.

\begin{caixa2}
\texttt{\$\textbackslash begin \{cases\}}

\texttt{x + y = 1 \textbackslash \textbackslash  x - y = 3}

\texttt{\textbackslash end \{cases\}\$}& $\begin{cases} x+y = 1 \\ x - y = 3\end{cases}$
\end{caixa2}

\begin{caixa2}
\texttt{\$\textbackslash begin \{matrix\}}

1 \& 2 \textbackslash \textbackslash  3 \& 4

\texttt{\textbackslash end \{matrix\}\$}& $\begin{matrix}1 & 2 \\ 3 & 4 \end{matrix}$
\end{caixa2}

\begin{caixa2}
\texttt{\$\textbackslash left ( \textbackslash begin \{matrix\}}

1 \& 2 \textbackslash \textbackslash  3 \& 4

\texttt{\textbackslash end \{matrix\} \textbackslash right ) \$}& $\left( \begin{matrix}1 & 2 \\ 3 & 4 \end{matrix} \right)$
\end{caixa2}

Para fazer tabelas, não é necessário usar o ambiente matemático. Temos o seguinte código:

\texttt{ \textbackslash begin \{tabular\}\{c||c|c|||\}}

\texttt{Eu \& Você \& Zooboomafoo \textbackslash\textbackslash  \textbackslash hline}

\texttt{Abra \& Kadabra \& Alakazam}

\texttt{\textbackslash end \{tabular\} }

\vspace{5mm}

Isso gera a tabela abaixo:

\begin{tabular}{c||c|c|||}
Eu & Você & Zooboomafoo \\ \hline Abra & Kadabra & Alakazam
\end{tabular}

\vspace{5mm}

Observe que todos os ambientes estão limitados por \texttt{\textbackslash begin\{\}   \textbackslash end\{\}}.

É possível, ainda, criar seu próprio environment, inserindo no preâmbulo o comando \texttt{\textbackslash newenvironment\{\}\{\}}, caso queira personalizar alguma tabela ou parte do texto.

\subsubsection{Imagens com \texttt{graphicx}}

Para inserir imagens, devemos utilizar o ambiente \texttt{figure}. A imagem da capa, por exemplo, foi colocada assim:

\noindent\texttt{\textbackslash begin\{figure\}[h]}

\texttt{\textbackslash centering}

\texttt{\textbackslash includegraphics[width=8cm]\{logo.png\}}

\noindent\texttt{\textbackslash end\{figure\}}



A imagem \textit{logo.png}, entretanto, deve estar disponível no diretório do projeto (no caso do Overleaf, fizemos o \textit{upload} do arquivo, em \textsf{Project $>$ file\dots $>$ upload from\dots}).

Os argumentos utilizados estão relacionados ao posicionamento da imagem e ao tamanho, e toda a explicação da package pode ser encontrada \href{https://pt.sharelatex.com/learn/Inserting_Images}{aqui}.

No caso do Compendium, foi feita uma macro para facilitar a inserção da imagem, \texttt{\textbackslash imgh\{nomeDoArquivoEmPNG\}\{tamanho\}}. Por exemplo, bastaria ter utilizado \texttt{\textbackslash imgh\{logo\}\{8\}} para inserir a imagem da capa.

\begin{figure}[h]
\centering
\href{http://desciclopedia.org/wiki/Instituto_Militar_de_Engenharia}{\includegraphics[width=1cm]{logo.png}}
\end{figure}

\subsubsection{Ancoragem com \texttt{hyperref}}

O \texttt{hyperref} possui utilidades muito bacanas, e que são essenciais para que o Compendium seja \hypertarget{ops}{prático}. Ele \textbf{permite a utilização de links, faz com que os itens do sumário tornem-se hiperlinks, permite ancorar partes diferentes do documento} etc.

Por exemplo, clique no \hyperlink{botao}{BOTÃO} e veja onde vai parar\dots

Os comandos dessa package são os seguintes:

\begin{itemize}
\item \texttt{\textbackslash href\{link.com\}\{texto-âncora do link\}} - torna uma palavra um link. Por exemplo, \href{google.com}{texto-âncora do link}.

\item \texttt{\textbackslash hyperref\{Id da Âncora\}\{texto clicável\}} - torna um texto um caminho para outro ponto do documento, o \textbf{hypertarget} de mesmo id.

\item \texttt{\textbackslash hypertarget\{Id da Âncora\}\{Destino\}} - é o destino do \textbf{hyperref} de mesmo id, isto é, o local que surge na tela ao clicar no link.
\end{itemize}

É possível modificar algumas configurações da package. No caso do Compendium, inserimos no preâmbulo o seguinte código:

\noindent\texttt{\textbackslash hypersetup\{}

\texttt{    colorlinks=true, }
    
\texttt{    linktoc=all,}
    
\texttt{    linkcolor=black,}
    
\texttt{    urlcolor=blue,}
    
\texttt{\}}

A ideia dos argumentos utilizados é intuitiva (\texttt{linktoc} diz respeito ao \textit{table of contents}, ou seja, sobre o sumário clicável), e podem ser alterados como bem quiserem.

\newpage
\subsection{Utilizando o \texttt{newcommand}}

Alguns comandos corriqueiros, como integrais, podem ser muito extensos para escrever. Para minimizar o esforço, é útil criar atalhos para os comandos, atráves do \texttt{newcommand}. A sintaxe é simples: basta escrever:

\texttt{\textbackslash newcommand\{ \textbackslash comando\}[numeroDeArgs]\{ o que faz \}}

No Compendium, utilizamos várias macros, à medida que foi sendo necessário. Listaremos:

\begin{itemize}
\item \texttt{\textbackslash newcommand\{\textbackslash R\}\{\textbackslash mathbb\{R\}\}} - para o símbolo dos reais $\mathbb{R}$, basta usar \texttt{\textbackslash R}

\item \texttt{\textbackslash newcommand\{\textbackslash modu\}[1]\{\textbackslash vert \#1 \textbackslash vert\}} - é o símbolo para módulo, basta fazer \texttt{\textbackslash modu\{x\}} para ter $\modu{x}$

\item \texttt{\textbackslash newcommand\{\textbackslash norm\}[1]\{\textbackslash Vert \#1 \textbackslash Vert\}} - é o símbolo para norma, basta fazer \texttt{\textbackslash norm\{x\}} para ter $\norm{x}$

\item \texttt{\textbackslash newcommand\{\textbackslash sen\}\{\textbackslash mathrm\{sen\}\}} - insere o símbolo de seno em Roman: $\sen\, x$ (ficaria $senx$ sem o comando)

\item \texttt{\textbackslash newcommand\{\textbackslash Ln\}\{\textbackslash ell\textbackslash mathrm\{n\}\}} - facilita a inserção do símbolo de $\Ln\, x$

\item \texttt{\textbackslash newcommand\{\textbackslash dd\}\{\textbackslash mathrm\{d\}\}} - para o símbolo do diferencial

\item \texttt{\textbackslash newcommand\{\textbackslash intdef\}[2]\{\textbackslash displaystyle\textbackslash int\textbackslash limits\_\{\#1\} $\widehat{}$ \{\#2\}\}} - este é, de longe, o melhor exemplo de como as macros facilitaram nossa vida. A integral $$\intdef{0}{e} \Ln x \dd x$$ pode ser escrita como \texttt{\$\$\textbackslash intdef\{0\}\{e\} \textbackslash Ln x \textbackslash dd x\$\$}, mas sem as macros ficaria:

\texttt{\$\$\textbackslash displaystyle\textbackslash int\textbackslash limits\_\{0\} $\widehat{}$ \{e\} \textbackslash ell\textbackslash mathrm\{n\} x \textbackslash mathrm\{d\}x\$\$}

Quando se escreve por volta de 5 integrais por prova, parece bizu ter macros a seu favor!


\item \texttt{\textbackslash newcommand\{\textbackslash original\}[1]\{\textbackslash tiny\textbackslash href\{\#1\}\{Original\}\textbackslash normalsize\}} - insere a palavra "original" em tamanho Tiny (pequeno), com o link para algo (utilizamos para linkar com as provas originais no Drive Compendium)

\item \texttt{\textbackslash newcommand\{\textbackslash img\}[3]\{}

\texttt{\textbackslash begin\{figure\}[\#3]}

\texttt{\textbackslash centering}

\texttt{\textbackslash includegraphics[width=\#2cm]\{\#1.png\}}

\texttt{\textbackslash end\{figure\}}

\texttt{\}}

Insere imagens mais facilmente.

\end{itemize}



\newpage

\subsection{Como colocar as provas no Compendium?}

As provas do \textit{Compendium} foram organizadas em disciplinas por seções, tipo de prova em subseções e ano em subsubseções. Para inserir uma prova, basta colocá-la na seção e subseção corretas, e abrir a nova subseção (que corresponde à nova prova).

Por exemplo, ao inserir a VE 1 de Cálculo 2018, você abrirá uma \textbackslash subsubsection\{2018\} na subseção "VE 1" da seção "Cálculo", e escreverá o texto. Para linkar com a prova original, use a macro \textbackslash original\{linkdaprova.com\}.

\begin{caixa}{Código da VE 1 de Cálculo 2018}
{\textbackslash subsubsection\{2018 \textbackslash original\{https://drive.google.com/drive/u/\} \}



\textbackslash paragraph\{Questão 1:\}(X pontos)\textbackslash\textbackslash

Texto da questão

\textbackslash paragraph\{Questão 2:\}(X pontos)\textbackslash\textbackslash

Texto da questão

\textbackslash paragraph\{Questão 3:\}(X pontos)\textbackslash\textbackslash

Texto da última questão

\textbackslash newpage}

\end{caixa}

Voilà! Agora é colocar a mão na massa, e manter esse documento sempre atualizado!

\newpage

\section{Soluções}

\subsection{Cálculo I}

\subsubsection {VE 1 - 2019}

\paragraph{Questão 1}

\begin{tasks}(1)
    
\task $$\dfrac{\sen{(x)}}{x - \sen{(x)}} = \dfrac{\dfrac{\sen{(x)}}{x}}{1 - \frac{\sen{(x)}}{x}} = \dfrac{1 - t}{t}$$
Fazendo $t = 1 - \dfrac{\sen{(x)}}{x}$; por limite fundamental: $x \rightarrow 0 , t \rightarrow 0$.
Logo: $$ \lim\limits_{x \to 0} \left(\dfrac{\sen{(x)}}{x}\right)^{\frac{\sen{(x)}}{x - \sen{(x)}}} = \lim\limits_{t \to 0} (1 - t)^{\frac{1 - t}{t}} = \lim\limits_{t \to 0} (1 - t)^{\frac{1}{t} - 1} = \lim\limits_{t \to 0} \dfrac{(1 - t)^\frac{1}{t}}{1 - t} $$ 
Fazendo: $ v = -t$ ;  $t \rightarrow 0$ ; $ v \rightarrow 0 $

Logo: $$ \lim\limits_{t \to 0} \dfrac{(1 - t)^\frac{1}{t}}{1 - t} = \lim\limits_{v \to 0} \dfrac{1}{(1 + v)(1 + v)^{\frac{1}{v}}} = \dfrac{1}{e} $$
Pois: $(1 + v)^\frac{1}{v} \rightarrow e $ quando $v \rightarrow 0 $ (Limite Fundamental)

\task Analisando os limites laterais: $$ \lim\limits_{x \to 0^{+}} \dfrac{\sen {(4 \lfloor x \rfloor + x)}}{x} = \lim\limits_{x \to 0^{+}} \dfrac{\sen {(x)}}{x} = 1$$  (Limite Fundamental)
$$ \lim\limits_{x \to 0^{-}} \dfrac{\sen {(4 \lfloor x \rfloor + x)}}{x} =  \lim\limits_{x \to 0^{-}} \dfrac{\sen {(-4 + x)}}{x} = - \infty $$
Logo: $\nexists \lim\limits_{x \to 0^{-}} \dfrac{\sen {4 \lfloor x \rfloor + x}}{x}$


\end{tasks}


\paragraph{Questão 2:}

\begin{tasks}(1)
\task Reescrevendo a expressão do limite: $$ax - b \left( \dfrac{x^{1000} + 1}{x^{999} + 1} \right) = \dfrac{ax^{1000} + bx^{999} + ax + b - 1}{x^{999} + 1} $$
Mas : $$\lim\limits_{x \to +\infty } \dfrac{P(x)}{Q(x)} = 0 \textnormal{   se   } \delta(Q) > \delta(P). $$
Logo, $(a - 1) = 0 \Rightarrow a = 1 \textnormal{ e } b = 0.$

\task Falso. \\
Seja: $$f(x) = x^{4/3} \Rightarrow f'(x) = \dfrac{4}{3} x^{1/3} \Rightarrow f''(x) = \dfrac{4}{9} x^{-1/3}$$
$f'(x)$ existe e é continua $\forall x \in \mathbb{R}$. 
Porém , $f''(x)$ não é derivável em $x = 0$.
$$f''(0) = \lim\limits_{h \to 0} \dfrac{f'(0 + h) - f'(0)}{h} = \lim\limits_{h \to 0} \dfrac{f'(h)}{h} = \lim\limits_{h \to 0} \dfrac{4}{3}\dfrac{h^{1/3}}{h}$$ Absurdo!!!
    
\end{tasks}

\paragraph{Questão 4:}
Como $P''(a) \neq 0, P$ é no mínimo de grau 3.
$$Q(x) = P(x+a) - P(a) - xP'(xy + a)$$ Derivando duas vezes, temos:
$$Q''(x) = P''(x+a) - 2yP''(xy + a) - xy^2 P'''(yx + a)$$
$$Q(0) = Q'(0) = 0 \textnormal{ e } Q''(0) = (1 - 2y) P''(a)$$
Como $P''(a) \neq 0 \textnormal{  e  } (1 -2y) \neq 0, \forall y \in \left( 0, \frac{1}{2}\right)$; então $Q''(0) \neq 0$.\\
Consequentemente $Q(x) = x^2 R(x)$, onde $R(x)$ é um polinomio de grau ímpar.\\ 
Note também que $R(0) \neq 0$, pois, caso contrário $R(x)$ não teria termo independente, o que é condição necessária para que $Q''(0) \neq 0$. \\
Desde que $R(x)$ é um polinomio de grau ímpar, ele tem pelo menos um zero real,  e desde que $R(0) \neq 0$ , então garante-se que existe $\delta$ tal que $R(\delta) = 0$.\\
Segue então que $Q(\delta) = \delta R(\delta) = 0.$ \\
Substituindo a expressão de $Q(x) = P(x+a) - P(a) -xP'(xy +a) $, temos: $$\dfrac{P(\delta + a) - P(a)}{\delta} = P'(y \delta + a)$$
Basta fazer $b = \delta + a $ e a prova se faz.

\newpage



\subsubsection{VE 1 - 2017}

\paragraph{Questão 4} \hfill \scriptsize \hypertarget{calculove12017q4ida}{* }\hyperlink{calculove12017q4volta}{Voltar ao enunciado} \normalsize

\begin{tasks}(1)
\task Observe que, em $\lim\limits_{x\to 0^+} \dfrac{f(x)-1}{x} = k$, temos uma fração cujo denominador tende a zero, mas cujo limite é um número real. Logo, deve-se ter, necessariamente, $\lim\limits_{x\to 0^+}f(x) - 1 = 0$, ou seja, $$\lim\limits_{x\to 0^+} f(x) = 1$$
\task Observe que a fração pedida possui uma derivada envolvida. Lembre-se que $$f^{\prime}(x) = \lim\limits_{h\to 0} \dfrac{f(x+h) - f(x)}{h} = \lim\limits_{h\to 0^+} \dfrac{f(x+h) - f(x)}{h}$$ e que $$f(x+h) = f(x) f(hf(x))$$
Logo: $$f^{\prime}(x) = \lim\limits_{h\to 0^+} \dfrac{f(x) f(hf(x)) - f(x)}{h} = \lim\limits_{h\to 0^+} f(x)\dfrac{f(hf(x)) - 1}{h}$$ $$ = f^2(x)\lim\limits_{hf(x)\to 0^+} \dfrac{f(hf(x)) - 1}{hf(x)} = f^2(x)$$ Logo, $f^2(x) = f^{\prime}(x)$, ou seja,  $$\dfrac{f^2(x)}{f^{\prime}(x) } = 1$$

\end{tasks}

\newpage

\subsubsection{VC - 2018}

\paragraph{Questão 2:}

O Teorema do Valor Médio (TVM) é aplicável se: 
    \begin{tasks}(2)
        \task $x^{2/3}$ contínua em $\left[-a, a\right]$
        \task $x^{2/3}$ derivável em $\left(-a, a\right)$        
    \end{tasks}

    De (a), sabemos que funções do tipo $x^{a/b}$, com $a$ e $b \in \mathbb{Z}$, são contínuas em $\mathbb{R}$.\\
    De ii, $$x^{2/3} = \dfrac{2}{3}x^{-1/3} = \dfrac{2}{3 \sqrt[3]{x}}. $$
    Não há $f'(x)$ para $x=0$, mas $0 \in \left(-a, a\right)$.
    Portanto, ii não é satisfeito.\\
    Logo, o TVM não é aplicável!

\paragraph{Questão 3:}
$$\dfrac{x}{a} + \dfrac{y}{b} = 1 \Rightarrow \dfrac{1}{a} + \dfrac{2}{b} = 1 \Rightarrow a = \dfrac{1}{1 - \dfrac{2}{b}}$$
Seja $d$ a distância entre os pontos $A$ e $B$, temos:
$$d^2 = (a-0)^2 + (0-b)^2 \Rightarrow d^2 = \dfrac{b^2}{(b-2)^2} + b^2 $$
Derivando a expressão:
$$2dd' = 2 \left(\dfrac{b}{b-2}\right)\left(\dfrac{b-2-b}{(b-2)^2}\right)  + 2b$$ 
Queremos $d' = 0$:
$$2 \left(\dfrac{b}{b-2}\right)\left(\dfrac{-2}{(b-2)^2}\right)  + 2b = 0 \Rightarrow b - \dfrac{2b}{(b-2)^3} + 2b = 0$$

Para $b \neq 0:$ $$1 - \dfrac{2}{(b-2)^3} = 0$$
Logo: $b = 2 + \sqrt[3]{2}$

Para provar que o ponto gera um mínimo na função, analisamos se $d'' \leq 0 $ (A cargo do leitor).

Daí: $$a = \dfrac{b}{b-2} = \dfrac{2 + \sqrt[3]{2}}{\sqrt[3]{2}}$$
Assim: $$r: \dfrac{x}{\dfrac{2 + \sqrt[3]{2}}{\sqrt[3]{2}}} + \dfrac{y}{2 + \sqrt[3]{2}} = 1$$

\paragraph{Questão 4:}
Suponha que exista a reta, então as duas relações abaixo são verdadeiras:
\begin{tasks}(2)
    \task $m_r = \dfrac{-1}{m_a}; m_r = \dfrac{-1}{m_b}$
    \task $\dfrac{\cosh{a} - \sinh{b}}{a-b} = m_r$    
\end{tasks}

De (a) temos $m_a = m_b \Rightarrow \sinh{a} = \cosh{b}$. Abrindo as contas:

$$\dfrac{e^a - e^{-a}}{2} = \dfrac{e^b + e^{-b}}{2} \Leftrightarrow e^a - \dfrac{1}{e^a} = e^b + \dfrac{1}{e^b} \Leftrightarrow e^a - e^b = \dfrac{1}{e^a} + \dfrac{1}{e^b} \Leftrightarrow e^a - e^b = \dfrac{e^a + e^b}{e^a \cdot e^b} $$

Analisando $K = \dfrac{e^a + e^b}{e^a \cdot e^b} $, temos que $K > 0$  $\forall a,b \in \mathbb{R}.$
Logo $e^a - e^b > 0 \Rightarrow a > b .$

De (b) e (a), temos:$$\dfrac{\cosh{a} - \sinh{b}}{a-b} = \dfrac{-1}{\cosh{b}} \Leftrightarrow \cosh{a}\sinh{b} - \sinh{b}\cosh{b} = b - a $$
Ajeitando a expressão, encontramos: $$\dfrac{1}{2}\cosh{(a+b)} + \dfrac{1}{2}\cosh{(a-b)} - \dfrac{1}{2} \sinh{2b} = b-a$$
Contudo, $\cosh{x}$ é crescente em $\mathbb{R}^+$ e $\cosh{x} > \sinh{x}, \forall x \in \mathbb{R}$ o que faz $\cosh{a+b} > \sinh{2b}.$ Seja $D$ o lado direito da equação temos então $D>0$ contudo pelo passo (a) concluimos que $(b-a)<0$  e então temos um absurdo!


\newpage

\subsubsection{VF - 2017}

\paragraph{Questão 2} \hfill \scriptsize \hypertarget{calculovf2017q2ida}{* }\hyperlink{calculovf2017q2volta}{Voltar ao enunciado}\normalsize

 Chame $u = \sen x$ e $v = \sen y$, e faça $f(t) = \cos \left( \mathrm{arcsen} t  \right)$, de forma que $f(u) = \cos x$ e $f(v) = \cos y $. Logo: $$uf(v) + vf(u) = \sen x \cos y + \sen y \cos x = sen(x+y) \le 1$$

Logo, para resolver a integral, façamos $t = \sen x \implies \dd t = \cos x dx$, e $f(t) = \cos x$. Ou seja: $$\int\limits_0^1 f(t) \dd t = \int\limits_0^{\pi / 2} \cos^2 x \dd x = \left(\dfrac{\sen (2x)}{4} + \dfrac{x}{2} \right)\Biggr |_0^{\pi / 2}$$ $$= \dfrac{\pi}{4} \le \dfrac{4}{5}$$

\newpage


\subsection{Química I}

\subsubsection{VE 1 - 2019}

\paragraph{1ª Questão:} (2,0 pontos) \\ [0.2cm]


Se a Equação dos Gases Ideias reflete um princípio de conservação de energia, é correto afirmar que a EOS para líquidos expressa por 
\begin{equation*}
    V_{m}(T, p) = c_{1} + c_{2}T + c_{3} T^{2} - c_{4} p - c_{5}pT    
\end{equation*}

\begin{enumerate} [label = (\alph*)]
    \item descreve o volume ocupado por uma fase condensada em função da temperatura;
    \item descreve o volume ocupado por uma fase condensada em função da temperatura e pressão;
    \item \textcolor{red}{é baseada no princípio de conservação de energia e descreve a relação entre funções termodinâmicas de estado};
    \item correlaciona variáveis termodinâmicas e prevê uma relação entre temperatura de fusão e ebulição; ou 
    \item associa constantes determinadas experimentalmente com o comportamento de fases.
\end{enumerate} \\ 

\textcolor{red}{Resposta: \\
Evidencia-se a correlação entre ambas as Equações de Estado, tanto dos Gases Ideais quanto dos líquidos, uma vez que que utilizam conservação de energia.\\
Assim, a EOS: $ V_{m}(T, p) = c_{1} + c_{2}T + c_{3} T^{2} - c_{4} p - c_{5}pT$ descreve a relação entre as funções termodinamicas de estado: pressão (P), temperatura (T), volume (V) e numero de mols(n), as quais tem medida relativamente fácil e, portanto, são empregadas na descrição de estado termodinamico.}

\newpage

\paragraph{2ª Questão:} (2,0 pontos) \\ [0.3cm]


A forma funcional $T = T \left(\langle \epsilon_{tr} \rangle\right)$, pode ser corretamente interpretada como
\begin{enumerate} [label = (\alph*)]
    \item uma relação entre a energia translacional média expressa em termos de temperatura;
    \item \textcolor{red}{a definição de temperatura como uma função de energia translacional média};
    \item uma expressão de proporcionalidade entre a temperatura e energia translacional;
    \item uma relação entre a temperatura média de um gás e sua energia translacional; ou
    \item uma forma derivada do princípio de conservação de energia.  \\ 
\end{enumerate}

\textcolor{red}{Resposta: \\
A definição de temperatura é  a capacidade de transmissão de Energia Térmica do corpo mais quente para o menos quente. Assim, utilizando do princípio de conservação de energia, a forma funcional do enunciado explicita a definição de temperatura como função da Energia Translacional média.} \\[0.1cm]



\paragraph{3ª Questão:} (2,0 pontos) \\ [0.3cm]


Sobre as Leis de difusão e efusão de Graham, as quais relacionam 
fluxo de partículas de um gás com a sua massa (molar ou molecular), 
é correto afirmar que 
\begin{enumerate}[label = (\alph*)]
    \item somente podem ser obtidas empiricamente, a partir de séries de experimentos com gases de massas diferentes;
    \item uma vez obtidas empiricamente ou a partir de TCG, apresentam formas matematicamente incompatíveis;
    \item somente podem ser aplicadas a sistemas cujo comportamento é ideal; 
    \item \textcolor{red}{relacionam fluxo, coeficiente de difusão e massa molar dos gases, bem como suas velocidades médias}; ou
    \item assumem que os gases são indistinguíveis entre si e os tratam como partículas da mesma natureza química. \\    
\end{enumerate}

\textcolor{red}{Resposta: \\
As leis de difusão e efusão de Graham são utilizadas para descrever os movimentos dos gases. Por meio delas temos uma relaçao de proporcionalidade entre as velocidades médias de efusão dos gases e suas massas molares, além de na difusão existir o fluxo de determinados gradientes, o coeficiente de difusão e as características do movimento do gás que é descrito na alternativa D.} \\ [0.5cm]


\paragraph{4ª Questão:} (2,0 pontos) \\ [0.5cm]


Observando a função de distribuição radial entre as distâncias interatômicas em água líquida, percebe-se que a intensidade de $g(r)$ é muito maior para as 
distâncias \chemfig{O-H} que para \chemfig{H-H} e \chemfig{O-O}, apresentando um máximo a $\SI{0,98}{ \angstrom}$. Esse comportamento é uma consequência

\begin{enumerate}[label = (\alph*)]
    \item da repulsão entre átomos do mesmo elemento;
    \item de flutuações de densidade eletrônica induzidas por momentos de dipolo transientes;
    \item da orientação das moléculas sob um campo eletromagnético externo;
    \item \textcolor{red}{da direcionalidade das interações específicas entre moléculas de água no líquido}; ou
    \item da auto-ionização da água, a qual gera íons hidrônio nas formas Eigen e Zundel. \\ 
    
\end{enumerate}

\textcolor{red}{Resposta: \\
O que provoca a variação na função de distribuição radial entre as distâncias interatômicas da água é a variação entre as intensidades das interações específicas entre as moléculas de água. A ligação de \chemfig{O-H} é uma ponte de hidrogênio que é uma interação muito mais forte do que as forças existentes entre átomos do mesmo elemento, como \chemfig{H-H} e \chemfig{O-O}, o que explica a mudança na intensidade de $g(r)$. } \\[0.5cm]

\newpage

\paragraph{5ª Questão:} (2,0 pontos)\\ [0.5cm]

Sabendo que uma avaliação geométrica da função de distribuição de Maxwell-Boltzmann permite relacionar a taxa das colisões moleculares $\dfrac{dN_{w}}{dt}$ com uma parede de área $A$, por
\begin{equation*}
    \dfrac{1}{A} \dfrac{dN_{w}}{dt} = \dfrac{1}{4} \dfrac{N}{V} \langle v \rangle = \dfrac{1}{4} \dfrac{pN_{a}}{RT} \left( \dfrac{8RT}{\pi M} \right)^{1/2},
\end{equation*} é correto afirmar que:

\begin{enumerate} [label = (\alph*)]
    \item Um maior tempo de observação causa um aumento da taxa de colisões;
    \item Dois gases com a mesma massa molar apresentarão sempre a mesma taxa de colisões;
    \item O sentido físico da equação não é alterado substituindo-se $\langle v \rangle$ por $\langle v^{2} \rangle^{1/2}$ ;
    \item \textcolor{red}{A geometria da parede impactada pelas moléculas de um certo gás não altera a relação entre as variáveis $p$ e $T$}; ou
    \item Mantendo-se a taxa de colisões constante, um aumento na temperatura está relacionado inversamente a um aumento de pressão.
    
\end{enumerate}

\textcolor{red}{Analisando a última igualdade da relação acima, é possível perceber que a relação entre as variáveis $p$ e $T$ não dependem da geometria da parede, ela se relaciona com as constantes $R, \pi $ e $N_{A}$ e varia com $M, V_{gás}$, o que é indicado na letra D.\\
$\cdot$ A taxa de colisões não depende do tempo. (A) incorreta\\
$\cdot$ Depende das condições $(V, T)$ em que forem analisados. (B) incorreta\\
$\cdot$ $\langle v \rangle \neq \langle v^{2} \rangle^{1/2}$ pois $\langle v \rangle$ é a média dos valores. (C) incorreta}

\end{document}



\end{document}